\documentclass[a4paper]{article}

\usepackage[T1]{fontenc}
\usepackage[latin1]{inputenc}
\usepackage[ngerman]{babel}
\usepackage{times, graphicx, currvita, hyperref, longtable, float, caption, subcaption}
\usepackage{amsmath,amsfonts,amssymb, amsthm}%, dsfont, bbm, mathtools, stmaryrd}
%\usepackage{fancyhdr}
%\usepackage{color}
%\usepackage[english]{babel}
\usepackage{paralist}
%\usepackage{algorithmic}
\usepackage{wasysym}	% verschiedene Symbole, siehe http://rpi.edu/dept/arc/training/latex/LaTeX_symbols.pdf
\graphicspath{{Bilder/}}

\theoremstyle{definition}
\newtheorem{definition}{Definition}[section]
\newtheorem*{Def1}{Definition}
\newtheorem*{ex}{Example}

\newtheoremstyle{Tobi}{10pt}{}{}{}{\bf}{ }{\newline}{}
% die parameter sind: {name}{platz drueber}{platz drunter}{schriftart text}{einrueckung}{schriftart kopf}{Punktierung des Kopfes}{Platz zwischen kopf und text}

\theoremstyle{Tobi}
\newtheorem{prop}[definition]{Proposition}
\newtheorem*{Pro}{Proposition}
\newtheorem{theorem}[definition]{Satz}
%\newtheorem*{le}{Lemma}
\newtheorem{lemma}[definition]{Lemma}
\newtheorem{cor}[definition]{Korollar}
\newtheorem*{conj}{Conjecture}
\newtheorem{obs}[definition]{Beobachtung}

\theoremstyle{remark}
\newtheorem*{que}{Questions}
\newtheorem*{claim}{Claim}
\newtheorem*{note}{Note:}
\newtheorem*{remark}{Remark}


\newcommand{\R}{\mathbb{R}}
\newcommand{\N}{\mathbb{N}}
\newcommand{\F}{\mathbb{F}}
\newcommand{\1}{\mathbbm{1}}
\newcommand{\Rn}{\mathbb{R}^n}
\newcommand{\La}{\mathcal{L}}
\newcommand{\D}{\mathcal{D}}
\newcommand{\Om}{\Omega}
\newcommand{\pa}{\partial}
\newcommand{\C}{\mathcal{C}}
\newcommand{\ph}{\varphi}
\newcommand{\sub}{\subseteq}

\setlength{\parindent}{0pt}

\begin{document}

\title{Domino Tilings auf dem Torus}

\author{Tobias Buchwald}
%\maketitle

%\large{Bachelorarbeit bei Prof. Dr. Stefan Felsner}


\begin{center} 
\Huge{Domino Tilings auf dem Torus}\\ \vspace{12 cm}
\Large{Bachelorarbeit\\ bei Prof. Dr. Stefan Felsner}\\ \vspace{1cm}
\large{Vorgelegt von Tobias Buchwald}\\
\large{am Fachbereich Mathematik der \\Technischen Universit�t Berlin}\\
\vspace{2cm}
\large{Berlin,  \today}

\end{center} 


\textbf{Erkl\"arung}\\

Hiermit versichere ich an Eides statt, dass ich die vorliegende Masterarbeit selbst\"andig und eigenh\"andig sowie
ausschlie\ss lich unter Verwendung der aufgef\"uhrten Quellen und Hilfsmittel angefertigt habe. \\


Berlin, den \today
\newline

\rule[-0.2cm]{10cm}{0.5pt}

\textsl{Tobias Buchwald} 
\newpage

\tableofcontents
\newpage
\section{Einleitung} Der Gegenstand dieser Bachelorarbeit sind Domino Tilings. Pflasterungen/Tilings von Fl�chen mit regelm��igen Polygonen kann man in der Mathematik aus unterschiedlichen Blickpunkten betrachten. Domino Tilings sind ein spezieller Fall von regul�ren Tilings - ein regelm��iger Gittergraph (dreieckig, quadratisch oder hexagonal)  soll durch eine Menge von entsprechenden Polyominos �berdeckt werden, in unserem Fall geht es  also um die �berdeckung eines quadratisches Gitters mit Dominos. 

Man kann Tilings aus unterschiedlich Blickwinkeln betrachten. So haben Kasteleyn \cite{Kasteleyn19611209} sowie Temperley und Fisher \cite{TempFish} die Anzahl m�glicher Domino Tilings (bzw perfekter Matchings auf dem Gittergraphen) durch die Forschung �ber Dimere (Zweiatomige Molek�le) motiviert berechnet, w�hrend sich z.B. Conway und Lagarias dem kombinatorischen Problem der Existenz von Tilings in \cite{ConwayLagar} mittels gruppentheoretischen Mitteln n�hern. Bestimmte Varianten von Tiling-Problemen sind sogar als NP-vollst�ndig oder gar unberechenbar (Wang-Tilings).\\

Ein bekanntes Resultat �ber Domino Tilings einer einfach zusammenh�ngenden Region in der Ebene ist, dass diese die algebraische Struktur eines distributiven Verbands tragen (siehe \cite{Remila2004409}). Dabei sind die �berg�nge lokale Transformationen (Flips) bei denen zwei benachbarte Dominos gedreht werden. Es wurden auch bereits Verallgemeinerungen davon untersucht, so z.B. die Struktur von  Domino Tilings in der Ebene, die nicht beschr�nkt sind, siehe \cite{fernique}. Insbesondere tritt dort auch das Problem auf, dass die lokalen Flips aus dem einfachen planaren Fall nicht mehr ausreichen um jede Orientierung zu erzeugen. 

Es gibt f�r Domino Tilings verschiedene Verallgemeinerungen, so werde ich die Beziehung zu Matchings und zu $\alpha$-Orientierungen betrachten, f�r die man im planaren Fall ebenfalls einen distributiven Verband erhalten kann. 
Betrachtet man Domino Tilings auf dem Torus, stellt man fest dass die lokalen Flips hier leider nicht mehr ausreichen, um �berhaupt jede Orientierung zu erzeugen. In einem �hnlichen Spezialfall torischer $\alpha$-Orientierungen (in dem Fall dass jeder Knoten im Gitter Ausgrad 2 hat) hat Kolja Knauer in \cite{Knauer07partialorders} gezeigt, dass dort schon eine Basis des Kreisraumes gen�gt. Das motiviert die Frage, ob man in �hnlicher Weise eine m�glichst kleine Menge von Transformationen auf torischen Domino Tilings angeben kann, durch die man jede Orientierung erh�lt. Ich zeige einige Beispiele auf, die zeigen dass bestimmte Klassen von Kreisen dazu ungeeignet/nicht ausreichend sind, und gebe ein Kriterium an mit dem man dies nachweisen kann. 
\section{Einf\"uhrende Definitionen}Wir betrachten das zweidimensionale ganzzahlige Gitter, d.h. den (unendlichen) planaren Graphen $\mathcal{G}=(V,E)$ mit den Knoten $V=\mathbb{Z}^2$ und Kanten $E=\{e=\{x,y\}| x,y\in V,\, (|x_1-y_1|=1\land x_2=y_2)\lor (x_1=y_1\land |x_2-y_2|=1)\}$.

Wir nennen den Raum zwischen den vier benachbarten Punkten $(x,y),(x+1,y),(x,y+1),(x+1,y+1) $ mit $x,y\in \N$  eine \textit{Zelle} dieses Gitters, die Verbindungsgerade zwischen zwei ganzzahligen Punkten eine \textit{Seite} der entsprechenden Zelle. Ein \textit{Domino} ist nun die Vereinigung von genau zwei Zellen, die sich eine gemeinsame Seite teilen. Mit anderen Worten ist ein Domino ein achsenparalleles Rechteck mit ganzzahligen Ecken und Seitenl�ngen $1$ und $2$. \\
Ein \textit{Domino Tiling} ist nun die �berdeckung einer Menge $\mathcal{Z}$ von Zellen durch Dominos mit disjunktem Inneren, wobei keine Zelle $z\notin \mathcal{Z}$ �berdeckt sein darf. Wir werden im Folgenden immer davon ausgehen, dass die Menge $\mathcal{Z}$ die einem Tiling zugrundeliegt endlich ist.

%�quivalenz mit matchings
Es gibt von diesen Domino Tilings eine �quivalenz zu perfekten Matchings von Gittergraphen:

Sei ein Domino Tiling $T$ gegeben. Konstruiert man einen Graphen, dessen Knoten jeweils in der Mitte der Zellen der Dominos aus $T$ liegen, und verbindet diese Knoten jeweils mit den direkten horizontalen und vertikalen Nachbarn, so erh�lt man einen Gittergraphen $G$.
\begin{obs}
Die Kanten von $G$, die komplett innerhalb eines Dominos liegen, sind die Kanten eines perfekten Matchings. Ist andererseits ein Gittergraph $G$ gegeben mit einem perfekten Matching $M$, l�sst sich daraus ein eindeutiges Domino Tiling konstruieren.
\end{obs}

\begin{proof}
Jeder Knoten von $G$ geh�rt per Definition zu einem Domino, also hat jeder Knoten mindestens eine Kante aus $G$, die eine Matchingkante ist. Andererseits ist laut der Definition der Domino Tilings das Innere der Dominos disjunkt, also kann jeder Knoten nur zu einem Domino aus $T$ geh�ren. Somit geh�rt zu jedem Knoten genau ein Matchingkante.
Ist ein Gittergraph mit einem perfekten Matching gegeben, dann k�nnen wir andererseits die Zellen so pflastern, dass ein Domino immer eine Matchingkante �berdeckt. Dabei sind die Dominos disjunkt, da jeder Knoten nur eine Matchingkante hat, und sie decken alles ab, da in einem perfekten Matching jeder Knoten ein Matchingkante hat. 
\end{proof}

\begin{obs}
\label{notwendigeBedFTiling}
(Notwedige Bedingung) Wenn $|\mathcal{Z}|$ ungerade ist, dann kann es kein Domino Tiling f�r $\mathcal{Z}$ geben.
\end{obs}
\begin{proof}
Da ein Matching immer genau eine Kante zu genau zwei Knoten zuordnet, muss ein perfektes Matching eine gerade Anzahl Knoten haben. Dementsprechend muss auch die Anzahl der Zellen, auf denen der Gittergraph $G$ definiert ist gerade sein.
\end{proof}

Dementsprechend gilt f�r Domino Tilings, dass z.B. bei einer rechteckigen Region nicht beide Seitenl�ngen ungerade sein d�rfen, damit ein Tiling der Region existiert. %TODO hinreichende Bedingung??? (satz von hall, thurstons h�henfunktion...)
Eine weitere �quivalenz kann man zu den $\alpha$-Orientierungen des gleichen, nun aber gerichteten, Gittergraphen $\overrightarrow{G}$ finden. \\

%alpha-or
Eine \textit{$\alpha$-Orientierung} ist eine Orientierung der Kanten eines Graphen, so dass zu einer Abbildung $\alpha : V\to \N $ der Ausgrad $out(v)=\alpha(v) \forall v\in V$ ist, $\alpha$ gibt also die Anzahl der ausgehenden Kanten an. 
Die Frage ob es f�r einen gegebenen Graphen eine $\alpha$-Orientierung gibt, l�sst sich in polynomieller Zeit beantworten indem man auf einem leicht modifiziertem Graphen einen maximalen Fluss berechnet, siehe dazu  \cite{Felsner04latticestructures}.
%TODO ALGO???

%Betrachten wir unseren Graphen $\overrightarrow{G}$ als Gitter von ganzzahligen Punkten $v=(v_1,v_2)$, indem wir der Einfachheit halber die Koordinaten aus der obigen Definition abrunden.
Wir haben oben $G$ bzw. $\overrightarrow{G}$ so definiert, dass die Knoten in der Mitte der Zellen $z\in \mathcal{Z}$ liegen. Daher haben alle Knoten von $G$ den Wert $n+\frac{1}{2}$ f�r ein $n\in \mathbb{Z}$. Die Summe zweier Koordinaten ist daher immer ganzzahlig.

Sei $k$ die Anzahl der benachbarten Zellen, die nicht in der zu �berdeckenden Region liegen. Dann k�nnen wir unser $\alpha$ definieren als:
$$\alpha(v)=
\begin{cases}
1 & v_1+v_2 \text{ ungerade, wir sagen v ist ungerader Knoten}\\
3-k & v_1+v_2 \text{ gerade, wir sagen v ist gerader Knoten}
\end{cases}
$$
Jeder Knoten (im Inneren) des Graphen hat nun entweder eine eingehende und $3$ ausgehende oder eine ausgehende und $3$ eingehende Kanten, am Rand entsprechend weniger. Insbesondere gibt es f�r jeden Knoten eine eindeutige Kante, die einzige eingehende oder einzige ausgehende Kante des Knotens. 
Im folgenden wird eine solche $\alpha$-Orientierung als $(3,1)$-Orientierung bezeichnet.

\begin{figure}[h!]
  \centering
  \scalebox{1}{\input{Bilder/einlbijektion.pstex_t}}
  \caption{Ein Domino Tiling, das zugeh�rige Matching und die $(3,1)$-Orientierung}
\end{figure}
%
%\begin{obs}
%Sei ein Gittergraph $\overrightarrow{G}$ mit $\alpha$ wie oben gegeben. Dann gibt es eine Bijektion zwischen den perfekten Matchings von $G$ und den $(3,1)$-Orientierungen auf $\overrightarrow{G}$. 
%\end{obs}
%
%\begin{proof}
%Betrachte eine $(3,1)$-Orientierung auf $\overrightarrow{G}$. Durch die Definition von $\alpha$ ist f�r jeden Knoten $v$ genau eine eindeutige Kante gegeben. Wenn $v_1+v_2$ gerade ist, ist es die einzige ausgehende Kante, ansonsten ist es die einzige eingehende Kante in $v$. Da die geraden und ungeraden Knoten alternieren, hat jeder gerade Knoten nur ungerade Nachbarn und umgekehrt. Daher ist eine Kante, die f�r einen der inzidenten Knoten die eindeutige ist, immer auch f�r beide die eindeutige Kante. Dadurch k�nnen wir diese Kanten als Matchingkanten w�hlen und w�hlen tats�chlich genau eine Matchingkante f�r jeden Knoten aus, wir erhalten also tats�chlich ein perfektes Matching f�r $G$.\\
%
%Betrachte andererseits ein perfektes Matching $M$ auf $G$. $G$ ist ein bipartiter Graph mit den Bipartitionsklassen $A:=$"'gerade Knoten"' und $B:=$"'ungerade Knoten"'. Orientiere die Kanten folgenderma�en:\\
%Falls $e\in M$, dann orientiere $e$ von $A$ nach $B$, falls $e\notin M$ orientiere $e$ von $B$ nach $A$. Da jeder Knoten zu genau einer Matchingkante geh�rte, hat nun jeder gerade Knoten genau eine ausgehende Kante und jeder ungerade Knoten hat genau eine eingehende Kante, wir haben also eine $(3,1)$-Orientierung auf $G$ definiert. 
%\end{proof}
%
%Man kann unter bestimmten Voraussetzungen dieses Ergebnis auch verallgemeinern:

\begin{lemma}
Sei $G=(A\cup B,E), \, A\cap B =\emptyset$ ein bipartiter Graph mit festgelegten Bipartitionsklassen $A,B$. Dann gibt es eine Bijektion zwischen den perfekten Matchings von $G$ und den $\alpha$-Orientierungen mit 
$$\alpha(v)=
\begin{cases}
1 & \text{ wenn }v\in A \\
deg(v)-1 & \text{ wenn }v\in B
\end{cases}
$$
\end{lemma}

\begin{proof}
Betrachte eine entsprechende $\alpha$-Orientierung auf $\overrightarrow{G}$. Durch die Definition von $\alpha$ ist f�r jeden Knoten $v$ genau eine eindeutige Kante gegeben. Wenn $v\in A$, ist es die einzige ausgehende Kante (ansonsten ist es die einzige eingehende Kante in $v$). Diese Kante w�hlen wir als Matchingkante $m_v$ von $v$. Nimm an, $v$ ist in $A$. Dann sind alle Nachbarn von $v$ in $B$.  Daher ist die Kante $m_v$ die eindeutige eingehende Kante f�r einen Knoten in $B$. Dadurch k�nnen wir diese Kanten als Matchingkanten w�hlen und w�hlen tats�chlich auch h�chstens eine Matchingkante f�r jeden Knoten aus, kein Nachbar kann eine weitere zu $v$ inzidente Kante als Matchingkante w�hlen. Das gleiche Argument mit umgekehrter Kantenrichtung gilt falls $v\in B$.

Betrachte andererseits ein perfektes Matching $M$ auf $G$. $G$ ist ein bipartiter Graph mit den Bipartitionsklassen $A$ und $B$. Orientiere die Kanten folgenderma�en:\\
Falls $e\in M$, dann orientiere $e$ von $A$ nach $B$, falls $e\notin M$ orientiere $e$ von $B$ nach $A$. Da jeder Knoten zu genau einer Matchingkante geh�rte, hat nun jeder Knoten $v\in A$ genau eine ausgehende Kante und jeder Knoten $v\in B$ hat genau eine eingehende Kante, wir haben also eine $(deg(v)-1,1)$-Orientierung auf $G$ konstruiert, in der jeder Knoten aus $A$ genau eine ausgehende und jeder in $B$ genau eine eingehende Kante hat. 
\end{proof}

\begin{cor}
Zwischen den Domino Tilings einer Region und den $(3,1)$-Orientierungen des zugeh�rigen Graphen $\overrightarrow{G}$ existiert eine Bijektion.
\end{cor}
\begin{proof}
Wir wissen bereits, dass Domino Tilings in Bijektion zu den perfekten Matchings auf dem zugeh�rigen Gittergraph $G$ stehen. Zeige also noch, dass $G$ die Voraussetzungen f�r das obige Lemma erf�llt.\\
W�hle als Bipartitionsklassen $A:=$"'gerade Knoten"' und $B:=$"' ungerade Knoten"' und $$\alpha(v)=
\begin{cases}
1 & v_1+v_2 \text{ gerade, wir sagen v ist gerader Knoten}\\
3-k & v_1+v_2 \text{ ungerade, wir sagen v ist ungerader Knoten}
\end{cases}
$$
Dies erf�llt die Bedingungen des Lemmas, wir bekommen daher eine Bijektion zwischen den Domino Tilings, perfekten Matchings und $(3,1)$-Orientierungen.
\end{proof}

$\alpha$-Orientierungen sind insofern also eine Verallgemeinerung von Matchings.
Nun lassen sich verschiedene Fragen zu diesen Tilings/Matchings/$\alpha$-Orientierungen stellen. 
Die Existenz kann man, wie bereits erw�hnt, einfach (polynomiell) �berpr�fen, auch schon f�r allgemeine $\alpha$-Orientierungen auf beliebigen Graphen. Betrachtet man rechteckige Gittergraphen, so muss man aber noch nicht mal diesen Algorithmus ausf�hren, um herauszufinden ob eine $(3,1)$-Orientierung existiert - sofern die notwendige Bedingung erf�llt ist, dass h�chstens eine der beiden Seitenl�ngen ungerade ist (siehe Beob. \ref{notwendigeBedFTiling}), existiert auch ein entsprechendes Domino Tiling:
Und zwar kann man an einer Ecke anfangen die Fl�che zu pflastern, so dass die l�ngere Seite der Dominos jeweils zu der geraden Seite zeigt. Da jede Schicht so l�ckenlos ausgef�llt werden kann, erhalten wir immer ein zul�ssiges Domino Tiling. 

In dieser Arbeit soll es insbesondere auch um Domino Tilings auf dem Torus gehen. Ein Torus entsteht, wenn man die gegen�berliegenden Seiten eines Rechtecks jeweils miteinander identifiziert. Dementsprechend wollen wir ein torisches Domino Tiling bzw. einen torischen Gittergraphen definieren.
Betrachte in dem zu Beginn definierten Gittergraphen $\mathcal{G}$ der ganzzahligen Punkte in der Ebene eine rechteckige Region von $n\times m$ Zellen, formaler:\\

Die Knotenmenge sei gegeben als $$V=\{(x_1+i,x_2+j)\in V(\mathcal{G}) | i=\{1\dots n\}, j=\{1\dots m\}\}$$ f�r ein festes $x=(x_1,x_2)\in \N$, definiere $\mathcal{R}$ als den von $V$ in $\mathcal{G}$ induzierten Graph. Identifiziere die Punkte folgenderma�en: $ (x_1,x_2)=(x_1+n,x_2)$ falls $(x_1,x_2)\in \mathcal{R}$ und $ (x_1+n,x_2)\in \mathcal{R} $, genauso $(x_1,x_2)=(x_1,x_2+m)$ falls $(x_1,x_2)\in \mathcal{R}$ und $(x_1,x_2+m)\in \mathcal{R}$, nenne den entstandenen torischen Gittergraphen $\mathcal{R}^T$.

Auch hier besteht ein Domino wieder aus der Vereinigung zweier Zellen, die eine gemeinsame Kante teilen, ein \textit{torisches Domino Tiling } ist eine vollst�ndige �berdeckung von $\mathcal{R}^T$ durch Dominos mit disjunktem Innern. 

Der Gittergraph $G$ des zugeh�rigen perfekten Matchings bzw $\overrightarrow{G}$ der zugeh�rigen $(3,1)$-Orientierung wird hier ebenfalls konstruiert, indem in die Mitte der Zellen von $\mathcal{R}^T$ Knoten gesetzt werden und die Knoten benachbarter Zellen verbunden (hier ergibt dies tats�chlich den Dualgraph). Eine $(3,1)$-Orientierung so dass jede Bipartitionsklasse Ausgrad 1 oder 3 hat, ist hier allerdings nur m�glich, wenn sowohl $m$ als auch $n$ gerade ist. 
Aus der Existenz von Domino Tilings einer rechteckigen Region folgt mit dieser Definition auch unmittelbar die Existenz von torischen Domino Tilings, da das Tiling einer rechteckigen Region $\mathcal{R}$ direkt zu einem Tiling von $\mathcal{R}^T$ wird. 

Eine nat�rliche Frage in diesem Zusammenhang ist die der Anzahl m�glicher Tilings. 
F�r Domino Tilings quadratischer Regionen sowie aus solchen entstandene torische Domino Tilings wurde diese Frage bereits 1961 durch Pieter Kasteleyn beantwortet (siehe \cite{Kasteleyn19611209}). %TODO Formel angeben?

Worum es hier aber im Kern gehen soll, ist die Transformierbarkeit von Tilings mittels Flips, insbesondere Face-Flips. 

\begin{definition}
Eine \textit{(kreuzungsfreie) Einbettung} des Graphen $G$ auf eine orientierbare Fl�che $S$ ist eine Abbildung $\phi: G\to S$ so dass f�r Knoten $v\neq v'\Rightarrow \phi(v)\neq \phi(v')$ gilt, und Kanten $e=\{v,w\}$ auf einfache Kurven $\gamma_e :[0,1] \to S$ mit $\gamma_e(0)=\phi(v),\, \gamma_e(1)=\phi(w),\, \gamma_e(x)\cap \gamma_f=\emptyset \, \forall f\in E, x\in (0,1)$ abgebildet werden.


Dabei betrachten wir 2-Zell-Einbettungen, d.h. solche Einbettungen, in denen $S \setminus \phi(G)$ zu der Vereinigung einer Menge offener Kreisscheiben hom�omorph ist. Der Rand des Bildes einer solchen Kreisscheibe ist insbesondere ein einfacher Kreis in $\phi(G)$. Wir nennen diesen Kreis (bzw. sein Urbild) einen Face-Kreis, sein Inneres ein Face von $G$.

Ein \textit{Flip} ist nun die Operation, die einen gerichteten Kreis in einer $\alpha$-Orientierung in $\overrightarrow{G}$ umdreht, d.h. jede Kante des Kreises wird umgedreht. Ein \textit{Face-Flip} ist der Flip eines Face-Kreises. In der Ebene kann man das $G$ umgebende �u�ere als unbeschr�nktes Face definieren, dieses erlauben wir aber ausdr�cklich nicht zu flippen.

\end{definition}
%
%Ein \textit{Face} in einem planaren bzw. auf einer Fl�che kreuzugsfrei eingebetteten Graphen ist der Raum, der zwischen mehreren Kanten eingeschlossen liegt. Auch das (in der Ebene unbeschr�nkte) �u�ere um einen planaren Graphen wird als Face bezeichnet, genauer als das unbeschr�nkte Face. Wir erlauben hier allerdings nicht dieses zu flippen. Ein \textit{Face-Flip} ist also das drehen eines gerichteten Kreises, der im Inneren keine Kanten hat.

Es ist von der konkreten festen Einbettung abh�ngig, was die Faces eines Graphen sind! Flips sind mit der $\alpha$-Orientierung des Graphen vertr�glich, das $\alpha$ bleibt bei einem Flip in jedem Knoten gleich. Genauer:

\begin{lemma}
\label{allgFlipZshg}
Zwei Orientierungen eines Graphen $G$ geh�ren genau dann zum gleichen $\alpha$, d.h. sie haben in jedem Knoten den gleichen Ausgrad, wenn sie durch Flips einfacher Kreise ineinander zu �berf�hren sind. 
\end{lemma}
\begin{proof}
Betrachte zwei $\alpha$-Orientierungen $X_1,X_2$ in $G$ zum gleichen $\alpha$ und den Differenzgraph der beiden Orientierungen: das ist der Graph, in dem nur Kanten enthalten sind, die in $X_1$ und $X_2$ unterschiedlich gerichtet sind, wir geben den Kanten dabei die Orientierung von $X_1$. Dieser Graph hat in jedem Knoten gleich viele eingehende wie ausgehende Kanten, denn f�r jede Kante die in $X_2$ ausgehand statt eingehend zu einem Knoten $v$ gerichtet ist, muss eine andere eingehend statt ausgehend gerichtet sein um den selben Wert $\alpha(v)$ zu erreichen. Der Differenzgraph von $X_1$ und $X_2$ ist somit eulersch. Eulersche Graphen lassen sich in kantendisjunkte einfache Kreise zerlegen: \\
($\rightarrow$ Algorithmus: Laufe auf vorw�rts gerichteten Kanten durch den Graph bis ein Knoten zum zweiten Mal besucht wurde. Der Weg vom ersten zum zweiten Besuch des Knotens ergibt einen Kreis $C$. Es gibt keine Sackgassen au�er schon besuchten Knoten, denn wenn wir zum ersten mal einen Knoten besuchen, hat der ja immer noch eine unbenutzte ausgehende Kante. Da es aber nur endlich viele Knoten und Kanten gibt, kommen wir immer zu einem schon besuchten Knoten. L�sche die Kanten des Kreises $C$, dann ist der Restgraph wieder eulersch, da wir einen endlichen Graph haben k�nnen wir rekursiv fortfahren bis der Graph leer ist. Die jeweils in einem Schritt gel�schten Kreise ergeben eine kantendisjunkte Zerlegung des Graphen in einfache Kreise. )\\
F�r jeden Kreis aus dieser Kreiszerlegung k�nnen wir nun die Richtung jeder Kante umdrehen, damit wird das $\alpha$ nicht ver�ndert da f�r jeden Knoten eine ausgehende und eine eingehende Kante geflippt wird. Nachdem wir alle diese Kreise geflippt haben, haben wir $X_1$ in $X_2$ �berf�hrt.\\


%W�hle eine Kante $e=(u,v)$ die oBdA in $X_1$ von $u$ nach $v$ und in $X_2$ von $v$ nach $u$ gerichtet ist. Durch das Umdrehen von $e$ in $X_1$ gibt es im Knoten $v$ eine ausgehende Kante zuviel $\Rightarrow$ drehe also eine andere ausgehende Kante $e'=(v,w)$ um, die in beiden Orientierungen unterschiedlich gerichtet ist. Diese Kante existiert, da beide zum gleichen $\alpha$ definiert wurden. Dies f�hrt in $w$ zu der gleichen Situation, drehe also iterativ Kanten um, die in $X_1$ und $X_2$ unterschiedlich gerichtet sind. Dies kann nur terminieren, wenn wir bei $u$ ankommen, da w�hrend des Algorithmus $u$ der einzige Knoten ist, der eine ausgehende Kante zu wenig hat. Da es insgesamt nur endlich viele Kanten gibt, muss der Algorithmus aber auch in endlicher Zeit terminieren. Er kommt daher zwangsl�ufig wieder beim Ausgangsknoten $u$ an, nachdem er einen gerichteten Kreis geflippt hat. Wiederhole dies solange, bis die beiden Orientierungen gleich sind, das tritt in endlicher Zeit ein, da nur endlich viele Kanten vorhanden sind. Der Differenzgraph von $X_1$ und $X_2$ ist eulersch, da in jedem Knoten gleich viele eingehende wie ausgehende Kanten unterschiedlich gerichtet sind. Daher l�sst er sich auch in einfache kantendisjunkte Kreise zerlegen. Wir k�nnen also mit einer Folge von Flips einfacher Kreise $X_1$ in $X_2$ �berf�hren.\\

F�r die R�ckrichtung sind wieder $X_1$ und $X_2$ gegeben, dazu eine Menge von einfachen Kreisen, deren Flips $X_1$ in $X_2$ �berf�hren. 

Behauptung: Das flippen eines einfachen Kreises �ndert in keinem Knoten den Ausgrad. Grund: In einem einfachen gerichteten Kreis hat jeder Knoten eine eingehende und eine ausgehende Kante. Wenn alle Kanten des Kreises umgedreht werden, dann tauschen die beiden nur ihre Rolle, der Ausgrad des Knotens bleibt aber unver�ndert. \\
Wenn wir aber bei keinem Flip das $\alpha$ ver�ndern, dann sind $X_1$ und $X_2$ Orientierungen zum gleichen $\alpha$.
\end{proof}

Wir k�nnen Flips auch direkt f�r Tilings oder Matchings definieren. Ein Face-Flip ist dabei die Operation, die zwei benachbarte, sich die lange Seite teilende Dominos um 90 Grad dreht, bzw die Matchingkanten in einem Facekreis mit den Nichtmatchingkanten austauscht. In der zugeh�rigen $(3,1)$-Orientierung entspricht dies gerade dem umdrehen eines gerichteten Face-Kreises.

F�r planare Graphen ist es m�glich, alle Tilings einer Fl�che mittels solchen Face-Flips zu erreichen. Mehr noch, man kann den Tilings einer planaren Fl�che eine Ordnungsrelation geben, die einen distributiven Verband bildet.

%Definition eines distributiven Verbandes
\begin{definition}
\label{defDistribVerband}
Sei $P=(X,\le )$ ein Poset/Halbordnung. $P$ ist ein distributiver Verband, wenn $\forall x,y\in X \exists x\lor y, x\land y \in X$ so dass gilt
\begin{itemize}
\item $x\lor y \ge x \ge x\land y$ und $x\lor y \ge y \ge x\land y$
\item $\forall z\in X: z\ge x,y\Rightarrow x\lor y\le z$
\item $\forall z\in X: z\le x,y \Rightarrow x\land y\ge z$
\item $\forall x,y,z \in X: x\land(y\lor z)=(x\land y)\lor (x\land z)$ (Distributivit�t)
\end{itemize}
\end{definition}
 

%%beweis planarer fall
%\begin{theorem}
%Auf der Menge der $\alpha$-Orientierungen eines planaren Graphen wird durch Kreisflips ein distributiver Verband induziert. Insbesondere gen�gen in einem stark zusamenh�ngenden Graphen Face-Flips.
%\end{theorem}
%TODO wird rausgenomen
%Da es in dieser Arbeit vor allem um den Flip-Zusammenhang geht (auf dem Torus ist dieser nicht durch Face-Flips zu gew�hrleisten), will ich hier auch nur den Flip-Zusammenhang im planaren Fall beweisen. Der Beweis, dass man sogar einen distributiven Verband erh�lt, ist in [] nachzulesen.

%
%\begin{cor}
%Je zwei $\alpha$-Orientierungen eines Graphen (zu gleichem $\alpha$) lassen sich mit einer Folge von Flips essentieller Kreise ineinander �berf�hren. \\
%Ein Kreis $C$ hei�t essentiell, wenn er folgende Eigenschaften hat:
%\\Er ist 
%\begin{itemize}
%\item ein einfacher Kreis
%\item hat keine Sehne/keinen gerichteten Pfad im Inneren
%\item die Kanten, die vom Rand ins Innere des Kreises gehen sind in allen $\alpha$-Orientierungen gleich gerichtet, also nie in einem gerichteten Kreis enthalten
%\item Es gibt eine $\alpha$-Orientierung, in der $C$ gerichtet ist
%\end{itemize}
%Bemerkung: In stark zusammenh�ngenden Graphen sind die essentiellen gerade die Face-Kreise.
%\end{cor}

%TODO den allgemeineren Satz hier davorstellen???
\begin{theorem}
Auf der Menge der Domino Tilings einer einfach zusammenh�ngenden Region in der Ebene wird durch die Operation des Drehens benachbarter Dominos, also durch Face-Flips, ein distributiver Verband induziert. 
\end{theorem}

Es gibt verschiedene Beweise f�r diesen Satz von unterschiedlicher Allgemeinheit, die sich jeweils auf eine bestimmte Verallgemeinerung beziehen. Beispielsweise haben wir gesehen, dass perfekte Matchings und auch $\alpha$-Orientierungen als Verallgemeinerung von Domino Tilings verstanden werden k�nnen. F�r perfekte Matchings planarer Graphen wurde die Struktur des distributiven Verbandes direkt in \cite{LamZhang} bewiesen, f�r allgemeinere $k$-Faktoren in \cite{ProppLattice}. Den Beweis f�r $\alpha$-Orientierungen aus \cite{Felsner04latticestructures} werde ich sp�ter direkt angeben. Auch f�r noch allgemeinere Strukturen als $\alpha$-Orientierungen l�sst sich noch dieser distributive Verband definieren, in \cite{FelsnerKnauerULDlattices} wird dies durch $\Delta$-Bonds bewerkstelligt.

Ich zeige hier zun�chst einen direkt auf Domino Tilings zugeschnittenen Beweis. 

\section{Thurston's H�henfunktion f�r Domino Tilings}
Der hier angegebene Beweis ist aus \cite{Remila2004409} �bernommen. Die dort benutzte H�henfunktion stammt von William Thurston in \cite{Thurston1990}, wo allerdings nicht der Zusammenhang mit distributiven Verb�nden betrachtet wird, sondern nur die Existenz von Tilings einer Region.

Erinnerung: Wir haben Domino Tilings �ber das Gitter der ganzzahligen Punkte in der Ebene definiert. Wenn man ein Tiling gegeben hat, kann man so die R�nder der Dominos als Kanten interpretieren und die Ecken als Knoten. Auch in der Mitte der langen Seiten eines Dominos liegt jeweils ein Knoten des Graphen, und eine Kante zwischen diesen beiden Knoten f�r jedes Domino. 

Man erh�lt so einen Gittergraphen in der Gr��e der �berdeckten Fl�che. Wir orientieren die Kanten so, dass die an einer Gitterzelle anliegenden Kanten immer einen gerichteten Kreis bilden, abwechselnd im Uhrzeigersinn und gegen den Uhrzeigersinn und  wir nennen diesen gerichteten Graphen $D=(V,A)$. Stellt man sich das Gitter als Schachbrett vor, dann kann man sagen links von einer Kante liegt ein schwarzes Feld und recht ein wei�es.

\begin{figure}[h!]
  \centering
  \scalebox{1}{\input{Bilder/thurston1.pstex_t}}
  \caption{Beispiel eines Graphen aus einem planaren Domino Tiling}
\end{figure}

Bis hierhin ist die Konstruktion noch unabh�ngig vom konkreten Domino Tiling. Definiere daher eine H�henfunktion $ h_T: V\to \mathbb{Z}$ auf den Knoten dieses Graphen $D$:%, welche ein Domino Tiling $T$ kodiert: 
\begin{itemize}
\item W�hle einen beliebigen Knoten $v_0$ und setze $h_T(v_0)=0$.
\item Falls Kante $(v,v')$ ein Domino schneidet, also zwischen zwei Zellen des gleichen Dominos liegt, setze $h_T(v')=h_T(v)-3$. 
\item Sonst setze f�r die Kante $(v,v')$ iterativ den Wert $h_T(v')=h_T(v)+1$.
\end{itemize}
Bemerkung: Die Wahl von $v_0=0$ ist beliebig, da nur die relativen H�hendifferenzen eine Bedeutung haben. Im Zusammenhang mit Flips ist es aber sinnvoll, einen Knoten $v_0$ auf dem Rand von $D$ zu w�hlen, da er dort nicht durch Flips ver�ndert werden kann.\\

\begin{figure}[h!]
  \centering
  \scalebox{1}{\input{Bilder/thurston2.pstex_t}}
  \caption{Ein Domino Tiling mit der oben definierten H�henfunktion}
\end{figure}

Nun muss nat�rlich noch �berpr�ft werden, ob diese Definition auch immer konsistent ist, und dass sie ein Domino Tiling eindeutig kodiert. Wir brauchen noch eine kleine Definition:
\begin{definition}
Ein Weg in $D$ ist ein \textit{legaler Weg} bez�glich eines Domino Tilings $T$, wenn er keine Kanten benutzt, die ein Domino in $T$ schneiden.
Seien $v,v'\in V$ beliebige Knoten in $D$. Die \textit{H�hendifferenz} von $v$ zu $v'$ ist definiert als  \it Anzahl vorw�rts gerichteter Kanten auf einem legalem Weg von $v$ zu $v'$ - Anzahl r�ckw�rts gerichteter Kanten auf einem legalen Weg von $v$ zu $v'$ \rm.
\end{definition}

\begin{lemma}
\label{EindeutigkeitH�henfkt}
Die H�henfunktion $h_T$ zu einem zusammenh�ngenden Domino Tiling $T$ ist (bei gleicher Wahl von $v_0$) unabh�ngig davon, in welcher Reihenfolge die H�he der Knoten berechnet wird. 
\end{lemma}
\begin{proof}
Betrachte den Graphen $D'$, der entsteht wenn wir aus $D$ alle Kanten l�schen, die Dominos schneiden. Da das Domino Tiling zusammenh�ngend ist, ist $D'$ ebenfalls zusammenh�ngend. Um f�r jeden Knoten des Graphen einen Wert der Funktion $h_T$ zu berechnen gen�gt also die Bedingung, dass f�r jede gerichtete Kante $(v,v')\in V$ gilt $h_T(v')=h_T(v)+1$. Weiterhin sieht man leicht, dass die zweite Bedingung (falls Kante $(v,v')$ ein Domino schneidet, setze $h_T(v')=h_T(v)-3$) hieraus bereits folgt: Da jede Zelle von einem gerichteten Kreis umschlossen wird und zu genau einem Domino geh�rt, hat jede Zelle auch genau eine Kante die ihr Domino schneidet. Die anderen 3 Kanten haben zusammen eine H�hendifferenz von 3 (in Kreisrichtung), so dass die Differenz f�r die das Domino schneidende Kante automatisch -3 ist. $\Rightarrow$ Es gen�gt also zu zeigen, dass die Konstruktion �ber Randkanten der Dominos konsistent ist. Als H�he eines Knotens $v$ definieren wir dann genau die H�hendifferenz von $v_0$ zu $v$, dies erf�llt die lokalen Bedingungen f�r eine H�henfunktion. Wir m�ssen nun noch zeigen, dass dieser Wert unabh�ngig davon ist, welcher legale Weg zur Berechnung gebraucht wurde. \\

\it Behauptung: \rm Ein Kreis in $D$ ohne die Kanten, welche ein Domino schneiden, hat H�hendifferenz 0. \\
\it Beweis der Behauptung: \rm Induktion - man sieht leicht an einem einzelnen Domino dass die H�hendifferenz 0 ist, da je $3$ Kanten vorw�rts und $3$ Kanten r�ckw�rts gerichtet sind. Hat man einen gr��eren Kreis, findet man darin einen Weg, an dem man den Kreis in zwei kleinere aufteilen kann. Die Summe der H�hendifferenzen der beiden kleinen Kreise ergibt die H�hendifferenz des gro�en Kreises - da die kleineren nach Induktionsvoraussetzung H�hendifferenz 0 haben, gilt dies auch f�r den gro�en Kreis.\\

Betrachte zwei legale Wege $p_1$ und $p_2$ von $v_0$ zu $v$. Dann ist $p_1$ mit $\overleftarrow{p_2}$ ($p_2$ r�ckw�rts) ein legaler Kreis. Die H�hendifferenz des Kreises ist die Differenz beider H�hendifferenzen der beiden Pfade. Da die H�hendifferenz in Kreisen immer 0 betr�gt, muss sie also bei beiden Pfaden gleich gro� sein.
\end{proof}

Da sich die H�henfunktion �ber relative H�hendifferenzen im Graphen berechnet, der Startknoten und im Prinzip auch der Startwert aber beliebig sind, kann es zu einem Domino Tiling $T$ mehrere H�henfunktionen geben. Diese unterscheiden sich aber nur um eine additive Konstante, d.h. f�r alle H�henfunktionen $h_T, {h'}_T$ gibt es eine Zahl $c\in \R$ so dass gilt $h_T(v)+c={h'}_T(v) \, \forall v\in V$.\\

Wir k�nnen ebenfalls zeigen, dass zu jeder konsistenten H�henfunktion des Graphen ein Domino Tiling geh�rt:
\begin{lemma}\label{3bedinglemma}
Sei eine Funktion $h: V \to \mathbb{Z}$ gegeben, so dass
\begin{itemize}
\item Es gibt einen Knoten $v_0\in V$, so dass $v_0$ auf dem Rand von $D$ liegt, d.h. das unbeschr�nkte Face ber�hrt, und den Wert $h(v_0)=0$ besitzt. 
\item F�r jede gerichtete Kante $(v,v')$ gilt entweder $h(v')=h(v)+1$ oder $h(v')=h(v)-3$
\item F�r jede Kante auf dem Rand von $D$ gilt die Beziehung $h(v')=h(v)+1$.
\end{itemize}
Dann ist $h$ tats�chlich die H�henfunktion $h_T$ f�r ein Domino Tiling $T$.
\end{lemma}
\begin{proof}
Da in $D$ jede Zelle von einem gerichteten Kreis (der L�nge 4) umgeben ist, gibt es in jeder Zelle genau eine Kante $e=(v,v')$ mit $h(v')=h(v)-3$. Wegen der dritten Bedingung kann diese nicht auf dem Rand des Graphen liegen. Auf der anderen Seite dieser Kante liegt deshalb eine weitere Zelle, f�r die ebenfalls $e$ die eindeutige Kante mit  $h(v')=h(v)-3$ ist. Beide zusammen bilden ein Domino. Dieses Argument kann man nun f�r jede einzelne Zelle anwenden, und erh�lt daraus ein eindeutiges Domino Tiling $T$.
\end{proof}

\begin{cor}
\label{bijektDomTil_H�henfkt}
Zwischen der Menge der H�henfunktionen zu einem festen $v_0\in V$, das auf dem Rand von $D$ liegt, und den Domino Tilings von $D$ gibt es eine Bijektion.
\end{cor}
\begin{proof}
Aus Lemma \ref{EindeutigkeitH�henfkt} wissen wir, dass man zu einem Domino Tiling $T$ mit festem $v_0$ eine eindeutige H�henfunktion konstruieren kann. In Lemma \ref{3bedinglemma} haben wir hingegen bewiesen, dass mit einer solchen H�henfunktion ein eindeutiges Domino Tiling gegeben ist. 
\end{proof}

Man kann beobachten, dass die Werte eines Knotens in H�henfunktionen zu verschiedenen Tilings sich nur um Vielfache von 4 unterscheiden:

\begin{lemma}
Seien $T,T'$ Domino Tilings, und die H�henfunktionen bez�glich des gleichen Wertes f�r $v_0$ definiert. Dann ist f�r jeden Knoten $v$ die Differenz $h_T(v)-h_{T'}(v)$ ein Vielfaches von $4$.
\end{lemma}
\begin{proof}
F�r jede Kante $(v,v')$ gilt $h_T(v')=h_T(v)+1$ oder $h_T(v')=h_T(v)-3$. Wenn also der Unterschied f�r $v$ ein vielfaches von $4$ ist, dann muss das auch f�r seinen Nachbarn $v'$ gelten. Laut Voraussetzung ist aber $h_T(v_0)=h_{T'}(v_0)$, was ein Vielfaches von 4 ist. Induktiv muss diese Aussage also auch f�r alle andern Knoten gelten.
\end{proof}

Die H�henfunktion f�r Domino Tilings ist also wohldefiniert und legt intuitiv nahe, folgende Ordnung auf Domino Tilings zu definieren: $$T\le T' \iff h_T(v)\le h_{T'}(v)\, \forall v\in V$$
Wir wollen zeigen, dass das tats�chlich ein distributiver Verband ist. 

\begin{theorem}
\label{minmaxHoehenfkt}
Seien $T,T'$ Domino Tilings und die H�henfunktion so dass $v_0$ auf dem Rand liegt und $h_T(v_0)=h_{T'}(v_0)$. Dann sind die Funktionen $\min (h_T,h_{T'})$ und $\max (h_T,h_{T'})$ ebenfalls H�henfunktionen von Domino Tilings.
\end{theorem}
\begin{proof}
Wir werden hier nur die Aussage f�r $\min (h_T,h_{T'})$ zeigen, der Beweis f�r $\max (h_T,h_{T'})$ funktioniert analog. Zu zeigen ist, dass $\min (h_T,h_{T'})$ und $\max (h_T,h_{T'})$ eine H�henfunktion wie in Lemma \ref{3bedinglemma} ergeben, dann ist dies auch sofort die H�henfunktion eines Domino Tiling.\\
Betrachte ein Paar $v,v'$ von benachbarten Knoten mit der gerichteten Kante $(v,v')$. Wenn nun gilt $$h_T(v)<h_{T'}(v)$$ dann folgt sofort $h_T(v)\le h_{T'}(v)-4$. Andererseits muss im jeweiligen Tiling die H�henfunktion konsistent sein, d.h. es gilt insbesondere $h_T(v')\le h_T(v)+1$ und $h_{T'}(v')\ge h_{T'}(v)-3$.
Zusammen erhalten wir:
$$h_T(v')\le h_T(v)+1\le h_{T'}(v)-4+1\le h_{T'}(v')+3-4+1=h_{T'}(v')$$
Das bedeutet $h_T(v)<h_{T'}(v)\Rightarrow \min(h_T(v'),h_{T'}(v'))=h_T(v')$, die H�henfunktion setzt sich also auf de Nachbarn von $v$ fort, wenn die Werte in $v$ ungleich sind (der Fall $h_T(v)>h_{T'}(v)$ ist analog), dies sichert die Konsistenz der zweiten Bedingung aus Lemma \ref{3bedinglemma}. Wo die beiden H�henfunktionen die gleichen Werte annehmen, ist die Konsistenz sowieso klar, da $h_T, h_{T'}$ H�henfunktionen sind.
\end{proof}

\begin{lemma}
\label{distribDominanzordnung}
Sei eine Menge von Vektoren $W \subset \R^n$ gegeben, so dass $W$ unter komponentenweiser Minimums- bzw. Maximumsbildung abgeschlossen ist. Mit der Dominanzordnung $$x\le y\iff x_i\le y_i \,\forall i\in \{1\dots n\}$$ ist dann $(W,\le)$ ein distributiver Verband mit $x\lor y=\max(x,y),\, x\land y=\min(x,y)$ (komponentenweise).
\end{lemma}
\begin{proof}
Wir m�ssen die Eigenschaften aus Definition \ref{defDistribVerband} nachweisen. Seien also $x,y\in W$ beliebig. Es gilt
\begin{itemize}
\item $\max(x,y)\ge x\ge \min(x,y)$ und $\max(x,y)\ge y\ge \min(x,y)$ gilt nach Definition des komponentenweisen Maximums und Minimums. Das Maximum und Minimum von Vektoren aus $W$ sind laut Voraussetzung ebenfalls in $W$ enthalten, existieren also.
\item Sei $z\in W$ gegeben so dass $z\ge x,y$. Dann gilt komponentenweise $z_i\ge x_i$ und $z_i\ge y_i$, also auch $z_i\ge \max(x_i,y_i)\Rightarrow$ $z\ge \max(x,y)$
\item Sei $z\in W$ gegeben so dass $z\le x,y$. Dann gilt komponentenweise $z_i\le x_i$ und $z_i\le y_i$, also auch $z_i\le \min(x_i,y_i)\Rightarrow$ $z\ge \min(x,y)$
\item Distributivit�t: Komponentenweise gilt f�r $x,y,z\in W$ :
$$ \max(x_i, \min(y_i,z_i))=\min(\max(x_i,y_i),\max(x_i,z_i)), $$ dies l�sst sich mit einer einfachen Fallunterscheidung �ber die Gr��e von $x,y,z$ �berpr�fen. 
\end{itemize}
\end{proof}

\begin{cor}
Die oben definierte Ordnung auf planaren Domino Tilings $$T\le T' \iff h_T(v)\le h_{T'}(v)\, \forall v\in V$$ induziert einen distributiven Verband. 
\end{cor}
\begin{proof}
Fasse die H�henfunktion eines Domino Tilings als Vektor $w$ �ber $\N^{|V|}$ auf. Aus Satz \ref{minmaxHoehenfkt} folgt, dass die Menge dieser Vektoren bez�glich Minimums-/Maximumsbildung abgeschlossen ist. Die hier definierte Ordnung $T\le T' \iff h_T(v)\le h_{T'}(v)\, \forall v\in V$ entspricht genau der in Lemma \ref{distribDominanzordnung} benutzten Ordnung, da die einzelnen Knoten $v\in V$ genau den Komponenten des Vektors $w$ entsprechen. Da wir von den H�henfunktionen eine Bijektion zu den Domino Tilings hergestellt haben (Korollar \ref{bijektDomTil_H�henfkt}), bilden die Domino Tilings also ebenfalls einen entsprechenden distributiven Verband.
\end{proof}

% beziehung zu flips 
Nun kann man sich die Frage stellen, wie denn der �bergang zu einem jeweils n�chstgr��eren/kleineren Element in diesem Verband aussieht. Die Antwort lautet: Die �berg�nge sind die Face-Flips, wie wir sie auch in dem Beweis f�r allgemeine $\alpha$-Orientierungen verwenden werden. 
%Ein Flip (Face-Flip) ist dabei das ersetzen von zwei nebeneinanderliegenden horizontalen durch zwei  nebeneinanderliegende vertikale Dominos oder anders herum. 

\begin{figure}[h!]
  \centering
  \scalebox{1}{\input{Bilder/thurston3.pstex_t}}
  \caption{Darstellung der Flips von benachbarten Dominos, links ist der jeweils kleinere}
\end{figure}

Bei einem solchen Face-Flip wird genau der Wert des Knotens in der Mitte der beiden geflippten Dominos um $\pm 4$ ver�ndert, auf allen anderen Knoten bleibt die H�henfunktion dabei konstant - die H�henfunktion ist ja f�r jeden Knoten $v$ durch die H�hendiffernez auf einem legalen Weg von $v_0$ zu $v$ definiert, der bis auf den Knoten in der Mitte der beiden Dominos ja unber�hrt bleibt. Im Bild ist angenommen, dass die schraffierten Zellen von den Kanten des Graphen gegen den Uhrzeigersinn und die wei�en im Uhrzeigersinn umlaufen werden. Macht man sich klar, wie dann die H�hendifferenz auf dem letzten St�ck zum Mittelknoten sich �ndert, sieht man dass die dargestellten Flips in Pfeilrichtung "'aufw�rts"' gehen.\\

Da sich diese Arbeit mit "'Domino Tilings auf dem Torus"' besch�ftigen will, muss man hier nat�rlich fragen ob dieser Ansatz sich auf torische Domino Tilings ebenfalls anwenden l�sst. Wo wurde die Planarit�t explizit ausgenutzt? Wo ergeben sich Probleme, wenn man die H�henfunktion auf den Torus �bertragen will? \\
\begin{itemize}
\item Zum einen ist nicht klar, ob die H�henfunktion noch wohldefiniert ist. Bei ungerader Breite/H�he, gemessen in Anzahl der Zellen, ist die H�hendifferenz eines legalen Weges nicht mehr unabh�ngig vom Weg! Die Definition m�sste also angepasst werden.
\item Die Azyklizit�t der Ordnungsrelation ist auf dem Torus nicht mehr gegeben, siehe Abbildung. In planaren Graphen wurde die Azyklizit�t dadurch erreicht, dass es verboten ist, das �u�ere Face zu drehen. Man k�nnte dies also beheben, indem man auch hier ein bestimmtes Face w�hlt, welches nicht geflippt werden darf, und den Knoten in dessen Mitte als $v_0$ w�hlen.
\item Der Flipzusammenhang ist allein mit Facekreisen nicht mehr gegeben, siehe dazu das Kapitel 5. Wir werden sehen, dass man Beispiele konstruieren kann, in denen kein einziger Face-Kreis flipbar ist. 
\end{itemize}

\begin{figure}[h!]
  \centering
  \scalebox{0.7}
  {\input{Bilder/thurston4.pstex_t}}
  \caption{Eine Folge von Flips auf einem torischen Domino Tiling, die wieder zum urspr�nglichen Tiling bei echt gr��erer H�henfunktion f�hrt}
\end{figure}


%%%%%%%%%%%%%%%%%%%%%%%%%%%%%%%%%%%%%%%%%%%%%%%%%%%%%%%%%%%%%%%%%%%%%%%%%%%%%%%%


\section{Verallgemeinerung: $\alpha$-Orientierungen}

Wir haben bereits $\alpha$-Orientierungen von Graphen definiert und gesehen, dass diese ebenso wie H�henfunktionen Domino Tilings kodieren k�nnen. Dar�ber hinaus ist das Konzept der $\alpha$-Orientierungen ohne Probleme auf den Torus zu �bertragen, $(3,1)$-Orientierungen zumindest f�r torische Gitter gerader Breite und H�he. Deshalb scheint es sinnvoll, auch den Beweis f�r allgemeine Domino Tilings zu betrachten. Der hier angegebene Beweis folgt dem von Stefan Felsner in \cite{Felsner04latticestructures} angegebenen. \\

Wir beweisen in diesem Abschnitt, dass allgemeine $\alpha$-Orientierungen von planaren Graphen mit einer festen planaren Einbettung einen distributiven Verband bilden. Wir betrachten daher f�r dieses Kapitel auch nur planare Graphen $\overrightarrow{G}=(V,A)$ zusammen mit einer festen kreuzungsfreien Einbettung in die Ebene.


\begin{definition}
Ein Kreis $C\subseteq A$ hei�t essentiell, wenn er folgende Eigenschaften hat:\\
Er ist 
\begin{itemize}
\item ein einfacher Kreis
\item hat keine Sehne/keinen gerichteten Pfad im Inneren
\item die Kanten, die vom Rand ins Innere des Kreises gehen sind in allen $\alpha$-Orientierungen gleich gerichtet, also nie in einem gerichteten Kreis enthalten
\item Es gibt eine $\alpha$-Orientierung auf $\overrightarrow{G}$, in der $C$ gerichtet ist
\end{itemize}
Bemerkung: In stark zusammenh�ngenden Graphen $\overrightarrow{G}$ sind die essentiellen Kreise gerade die Face-Kreise. 
\end{definition}

Zun�chst ein kleines Lemma, das wir auch sp�ter noch ben�tigen werden.

\begin{lemma}

Jeder einfache gerichtete Kreis $C$ in $\overrightarrow{G}$ kann umgedreht werden, indem eine Folge von Drehungen/Flips essentieller Kreise im Inneren von $C$ ausgef�hrt wird. Erfolgt die Drehung von "'gegen den Uhrzeigersinn"' (ccw, counterclockwise) zu "'im Uhrzeigersinn"' (cw, clockwise), dann sind auch auch Drehungen der essentiellen Kreise von "'gegen den Uhrzeigersinn"' zu "'im Uhrzeigersinn"' . 
\end{lemma}

\begin{proof}
Wir nutzen Induktion �ber die Anzahl der im Inneren enthaltenen essentiellen Kreise: \\
Falls $C$ ein essentieller Kreis ist, sind wir fertig, da $C$ nach Voraussetzung gerichtet ist und somit selbst gedreht werden kann. \\
Falls $C$ kein essentieller Kreis ist, dann muss es einen gerichteten Pfad im Inneren geben (m�glicherweise der L�nge 1), d.h. ein Pfad $P$ von $u$ nach $v$, so dass $u,v$ auf $C$ liegen und der Rest von $P$ im Inneren von $C$:\\
Angenommen, es gibt in einer $\alpha$-Orientierung keinen gerichteten Pfad im Inneren von $C$. Dann betrachte die Menge der Kanten, die von Knoten auf $C$ ins Innere gehen. Da es keinen Pfad zur�ck oder aus $C$ hinaus gibt, muss diese Menge ein gerichteter Schnitt sein. F�r die Menge der Kanten die aus dem Inneren zu einem Knoten auf $C$ gehen gilt das gleiche. Es muss Kanten zum Inneren oder aus dem Inneren geben, da der Kreis nicht essentiell ist, somit ist mindestens einer dieser Schnitte nichtleer. Wenn diese Kanten aber in jeder $\alpha$-Orientierung gleich gerichtet sind, dann ist der Kreis essentiell.\lightning Wenn ein Kreis also nicht essentiell ist, dann gibt es einen gerichteten Pfad im Inneren.\\
An diesem gerichteten Pfad $p$ im Inneren von $u$ nach $v$, $u,v\in C$ k�nnen wir nun unseren Kreis aufspalten. Sei $p_1\subset C$ der Pfad in $C$ von $v$ zu $u$ und $p_2\subset C$ der Pfad in $C$ von $u$ nach $v$. 
Zerlege $C$ an diesem Pfad in $C_1=p+ p_1$ und $C_2=\overleftarrow{p}+ p_2$. Das Drehen von $C_1$ und anschlie�end $C_2$ ergibt dann den gleichen Graphen wie das Drehen von $C$, da alle Kanten von $C$ genau einmal umgedreht werden, und alle Kanten auf $p$ genau zweimal. Nach Induktionsvoraussetzung k�nnen wir die kleineren Kreise durch in ihrem Inneren enthaltene essentielle Kreise drehen. Wenn die Drehung von $C$ von ccw zu cw geht, dann ist offenbar auch $C_1$ und $C_2$ zu dem Zeitpunkt, wo sie gedreht werden jeweils ccw, und das gleiche gilt nach Induktionsvaraussetzung auch f�r die kleineren Kreise bis zu den essentiellen. Wie man erkennen kann, wird jeder essentielle Kreis im Inneren dabei �brigens genau einmal gedreht.

\end{proof}

Man kann sich also tats�chlich darauf zur�ckziehen, nur essentielle Kreise zu betrachten, dies ist nach obigem Lemma keine Einschr�nkung. Wir werden im folgenden ein $\alpha$-Potential definieren, dessen Werte von der L�nge einer Flipsequenz bis zum Minimum (Orientierung ohne gegen den Uhrzeigersinn gerichtete Kreise) abh�ngen. Daf�r brauchen wir die Eindeutigkeit einer solchen Sequenz. \\

Wir nennen eine Folge von essentiellen Kreisen $C_i$ eine \textit{zul�ssige Flipsequenz} f�r eine $\alpha$-Orientierung $X$, wenn $C_1$ in $X$ ccw gerichtet ist, und $C_i$ in $X^{C_1\dots C_{i-1}}$ jeweils ccw gerichtet ist, wobei $X^{C_1\dots C_{i-1}}$ der Graph ist, der durch die sukzessiven Flips von $C_1,\dots C_i$ aus X entsteht. 


\begin{lemma}
Eine Kante kann zu h�chstens zwei verschiedenen essentiellen Kreisen geh�ren, eine Kante am Rand des Graphen kann zu h�chstens einem geh�ren.
\end{lemma}
\begin{proof}
Angenommen eine Kante $e\in A$ geh�rt zu zwei essentiellen Kreisen, die beide auf der gleichen Seite, zum Beispiel links von $e$ liegen. Dann w�rden sich diese beiden Kreise mindestens die Kante $e=(u,v)$ teilen. Da sie auch beide auf der gleichen Seite liegen, muss das Innere eines der beiden zumindest teilweise im Inneren des andere enthalten sein. Somit gibt es aber einen gerichteten Pfad (von $v$ nach $u$) im Inneren und des anderen, dieser kann somit nicht essentiell sein.
\end{proof}

\begin{lemma}
\label{alternierungslemma}
Sei eine zul�ssige Flipsequenz $(C_1\dots C_k)$ f�r eine $\alpha$-Orientierung $X$ gegeben. Nenne f�r eine Kante $e$ in $X$ den essentiellen Kreis links von $e$ $C^{l(e)}$ und den rechts von $e$ $C^{r(e)}$. Dann kommen $C^{r(e)}$ und $C^{l(e)}$ in der Flipsequenz nur alternierend vor. Formaler: wenn $C_{i_1}=C_{i_2}=\C^{l(e)}$, $i_1<i_2$ dann existiert $i_3$ mit $i_1<i_3<i_2$ und $C_{i_3}=C^{r(e)}$, analog wenn rechts und links vertauscht sind.
\end{lemma}
\begin{proof}
Wenn der Kreis $C^{l(e)}$ geflippt wird, �ndert sich seine Drehrichtung von ccw zu cw. Damit ein zweiter Flip ausgef�hrt werden kann, m�ssen also erst wieder alle Kanten von $C^{l(e)}$ umgedreht werden, insbesondere die Kante $e$. Nach dem vorherigen Lemma kann $e$ aber nur umgedreht werden, indem $C^{r(e)}$ geflippt wird.
\end{proof}

\begin{lemma}
F�r jede Kante $e\in \overrightarrow{G}$ existiert eine nat�rliche Zahl $t_e$, so dass in einer zul�ssigen  Flipsequenz $e$ h�chstens $t_e$ mal umgedreht wird. Jeder essentielle Kreis $C$ wird in einer zul�ssigen  Flipsequenz maximal $\frac{\min_{e\in C}{t_e}}{2}+1$ mal geflippt. Andererseits kann, wenn ein essentieller Kreis in einer Flipsequenz $s$ mal geflippt wird, jede Kante in diesem Kreis h�chstens $2s+1$ mal umgedreht werden.\\

Folglich gilt f�r jeden essentiellen Kreis, dass der Wert $t_e$ seiner Kanten um maximal 2 auseinanderliegen kann.
\end{lemma}

\begin{proof}
Kanten, die in keinem essentiellen Kreis enthalten sind, lassen sich nicht drehen, haben also den Wert $t_e=0$. Dass jeder essentielle Kreis in einer zul�ssigen Flipsequenz maximal $\frac{\min_{e\in C}{t_e}}{2}+1$ oft geflippt wird, folgt aus Lemma \ref{alternierungslemma} : F�r jede Kante im Kreis m�ssen die beiden essentiellen Kreise in denen sie maximal enthalten ist alternierend geflippt werden. F�r jeden Flip, bis auf den letzten m�glicherweise, muss also die Kante ein weiteres mal umgedreht werden, und der Kreis kann nicht h�ufiger geflippt werden als jede seiner Kanten. Da dies auch f�r die Nachbarkreise gilt, kann jede Kante nur h�chstens $2s+1$ mal gedreht werden, was das Maximum der Summe seiner beiden benachbarten Kreise ist (nur einer der beiden kann der letzte in der Flipsequenz sein). \\
Nun k�nnen wir vom Rand des Graphen her eine Obergrenze f�r den Wert $t_e$ jeder Kante $e$ konstruieren. Kanten, die auf dem Rand des Graphen liegen, k�nnen h�chstens einmal gedreht werden. Kanten, die nicht auf dem Rand liegen, aber in einem essentiellen Kreis wo mindestens f�r eine Kante $t_e$ festgelegt ist, k�nnen h�chstens den Wert $t_e+2$ bekommen. Da es nur endlich viele essentielle Kreise gibt und da der Dualgraph zusammenh�ngend ist, so dass man jedes Face betrachtt, bekommt jede Kante einen endlichen Wert zugeordnet.
\end{proof}

Da die Anzahl der essentiellen Kreise sowie die Anzahl der Flips selbiger also begrenzt sind, folgt direkt dass auch die L�nge jeder zul�ssigen Flipsequenz begrenzt ist. Dies sichert die Azyklizit�t, welche aber sp�ter auf dem Torus nicht mehr so einfach gegeben ist, wie wir bereits in dem Beipiel am Ende des letzten Kapitels gesehen haben. 

\begin{lemma}
Es gibt eine eindeutige $\alpha$-Orientierung in $\overrightarrow{G}$, so dass alle Kreise cw, also "'im Uhrzeigersinn"' gerichtet sind.
\end{lemma}

\begin{proof}
Solange es noch einen Kreis gibt, der ccw gerichtet ist, k�nnen wir diesen mittels essentieller Kreise flippen. Da wir aber gezeigt haben, dass die L�nge einer solchen zul�ssigen Flipsequenz begrenzt ist, muss es mindestens eine Orientierung geben, so dass kein Kreis mehr ccw gerichtet ist. 
Angenommen es g�be zwei unterschiedliche Orientierungen, so dass darin alle Kreise cw gerichtet sind. Die in beiden unterschiedlich gerichteten Kanten m�ssen dann in jeder Orientierung eine Menge von gerichteten Kreisen bilden, da in jedem Knoten das $\alpha$ gleich ist ($\Rightarrow$ f�r jede eingehende Kante die gedreht wird muss auch eine ausgehende gedreht werden und andersherum - dies setzt sich auf den Nachbarknoten fort bis man einen Kreis schlie�t). Da es nur zwei M�glichkeiten gibt, wie Kreise gerichtet sein k�nnen, m�sste in einem Graph der Differenzkreis ccw gerichtet sein \lightning .\\
Also kann es nur eine Orientierung geben, in der alle Kreise cw gerichtet sind.
\end{proof}

\begin{lemma}
Seien zwei $\alpha$-Orientierungen $X,Y$ gegeben, so dass es eine zul�ssige  Flipsequenz gibt, die $X$ in $Y$ �berf�hrt. Dann wird jeder essentielle Kreis $C$ in allen zul�ssigen Flipsequenzen gleich oft geflippt.

\end{lemma}

\begin{proof}
Wir zeigen, dass die Orientierung einer Kante in $X$ und in $Y$ bereits festlegt, wie oft die beiden benachbarten essentiellen Kreise geflippt werden m�ssen.\\

\begin{itemize}
\item Falls die Kante $e$ in beiden Orientierungen gleich gerichtet ist, muss sie gerade viele Male umgedreht werden. Aus Lemma \ref{alternierungslemma} folgt, dass die Anzahl der Drehungen der beiden benachbarten Kreise sich um h�chstens 1 unterscheiden kann $\Rightarrow$ also m�ssen beide Kreise gleich oft gedreht werden. 
\item Falls $e$ in beiden unterschiedlich gerichtet ist: $C^{l(e)}$ ist der Kreis, der in $X$ links von $e$ liegt. Das hei�t insbesondere, dass $C^{l(e)}$ ccw gerichtet ist und deshalb zuerst gedreht werden muss. Folglich ist es auch $C^{l(e)}$, der einmal h�ufiger als $C^{r(e)}$ geflippt wird. 
\end{itemize}
Das �u�ere Face kann nie geflippt werden, und bekommt daher den Wert $0$ zugewiesen, von diesem Startpunkt aus k�nnen wir nun jedem essentiellen Kreis iterativ den Wert zuweisen, wie oft er in der Flipsequenz vorkommen muss - je nach Orientierung der Kanten k�nnen wir die Werte aller Nachbarkreise berechnen. Da der Dualgraph immer zusammenh�ngend ist, wird dabei auch kein Kreis ausgelassen.
\end{proof}
Nun k�nnen wir das $\alpha$-Potential definieren.

\begin{definition}
Sei $\mathcal{E}$ die Menge der essentiellen Kreise in einer $\alpha$-Orientierung $X$. Eine Abbildung $\mathcal{P}:\mathcal{E}\to \mathbb{N} $, die jedem essentiellen Kreis eine nat�rlich Zahl zuweist, ist ein $\alpha$-Potential, wenn
\begin{itemize}
\item $\mathcal{P}(C)\le 1$ wenn es eine Kante gibt, f�r die $C$ der einzige essentielle Kreis ist
\item $|\mathcal{P}(C)-\mathcal{P}(C')|\le 1$ falls $C$ und $C'$ eine gemeinsame Kante haben
\item Seien $C^{l(e)}$ bzw. $C^{r(e)}$ der in der minimalen Orientierung (in der alle Kreise cw gerichtet sind) links bzw rechts von $e$ liegende essentielle Kreis. Dann gilt  $\mathcal{P}(C^{l(e)})\le \mathcal{P}(C^{r(e)})$
\end{itemize}
Wir definieren $z_X:\mathcal{E}\to \mathbb{N},\, z_X(C)=$ Anzahl der Flips von $C$ in einer zul�ssigen  Flipsequenz die von $X$ zu der Orientierung ohne ccw gerichtete Kreise f�hrt. Das vorherige Lemma sichert die Konsistenz dieser Definition.
\end{definition}

Wir sehen sofort, dass die ersten beiden Bedingungen  von $z_X$ wegen der Alternierung  (Lemma \ref{alternierungslemma}) erf�llt werden. Auch die dritte Bedingung gilt: \\
Da in $X_{min} C^{l(e)}$ links von $e$ liegt, muss der letzte Flip, von dem $e$ betroffen war der von $C^{r(e)}$ gewesen sein. Also gilt $z_X(C^{l(e)})\le z_X(C^{r(e)})$.
F�r die Bijektion ben�tigen wir noch die andere Richtung:

\begin{lemma}
F�r jedes $\alpha$-Potential $\mathcal{P}$ gibt es eine $\alpha$-Orientierung $X$ mit $z_X=\mathcal{P}$.
\end{lemma}

\begin{proof}
Wir definieren eine Orientierung $X_\mathcal{P}$ der Kanten von $\overrightarrow{G}$ ausgehend von unserem $\alpha$-Potential $\mathcal{P}$.
\begin{itemize}
\item Kanten, die in keinem essentiellen Kreis enthalten sind, m�ssen in allen Orientierungen gleich gerichtet sein, diese bekommen also die gleiche Orientierung wie in der eindeutigen minimalen Orientierung $X_{min}$.
\item Falls eine Kante $e$ in genau einem essentiellen Kreis $C$ enthalten ist, und dieser Kreis das Potential $\mathcal{P}(C)=0$ hat, orientiere $e$ genau wie in $X_{min}$, ansonsten (im Fall $\mathcal{P}(C)=1$, da auf jeden Fall $\mathcal{P}(C)\le 1$) orientiere $e$ genau andersherum.
\item Falls Kante $e$ in zwei essentiellen Kreisen $C^{l(e)}$ und $C^{r(e)}$ , links und rechts bez�glich $X_{min}$, ist, dann wird $e$ wie in $X_{min}$ gerichtet, falls $\mathcal{P}(C^{l(e)})=\mathcal{P}(C^{r(e)})$, andernfalls genau andersherum.
\end{itemize}
Es ist nun zu zeigen, dass dies tats�chlich eine $\alpha$-Orientierung ist:\\
Betrachte einen Knoten $v$ und das Potential $\mathcal{P}$. Wenn wir in einer beliebigen inzidenten Kante starten und im Uhrzeigersinn um $v$ herumlaufen, sehen wir jede zu $v$ inzidente Kante und alle ihre benachbarten essentiellen Kreise. Aufgrund der Bedingungen $\mathcal{P}(C^{l(e)})\le \mathcal{P}(C^{r(e)})$ und $|\mathcal{P}(C)-\mathcal{P}(C')|\le 1$ f�r benachbarte Kreise gilt: wenn wir im Uhrzeigersinn eine in $X_{min}$ von $v$ ausgehende Kante �berqueren, kann das Potential des Kreises in dem wir uns dann befinden nur gleich oder um eins h�her sein als im vorherigen, wenn wir eine in $X_{min}$ eingehende Kante �berqueren kann das Potential gleich oder um eins kleiner sein. Da wir wieder am Ausgangspunkt ankommen (also beim gleichen Potential), muss es in $\mathcal{P}$ gleich viele Kanten geben, an denen sich das Potential f�r uns erh�ht oder senkt. Daraus folgt, dass es genauso viele eingehende Kanten gibt, an denen das Potential der beiden benachbarten Kreise unterschiedlich ist, wie es ausgehende Kanten mit unterschiedlichem Potential auf beiden Seiten gibt. Genau diese Kanten werden nach unserer Konstruktion aber umgedreht $\Rightarrow$ also werden gleich viele eingehende wie ausgehende Kanten im Vergleich zu $X_{min}$ umgedreht. Da $X_{min}$ eine zul�ssige $\alpha$-Orientierung ist, muss $X_\mathcal{P}$ ebenfalls eine solche sein.
%TODO Bild malen zum beweis?
\end{proof}

\begin{theorem}
Sei ein planarer Graph $\overrightarrow{G}$ mit einer festen, kreuzungsfreien Einbettung in die Ebene gegeben. Auf der Menge der $\alpha$-Orientierungen von $\overrightarrow{G}$ definieren die Flips von essentiellen Kreisen einen distributiven Verband.
\end{theorem}


\begin{proof}
Wir haben gesehen, dass es eine Bijektion zwischen den $\alpha$-Orientierungen und den $\alpha$-Potentialen gibt. Es reicht also zu zeigen, dass die $\alpha$-Potentiale einen distributiven Verband bilden. $\alpha$-Potentiale lassen sich als Vektoren darstellen, definiere $\mathcal{P}_1\lor \mathcal{P}_2=\max\{\mathcal{P}_1,\mathcal{P}_2\}$ und $\mathcal{P}_1\land \mathcal{P}_2=\min\{\mathcal{P}_1,\mathcal{P}_2\}$ , Maximum und Minimum sind dabei jeweils komponentenweise definiert. Die Distributivit�t des komponentenweisen Maximums und Minimums folgt aus \ref{distribDominanzordnung}, wir m�ssen also nur zeigen, dass das komponentenweise Maximum und Minimum wieder ein $\alpha$-Potential ist. \\
Betrachte eine Kante $e$ und dazu $C^{l(e)}$ und $C^{r(e)}$. Da in $\mathcal{P}_1$ sowie $\mathcal{P}_2$ jeweils $C^{l(e)}\le C^{r(e)}$ gilt, gilt das gleiche auch f�r $\mathcal{P}_1\lor \mathcal{P}_2$. \\
Sei nun oBdA $\mathcal{P}_1(C^{r(e)})=\max\{\mathcal{P}_1(C^{r(e)})\lor \mathcal{P}_2(C^{r(e)}) \}$, dann ist $(\mathcal{P}_1\lor \mathcal{P}_2)(C^{l(e)})\ge \mathcal{P}_1(C^{l(e)})\ge \mathcal{P}_1(C^{r(e)})-1$. Demnach gilt die Bedingung  $|\mathcal{P}(C)-\mathcal{P}(C')|\le 1$ f�r benachbarte Kreise. Die Bedingung "'$\mathcal{P}(C)\le 1$ wenn es eine Kante gibt, f�r die $C$ der einzige essentielle Kreis ist"' folgt sofort aus der Definition. Der Beweis f�r das Minimum ist analog.
\end{proof}

Die planaren $\alpha$-Orientierungen bilden also einen distributiven Verband. Wie verh�lt dieser sich zu dem aus dem ersten Beweis speziell f�r Domino Tilings? Es ist genau der selbe, wenn wir die $(3,1)$-Orientierung des selben Graphen betrachten! Wenn wir beim ersten Beweis annehmen, dass die in den Bildern schraffierten Fl�chen diejenigen sind, dessen Knoten in der zugeh�rigen $\alpha$-Orientierung genau eine ausgehende Kante haben, dann sind die dort definierten "'aufw�rts gerichteten"' Flips genau die hier von ccw zu cw gerichteten, andernfalls sind es genau die von cw zu ccw gehenden. In jedem Fall ergibt sich bis auf Umkehrung der Relation der genau gleiche distributive Verband.\\

Auch hier m�ssen wir uns fragen, wie sich der Beweis auf den Torus verallgemeinern l�sst oder warum nicht. Die Antworten sind wiederum �hnlich: genau wie im ersten Beweis ben�tigt die Azyklizit�t der Relation einen Rand des Graphen, der (als �u�eres Face m�glicherweise durchaus gerichtet) nicht geflippt werden darf. Dieser Rand ist auf dem Torus nicht nat�rlich vorhanden, man m�sste das entsprechende Face also willk�rlich festhalten.\\
Auf dem Torus gibt es aber insbesondere auch Orientierungen, die keine gerichteten Facekreise besitzen, der Flipzusammenhang fehlt also. Mit eben diesem besch�ftigt sich das n�chste Kapitel.

\newpage
\section{Situation auf dem Torus}Wie wir gesehen haben, ist es auf planaren $\alpha$-Orientierungen, also auch auf planaren Domino-Tilings m�glich, nur mittels drehen von essentiellen Kreisen (in Domino-Tilings tats�chlich Face-Flips) jede m�gliche Konfiguration zu erreichen.

Dabei stellt sich die Frage, ob und wie man dieses Resultat verallgemeinern kann. $\alpha$-Orientierungen lassen sich problemlos auf dem Torus definieren. Eine $(3,1)$-Orientierung wie im vorherigen Kapitel k�nnen wir zumindest auf torischen Gittergraphen gerader Breite und H�he definieren - eine ungerade Breite oder H�he macht �hnlich wie bei der H�henfunktion Probleme, da es dann benachbarte Knoten geben m�sste, die den gleichen Ausgrad haben. Wir betrachten hier daher nur torische Gittergraphen, die sowohl gerade H�he als auch Breite haben.

Eine besondere Rolle spielt auf dem Torus die topologische Struktur. 

\begin{definition}
Wir k�nnen Kreise in $\overrightarrow{G}$ als Vektoren des Kreisraums in $\{-1,0,1\}^{|A|}$ betrachten, wobei die $i$-te Komponente des Vektors der Kante $e_i$ entspricht und den Wert 1 hat, falls die Kante im Kreis enthalten und bez�glich einer festen Orientierung $X$ gleich gerichtet ist, 0 falls $e_i$ nicht im entsprechenden Kreis enthalten ist und -1 sonst. \\
Wenn $v_C$ der Vektor einer Menge von gerichteten Kreisen ist und sich als Linearkombination von Vektoren in $\{-1,0,1\}$ von Face-Kreisen schreiben l�sst, nennen wir $C$ nullhomolog.
\end{definition}

\begin{figure}[h!]
  \centering
  \scalebox{1}{\input{Bilder/nullhomolog.pstex_t}}
  \caption{Betrachte die beiden roten Kreise: gemeinsam sind sie in dem linken Bild nullhomolog, im rechten sind sie es nicht mehr}
\end{figure}

 Kolja Knauer hat in \cite{Knauer07partialorders} gezeigt, dass es f�r einen torischen Gittergraphen mit $\alpha(v)=2 \forall v \in V$ ausreicht, wenn man alle Faces bis auf eines, und zus�tzlich noch zwei nicht nullhomologe Kreise erlaubt, um den Zusammenhang und die Poset-Struktur zu erreichen. Durch das Festhalten eines Faces schafft man es insbesondere, die Azyklizit�t zu sichern.
Zugleich muss die ben�tigte Menge von Kreisen eine Basis des Kreisraumes enthalten, daher ist diese Menge in seinem Beispiel auch optimal. 

Wir wollen das "'erzeugen einer Orientierung"' formaler machen: Definiere eine \textit{zul�ssige Flipsequenz} in einer Orientierung $X$ von $\overrightarrow{G}$ auf dem Torus als eine Folge $C_1,\dots ,C_n$ von Kreisen, so dass $C_1$ in $X$ gerichtet ist, und $C_i, \, i=\{2,\dots,n\}$ jeweils in dem Graphen gerichtet ist, der entsteht wenn man nacheinander die Kreise $C_1,\dots ,C_{i-1}$ flippt. 
Wir erlauben hier also beliebige Kreise mit beliebiger Drehrichtung zu flippen. Wenn durch eine Folge von Flips von $C_1\dots C_n$  genau die Kantenmenge eines Kreises $C$ umgedreht wird, der selber nicht geflippt wurde, sagen wir dass $C$ sich aus $C_1\dots C_n$ zusammensetzen oder erzeugen l�sst. 

Beispielsweise haben wir bereits gesehen, dass sich jeder Kreis in planaren $\alpha$-Orientierungen durch Folgen von Face-Flips erzeugen l�sst. Teilt ein Kreis $C$ die Fl�che $S$ auf der $\overrightarrow{G}$ eingebettet ist in zwei Regionen, von denen eine hom�omorph zu einer offenen Kreisscheibe ist, dann ist der $C$ mit dem auf dieser Region engebetteten Teil des Graphen ein planarer Teilgraph von $\overrightarrow{G}$, und wir k�nnen $C$ wie bereits gezeigt durch Face-Flips in seinem Inneren zusammensetzen. 
 
Kann man also �hnliches auch f�r den Fall der torischen Domino Tilings, also der $(3,1)$-Orientierungen auf dem Torus zeigen? Kann man eine Menge von Kreisen ausw�hlen, mit denen man tats�chlich jede Orientierung auf dem Graphen mittels Kreisflips erzeugen kann, sozusagen eine Art Erzeugendensystem f�r alle Orientierungen? Wir werden sehen, dass wir hier eine deutlich gr��ere Menge von Kreisen brauchen, eine Basis des Kreisraumes also nicht ausreicht.

Betrachten wir also eine $(3,1)$-Orientierung auf dem Torus. L�sst man tats�chlich das Drehen aller denkbaren Kreise zu, so ist es sofort m�glich alle Orientierungen ausgehend von einer beliebigen Orientierung zu erzeugen, wie wir in Lemma \ref{allgFlipZshg} gezeigt haben. 

Da wir gerade gesehen haben, dass Face-Flips auch hier gr��ere Kreise erzeugen k�nnen, ist klar dass nicht alle Kreise n�tig sind. Kann man also m�glichst kleine Kreismengen finden um alle Orientierungen zu erzeugen?


\subsection{Gegenbeispiele}
Hier soll es darum gehen, welche Klassen von Kreisen sich nicht eignen bzw nicht ausreichen um �berhaupt den Flipzusammenhang zu gew�hrleisten - und wie man das jeweils beweisen kann.\\
Zur Notation: Je nachdem was man zeigen will, eignen sich verschiedene Modelle besser um zu verstehen, was geschieht. Ich werde hier im wesentlichen von den $\alpha$-Orientierungen ausgehen, die auf dem zugrundeliegenden torischen Gittergraphen das Domino Tiling des Torus eindeutig bestimmen. Immer wenn von Kanten die Rede ist, sind daher Kanten in diesem Graph gemeint. "'Matchingkanten"' sind darin die Kanten, die komplett in einem jeweiligen Domino liegen, d.h. die f�r ihre inzidenten Knoten die jeweils einzige ausgehende bzw einzige eingehende Kante bilden. In den Bildern sind die jeweils gegen�berliegenden R�nder identifiziert sofern nicht explizit etwas anderes angegeben ist, so dass Dominos/Kanten des Graphen "'�ber den Rand"' gehen k�nnen.\\

Als einfaches Beispiel daf�r, dass allein Face-Kreise nicht ausreichen kann das Muster einer "'Ziegelmauer"' dienen. Dabei sind die Matchingkanten im Graphen horizontal angeordnet, aber wie bei einer Mauer um jeweils eins versetzt. 
\begin{figure}[h!]
  \centering
  {\input{Bilder/ex0.pstex_t}}
\caption{Ein Domino Tiling des Torus (gegen�berliegende Seiten identifiziert) , bei dem kein Face-Flip m�glich ist}
\end{figure}
Bei dieser Anordnung gibt es allerdings keinen einzigen gerichteten Face-Kreis, bzw. f�r Tilings keine zwei nebeneinanderliegenden Dominos, die man drehen k�nnte. Allein mit Face-Flips kommt man hier also nicht weiter, es braucht also offensichtlich mehr Kreise. F�r dieses Beispiel sticht eine L�sung sofort ins Auge - man kann gerade horizontale (und wegen Rotationssymmetrie vertikale) Kreise erlauben, d.h. Kreise in denen alle Kanten in die gleiche Richtung zeigen. Damit kann man zumindest andere Orientierungen erzeugen, es ist jedoch nicht klar ob man alle erh�lt. (Tats�chlich erh�lt man nicht alle wie wir gleich sehen werden!)\\

Wenn man erlaubt, einen nicht nullhomologen Kreis umzudrehen, kann dies durchaus das Flippen weiterer Kreise erm�glichen wenn der Kreis denn gerichtet ist. Betrachtet man allerdings den gleichen Graph, allerdings etwas verschoben, dann ist dieser Kreis im allgemeinen nicht mehr gerichtet und kann deshalb nicht umgedreht werden. Aus diesem Grund werde ich hier im weiteren Kreisklassen behandeln: 
\begin{definition}
Eine Kreisklasse ist eine Menge von Kreisen, f�r die es f�r jeden Kreis $C$ darin einen Knoten $v_0^C$ gibt, so dass von $v_0^C$ ausgehend die Folge der Kantenrichtungen ( $\in$ \{Hoch, Runter, Links, Rechts\} ) die gleiche ist. Wir erlauben also eine bestimmte Form von Kreisen mit jedem m�glichem Startpunkt im Graphen zu flippen, wenn wir davon sprechen eine Kreisklasse zu erlauben.
\end{definition}

Welche nat�rlichen Kreise bieten sich noch an, nachdem Face-Flips, wie wir gesehen haben, nicht reichen? Reicht es beispielsweise aus, wenn man gerade und treppenf�rmige Kreise w�hlt? Auch hier ist die Antwort nein. Nehmen wir ein quadratisches Gitter, horizontale und vertikale gerade Kreise und dazu die folgenden beiden Kreistypen ("'Treppen"'):\\
%TODO Bilder verbessern
\begin{figure}[h!]
  \centering
  \scalebox{1}{\input{Bilder/treppen1.pstex_t}}
\end{figure}

Erlauben wir nun das Flippen dieser drei Kreisklassen. Im Gegensatz zu dem obigen Beispiel kann man hier aufgrund der Form der Kreise nicht mehr erwarten, eine Orientierung/ein Tiling zu finden in der alle in Frage kommenden Kreise (insbes. Faces) nicht gerichtet sind. Betrachte beispielsweise folgendes Beispiel: in Abb 7 sind zwar einzelne Faces flipbar, allerdings ist nicht klar ob man hiermit alle anderen Orientierungen/Tilings erreichen kann. Jedoch sind s�mtliche nun erlaubten nicht nullhomologen Kreise in diesem Beispiel nicht gerichtet, also nicht flipbar.\\

\begin{figure}[h!]
\label{ex1}
  \centering
  \scalebox{1}{\input{Bilder/ex1.pstex_t}}
\caption{Beispiel, in dem Faces (gelb markiert) drehbar sind. Lassen sich hieraus mit den obigen Kreisen alle Orientierungen erzeugen?}
\end{figure}
 
Das folgende Beispiel zeigt uns, dass auch mit Face-Flips mitunter nur sehr wenige Orientierungen erreichbar sind. Wir k�nnen also nicht einfach durch Face-Flips immer in eine g�nstigere Position kommen, in der wir gr��ere nicht nullhomologe Kreise gerichtet haben, sondern es kann passieren, dass wir immer wieder in �hnlich schlechte Orientierungen kommen (Abb 8).
 
 \begin{figure}[h!]
\label{ggbsp1}  
  \centering
  \scalebox{0.9}{\input{Bilder/ggbsp1.pstex_t}}
  \caption{Die Ausgangskonfiguration und die Endkonfiguration sind bis auf eine Verschiebung gleich.}
\end{figure}

Da wir offenbar viele Beispiele finden k�nnen, bei denen man nicht sofort sehen kann ob alle Orientierungen erreicht werden k�nnen, brauchen wir eine aussagekr�ftige Bedingung, mit der man (m�glichst mit geringem Aufwand) pr�fen kann, ob man wirklich nicht alle anderen Orientierungen mit den gegebenen Kreisen erzeugen kann. Das hei�t wir wollen pr�fen, ob man an einer konkreten Orientierung sehen kann, dass die gegebenen Kreisklassen nicht ausreichen um den Flipzusammenhang herzustellen.\\

%Notation: In der zu einem DominoTiling geh\"orenden $\alpha$-Orientierung hat jeder Knoten eine eindeutige Kante - die einzige eingehende oder einzige ausgehende. Diese nennen wir hier die Matchingkante des Knotens. 
\begin{definition} 
Sei ein Kreis $C$ und ein Knoten $v$ auf diesem Kreis gegeben. Sind die beiden in $C$ zu $v$ inzidenten Kanten in Durchlaufrichtung des Kreises entgegengesetzt gerichtet, so nennen wir diesen Knoten eine Konfliktstelle von $C$.
\end{definition}

\begin{lemma}
\label{notwBed}
Sei ein Kreis $C$ in einer torischen $(deg-1,1)$-Orientierung gegeben. 
Sei $M_{ein}$ die Anzahl der (in Durchlaufrichtung) auf der rechten Seite in $C$ eingehenden Matchingkanten des Graphen und $M_{aus}$ die Anzahl der auf dieser Seite aus $C$ ausgehenden Matchingkanten an Konfliktstellen. \\
Dann ist f�r jede Folge von Face-Flips im Graphen die Differenz $M_{ein}-M_{aus}$ f�r $C$ konstant.
\end{lemma}
Da in einem gerichteten Kreis die Differenz $M_{ein}-M_{aus}=0 $ ist, f�hrt uns dies direkt zu folgendem, �u�erst n�tzlichen Korollar:

\begin{cor}
Wenn zu einer Seite des Kreises, oBdA in Durchlaufrichtung rechts, die Anzahl der in diese Richtung an den Konfliktstellen ausgehenden Matchingkanten ungleich der aus dieser Richtung an Konfliktstellen eingehenden Matchingkanten ist, so kann man den Kreis nicht durch flippen einer Folge von Face-Kreisen in einen gerichteten \"uberf\"uhren.
\end{cor}

\begin{proof}[Beweis des Lemmas]
Betrachte einen Face-Flip. Falls das Face keinen Knoten mit $C$ gemeinsam hat, so ist die Bahauptung klar. Andernfalls betrachten wir also ein gerichtetes Face, das zur rechten Seite von $C$ liegt und mindestens einen gemeinsamen Knoten mit $C$ hat. Der Teil vom Face, der nicht zu $C$ geh�rt, ist ein Pfad der zur rechten Seite von $C$ ein- und ausgeht und auf dem die Richtung der Kanten geflippt wird. Da er gerichtet ist, wechseln sich auf ihm Matchingkanten mit Nichtmatchingkanten ab.
\begin{enumerate}
\item Fall 1: Der Pfad hat eine gerade Anzahl von Kanten:\\
Dann ist entweder die erste oder die letzte Kante eine Matchingkante - egal wierum der Pfad gerichtet ist. Ist vor dem flippen des Kreises die erste Kante des Pfades eine Matchingkante, so ist es danach die (nun erste) Kante, die vorher letzte des Pfades war. Das flippen �ndert also nichts an den Werten $M_{ein}$ und $M_{aus}$. 
\item Fall 2: Der Pfad hat ungerade Anzahl Kanten: \\
Dann sind beide oder keine der zu $C$ inzidenten Kanten im Matching. Wenn es beide sind, dann sind sie auf dem Pfad gleich gerichtet, und deshalb ist eine von $C$ aus gesehen eine eingehende und die andere eine ausgehende Kante. Die Differenz $M_{ein}-M_{aus}$ bleibt also genauso wie wenn keine der beiden Kanten im Matching ist.
\item es bleibt noch zu zeigen, warum wir nur Matchingkanten an Konfliktstellen betrachten:\\
Ist in einem Knoten eine Kante im Matching, dann sind die restlichen Kanten entweder alle im Bezug zum Knoten ausgehend oder alle eingehend. Dadurch sind sie aber entgegengesetzt gerichtet, wenn also von einem Knoten auf $C$  die Matchingkante nicht zu $C$ geh�rt, dann ist dieser Knoten immer ein Konfliktstelle. Man w�rde also auch wenn man es nicht explizit sagt nur die Matchingkanten an Konfliktstellen z�hlen.
\end{enumerate}

\end{proof}


Was hilft uns nun dieses Lemma? Wir haben nun ein leicht zu �berpr�fendes Kriterium an der Hand, um zu entscheiden ob eine gegebene Menge von Kreisen \textit{nicht} ausreicht um alle Domino-Tilings mittels Flips zu erzeugen. Dazu reicht es, ein Beispiel anzugeben, in dem s�mtliche gegebenen Kreise nicht gerichtet sind, und nachzuweisen dass au�erdem f�r jeden dieser Kreise $M_{ein}-M_{aus} \neq 0$ ist. 
Es gibt uns also eine notwendige Bedingung daf�r, dass man ungerichtete Kreise mittels Face-Flips in gerichtete �berf�hren kann. 
%TODO Ist diese Bedingung aber auch hinreichend? hab leider kein richtiges Beispiel...


Nun stellt sich nat�rlich die Frage nach konkreten Beipielen, die diese Bedingung erf�llen. Wir haben bereits einige kleine Beipiele gesehen, aber konnten noch nicht beweisen, welche Kreise in diesen Beispielen jeweils nicht ausreichend sind.
Der n�chste Abschnitt wird sich daher genauer damit besch�ftigen.

\subsection*{Einige konkrete Gegenbeispiele}
Um konkrete Beispiele zu finden gibt es verschiedene M�glichkeiten. Man kann zum Beispiel ein beliebiges Domino Tiling hernehmen und alle m�glichen Kreise �berpr�fen ob sie die Bedingungen des Lemmas erf�llen. Die Wahrscheinlichkeit ist hoch, dabei einige Kreisklassen oder Kreise zu entdecken.
Man kann sich andererseits auch eine Menge von Kreisen/Kreisklassen definieren, die man ausschlie�en will. Anschlie�end konstruiert man ein entsprechendes Domino Tiling per Hand (falls das m�glich ist) und �berpr�ft es dann mit Hilfe des Lemmas. Auf diese Weise bin ich bei den folgenden Beispielen vorgegangen:

\begin{figure}[h!]
  \centering
  \scalebox{1}{\input{Bilder/ex1.pstex_t}}
  \caption{Beipiel 1 }
\end{figure}

\begin{figure}[h!]
  \centering
  \scalebox{1}{\input{Bilder/ex2.pstex_t}}
  \caption{Beipiel 2 }
\end{figure}

\begin{figure}[h!]
  \centering
  \scalebox{1}{\input{Bilder/ex2orientierung.pstex_t}}
  \caption{nochmal Beispiel 2 mit dem zugeh�rigen gerichteten Graph}
\end{figure}

Folgende Klassen von Kreisen sind in diesem Domino-Tiling nicht flipbar: 

\begin{figure}[h!]
  \centering
  \scalebox{1}{\input{Bilder/treppen3.pstex_t}}
  \caption{Kreisklassen die in Bsp 2 nicht durch Face-Flips gerichtet gemacht werden k�nnen}
\end{figure}

Zur �berpr�fung habe ich ein kleines Java-Programm genutzt. %Das Programm ranh�ngen???


Sind diese Beispiele aber auch aussagekr�ftig f�r allgemeinere F�lle, oder k�nnen es m�glicherweise nur Ausnahmen sein, die auf besonders kleinen Instanzen auftreten? Auch hierf�r hilft uns unser Lemma weiter.

\begin{definition}
\label{defT2}
Sei ein torisches Domino Tiling $T$ gegeben. Wir haben torische Domino Tilings definiert �ber Domino Tilings einer rechteckigen Region, bei der die jeweils gegen�berliegenden Seiten identifiziert werden. Bezeichne die R�nder dieser rechteckigen Region als $a,b,c,d$ wobei $a,c$ und $b,d$ jeweils die gegen�berliegenden Seiten sind, die urspr�nglich identifiziert wurden. Konstruiere nun folgenderma�en ein doppelt so gro�es Domino Tiling aus $T$: Erstelle eine Kopie $T'$ von $T$, mit den Seiten $a',b',c',d'$, wobei $a'$ die Kopie von $a$ ist usw. Identifiziere nun die Seitenpaare: $(b,d),(b',d'),(a,c'),(a',c)$. Das entstandene torische Domino Tiling nenne $T^2$. 
\end{definition}

Nun k�nnen wir folgendes beobachten: 

\begin{lemma}
Sei ein torisches Domino Tiling $T$ gegeben, so dass $k$ Kreise die Bedingung aus Lemma \ref{notwBed} nicht erf�llen,  um durch Face-Flips in gerichtete Kreise �berf�hrt zu werden, die also an Konfliktstellen ungleich viele eingehende und ausgehende Kanten haben. Dann hat $T^2$ mindestens genauso viele Kreise mit dieser Eigenschaft.
\end{lemma}

\begin{proof}
Durch die Konstruktion ergibt sich, dass jeder Kreis der �ber den Rand $(a,c)$ verl�uft, zu einem gr��eren Kreis aus zwei gleichartigen Kopien verl�ngert wird. Die Anzahl der ein- bzw. ausgehenden Matchingkanten an Konfliktstellen wird dabei jeweils verdoppelt, da ja der gleiche Kreis mit der gleichen Orientierung nun zwei mal durchlaufen wird. Insbesondere bleiben die Zahlen unterschiedlich, wenn sie es vorher waren.
\end{proof}


Haben wir also mit Hilfe des Lemmas \ref{notwBed} ein Beispiel gefunden bei dem wir mindestens $k$ verschiedene Kreise ben�tigen, so finden wir auch beliebig gr��ere Beispiele die mindestens $k$ verschiedene Kreise ben�tigen. In der Tat scheint es plausibel, dass die Anzahl der ben�tigten Kreisklassen mit der Gr��e der Fl�che monoton wachsend ist. 



Es ist, zumindest wenn man das Wachstum der ben�tigten Kreise betrachtet, daf�r wie gesagt n�tig sich ganze Kreisklassen anzuschauen, da wir nach obiger Beobachtung auf jeden Fall immer gr��ere Beispiele mit im Prinzip den gleichen Kreisklassen bauen k�nnen. Die Anzahl der Kreise in einer Kreisklasse ist aber offensichtlich monoton wachsend, da bei einer gr��eren Fl�che nat�rlich auch mehr Startpositionen zur Verf�gung stehen, an denen gleichgeformte Kreise beginnen k�nnen.

%TODO TODO TODO!!!
\begin{prop}
Die Anzahl der Kreisklassen, f�r die man zeigen kann, dass sie gemeinsam mit Face-Flips nicht gen�gen um alle Orientierungen zu erzeugen, ist unbeschr�nkt, d.h. sie w�chst mit der Gr��e des Gitters.
\end{prop}

\begin{proof}
Wir haben im Beweis des vorherigen Lemmas gesehen, dass man die L�cke zwischen der Zahl der eingehenden Matchingkanten und der ausgehenden Matchingkanten auf einer Seite des Kreises beliebig gro� machen kann. Die Idee ist: Wenn eine lokale �nderung an der Kreisklasse klein genug und die Differenz $M_{ein}-M_{aus}$ gro� genug ist, dann kann es durch die lokale �nderung nicht genug Matchingkanten geben, um den Unterschied auf 0 zu bringen.

Da wir die Differenz $M_{ein}-M_{aus}$ beliebig gro� machen k�nnen, m�ssen wir nur zeigen, dass es immer eine lokale �nderung einer Kreisklasse gibt, durch die eine noch nicht verwendete Kreisklasse entsteht. Dies k�nnen wir folgenderma�en erreichen:\\
Sei $K$ die Menge der Kreisklassen, f�r die wir wissen dass sie in einger gegebenen Orientierung $T$ die Bedingung $M_{ein}-M_{aus}\ne 0$ erf�llen und deren Kreise horizontal verlaufen (d.h. formell: in der Folge der Kantenrichtungen der Kreisklasse kommt f�r ein Gitter der Breite $n$ die Richtung "'Rechts"' $r$ mal auf f�r ein $r\in \N,\, r\le n$ und die Richtung "'Links"' nur $r-n$ mal).
Analog zu der in \ref{defT2} beschriebenen Methode erzeuge eine Orientierung $T^3$ die aus 3 Kopien von $T$ besteht, jeweils an der selben Stelle mit der Nachbarkopie identifiziert. Die Identifizierung soll so erfolgen, dass die horizontalen Kreise aus $K$ jeweils zu einer neuen Kreisklasse verschmelzen. Die Menge der hierdurch entstandenen Kreisklassen nennen wir $K'$.  In $T^3$ hat jede Kreisklasse aus $K'$ nun $|M_{ein}-M_{aus}|\le 3$ in allen enthaltenen Kreisen. F�hre folgende lokale �nderung f�r eine Kreisklasse durch: Da die Kreise horizontal sind, gibt es eine Kante $e=(u,v)$, so dass die Kante zwischen den beiden Knoten, die von $u,v$ jeweils in Richtung "'Runter"' liegen, nicht zu der Kreisklasse geh�rt. Ersetze nun in der Folge der Kantenrichtungen "'Richtung von $e$"' durch "'Runter, Richtung von $e$, Hoch"'. Dies ist tats�chlich eine neue horizontale Kreisklasse, denn alle in $K'$ haben die gleiche Richtungsfolge 3 mal hintereinander, w�hrend die neue Kreisklasse als einzige eine Variation enth�lt. Da nur vier benachbarte Knoten betroffen sind, die als nur insgesamt zwei Matchingkanten haben k�nnen, kann sich $M_{ein}-M_{aus}$ h�chstens um $\pm 2$ ver�ndern, somit folgt $|M_{ein}-M_{aus}|\le 1$ und wir haben nun mindestens $|K|+1$ viele Kreisklassen die Probleme machen. 

\end{proof}
\newpage
\subsection{Einschr\"ankung der ben\"otigten Kreisklassen} 
Wie wir nun wissen, kann man nicht unbedingt erwarten mit wenigen Kreisklassen den Flipzusammenhang wieder herzustellen. %TODO Es waere gut nen Beweis zu haben dass es nicht mit konstant vielen Kreisklassen geht...
Allerdings wissen wir auch, dass sich bestimmte Kreisflips durch mehrere andere erzeugen lassen. Wir betrachten deshalb die Kombination von Faces und anderen Kreisklassen.
 Dabei ist das, was in diesem Kapitel zu finden ist, nur ein Anfang, in dem per Hand einige Kreise aus anderen konstruiert werden. W\"unschenswert w\"are nat\"urlich eher ein Ergebnis, dass allgemein angibt wie viele Kreise man sparen kann wenn man eine bestimmte Menge von Kreisen zu flippen erlaubt - leider ist mir so etwas nicht gelungen. Au�erdem ist zu beachten, dass nicht klar ist, ob eine Menge von Kreisen mit denen man den Flipzusammenhang nachweisbar nicht herstellen kann auch bedeutet, dass man dies nicht doch mit einer kleinen Menge von Kreisen bewerkstelligen kann.\\

Face-Flips sind sehr nat\"urlich und erm\"oglichen uns bereits eine sehr gro\ss e Anzahl von Kreisklassen zu flippen, au�erdem ein nat�rlicher Teil der Kreisraumbasis. Deshalb werden wir immer Face-Flips zulassen. Da diese wie gezeigt nicht ausreichen, brauchen wir mindestens eine weitere Klasse von Kreisen deren Flips wir erlauben. Das einfachste Beispiel hierf�r sind die Kreise, die ohne einen Knick immer geradeaus gehen, aus Symmetriegr\"unden nehmen wir gleich sowohl die horizontalen als auch die vertikalen Kreise.

\begin{definition}
Die Menge der Knoten $(x,y)$, die im torischen Gittergraphen der $(3,1)$-Orientierung die jeweils gleiche Koordinate $x$ oder die gleiche Koordinate $y$ haben, nennen wir horizontale oder vertikale Achse des Graphen. Ein Kreis in diesem Graphen hei�t gerade, wenn alle seine Knoten auf der gleichen Achse liegen.\\

Als Beule (bez�glich einer Achse) bezeichnen wir einen Pfad $P$, so dass $P$ nur Start- und Endknoten auf der Achse hat und mit Kanten des geraden Kreises dieser Achse zu einem einfachen nullhomologen Kreis erweitert werden kann. Die L�nge der Beule ist dabei der maximale Abstand eines Punktes auf $P$ zur Achse bez�glich der Seite, zu der $P$ die Achse verl�sst und betritt. Die Breite ist die Anzahl der Kanten, die auf dem geraden Achsenkreis ben�tigt werden, um $P$ zu einem einfachen nullhomologen Kreis zu erweitern.
\end{definition}

%Bild?
Was bringt uns das nun? Wir werden sehen, dass sich Kreise mit schmalen Beulen sich durch gerade und Face-Kreise erzeugen lassen. 
%[schon oben definiert]Eine Ausst�lpung des Kreises ist dabei ein Teil des Kreises, der von der Hauptrichtung zu einer Seite des Kreises abzweigt und auf der gleichen Seite wieder in den Hauptkreis einm\"undet. 

Betrachte als Hauptkreis einen geraden Kreis mit einer Beule der Breite 1, der gerichtet ist, also theoretisch im ganzen flipbar w\"are. Im folgenden Bild sieht man sehr gut, wie leicht sich diese aus Facekreisen und dem Hauptkreis zusammensetzen l�sst. 
\begin{figure}[h!]
  \centering
  \scalebox{1}{\input{Bilder/beule1.pstex_t}}
\caption{Beispiel wie eine Ausst�lpung durch eine Kombination mit nullhomologen Kreisen geflippt wird}
\end{figure}

Gleiches funktioniert nicht nur in diesem Beispiel, sondern f�r Beulen der Breite 1 bei beliebiger L�nge: die Beule an sich ist ja ein nullhomologer Kreis, der bis auf die mit dem Hauptkreis geteilte Kante gerichtet sein muss. Diese Kante kann wenn sie entgegen den anderen gerichtet ist durch das flippen des Hauptkreises umgedreht werden, andernfalls ist sie f�r den Hauptkreis ungerichtet und wird durch den Flip des nullhomologen Kreises gedreht. Dabei muss man nat�rlich die Reihenfolge beachten, das ganze ist in Abbildung \ref{beule2}
schematisch dargestellt.\\

\begin{figure}[h!]

  \centering
  \scalebox{1}{\input{Bilder/beule2.pstex_t}}
\caption{Unterschiedliche Reihenfolge des der Flips, abh�ngig von der Richtung der (roten) Zwischenkante}
\label{beule2}
\end{figure}

Dieses Vorgehen wird schwieriger, wenn die Breite der Beulen zunimmt. Die hier rot gemalte Zwischenkante wird zu einem Pfad von mehreren Zwischenkanten. Diese sind dann aber m�glicherweise nicht gerichtet!  Abbildung \ref{2erbeule}
verdeutlicht das Problem:\\

\begin{figure}[h!]
  \centering
  \scalebox{1}{\input{Bilder/2erbeule.pstex_t}}
\caption{Bei dieser $2\times 1$ Beule ist es nicht mehr direkt m�glich, entlang des schwarz eingezeichneten Kreises einen Flip aus dem geraden Kreis und einem nullhomologen Kreis zusammenzusetzen. Im Domino Tiling (rechte Darstellung) erkennt man, dass es nicht mal klar ist ob �berhaupt Face-Flips m�glich sind.}
\label{2erbeule}
\end{figure}


Wie es scheint k�nnen wir bei dieser Form von Beulen zumindest mit der obigen Methode nichts erreichen. Die gute Nachricht ist aber: Wir sind nicht bei allen gr��eren Ausst�lpungen verloren. H�ufig ist es m�glich, zumindest einen Teil des Kreises �ber Faceflips zu drehen. % und das was �brigbleibt ist dann eben eine solche $2\times 1$ Beule.? ->NEIN
Betrachte die Situation in einer Ecke (Abb \ref{abkuerzung}) 


\begin{figure}[h!]
  \centering
  \scalebox{1}{\input{Bilder/beule3.pstex_t}}
\caption{Eine der beiden gr�nen Kanten muss eine eingehende Kante in den gemeinsamen Knoten sein. Alle nichtgr�nen Kanten folgen zwangsweise aus der Form des Pfades. Je nachdem wie man die Kantenrichtung w�hlt, erh�lt man entweder eine Abk�rzung des gro�en Kreises, oder man erh�lt einen kleinen nullhomologen Kreis aus zwei Faces den man flippen kann. In beiden F�llen reduziert sich die Gr��e des ben�tigten Kreises.}
\label{abkuerzung}
\end{figure}

Analog dazu ist das Bild, wenn der Eckknoten 3 eingehende und eine ausgehende Kante hat, nur dass die Wege dann jeweils in die andere Richtung laufen. Es gen�gt also, einen kleineren Kreis zuzulassen um den gr��eren zu erzeugen. Dieses Argument kann man nun induktiv immer weiter fortf�hren, da in der Regel auch der entsprechende kleinere Kreis eine solche Ecke besitzt. 
Wie weit kann man die Gr��e der Beulen so einschr�nken? Leider kann das Domino Tiling durchaus so aussehen, dass ein gro�er Teil der Beule �brig bleibt. Wir k�nnen also damit nur Ecken ausschlie�en, die aus "' hoch, hoch, rechts, rechts"' oder entsprechend gedrehten/gespiegelten Kanten bestehen.\\ Betrachte dazu Abbildung \ref{problembeule}

\begin{figure}[h!]
  \centering
  \scalebox{1}{\input{Bilder/beule4.pstex_t}}
\caption{Beispiel einer Beule, in der man keinen Face-Flip mehr anwenden kann um die Gr��e zu reduzieren}
\label{problembeule}
\end{figure}

sie zeigt ein Beispiel an dem keine solchen Ecken mehr vorhanden sind. Mit dem geraden Kreis als Grundkreis sind wir an dieser Stelle also auch am Ende, da wir im Gegensatz zu den Beulen der Breite 1 diese breiteren nicht durch Facekreise zusammensetzen k�nnen. Bei Breite 2 kommt es bereits auf die L�nge an: ist die L�nge gerade, dann lassen sie sich ebenso wie Beulen der Breite 1 durch Facekreise und den Grundkreis zusammensetzen, ist die L�nge ungerade funktioniert das allerdings auch schon nicht mehr.


%%% FAZIT
\section{Fazit}

Wie wir gesehen haben, lassen sich die positiven Resultate aus dem planaren Fall hier nicht direkt auf den Fall der torischen Domino Tilings erweitern. Insbesondere steigt die Zahl der ben�tigten Kreisklassen von einer (Face-Kreise) auf m�glicherweise beliebig viele abh�ngig von der Gittergr��e. Dadurch ist insbesondere nicht klar, wie man m�glichst g�nstig Flipzusammenhang konstruieren kann. Gleichzeitig konnte ich nicht zeigen, dass eine gro�e Menge ung�nstiger Kreisklassen auch implizieren, dass es keine kleinen Mengen gibt die ausreichend sind. Wir haben im letzten Kapitel gesehen, dass man durchaus Kreise aus anderen zusammensetzen kann, zumindest f�r den Fall von Faces und einem Grundkreis der nicht nullhomolog ist. 

Eine interessante Fragestellung w�re nun zum Beispiel, ob durch das Zusammenspiel mehrerer nicht nullhomologer Kreise mit Facekreisen deutlich mehr zu erreichen ist. Auf allgemeinen $\alpha$-Orientierungen stimmt es leider auch nicht, dass alle nullhomologen Kreise sich immer durch Faces zusammensetzen lassen, es w�re interessant zu sehen ob das auch f�r die Orientierungen gilt, die durch Domino Tilings induziert werden. %TODO geht da nicht mehr???
Ebenfalls konnte ich f�r aus Domino Tilings entstandene Graphen nicht zeigen, ob die Bedingung aus Lemma \ref{notwBed} auch hinreichend ist. 

\subsection*{Dank}

An Kolja Knauer, der sich pers�nlich wie per Mail die Zeit genommen hat um mich zu betreuen, sich immer wieder meine Ideen angeh�rt und dazu wertvolle Ideen entwickelt und mir Hinweise gegeben hat, insbesondere Lemma \ref{notwBed} betreffend - Danke!

%TODO hier weitergehen und genau sagen, wie weit die Gr��e sich reduziert!!!



%TODO anzahl der toricshen Domino Tilings???
\newpage
\newpage
\addcontentsline{toc}{section}{Literaturverzeichnis}
\bibliographystyle{amsplain}
\bibliography{literatur}
%\nocite{*}

\end{document}
