Der Gegenstand dieser Bachelorarbeit sind Domino Tilings. Pflasterungen/Tilings von Fl�chen mit regelm��igen Polygonen kann man in der Mathematik aus unterschiedlichen Blickpunkten betrachten. Domino Tilings sind ein spezieller Fall von regul�ren Tilings - ein regelm��iger Gittergraph (dreieckig, quadratisch oder hexagonal)  soll durch eine Menge von entsprechenden Polyominos �berdeckt werden, in unserem Fall geht es  also um die �berdeckung eines quadratisches Gitters mit Dominos. 

Man kann Tilings aus unterschiedlich Blickwinkeln betrachten. So haben Kasteleyn \cite{Kasteleyn19611209} sowie Temperley und Fisher \cite{TempFish} die Anzahl m�glicher Domino Tilings (bzw perfekter Matchings auf dem Gittergraphen) durch die Forschung �ber Dimere (Zweiatomige Molek�le) motiviert berechnet, w�hrend sich z.B. Conway und Lagarias dem kombinatorischen Problem der Existenz von Tilings in \cite{ConwayLagar} mittels gruppentheoretischen Mitteln n�hern. Bestimmte Varianten von Tiling-Problemen sind sogar als NP-vollst�ndig oder gar unberechenbar (Wang-Tilings).\\

Ein bekanntes Resultat �ber Domino Tilings einer einfach zusammenh�ngenden Region in der Ebene ist, dass diese die algebraische Struktur eines distributiven Verbands tragen (siehe \cite{Remila2004409}). Dabei sind die �berg�nge lokale Transformationen (Flips) bei denen zwei benachbarte Dominos gedreht werden. Es wurden auch bereits Verallgemeinerungen davon untersucht, so z.B. die Struktur von  Domino Tilings in der Ebene, die nicht beschr�nkt sind, siehe \cite{fernique}. Insbesondere tritt dort auch das Problem auf, dass die lokalen Flips aus dem einfachen planaren Fall nicht mehr ausreichen um jede Orientierung zu erzeugen. 

Es gibt f�r Domino Tilings verschiedene Verallgemeinerungen, so werde ich die Beziehung zu Matchings und zu $\alpha$-Orientierungen betrachten, f�r die man im planaren Fall ebenfalls einen distributiven Verband erhalten kann. 
Betrachtet man Domino Tilings auf dem Torus, stellt man fest dass die lokalen Flips hier leider nicht mehr ausreichen, um �berhaupt jede Orientierung zu erzeugen. In einem �hnlichen Spezialfall torischer $\alpha$-Orientierungen (in dem Fall dass jeder Knoten im Gitter Ausgrad 2 hat) hat Kolja Knauer in \cite{Knauer07partialorders} gezeigt, dass dort schon eine Basis des Kreisraumes gen�gt. Das motiviert die Frage, ob man in �hnlicher Weise eine m�glichst kleine Menge von Transformationen auf torischen Domino Tilings angeben kann, durch die man jede Orientierung erh�lt. Ich zeige einige Beispiele auf, die zeigen dass bestimmte Klassen von Kreisen dazu ungeeignet/nicht ausreichend sind, und gebe ein Kriterium an mit dem man dies nachweisen kann. 