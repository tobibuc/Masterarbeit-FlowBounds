In this chapter we will discuss the computational complexity of the \textit{Acyclic Flowbound Problem}. We show that 
Cycle-free Min-Cost-Flow with negative weights is $\mathcal{NP}$-complete, and that our problem is just a special case 
of this problem. \\

$\mathcal{NP}$-completeness is maybe the most important concept in the classification of combinatorial optimization 
problems (we will define it in a second). The $\mathcal{P-NP}$-Problem is an open question for many years now. Many 
computer scientists conjecture that in fact $\mathcal{P}\neq \mathcal{NP}$ and the $\mathcal{NP}$-complete algorithmical 
problems are hard to solve. 

The classical formal definition of the class $\mathcal{NP}$ in complexity theory uses formal languages and automata 
theory (namely Turing Machines). For this formal definition we refer to the textbook of Korte and Vygen 
\cite{KorteVygenCombOpt2007}. If we accept to be a bit less formal we can define it the following way:
\begin{definition}
%  A decision problem is an algorithmic problem that has yes or no as solution.
%  A polynomial algorithm is a procedure that 

% A combinatorial decision problem is in the complexity class $\mathcal{NP}$ if there is any nondeterministic algorithm 
% which computes the output "yes" in a polynomial number of steps.

A combinatorial decision problem (a problem with the possible solutions "yes" and "no") is in the complexity class 
$\mathcal{NP}$ iff there exists an algorithm that, given the problem and an additional input (the certificate), 
computes "yes" in a polynomial bounded number of steps if and only if the instance is a "yes"-instance of the problem 
\textit{and} the right certificate is given, otherwise false.\\
The set of problems where there exists an algorithm that always finds the correct answer in a polynomial (in the input 
size) bounded number of steps is called the complexity class $\mathcal{P}$. 
\end{definition}

Obviously all problems for which polynomial algorithms are known are in $\mathcal{NP}$, so 
$\mathcal{P}\subseteq\mathcal{NP}$. We do not know if polynomial algorithms (without the "certificate") exist for all 
problems in $\mathcal{NP}$. There are problems in $\mathcal{NP}$ for which no subexponential algorithm is known so far, 
for example the SAT problem. For some of them one can show that a polynomial algorithm for this problem already would 
imply a polynomial algorithm for all others. 

\begin{definition}
 A problem \textit{P} is called $\mathcal{NP}$-complete if \textit{P}$\in\mathcal{NP}$ and for every problem 
\textit{P'} $\in \mathcal{NP}$ there exists an algorithm with polynomial bounded running time if it can call an 
oracle for \textit{P}, i.e. a blackbox algorithm solving \textit{P} in a single step.
\end{definition}

Stephen Cook proved that every problem in $\mathcal{NP}$ can be solved by a polynomial algorithm if the SAT (boolean 
satisfiability) problem can be solved polynomially \cite{Cook:1971:CTP:800157.805047}. 
Richard Karp presented the result that the hamiltonian cycle problem (and by this also the hamiltonian path problem) 
can be reduced to SAT (\cite{Karp1972}). Thus these problems are $\mathcal{NP}$-complete.

We want to show that the acyclic minimum cost flow problem, of which the acyclic flowbound problem is a special case, 
is an $\mathcal{NP}$-complete problem. This justifies to deal with algorithms that either are not provable to have 
polynomial worst case running time or else are no exact algorithms but rather heuristics to find a weaker bound.


%TODO kann man für das Unterproblem selbst auch etwas zeigen? Das 
%könnte ja durchaus auch Polynomiell sein!

\begin{definition}
 We call the problem of finding a Minimum Cost Flow as in Definition \ref{def:mincostflow} in a given 
network $G$ with the additional constraint  that there is no cyclic flow the \textit{Acyclic Minimum Cost Flow} 
Problem. 

The corresponding decision problem is to decide the question if there can be any acyclic flow with total cost of at 
most $x$ in the network. With binary search on the values of $x$ we can solve any optimization problem by iteratively 
solving the corresponding decision problem - so since both have the same complexity, we can speak of $\mathcal{NP}$ 
completeness of the optimization problem as well . 
\end{definition}

\begin{theorem}
 The \textit{Acyclic Minimum Cost Flow Problem} with arbitrary arc weights is $\mathcal{NP}$ complete.
\end{theorem}
\begin{proof}
 We have to show two things: The problem is in the complexity class $\mathcal{NP}$ , and the problem is indeed 
$\mathcal{NP}$ hard. We do this via two lemmas:
\begin{lemma}
 The Acyclic Minimum Cost Flow Problem is in the complexity class $\mathcal{NP}$
\end{lemma}
\begin{proof}
It is easy to see that the problem is in $\mathcal{NP}$. Given a valid solution to the decision problem we can check if 
it is a feasible network flow by just checking the flow balances on all nodes, which needs only linear time 
$\mathcal{O}(|V|+|A|)$. We can also easily check if the total cost is smaller or equal than demanded and that all 
capacities are respected by running over all arcs and checking them, thus this part is also linear.
We can also check the acyclicity in polynomial time: On each of the arcs we can start a graph search, considering arcs 
only in their flow direction. If this graph search comes back to the original arc again, there is a cycle. %TODOprove?
If graph search does not find a way to the tail of the original arc, there is no cyclic flow on this arc. The graph 
search is also a linear algorithm. If we do this for each arc, the running time is still quadratic.
\end{proof}

\begin{lemma}
 The Acyclic Minimum Cost Flow Problem is $\mathcal{NP}$-hard.
\end{lemma}
\begin{proof}

To show the $\mathcal{NP}$ hardness, we reduce the problem to the \textit{undirected $s-t$-Hamiltonian Path Problem} 
which is well known as a standard example for $\mathcal{NP}$ completeness. Hamiltonian cycle is $\mathcal{NP}$ 
complete both in the directed and undirected case, see \cite{Karp1972}. To find an $s-t$-hamiltonian path in a graph 
$G$ for some vertices $s,t\in V(G)$ we add one extra vertex $x$ and the edges $(t,x)$ and $(x,u)$ to $G$. This new 
graph has a hamiltonian cycle if and only if $G$ has a hamiltonian path from $s$ to $t$. \\
%done zitieren wo hampath vorkommt,NP-haertebeweis

Given an instance of the $s-t$-Hamiltonian Path problem, we have to transform it into an instance of our problem. Then 
we show that the reduced problem is a yes-instance of the Acyclic Min Cost Flow decision problem if and only if the 
original instance of the Hamiltonian Path problem is a yes-instance. Further the reduction must be computable in 
polynomial time.
 
\begin{figure}[h!]
\centering
\begin{tikzpicture}
\node (s) at (0,0)[vertex, label=left:s]{};
\node (t) at (0,1)[vertex, label=left:t]{} ;
\node (3) at (1,0)[vertex]{};
\node (4) at (1,1.7)[vertex]{};
\node (5) at (1.7,1)[vertex]{};

% \draw[edgeflow] (s)--(3)--(4)--(5)--(t);
\draw[edge] (s)--(3)--(4)--(5)--(t);
\draw[edge] (t)--(s)--(5)--(3);

%now draw the transformed graph
\node (s2) at (3,0)[vertex, label=left:s]{};
\node (t2) at (3,1)[vertex, label=left:t]{} ;
\node (32) at (4,0)[vertex]{};
\node (42) at (4,1.7)[vertex]{};
\node (52) at (4.7,1)[vertex]{};

\draw[edgeflow] (s2)--(32)--(42)--(52)--(t2);
\draw[edge] (s2)--(32)--(42)--(52)--(t2);
\draw[edge] (t2)--(s2)--(52)--(32);
\end{tikzpicture}
\caption{This graph has an $s-t$-hamilton path contained, marked in blue in the second picture}
\label{bild:hampath}
\end{figure}

An $s-t$-Hamiltonian Path is a path in a network starting at node $s$ and ending at node $t$ that is visiting every 
node of $G$ exactly once like in the example in figure \ref{bild:hampath}. 

This can only work in connected graphs, so we know that if $n$ is the number of nodes in the 
graph, the Hamiltonian Path has to have size $n-1$. Our reduction is the following:

Transform the $s-t$-Hamiltonian Path instance into an instance of a flow network where for each edge $\{uv\}$ there 
are two directed arcs $(u,v),(v,u)$, the arcs $a\in A$ all have capacities $c(a)=[0,1]$ and costs of $-1$ 
with the same underlying graph as before. The nodes $s$ and $t$ become the source and sink of the flow.
The decision we ask for is whether there is a flow with total costs of at most $-(n-1)$ or not. (Figure 
\ref{bild:hampathreduct} shows an example of such a transformation)

 
\begin{figure}[h!]
\centering
\begin{tikzpicture}
\node (s) at (0,0)[vertex, label=left:s]{};
\node (t) at (0,1)[vertex, label=left:t]{} ;
\node (3) at (1,0)[vertex]{};
\node (4) at (1,1.7)[vertex]{};
\node (5) at (1.7,1)[vertex]{};

\draw[edgeflow] (s)--(3)--(4)--(5)--(t);
\draw[edge] (s)--(3)--(4)--(5)--(t);
\draw[edge] (t)--(s)--(5)--(3);

%now draw the transformed graph
\node (s2) at (3,0)[vertex, label=left:s]{};
\node (t2) at (3,1)[vertex, label=left:t]{} ;
\node (32) at (4,0)[vertex]{};
\node (42) at (4,1.7)[vertex]{};
\node (52) at (4.7,1)[vertex]{};

\draw[curvedarcflow](s2)to(32);
\draw[curvedarcflow](32)to(42);
\draw[curvedarcflow](42)to(52);
\draw[curvedarcflow](52)to(t2);
\draw[curvedarc](s2)to(32);
\draw[curvedarc](32)to(s2);
\draw[curvedarc](42)to(32);
\draw[curvedarc](32)to(42);
\draw[curvedarc](42)to(52);
\draw[curvedarc](52)to(42);
\draw[curvedarc](t2)to(52);
\draw[curvedarc](52)to(t2);
\draw[curvedarc](s2)to(32);
\draw[curvedarc](s2)to(52);
\draw[curvedarc](52)to(s2);
\draw[curvedarc](52)to(32);
\draw[curvedarc](32)to(52);
\draw[curvedarc](t2)to(s2);
\draw[curvedarc](s2)to(t2);

\end{tikzpicture}
\caption{A graph with an $s-t$-hamilton path is transformed to an acyclic minimum cost flow instance with costs of $-1$ 
per arc. Hence in this example we find a minimum acyclic flow of cost $-4=-n+1$}
\label{bild:hampathreduct}
\end{figure}

It is obviously a polynomial time transformation because only a linear number of flags and arcs is added to the 
network, while the edges are deleted when they get replaced with the new arcs.

It yields the same decision as the original Hamiltonian Path instance:
\begin{itemize}
 \item $\Rightarrow:$ Given an instance of the Acyclic Min Cost Flow which has a acyclic flow from $s$ to $t$ with cost 
$-n+1$.
Flow can have a value of at most $1$ on each arc, so each arc the flow is going through can only contribute up to $-1$ 
costs to the cost function. The flow could be split up into different paths from $s$ to $t$, but we know that there are 
no cycles. That means on every of these paths every node is visited only once - if we see it twice we have closed a 
cycle. The cost of the flow is minus the average length of the paths of flow. So we can conclude there is at least one 
path with length $n-1$ that is seeing no node twice - hence it has to see every node exactly once, so we found (at 
least) one $s-t$-Hamiltonian Path and know it is a yes-instance of Hamiltonian Path too.

\item $\Leftarrow:$ Given a graph together with an $s-t$-Hamiltonian Path. In the transformed instance of Min Cost Flow 
we get immediately a feasible $s-t$-Flow by setting the flow value on all forward arcs on the given hamiltonian path 
equals $1$. This flow also has total costs of $-(n-1)$.
\end{itemize}
\end{proof}

These two lemmas together show that our problem of finding an acyclic flow with minimum costs and arbitrary weight 
function is in general $\mathcal{NP}$-complete.
\end{proof}
