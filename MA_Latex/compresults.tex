% The models we described have also been implemented and tested on networks of different size.
In this section we will discuss practical implementation and results of the computation. For relevant real-world 
problem sizes the exact algorithm (even with separation of the acyclicity constraints) is too slow. Hence we also 
describe how to relax the problem in order to get useable (though in general not optimal) results for the flow bounds. 
We compare the bounds found by different approaches/settings in different networks and nominations to see which 
approach is suited best to actually compute acyclic flow bounds in a network. 

At last we compare the impact of bounds and the number of fixed variables fed into the solver on the model sizes of the 
discretization. \\

\subsection{Implemented and Tested Algorithms}

We described the theoretical backgrounds of different algorithms to obtain flow bounds for our problem. The 
computational results of different implementations of them may vary widely, and the performance is influenced by lots 
of details regarding the machines, programming languages, data structures or just code quality. Nevertheless we want 
to give an overview what worked well in the implementation and what did not.\\

All algorithms where tested on a PC with instances of the \textit{"gaslib"} gasnet test library. The library can be 
downloaded on \url{http://gaslib.zib.de/}. A detailed description of this problem library can be found in 
\cite{HumpolaJoormannOucherifPfetschScheweSchmidtSchwarz:2015}.

\subsubsection{MIP with Node Potentials and Binary Direction Variables}
The MIP model described in \ref{model:nodePotential} was implemented in Lamatto, using Gurobi as solver for this 
model. However,the two layers of indicator constraints seem to be a heavy problem of the model in 
practice. The folowing model was implemented:
\begin{align*}
  &\min \sum_{a\in A} w_a\cdot q_a & \\
 s.t. & \sum_{a\in \delta^+(v)}q_a - \sum_{a\in\delta^- (v)}q_a &=& d_v\ &\forall v\in V \\
 & q_a &\le& c_u(a) & \forall a\in A\\
 & q_a &\ge& c_l(a) & \forall a\in A\\
 & \rho_{vw}=1 &\Rightarrow &\pi_v\ge\pi_w +1 & \forall (v,w)\in A\\
 & \rho_{vw}=0 &\Rightarrow &\pi_w\ge\pi_v +1& \forall (v,w)\in A\\
 & \rho_{vw}=1 &\Rightarrow &q((v,w))\ge 0& \forall (v,w)\in A\\
 & \rho_{vw}=0 &\Rightarrow &q((v,w))\le 0& \forall (v,w)\in A\\
 & q_a \in \R & & &\forall a\in A\\
 & \pi_v \in \R & & & \forall v\in V\\
 & d_a \in \{0,1\} & & &\forall a\in A\\
 & \rho_{vw} \in \{0,1\}&&& \forall a=(v,w)\in A\\
\end{align*}
Integer node potentials imply the direction of the arc between the incident nodes. These binary direction 
variables again imply the sign of the actual flow on an arc. Indicator constraints are usually implemented via 
Big-M-constraints of the form 
\begin{align*}
 &\pi_v-\pi_w &\ge & N(\rho_{vw}-1)+1\\
 &\pi_v-\pi_w&\le &N\rho_{vw}-1 \\
 &q((v,w))&\ge& M(\rho_{vw}-1)\\
 &q((v,w))&\le & M\rho_{vw}\\
 &\rho_{vw} \in \{0,1\}&&\\
\end{align*}
with sufficient big values for $M$ and $N$. This model heavily relies on the Big-M-constraints.
For any MIP solver these indicator or big-M constraints are difficult to handle, because the optimal vertex of the LP 
polyhedron can be quite far from the optimal mixed integer solution.\\

The model was tested on the smallest network named "gaslib-40" of the gaslib package. After running for 
%TODO testergebnisse (vermutlich fail nach etlichen tagen)

\subsubsection{MIP with Binary Direction Variables and separated Acyclicity Constraints}
In section \ref{model:AcyclicityConstraints} we described a model using acyclicity constraints on the binary directions 
for cycle that was found during the solving process.

