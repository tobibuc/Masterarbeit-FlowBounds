\documentclass[a4paper]{article}

\usepackage[T1]{fontenc}
\usepackage[utf8]{inputenc}
\usepackage[ngerman]{babel}
\usepackage{times, graphicx, currvita, hyperref, longtable, float, caption, subcaption}
\usepackage{amsmath,amsfonts,amssymb, amsthm}%, dsfont, bbm, mathtools, stmaryrd}
%\usepackage{fancyhdr}
%\usepackage{color}
%\usepackage[english]{babel}
\usepackage{paralist}
%\usepackage{algorithmic}
\usepackage{wasysym}	% verschiedene Symbole, siehe http://rpi.edu/dept/arc/training/latex/LaTeX_symbols.pdf
\graphicspath{{Bilder/}}

\theoremstyle{definition}
\newtheorem{definition}{Definition}[section]
\newtheorem*{Def1}{Definition}
\newtheorem*{ex}{Example}

\newtheoremstyle{Tobi}{10pt}{}{}{}{\bf}{ }{\newline}{}
% die parameter sind: {name}{platz drueber}{platz drunter}{schriftart text}{einrueckung}{schriftart kopf}{Punktierung des Kopfes}{Platz zwischen kopf und text}

\theoremstyle{Tobi}
\newtheorem{prop}[definition]{Proposition}
\newtheorem*{Pro}{Proposition}
\newtheorem{theorem}[definition]{Satz}
%\newtheorem*{le}{Lemma}
\newtheorem{lemma}[definition]{Lemma}
\newtheorem{cor}[definition]{Korollar}
\newtheorem*{conj}{Conjecture}
\newtheorem{obs}[definition]{Beobachtung}

% \theoremstyle{remark}
% \newtheorem*{que}{Questions}
% \newtheorem*{claim}{Claim}
% \newtheorem*{note}{Note:}
% \newtheorem*{remark}{Remark}


\newcommand{\R}{\mathbb{R}}
\newcommand{\N}{\mathbb{N}}
% \newcommand{\F}{\mathbb{F}}
% \newcommand{\1}{\mathbbm{1}}
\newcommand{\Rn}{\mathbb{R}^n}
% \newcommand{\La}{\mathcal{L}}
% \newcommand{\D}{\mathcal{D}}
\newcommand{\Om}{\Omega}
\newcommand{\pa}{\partial}
\newcommand{\C}{\mathcal{C}}
\newcommand{\ph}{\varphi}

\setlength{\parindent}{0pt}

\begin{document}

\title{Entwurf MA - Flussschranken in Netzen ohne Kreisfluss}

\author{Tobias Buchwald}
%\maketitle

%\large{Bachelorarbeit bei Prof. Dr. Stefan Felsner}


% \begin{center} 
%TODO auf Masterarbeit anpassen
% \Huge{Domino Tilings auf dem Torus}\\ \vspace{12 cm}
% \Large{Bachelorarbeit\\ bei Prof. Dr. Stefan Felsner}\\ \vspace{1cm}
% \large{Vorgelegt von Tobias Buchwald}\\
% \large{am Fachbereich Mathematik der \\Technischen Universit�t Berlin}\\
% \vspace{2cm}
% \large{Berlin,  \today}
% 
% \end{center} 
% 
% 
% \textbf{Erkl\"arung}\\

Hiermit versichere ich an Eides statt, dass ich die vorliegende Masterarbeit selbst\"andig und eigenh\"andig sowie
ausschlie\ss lich unter Verwendung der aufgef\"uhrten Quellen und Hilfsmittel angefertigt habe. \\


Berlin, den \today
\newline

\rule[-0.2cm]{10cm}{0.5pt}

\textsl{Tobias Buchwald} 
% \newpage

\tableofcontents
\newpage
\section{Introduction} \subsection{Motivation and Outline}
Today more and more real-world problems in the areas of simulation and optimization are solved by mathematical and 
computational methods. A growing number of these problems can be solved without problems, i.e. even huge instances give 
an optimal or near optimal solution within seconds. Still, there remain problems that even on modern computers are hard 
to solve. For these problems it is important to find ways to increase the efficiency of the algorithms. 

The topic of this thesis arises from the computation of flow in natural gas networks, which is currently developed 
in the FORNE Project in a cooperation of OGE with universities and research insitutes including ZIB.
%TODO genaueres zu FORNE? genaueres zum aufbau des Gasnetzes?
The flow of natural gas in a network is described by nonlinear equations and depends on many parameters, which makes 
the problem hard to solve. If we can find good upper and lower bounds for the flow on an arc during the preprocessing, 
we can hope to improve the behavior of the nonlinear solver by giving these tigther bounds. 

The flow is induced by pressure differences, so in reality there can't be cyclic flow (if we exclude compressor 
stations). Without the condition of acyclic flow, it is sufficient to run a standard min-cost-flow algorithm where the 
maximized arc $e$ gets weight $w_e = -1$ and all others are 0. However, the arising bounds are far from optimal. If arc 
$e$ is contained in any cycle we could decrease the cost by pushing more and more flow around this cycle until the arcs 
capacity is at its limits.

This master thesis will deal with the problem of finding a network flow with no directed cycles (acyclic flow), which at 
the same time maximizes the amount of flow on a specified arc $e$ of the network. We will discuss the complexity, an 
exact algorithm based on a mixed integer program with separation of inequalities that forbid cycles and also a heuristic 
approach that yields results much faster (but not optimal).%TODO am ende genauer schreiben was wirklich gemacht wurde

% \subsection{The gas flow problem}
% Although it is mainly the motivation, not really the topic of this thesis, we want to briefly introduce the gas 
% transport problem. For more a detailed description we refer to LINK 
% %TODO link auf ein entsprechendes ZIB-paper? Nee, Jesco meinte soll nicht so rein 
% 

\newpage
\section{Basic Notation and Definitions}% Als Referenz und Grundlage fuer die Definitionen nutze ich das Lehrbuch Combinatorial Optimization:Theory and 
%Algorithms von Bernhard Korte u Jens Vygen - definiere die wichtigsten Sachen aber erstmal auch selbst

Since there are many definitions, which may differ slightly, we want to introduce now the basic notation and 
definitions that we use throughout this thesis. The definitions in this chapter are mainly taken from the textbook 
about combinatorial optimization from Korte and Vygen \cite{KorteVygenCombOpt2007}.

An undirected graph is a triple (V, E, $\Psi$), where $V$ and $E$ are finite sets and
$\Psi: E\to \{X \subseteq V: |X| = 2\}$. 
A directed graph or digraph is a triple $(V, E, \Psi)$,
where $V$ and $E$ are finite sets and $\Psi : E \to \{(v, w) \in V \times V : v \neq w\}$. In this thesis by a
graph we mean normally the directed graph. If we talk about undirected graphs it will be stated 
explicitly. The elements of $V$ are called vertices, the elements of $E$ are the edges. Edges of undirected graphs can 
also be called arcs to make clear that they are directed.

Two edges $e, e'$ with $\Psi(e) = \Psi ( e')$ are called parallel. Graphs without parallel
edges are called simple. For simple graphs we usually identify an edge $e$ with its
image $\psi(e)$ and write $G = (V(G), E(G))$, where $E(G) \subseteq \{X \subseteq V(G) : |X| = 2\}$
or $E(G) \subseteq V(G) \times V(G)$. We often use this simpler notation even in the presence
of parallel edges, then the ``set'' $E (G)$ may contain several ``identical'' elements. In this thesis all graphs 
are considered simple if nothing different is said. %TODO rausnehmen falls gar keine parallelen Kanten gebraucht werden
$|E(G)|$ denotes the number of edges, and for two edge sets $E$ and $F$ we always
have $|E \cup F | = |E | + |F |$ even if parallel edges arise.

We say that an edge $e = \{v, w\}$ or $e = (v, w)$ joins $v$ and $w$. In this case, $v$ and $w$ are adjacent. $v$ is a 
neighbour of $w$ (and vice versa). $v$ and $w$ are the endpoints of $e$. If $v$ is an endpoint of an edge $e$, we say 
that $v$ is incident with $e$. 
In the directed case we say that $( v, w)$ leaves $v$ and enters $w$, $v$ is the tail and $w$ is the head of the arc 
$e$. Two edges which share at least one endpoint are called adjacent.

For a digraph $G$ we sometimes consider the underlying undirected graph, i.e. the undirected graph $G'$ on the same 
vertex set which contains an edge $\{v, w\}$
for each edge $(v, w)$ of $G$. We also say that $G$ is an orientation of $G'$.
A subgraph of a graph $G = (V(G), E(G))$ is a graph $H = (V(H), E(H))$
with $V(H) \subset V(G)$ and $E(H) \subset E(G)$. We also say that $G$ contains $H$. $H$ is an
induced subgraph of $G$ if it is a subgraph of $G$ and $E (H) = \{ \{x, y\} \textrm{ resp. } (x, y) \in
E(G) : x, y \in V(H)\}$. Here $H$ is the subgraph of $G$ induced by $V(H)$. We also
write $H = G[V(H)]$. A subgraph $H$ of $G$ is called spanning if $V(H) = V(G)$.
If $v \in V(G)$, we write $G- v$ for the subgraph of $G$ induced by $V(G) \setminus {v}$.
If $e \in E(G)$, we define $G- e := (V(G), E(G) \setminus \{e\})$. Furthermore, the addition
of a new edge $e$ is abbreviated by $G + e := (V(G), E(G) \cup {e})$. If $G$ and $H$
are two graphs, we denote by $G + H$ the graph with $V(G +H)= V(G) \cup V(H)$
and $E(G +H)$ being the disjoint union of $E(G)$ and $E(H)$ (parallel edges may arise).

For a graph $G$ and $X, Y\subseteq V(G)$ we define $E(X, Y) := \{\{x, y\} \in E(G) : x \in
X \setminus Y, y \in Y \setminus X\}$ resp. $E^+(X, Y) := \{(x, y) \in E(G) : x \in X\setminus Y, y \in Y \setminus 
X\}$.
For undirected graphs $G$ and $X \subseteq V(G)$ we define $\delta(X) := E(X, V(G) \setminus X)$. The
set of neighbours of $X$ is defined by $ \Gamma(X) := \{v \in V(G) \setminus X : E(X, \{v\})  \neq \emptyset\}$.
For digraphs $G$ and $X \subseteq V(G)$ we define $\delta^+(X) := E^+(x, V(G) \setminus X)$, $\delta^-(x) :=
\delta^+(V(G) \setminus X)$ and $\delta(X) := \delta^+(x) \cup \delta^-(x)$. We use subscripts (e.g. $\delta_G(X)$) to
specify the graph $G$ if necessary.

For singletons, i.e. one-element vertex sets $\{v\} (v \in V(G))$ we write $\delta(v) :=
\delta(\{v\})$, $\Gamma(v) := \Gamma(\{v\}), \delta^+(v) := \delta^+(\{v\})$ and $\delta^-(v) := \delta^-(\{v\})$. The 
degree of a vertex $v$ is $|\delta(v)|$, the number of edges incident to $v$. In the directed case, the
in-degree is $|\delta^-(v)|$, the out-degree is $|\delta^+(v)|$, and the degree is $|\delta^+(v)|+ |\delta^-(v)|$.
A vertex $v$ with zero degree is called isolated. A graph where all vertices have
degree $k$ is called $k$-regular.

An edge progression $W$ in $G$ is a sequence $v_1, e_1, v_2, \dots , v_k, e_k, v_{k+1}$ such that $k \ge 0$,
and $e_i = (v_i, v_{i+ 1}) \in E(G)$ resp. $e_i = \{v_i, v_{i+1}\}\in E(G)$ for $i = 1, \dots , k$. If in
addition $e_i \ne e_j \,\forall\, 1 \le i < j \le k$, $W$ is called a walk in $G$. $W$ is closed if
$v_1 = v_{k+1}$. A path is a graph $P = (\{v_1, ... , v_{k+1}\}, \{e_1, ... , e_k\})$ such that $v_i \ne v_j$ for
$1 \le i < j \le k + 1$ and the sequence $v_1 , e_1 , v_2, \dots , v_k, e_k, v_{k+1}$ is a walk. $P$ is
also called a path from $v_1$ to $v_{k+1}$ or a $v_1 - v_{k+1}$-path. $v_1$ and $v_{k+1}$ are the endpoints
of $P$. By $P_{[x,y]}$ with $x, y \in V(P)$ we mean the (unique) subgraph of $P$ which is
an $x-y$-path. Evidently, there is an edge progression from a vertex $v$ to another
vertex $w$ if and only if there is a $v-w$-path.

A cycle is a graph $(\{v_1, \dots , v_k\}, \{e_1, \dots, e_k\})$ such that the sequence $v_1, e_1, v_2, 
\dots , v_k,e_k,v_1$ is a (closed) walk and $v_i \ne v_j$ for $1 \le i < j\le k$.
An easy induction argument shows that the edge set of a closed walk can be
partitioned into edge sets of cycles.

The length of a path or cycle is the number of its edges. If it is a subgraph
of $G$, we speak of a path or cycle in $G$. A spanning path in $G$ is called a
Hamiltonian path while a spanning cycle in $G$ is called a Hamiltonian cycle
or a tour. A graph containing a Hamiltonian cycle is a Hamiltonian graph.
For two vertices $v$ and $w$ we write $dist(v, w)$ or $dist_G (v, w)$ for the length of
a shortest $v-w$-path (the distance from $v$ to $w$) in $G$. If there is no $v-w$-path at all,
i.e. $w$ is not reachable from $v$, we set $dist(v, w) := \inf$. In the undirected case,
$dist(v, w) = dist(w, v)$ for all $v, w \in V(G)$.

We shall often have a cost function $c : E(G) \to \R$. Then for $F \subseteq E(G)$ we
write $c(F) := \sum_{e\in F} c(e)$ (and $c(\emptyset) = 0$). This extends $c$ to a modular function
$c : 2^{E(G)}\to \R$. Moreover, $dist_{(G,c)}(v, w)$ denotes the minimum $c(E(P))$ over all
$v-w$-paths $P$ in $G$.

\newpage
\section{The Acyclic Flowbound Problem}
\newpage
\section{Complexity}
\newpage
\section{Implementation}
\newpage
\section{Practical Results}
\newpage
\section{Conclusion}
% \newpage
% \section*{Literature}

\newpage
\newpage
\addcontentsline{toc}{section}{Literaturverzeichnis}
\bibliographystyle{amsplain}
\bibliography{literatur}
\nocite{*}

\end{document}
