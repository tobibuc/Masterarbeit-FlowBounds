\documentclass[a4paper]{article}

\usepackage[T1]{fontenc}
\usepackage[utf8]{inputenc}
\usepackage[ngerman]{babel}
\usepackage{times, graphicx, currvita, hyperref, longtable, float, caption, subcaption}
\usepackage{amsmath,amsfonts,amssymb, amsthm}%, dsfont, bbm, mathtools, stmaryrd}
%\usepackage{fancyhdr}
%\usepackage{color}
%\usepackage[english]{babel}
\usepackage{paralist}
%\usepackage{algorithmic}
\usepackage{wasysym}	% verschiedene Symbole, siehe http://rpi.edu/dept/arc/training/latex/LaTeX_symbols.pdf
\graphicspath{{Bilder/}}

\theoremstyle{definition}
\newtheorem{definition}{Definition}[section]
\newtheorem*{Def1}{Definition}
\newtheorem*{ex}{Example}

\newtheoremstyle{Tobi}{10pt}{}{}{}{\bf}{ }{\newline}{}
% die parameter sind: {name}{platz drueber}{platz drunter}{schriftart text}{einrueckung}{schriftart kopf}{Punktierung des Kopfes}{Platz zwischen kopf und text}

\theoremstyle{Tobi}
\newtheorem{prop}[definition]{Proposition}
\newtheorem*{Pro}{Proposition}
\newtheorem{theorem}[definition]{Satz}
%\newtheorem*{le}{Lemma}
\newtheorem{lemma}[definition]{Lemma}
\newtheorem{cor}[definition]{Korollar}
\newtheorem*{conj}{Conjecture}
\newtheorem{obs}[definition]{Beobachtung}

% \theoremstyle{remark}
% \newtheorem*{que}{Questions}
% \newtheorem*{claim}{Claim}
% \newtheorem*{note}{Note:}
% \newtheorem*{remark}{Remark}


\newcommand{\R}{\mathbb{R}}
\newcommand{\N}{\mathbb{N}}
% \newcommand{\F}{\mathbb{F}}
% \newcommand{\1}{\mathbbm{1}}
\newcommand{\Rn}{\mathbb{R}^n}
% \newcommand{\La}{\mathcal{L}}
% \newcommand{\D}{\mathcal{D}}
\newcommand{\Om}{\Omega}
\newcommand{\pa}{\partial}
\newcommand{\C}{\mathcal{C}}
\newcommand{\ph}{\varphi}

\setlength{\parindent}{0pt}

\begin{document}

\title{Entwurf MA - Flussschranken in Netzen ohne Kreisfluss}

\author{Tobias Buchwald}
%\maketitle

%\large{Bachelorarbeit bei Prof. Dr. Stefan Felsner}


% \begin{center} 
%TODO auf Masterarbeit anpassen
% \Huge{Domino Tilings auf dem Torus}\\ \vspace{12 cm}
% \Large{Bachelorarbeit\\ bei Prof. Dr. Stefan Felsner}\\ \vspace{1cm}
% \large{Vorgelegt von Tobias Buchwald}\\
% \large{am Fachbereich Mathematik der \\Technischen Universit�t Berlin}\\
% \vspace{2cm}
% \large{Berlin,  \today}
% 
% \end{center} 
% 
% 
% \textbf{Erkl\"arung}\\

Hiermit versichere ich an Eides statt, dass ich die vorliegende Masterarbeit selbst\"andig und eigenh\"andig sowie
ausschlie\ss lich unter Verwendung der aufgef\"uhrten Quellen und Hilfsmittel angefertigt habe. \\


Berlin, den \today
\newline

\rule[-0.2cm]{10cm}{0.5pt}

\textsl{Tobias Buchwald} 
% \newpage

\tableofcontents
\newpage
\section{Introduction} \subsection{Motivation and Outline}
Today more and more real-world problems in the areas of simulation and optimization are solved by mathematical and 
computational methods. A growing number of these problems can be solved without problems, i.e. even huge instances give 
an optimal or near optimal solution within seconds. Still, there remain problems that even on modern computers are hard 
to solve. For these problems it is important to find ways to increase the efficiency of the algorithms. 

The topic of this thesis arises from the computation of flow in natural gas networks, which is currently developed 
in the FORNE Project in a cooperation of OGE with universities and research insitutes including ZIB.
%TODO genaueres zu FORNE? genaueres zum aufbau des Gasnetzes?
The flow of natural gas in a network is described by nonlinear equations and depends on many parameters, which makes 
the problem hard to solve. If we can find good upper and lower bounds for the flow on an arc during the preprocessing, 
we can hope to improve the behavior of the nonlinear solver by giving these tigther bounds. 

The flow is induced by pressure differences, so in reality there can't be cyclic flow (if we exclude compressor 
stations). Without the condition of acyclic flow, it is sufficient to run a standard min-cost-flow algorithm where the 
maximized arc $e$ gets weight $w_e = -1$ and all others are 0. However, the arising bounds are far from optimal. If arc 
$e$ is contained in any cycle we could decrease the cost by pushing more and more flow around this cycle until the arcs 
capacity is at its limits.

This master thesis will deal with the problem of finding a network flow with no directed cycles (acyclic flow), which at 
the same time maximizes the amount of flow on a specified arc $e$ of the network. We will discuss the complexity, an 
exact algorithm based on a mixed integer program with separation of inequalities that forbid cycles and also a heuristic 
approach that yields results much faster (but not optimal).%TODO am ende genauer schreiben was wirklich gemacht wurde

% \subsection{The gas flow problem}
% Although it is mainly the motivation, not really the topic of this thesis, we want to briefly introduce the gas 
% transport problem. For more a detailed description we refer to LINK 
% %TODO link auf ein entsprechendes ZIB-paper? Nee, Jesco meinte soll nicht so rein 
% 

\newpage
\section{Grundlagen und Definitionen}% Als Referenz und Grundlage fuer die Definitionen nutze ich das Lehrbuch Combinatorial Optimization:Theory and 
%Algorithms von Bernhard Korte u Jens Vygen - definiere die wichtigsten Sachen aber erstmal auch selbst

Since there are many definitions, which may differ slightly, we want to introduce now the basic notation and 
definitions that we use throughout this thesis. The definitions in this chapter are mainly taken from the textbook 
about combinatorial optimization from Korte and Vygen \cite{KorteVygenCombOpt2007}.

An undirected graph is a triple (V, E, $\Psi$), where $V$ and $E$ are finite sets and
$\Psi: E\to \{X \subseteq V: |X| = 2\}$. 
A directed graph or digraph is a triple $(V, E, \Psi)$,
where $V$ and $E$ are finite sets and $\Psi : E \to \{(v, w) \in V \times V : v \neq w\}$. In this thesis by a
graph we mean normally the directed graph. If we talk about undirected graphs it will be stated 
explicitly. The elements of $V$ are called vertices, the elements of $E$ are the edges. Edges of undirected graphs can 
also be called arcs to make clear that they are directed.

Two edges $e, e'$ with $\Psi(e) = \Psi ( e')$ are called parallel. Graphs without parallel
edges are called simple. For simple graphs we usually identify an edge $e$ with its
image $\psi(e)$ and write $G = (V(G), E(G))$, where $E(G) \subseteq \{X \subseteq V(G) : |X| = 2\}$
or $E(G) \subseteq V(G) \times V(G)$. We often use this simpler notation even in the presence
of parallel edges, then the ``set'' $E (G)$ may contain several ``identical'' elements. In this thesis all graphs 
are considered simple if nothing different is said. %TODO rausnehmen falls gar keine parallelen Kanten gebraucht werden
$|E(G)|$ denotes the number of edges, and for two edge sets $E$ and $F$ we always
have $|E \cup F | = |E | + |F |$ even if parallel edges arise.

We say that an edge $e = \{v, w\}$ or $e = (v, w)$ joins $v$ and $w$. In this case, $v$ and $w$ are adjacent. $v$ is a 
neighbour of $w$ (and vice versa). $v$ and $w$ are the endpoints of $e$. If $v$ is an endpoint of an edge $e$, we say 
that $v$ is incident with $e$. 
In the directed case we say that $( v, w)$ leaves $v$ and enters $w$, $v$ is the tail and $w$ is the head of the arc 
$e$. Two edges which share at least one endpoint are called adjacent.

For a digraph $G$ we sometimes consider the underlying undirected graph, i.e. the undirected graph $G'$ on the same 
vertex set which contains an edge $\{v, w\}$
for each edge $(v, w)$ of $G$. We also say that $G$ is an orientation of $G'$.
A subgraph of a graph $G = (V(G), E(G))$ is a graph $H = (V(H), E(H))$
with $V(H) \subset V(G)$ and $E(H) \subset E(G)$. We also say that $G$ contains $H$. $H$ is an
induced subgraph of $G$ if it is a subgraph of $G$ and $E (H) = \{ \{x, y\} \textrm{ resp. } (x, y) \in
E(G) : x, y \in V(H)\}$. Here $H$ is the subgraph of $G$ induced by $V(H)$. We also
write $H = G[V(H)]$. A subgraph $H$ of $G$ is called spanning if $V(H) = V(G)$.
If $v \in V(G)$, we write $G- v$ for the subgraph of $G$ induced by $V(G) \setminus {v}$.
If $e \in E(G)$, we define $G- e := (V(G), E(G) \setminus \{e\})$. Furthermore, the addition
of a new edge $e$ is abbreviated by $G + e := (V(G), E(G) \cup {e})$. If $G$ and $H$
are two graphs, we denote by $G + H$ the graph with $V(G +H)= V(G) \cup V(H)$
and $E(G +H)$ being the disjoint union of $E(G)$ and $E(H)$ (parallel edges may arise).

For a graph $G$ and $X, Y\subseteq V(G)$ we define $E(X, Y) := \{\{x, y\} \in E(G) : x \in
X \setminus Y, y \in Y \setminus X\}$ resp. $E^+(X, Y) := \{(x, y) \in E(G) : x \in X\setminus Y, y \in Y \setminus 
X\}$.
For undirected graphs $G$ and $X \subseteq V(G)$ we define $\delta(X) := E(X, V(G) \setminus X)$. The
set of neighbours of $X$ is defined by $ \Gamma(X) := \{v \in V(G) \setminus X : E(X, \{v\})  \neq \emptyset\}$.
For digraphs $G$ and $X \subseteq V(G)$ we define $\delta^+(X) := E^+(x, V(G) \setminus X)$, $\delta^-(x) :=
\delta^+(V(G) \setminus X)$ and $\delta(X) := \delta^+(x) \cup \delta^-(x)$. We use subscripts (e.g. $\delta_G(X)$) to
specify the graph $G$ if necessary.

For singletons, i.e. one-element vertex sets $\{v\} (v \in V(G))$ we write $\delta(v) :=
\delta(\{v\})$, $\Gamma(v) := \Gamma(\{v\}), \delta^+(v) := \delta^+(\{v\})$ and $\delta^-(v) := \delta^-(\{v\})$. The 
degree of a vertex $v$ is $|\delta(v)|$, the number of edges incident to $v$. In the directed case, the
in-degree is $|\delta^-(v)|$, the out-degree is $|\delta^+(v)|$, and the degree is $|\delta^+(v)|+ |\delta^-(v)|$.
A vertex $v$ with zero degree is called isolated. A graph where all vertices have
degree $k$ is called $k$-regular.

An edge progression $W$ in $G$ is a sequence $v_1, e_1, v_2, \dots , v_k, e_k, v_{k+1}$ such that $k \ge 0$,
and $e_i = (v_i, v_{i+ 1}) \in E(G)$ resp. $e_i = \{v_i, v_{i+1}\}\in E(G)$ for $i = 1, \dots , k$. If in
addition $e_i \ne e_j \,\forall\, 1 \le i < j \le k$, $W$ is called a walk in $G$. $W$ is closed if
$v_1 = v_{k+1}$. A path is a graph $P = (\{v_1, ... , v_{k+1}\}, \{e_1, ... , e_k\})$ such that $v_i \ne v_j$ for
$1 \le i < j \le k + 1$ and the sequence $v_1 , e_1 , v_2, \dots , v_k, e_k, v_{k+1}$ is a walk. $P$ is
also called a path from $v_1$ to $v_{k+1}$ or a $v_1 - v_{k+1}$-path. $v_1$ and $v_{k+1}$ are the endpoints
of $P$. By $P_{[x,y]}$ with $x, y \in V(P)$ we mean the (unique) subgraph of $P$ which is
an $x-y$-path. Evidently, there is an edge progression from a vertex $v$ to another
vertex $w$ if and only if there is a $v-w$-path.

A cycle is a graph $(\{v_1, \dots , v_k\}, \{e_1, \dots, e_k\})$ such that the sequence $v_1, e_1, v_2, 
\dots , v_k,e_k,v_1$ is a (closed) walk and $v_i \ne v_j$ for $1 \le i < j\le k$.
An easy induction argument shows that the edge set of a closed walk can be
partitioned into edge sets of cycles.

The length of a path or cycle is the number of its edges. If it is a subgraph
of $G$, we speak of a path or cycle in $G$. A spanning path in $G$ is called a
Hamiltonian path while a spanning cycle in $G$ is called a Hamiltonian cycle
or a tour. A graph containing a Hamiltonian cycle is a Hamiltonian graph.
For two vertices $v$ and $w$ we write $dist(v, w)$ or $dist_G (v, w)$ for the length of
a shortest $v-w$-path (the distance from $v$ to $w$) in $G$. If there is no $v-w$-path at all,
i.e. $w$ is not reachable from $v$, we set $dist(v, w) := \inf$. In the undirected case,
$dist(v, w) = dist(w, v)$ for all $v, w \in V(G)$.

We shall often have a cost function $c : E(G) \to \R$. Then for $F \subseteq E(G)$ we
write $c(F) := \sum_{e\in F} c(e)$ (and $c(\emptyset) = 0$). This extends $c$ to a modular function
$c : 2^{E(G)}\to \R$. Moreover, $dist_{(G,c)}(v, w)$ denotes the minimum $c(E(P))$ over all
$v-w$-paths $P$ in $G$.

\newpage
\section{Das Problem der Bestimmung azyklischer Flussschranken}
\newpage
\section{Implementierung}
\newpage
\section{Ergebnisse der praktischen Rechnungen}
\newpage
\section{Zusammenfassung}
\newpage
\section*{Literaturverzeichnis}
% \newpage
% \section{Situation auf dem Torus}Wie wir gesehen haben, ist es auf planaren $\alpha$-Orientierungen, also auch auf planaren Domino-Tilings m�glich, nur mittels drehen von essentiellen Kreisen (in Domino-Tilings tats�chlich Face-Flips) jede m�gliche Konfiguration zu erreichen.

Dabei stellt sich die Frage, ob und wie man dieses Resultat verallgemeinern kann. $\alpha$-Orientierungen lassen sich problemlos auf dem Torus definieren. Eine $(3,1)$-Orientierung wie im vorherigen Kapitel k�nnen wir zumindest auf torischen Gittergraphen gerader Breite und H�he definieren - eine ungerade Breite oder H�he macht �hnlich wie bei der H�henfunktion Probleme, da es dann benachbarte Knoten geben m�sste, die den gleichen Ausgrad haben. Wir betrachten hier daher nur torische Gittergraphen, die sowohl gerade H�he als auch Breite haben.

Eine besondere Rolle spielt auf dem Torus die topologische Struktur. 

\begin{definition}
Wir k�nnen Kreise in $\overrightarrow{G}$ als Vektoren des Kreisraums in $\{-1,0,1\}^{|A|}$ betrachten, wobei die $i$-te Komponente des Vektors der Kante $e_i$ entspricht und den Wert 1 hat, falls die Kante im Kreis enthalten und bez�glich einer festen Orientierung $X$ gleich gerichtet ist, 0 falls $e_i$ nicht im entsprechenden Kreis enthalten ist und -1 sonst. \\
Wenn $v_C$ der Vektor einer Menge von gerichteten Kreisen ist und sich als Linearkombination von Vektoren in $\{-1,0,1\}$ von Face-Kreisen schreiben l�sst, nennen wir $C$ nullhomolog.
\end{definition}

\begin{figure}[h!]
  \centering
  \scalebox{1}{\input{Bilder/nullhomolog.pstex_t}}
  \caption{Betrachte die beiden roten Kreise: gemeinsam sind sie in dem linken Bild nullhomolog, im rechten sind sie es nicht mehr}
\end{figure}

 Kolja Knauer hat in \cite{Knauer07partialorders} gezeigt, dass es f�r einen torischen Gittergraphen mit $\alpha(v)=2 \forall v \in V$ ausreicht, wenn man alle Faces bis auf eines, und zus�tzlich noch zwei nicht nullhomologe Kreise erlaubt, um den Zusammenhang und die Poset-Struktur zu erreichen. Durch das Festhalten eines Faces schafft man es insbesondere, die Azyklizit�t zu sichern.
Zugleich muss die ben�tigte Menge von Kreisen eine Basis des Kreisraumes enthalten, daher ist diese Menge in seinem Beispiel auch optimal. 

Wir wollen das "'erzeugen einer Orientierung"' formaler machen: Definiere eine \textit{zul�ssige Flipsequenz} in einer Orientierung $X$ von $\overrightarrow{G}$ auf dem Torus als eine Folge $C_1,\dots ,C_n$ von Kreisen, so dass $C_1$ in $X$ gerichtet ist, und $C_i, \, i=\{2,\dots,n\}$ jeweils in dem Graphen gerichtet ist, der entsteht wenn man nacheinander die Kreise $C_1,\dots ,C_{i-1}$ flippt. 
Wir erlauben hier also beliebige Kreise mit beliebiger Drehrichtung zu flippen. Wenn durch eine Folge von Flips von $C_1\dots C_n$  genau die Kantenmenge eines Kreises $C$ umgedreht wird, der selber nicht geflippt wurde, sagen wir dass $C$ sich aus $C_1\dots C_n$ zusammensetzen oder erzeugen l�sst. 

Beispielsweise haben wir bereits gesehen, dass sich jeder Kreis in planaren $\alpha$-Orientierungen durch Folgen von Face-Flips erzeugen l�sst. Teilt ein Kreis $C$ die Fl�che $S$ auf der $\overrightarrow{G}$ eingebettet ist in zwei Regionen, von denen eine hom�omorph zu einer offenen Kreisscheibe ist, dann ist der $C$ mit dem auf dieser Region engebetteten Teil des Graphen ein planarer Teilgraph von $\overrightarrow{G}$, und wir k�nnen $C$ wie bereits gezeigt durch Face-Flips in seinem Inneren zusammensetzen. 
 
Kann man also �hnliches auch f�r den Fall der torischen Domino Tilings, also der $(3,1)$-Orientierungen auf dem Torus zeigen? Kann man eine Menge von Kreisen ausw�hlen, mit denen man tats�chlich jede Orientierung auf dem Graphen mittels Kreisflips erzeugen kann, sozusagen eine Art Erzeugendensystem f�r alle Orientierungen? Wir werden sehen, dass wir hier eine deutlich gr��ere Menge von Kreisen brauchen, eine Basis des Kreisraumes also nicht ausreicht.

Betrachten wir also eine $(3,1)$-Orientierung auf dem Torus. L�sst man tats�chlich das Drehen aller denkbaren Kreise zu, so ist es sofort m�glich alle Orientierungen ausgehend von einer beliebigen Orientierung zu erzeugen, wie wir in Lemma \ref{allgFlipZshg} gezeigt haben. 

Da wir gerade gesehen haben, dass Face-Flips auch hier gr��ere Kreise erzeugen k�nnen, ist klar dass nicht alle Kreise n�tig sind. Kann man also m�glichst kleine Kreismengen finden um alle Orientierungen zu erzeugen?


\subsection{Gegenbeispiele}
Hier soll es darum gehen, welche Klassen von Kreisen sich nicht eignen bzw nicht ausreichen um �berhaupt den Flipzusammenhang zu gew�hrleisten - und wie man das jeweils beweisen kann.\\
Zur Notation: Je nachdem was man zeigen will, eignen sich verschiedene Modelle besser um zu verstehen, was geschieht. Ich werde hier im wesentlichen von den $\alpha$-Orientierungen ausgehen, die auf dem zugrundeliegenden torischen Gittergraphen das Domino Tiling des Torus eindeutig bestimmen. Immer wenn von Kanten die Rede ist, sind daher Kanten in diesem Graph gemeint. "'Matchingkanten"' sind darin die Kanten, die komplett in einem jeweiligen Domino liegen, d.h. die f�r ihre inzidenten Knoten die jeweils einzige ausgehende bzw einzige eingehende Kante bilden. In den Bildern sind die jeweils gegen�berliegenden R�nder identifiziert sofern nicht explizit etwas anderes angegeben ist, so dass Dominos/Kanten des Graphen "'�ber den Rand"' gehen k�nnen.\\

Als einfaches Beispiel daf�r, dass allein Face-Kreise nicht ausreichen kann das Muster einer "'Ziegelmauer"' dienen. Dabei sind die Matchingkanten im Graphen horizontal angeordnet, aber wie bei einer Mauer um jeweils eins versetzt. 
\begin{figure}[h!]
  \centering
  {\input{Bilder/ex0.pstex_t}}
\caption{Ein Domino Tiling des Torus (gegen�berliegende Seiten identifiziert) , bei dem kein Face-Flip m�glich ist}
\end{figure}
Bei dieser Anordnung gibt es allerdings keinen einzigen gerichteten Face-Kreis, bzw. f�r Tilings keine zwei nebeneinanderliegenden Dominos, die man drehen k�nnte. Allein mit Face-Flips kommt man hier also nicht weiter, es braucht also offensichtlich mehr Kreise. F�r dieses Beispiel sticht eine L�sung sofort ins Auge - man kann gerade horizontale (und wegen Rotationssymmetrie vertikale) Kreise erlauben, d.h. Kreise in denen alle Kanten in die gleiche Richtung zeigen. Damit kann man zumindest andere Orientierungen erzeugen, es ist jedoch nicht klar ob man alle erh�lt. (Tats�chlich erh�lt man nicht alle wie wir gleich sehen werden!)\\

Wenn man erlaubt, einen nicht nullhomologen Kreis umzudrehen, kann dies durchaus das Flippen weiterer Kreise erm�glichen wenn der Kreis denn gerichtet ist. Betrachtet man allerdings den gleichen Graph, allerdings etwas verschoben, dann ist dieser Kreis im allgemeinen nicht mehr gerichtet und kann deshalb nicht umgedreht werden. Aus diesem Grund werde ich hier im weiteren Kreisklassen behandeln: 
\begin{definition}
Eine Kreisklasse ist eine Menge von Kreisen, f�r die es f�r jeden Kreis $C$ darin einen Knoten $v_0^C$ gibt, so dass von $v_0^C$ ausgehend die Folge der Kantenrichtungen ( $\in$ \{Hoch, Runter, Links, Rechts\} ) die gleiche ist. Wir erlauben also eine bestimmte Form von Kreisen mit jedem m�glichem Startpunkt im Graphen zu flippen, wenn wir davon sprechen eine Kreisklasse zu erlauben.
\end{definition}

Welche nat�rlichen Kreise bieten sich noch an, nachdem Face-Flips, wie wir gesehen haben, nicht reichen? Reicht es beispielsweise aus, wenn man gerade und treppenf�rmige Kreise w�hlt? Auch hier ist die Antwort nein. Nehmen wir ein quadratisches Gitter, horizontale und vertikale gerade Kreise und dazu die folgenden beiden Kreistypen ("'Treppen"'):\\
%TODO Bilder verbessern
\begin{figure}[h!]
  \centering
  \scalebox{1}{\input{Bilder/treppen1.pstex_t}}
\end{figure}

Erlauben wir nun das Flippen dieser drei Kreisklassen. Im Gegensatz zu dem obigen Beispiel kann man hier aufgrund der Form der Kreise nicht mehr erwarten, eine Orientierung/ein Tiling zu finden in der alle in Frage kommenden Kreise (insbes. Faces) nicht gerichtet sind. Betrachte beispielsweise folgendes Beispiel: in Abb 7 sind zwar einzelne Faces flipbar, allerdings ist nicht klar ob man hiermit alle anderen Orientierungen/Tilings erreichen kann. Jedoch sind s�mtliche nun erlaubten nicht nullhomologen Kreise in diesem Beispiel nicht gerichtet, also nicht flipbar.\\

\begin{figure}[h!]
\label{ex1}
  \centering
  \scalebox{1}{\input{Bilder/ex1.pstex_t}}
\caption{Beispiel, in dem Faces (gelb markiert) drehbar sind. Lassen sich hieraus mit den obigen Kreisen alle Orientierungen erzeugen?}
\end{figure}
 
Das folgende Beispiel zeigt uns, dass auch mit Face-Flips mitunter nur sehr wenige Orientierungen erreichbar sind. Wir k�nnen also nicht einfach durch Face-Flips immer in eine g�nstigere Position kommen, in der wir gr��ere nicht nullhomologe Kreise gerichtet haben, sondern es kann passieren, dass wir immer wieder in �hnlich schlechte Orientierungen kommen (Abb 8).
 
 \begin{figure}[h!]
\label{ggbsp1}  
  \centering
  \scalebox{0.9}{\input{Bilder/ggbsp1.pstex_t}}
  \caption{Die Ausgangskonfiguration und die Endkonfiguration sind bis auf eine Verschiebung gleich.}
\end{figure}

Da wir offenbar viele Beispiele finden k�nnen, bei denen man nicht sofort sehen kann ob alle Orientierungen erreicht werden k�nnen, brauchen wir eine aussagekr�ftige Bedingung, mit der man (m�glichst mit geringem Aufwand) pr�fen kann, ob man wirklich nicht alle anderen Orientierungen mit den gegebenen Kreisen erzeugen kann. Das hei�t wir wollen pr�fen, ob man an einer konkreten Orientierung sehen kann, dass die gegebenen Kreisklassen nicht ausreichen um den Flipzusammenhang herzustellen.\\

%Notation: In der zu einem DominoTiling geh\"orenden $\alpha$-Orientierung hat jeder Knoten eine eindeutige Kante - die einzige eingehende oder einzige ausgehende. Diese nennen wir hier die Matchingkante des Knotens. 
\begin{definition} 
Sei ein Kreis $C$ und ein Knoten $v$ auf diesem Kreis gegeben. Sind die beiden in $C$ zu $v$ inzidenten Kanten in Durchlaufrichtung des Kreises entgegengesetzt gerichtet, so nennen wir diesen Knoten eine Konfliktstelle von $C$.
\end{definition}

\begin{lemma}
\label{notwBed}
Sei ein Kreis $C$ in einer torischen $(deg-1,1)$-Orientierung gegeben. 
Sei $M_{ein}$ die Anzahl der (in Durchlaufrichtung) auf der rechten Seite in $C$ eingehenden Matchingkanten des Graphen und $M_{aus}$ die Anzahl der auf dieser Seite aus $C$ ausgehenden Matchingkanten an Konfliktstellen. \\
Dann ist f�r jede Folge von Face-Flips im Graphen die Differenz $M_{ein}-M_{aus}$ f�r $C$ konstant.
\end{lemma}
Da in einem gerichteten Kreis die Differenz $M_{ein}-M_{aus}=0 $ ist, f�hrt uns dies direkt zu folgendem, �u�erst n�tzlichen Korollar:

\begin{cor}
Wenn zu einer Seite des Kreises, oBdA in Durchlaufrichtung rechts, die Anzahl der in diese Richtung an den Konfliktstellen ausgehenden Matchingkanten ungleich der aus dieser Richtung an Konfliktstellen eingehenden Matchingkanten ist, so kann man den Kreis nicht durch flippen einer Folge von Face-Kreisen in einen gerichteten \"uberf\"uhren.
\end{cor}

\begin{proof}[Beweis des Lemmas]
Betrachte einen Face-Flip. Falls das Face keinen Knoten mit $C$ gemeinsam hat, so ist die Bahauptung klar. Andernfalls betrachten wir also ein gerichtetes Face, das zur rechten Seite von $C$ liegt und mindestens einen gemeinsamen Knoten mit $C$ hat. Der Teil vom Face, der nicht zu $C$ geh�rt, ist ein Pfad der zur rechten Seite von $C$ ein- und ausgeht und auf dem die Richtung der Kanten geflippt wird. Da er gerichtet ist, wechseln sich auf ihm Matchingkanten mit Nichtmatchingkanten ab.
\begin{enumerate}
\item Fall 1: Der Pfad hat eine gerade Anzahl von Kanten:\\
Dann ist entweder die erste oder die letzte Kante eine Matchingkante - egal wierum der Pfad gerichtet ist. Ist vor dem flippen des Kreises die erste Kante des Pfades eine Matchingkante, so ist es danach die (nun erste) Kante, die vorher letzte des Pfades war. Das flippen �ndert also nichts an den Werten $M_{ein}$ und $M_{aus}$. 
\item Fall 2: Der Pfad hat ungerade Anzahl Kanten: \\
Dann sind beide oder keine der zu $C$ inzidenten Kanten im Matching. Wenn es beide sind, dann sind sie auf dem Pfad gleich gerichtet, und deshalb ist eine von $C$ aus gesehen eine eingehende und die andere eine ausgehende Kante. Die Differenz $M_{ein}-M_{aus}$ bleibt also genauso wie wenn keine der beiden Kanten im Matching ist.
\item es bleibt noch zu zeigen, warum wir nur Matchingkanten an Konfliktstellen betrachten:\\
Ist in einem Knoten eine Kante im Matching, dann sind die restlichen Kanten entweder alle im Bezug zum Knoten ausgehend oder alle eingehend. Dadurch sind sie aber entgegengesetzt gerichtet, wenn also von einem Knoten auf $C$  die Matchingkante nicht zu $C$ geh�rt, dann ist dieser Knoten immer ein Konfliktstelle. Man w�rde also auch wenn man es nicht explizit sagt nur die Matchingkanten an Konfliktstellen z�hlen.
\end{enumerate}

\end{proof}


Was hilft uns nun dieses Lemma? Wir haben nun ein leicht zu �berpr�fendes Kriterium an der Hand, um zu entscheiden ob eine gegebene Menge von Kreisen \textit{nicht} ausreicht um alle Domino-Tilings mittels Flips zu erzeugen. Dazu reicht es, ein Beispiel anzugeben, in dem s�mtliche gegebenen Kreise nicht gerichtet sind, und nachzuweisen dass au�erdem f�r jeden dieser Kreise $M_{ein}-M_{aus} \neq 0$ ist. 
Es gibt uns also eine notwendige Bedingung daf�r, dass man ungerichtete Kreise mittels Face-Flips in gerichtete �berf�hren kann. 
%TODO Ist diese Bedingung aber auch hinreichend? hab leider kein richtiges Beispiel...


Nun stellt sich nat�rlich die Frage nach konkreten Beipielen, die diese Bedingung erf�llen. Wir haben bereits einige kleine Beipiele gesehen, aber konnten noch nicht beweisen, welche Kreise in diesen Beispielen jeweils nicht ausreichend sind.
Der n�chste Abschnitt wird sich daher genauer damit besch�ftigen.

\subsection*{Einige konkrete Gegenbeispiele}
Um konkrete Beispiele zu finden gibt es verschiedene M�glichkeiten. Man kann zum Beispiel ein beliebiges Domino Tiling hernehmen und alle m�glichen Kreise �berpr�fen ob sie die Bedingungen des Lemmas erf�llen. Die Wahrscheinlichkeit ist hoch, dabei einige Kreisklassen oder Kreise zu entdecken.
Man kann sich andererseits auch eine Menge von Kreisen/Kreisklassen definieren, die man ausschlie�en will. Anschlie�end konstruiert man ein entsprechendes Domino Tiling per Hand (falls das m�glich ist) und �berpr�ft es dann mit Hilfe des Lemmas. Auf diese Weise bin ich bei den folgenden Beispielen vorgegangen:

\begin{figure}[h!]
  \centering
  \scalebox{1}{\input{Bilder/ex1.pstex_t}}
  \caption{Beipiel 1 }
\end{figure}

\begin{figure}[h!]
  \centering
  \scalebox{1}{\input{Bilder/ex2.pstex_t}}
  \caption{Beipiel 2 }
\end{figure}

\begin{figure}[h!]
  \centering
  \scalebox{1}{\input{Bilder/ex2orientierung.pstex_t}}
  \caption{nochmal Beispiel 2 mit dem zugeh�rigen gerichteten Graph}
\end{figure}

Folgende Klassen von Kreisen sind in diesem Domino-Tiling nicht flipbar: 

\begin{figure}[h!]
  \centering
  \scalebox{1}{\input{Bilder/treppen3.pstex_t}}
  \caption{Kreisklassen die in Bsp 2 nicht durch Face-Flips gerichtet gemacht werden k�nnen}
\end{figure}

Zur �berpr�fung habe ich ein kleines Java-Programm genutzt. %Das Programm ranh�ngen???


Sind diese Beispiele aber auch aussagekr�ftig f�r allgemeinere F�lle, oder k�nnen es m�glicherweise nur Ausnahmen sein, die auf besonders kleinen Instanzen auftreten? Auch hierf�r hilft uns unser Lemma weiter.

\begin{definition}
\label{defT2}
Sei ein torisches Domino Tiling $T$ gegeben. Wir haben torische Domino Tilings definiert �ber Domino Tilings einer rechteckigen Region, bei der die jeweils gegen�berliegenden Seiten identifiziert werden. Bezeichne die R�nder dieser rechteckigen Region als $a,b,c,d$ wobei $a,c$ und $b,d$ jeweils die gegen�berliegenden Seiten sind, die urspr�nglich identifiziert wurden. Konstruiere nun folgenderma�en ein doppelt so gro�es Domino Tiling aus $T$: Erstelle eine Kopie $T'$ von $T$, mit den Seiten $a',b',c',d'$, wobei $a'$ die Kopie von $a$ ist usw. Identifiziere nun die Seitenpaare: $(b,d),(b',d'),(a,c'),(a',c)$. Das entstandene torische Domino Tiling nenne $T^2$. 
\end{definition}

Nun k�nnen wir folgendes beobachten: 

\begin{lemma}
Sei ein torisches Domino Tiling $T$ gegeben, so dass $k$ Kreise die Bedingung aus Lemma \ref{notwBed} nicht erf�llen,  um durch Face-Flips in gerichtete Kreise �berf�hrt zu werden, die also an Konfliktstellen ungleich viele eingehende und ausgehende Kanten haben. Dann hat $T^2$ mindestens genauso viele Kreise mit dieser Eigenschaft.
\end{lemma}

\begin{proof}
Durch die Konstruktion ergibt sich, dass jeder Kreis der �ber den Rand $(a,c)$ verl�uft, zu einem gr��eren Kreis aus zwei gleichartigen Kopien verl�ngert wird. Die Anzahl der ein- bzw. ausgehenden Matchingkanten an Konfliktstellen wird dabei jeweils verdoppelt, da ja der gleiche Kreis mit der gleichen Orientierung nun zwei mal durchlaufen wird. Insbesondere bleiben die Zahlen unterschiedlich, wenn sie es vorher waren.
\end{proof}


Haben wir also mit Hilfe des Lemmas \ref{notwBed} ein Beispiel gefunden bei dem wir mindestens $k$ verschiedene Kreise ben�tigen, so finden wir auch beliebig gr��ere Beispiele die mindestens $k$ verschiedene Kreise ben�tigen. In der Tat scheint es plausibel, dass die Anzahl der ben�tigten Kreisklassen mit der Gr��e der Fl�che monoton wachsend ist. 



Es ist, zumindest wenn man das Wachstum der ben�tigten Kreise betrachtet, daf�r wie gesagt n�tig sich ganze Kreisklassen anzuschauen, da wir nach obiger Beobachtung auf jeden Fall immer gr��ere Beispiele mit im Prinzip den gleichen Kreisklassen bauen k�nnen. Die Anzahl der Kreise in einer Kreisklasse ist aber offensichtlich monoton wachsend, da bei einer gr��eren Fl�che nat�rlich auch mehr Startpositionen zur Verf�gung stehen, an denen gleichgeformte Kreise beginnen k�nnen.

%TODO TODO TODO!!!
\begin{prop}
Die Anzahl der Kreisklassen, f�r die man zeigen kann, dass sie gemeinsam mit Face-Flips nicht gen�gen um alle Orientierungen zu erzeugen, ist unbeschr�nkt, d.h. sie w�chst mit der Gr��e des Gitters.
\end{prop}

\begin{proof}
Wir haben im Beweis des vorherigen Lemmas gesehen, dass man die L�cke zwischen der Zahl der eingehenden Matchingkanten und der ausgehenden Matchingkanten auf einer Seite des Kreises beliebig gro� machen kann. Die Idee ist: Wenn eine lokale �nderung an der Kreisklasse klein genug und die Differenz $M_{ein}-M_{aus}$ gro� genug ist, dann kann es durch die lokale �nderung nicht genug Matchingkanten geben, um den Unterschied auf 0 zu bringen.

Da wir die Differenz $M_{ein}-M_{aus}$ beliebig gro� machen k�nnen, m�ssen wir nur zeigen, dass es immer eine lokale �nderung einer Kreisklasse gibt, durch die eine noch nicht verwendete Kreisklasse entsteht. Dies k�nnen wir folgenderma�en erreichen:\\
Sei $K$ die Menge der Kreisklassen, f�r die wir wissen dass sie in einger gegebenen Orientierung $T$ die Bedingung $M_{ein}-M_{aus}\ne 0$ erf�llen und deren Kreise horizontal verlaufen (d.h. formell: in der Folge der Kantenrichtungen der Kreisklasse kommt f�r ein Gitter der Breite $n$ die Richtung "'Rechts"' $r$ mal auf f�r ein $r\in \N,\, r\le n$ und die Richtung "'Links"' nur $r-n$ mal).
Analog zu der in \ref{defT2} beschriebenen Methode erzeuge eine Orientierung $T^3$ die aus 3 Kopien von $T$ besteht, jeweils an der selben Stelle mit der Nachbarkopie identifiziert. Die Identifizierung soll so erfolgen, dass die horizontalen Kreise aus $K$ jeweils zu einer neuen Kreisklasse verschmelzen. Die Menge der hierdurch entstandenen Kreisklassen nennen wir $K'$.  In $T^3$ hat jede Kreisklasse aus $K'$ nun $|M_{ein}-M_{aus}|\le 3$ in allen enthaltenen Kreisen. F�hre folgende lokale �nderung f�r eine Kreisklasse durch: Da die Kreise horizontal sind, gibt es eine Kante $e=(u,v)$, so dass die Kante zwischen den beiden Knoten, die von $u,v$ jeweils in Richtung "'Runter"' liegen, nicht zu der Kreisklasse geh�rt. Ersetze nun in der Folge der Kantenrichtungen "'Richtung von $e$"' durch "'Runter, Richtung von $e$, Hoch"'. Dies ist tats�chlich eine neue horizontale Kreisklasse, denn alle in $K'$ haben die gleiche Richtungsfolge 3 mal hintereinander, w�hrend die neue Kreisklasse als einzige eine Variation enth�lt. Da nur vier benachbarte Knoten betroffen sind, die als nur insgesamt zwei Matchingkanten haben k�nnen, kann sich $M_{ein}-M_{aus}$ h�chstens um $\pm 2$ ver�ndern, somit folgt $|M_{ein}-M_{aus}|\le 1$ und wir haben nun mindestens $|K|+1$ viele Kreisklassen die Probleme machen. 

\end{proof}
% \newpage
% \subsection{Einschr\"ankung der ben\"otigten Kreisklassen} 
Wie wir nun wissen, kann man nicht unbedingt erwarten mit wenigen Kreisklassen den Flipzusammenhang wieder herzustellen. %TODO Es waere gut nen Beweis zu haben dass es nicht mit konstant vielen Kreisklassen geht...
Allerdings wissen wir auch, dass sich bestimmte Kreisflips durch mehrere andere erzeugen lassen. Wir betrachten deshalb die Kombination von Faces und anderen Kreisklassen.
 Dabei ist das, was in diesem Kapitel zu finden ist, nur ein Anfang, in dem per Hand einige Kreise aus anderen konstruiert werden. W\"unschenswert w\"are nat\"urlich eher ein Ergebnis, dass allgemein angibt wie viele Kreise man sparen kann wenn man eine bestimmte Menge von Kreisen zu flippen erlaubt - leider ist mir so etwas nicht gelungen. Au�erdem ist zu beachten, dass nicht klar ist, ob eine Menge von Kreisen mit denen man den Flipzusammenhang nachweisbar nicht herstellen kann auch bedeutet, dass man dies nicht doch mit einer kleinen Menge von Kreisen bewerkstelligen kann.\\

Face-Flips sind sehr nat\"urlich und erm\"oglichen uns bereits eine sehr gro\ss e Anzahl von Kreisklassen zu flippen, au�erdem ein nat�rlicher Teil der Kreisraumbasis. Deshalb werden wir immer Face-Flips zulassen. Da diese wie gezeigt nicht ausreichen, brauchen wir mindestens eine weitere Klasse von Kreisen deren Flips wir erlauben. Das einfachste Beispiel hierf�r sind die Kreise, die ohne einen Knick immer geradeaus gehen, aus Symmetriegr\"unden nehmen wir gleich sowohl die horizontalen als auch die vertikalen Kreise.

\begin{definition}
Die Menge der Knoten $(x,y)$, die im torischen Gittergraphen der $(3,1)$-Orientierung die jeweils gleiche Koordinate $x$ oder die gleiche Koordinate $y$ haben, nennen wir horizontale oder vertikale Achse des Graphen. Ein Kreis in diesem Graphen hei�t gerade, wenn alle seine Knoten auf der gleichen Achse liegen.\\

Als Beule (bez�glich einer Achse) bezeichnen wir einen Pfad $P$, so dass $P$ nur Start- und Endknoten auf der Achse hat und mit Kanten des geraden Kreises dieser Achse zu einem einfachen nullhomologen Kreis erweitert werden kann. Die L�nge der Beule ist dabei der maximale Abstand eines Punktes auf $P$ zur Achse bez�glich der Seite, zu der $P$ die Achse verl�sst und betritt. Die Breite ist die Anzahl der Kanten, die auf dem geraden Achsenkreis ben�tigt werden, um $P$ zu einem einfachen nullhomologen Kreis zu erweitern.
\end{definition}

%Bild?
Was bringt uns das nun? Wir werden sehen, dass sich Kreise mit schmalen Beulen sich durch gerade und Face-Kreise erzeugen lassen. 
%[schon oben definiert]Eine Ausst�lpung des Kreises ist dabei ein Teil des Kreises, der von der Hauptrichtung zu einer Seite des Kreises abzweigt und auf der gleichen Seite wieder in den Hauptkreis einm\"undet. 

Betrachte als Hauptkreis einen geraden Kreis mit einer Beule der Breite 1, der gerichtet ist, also theoretisch im ganzen flipbar w\"are. Im folgenden Bild sieht man sehr gut, wie leicht sich diese aus Facekreisen und dem Hauptkreis zusammensetzen l�sst. 
\begin{figure}[h!]
  \centering
  \scalebox{1}{\input{Bilder/beule1.pstex_t}}
\caption{Beispiel wie eine Ausst�lpung durch eine Kombination mit nullhomologen Kreisen geflippt wird}
\end{figure}

Gleiches funktioniert nicht nur in diesem Beispiel, sondern f�r Beulen der Breite 1 bei beliebiger L�nge: die Beule an sich ist ja ein nullhomologer Kreis, der bis auf die mit dem Hauptkreis geteilte Kante gerichtet sein muss. Diese Kante kann wenn sie entgegen den anderen gerichtet ist durch das flippen des Hauptkreises umgedreht werden, andernfalls ist sie f�r den Hauptkreis ungerichtet und wird durch den Flip des nullhomologen Kreises gedreht. Dabei muss man nat�rlich die Reihenfolge beachten, das ganze ist in Abbildung \ref{beule2}
schematisch dargestellt.\\

\begin{figure}[h!]

  \centering
  \scalebox{1}{\input{Bilder/beule2.pstex_t}}
\caption{Unterschiedliche Reihenfolge des der Flips, abh�ngig von der Richtung der (roten) Zwischenkante}
\label{beule2}
\end{figure}

Dieses Vorgehen wird schwieriger, wenn die Breite der Beulen zunimmt. Die hier rot gemalte Zwischenkante wird zu einem Pfad von mehreren Zwischenkanten. Diese sind dann aber m�glicherweise nicht gerichtet!  Abbildung \ref{2erbeule}
verdeutlicht das Problem:\\

\begin{figure}[h!]
  \centering
  \scalebox{1}{\input{Bilder/2erbeule.pstex_t}}
\caption{Bei dieser $2\times 1$ Beule ist es nicht mehr direkt m�glich, entlang des schwarz eingezeichneten Kreises einen Flip aus dem geraden Kreis und einem nullhomologen Kreis zusammenzusetzen. Im Domino Tiling (rechte Darstellung) erkennt man, dass es nicht mal klar ist ob �berhaupt Face-Flips m�glich sind.}
\label{2erbeule}
\end{figure}


Wie es scheint k�nnen wir bei dieser Form von Beulen zumindest mit der obigen Methode nichts erreichen. Die gute Nachricht ist aber: Wir sind nicht bei allen gr��eren Ausst�lpungen verloren. H�ufig ist es m�glich, zumindest einen Teil des Kreises �ber Faceflips zu drehen. % und das was �brigbleibt ist dann eben eine solche $2\times 1$ Beule.? ->NEIN
Betrachte die Situation in einer Ecke (Abb \ref{abkuerzung}) 


\begin{figure}[h!]
  \centering
  \scalebox{1}{\input{Bilder/beule3.pstex_t}}
\caption{Eine der beiden gr�nen Kanten muss eine eingehende Kante in den gemeinsamen Knoten sein. Alle nichtgr�nen Kanten folgen zwangsweise aus der Form des Pfades. Je nachdem wie man die Kantenrichtung w�hlt, erh�lt man entweder eine Abk�rzung des gro�en Kreises, oder man erh�lt einen kleinen nullhomologen Kreis aus zwei Faces den man flippen kann. In beiden F�llen reduziert sich die Gr��e des ben�tigten Kreises.}
\label{abkuerzung}
\end{figure}

Analog dazu ist das Bild, wenn der Eckknoten 3 eingehende und eine ausgehende Kante hat, nur dass die Wege dann jeweils in die andere Richtung laufen. Es gen�gt also, einen kleineren Kreis zuzulassen um den gr��eren zu erzeugen. Dieses Argument kann man nun induktiv immer weiter fortf�hren, da in der Regel auch der entsprechende kleinere Kreis eine solche Ecke besitzt. 
Wie weit kann man die Gr��e der Beulen so einschr�nken? Leider kann das Domino Tiling durchaus so aussehen, dass ein gro�er Teil der Beule �brig bleibt. Wir k�nnen also damit nur Ecken ausschlie�en, die aus "' hoch, hoch, rechts, rechts"' oder entsprechend gedrehten/gespiegelten Kanten bestehen.\\ Betrachte dazu Abbildung \ref{problembeule}

\begin{figure}[h!]
  \centering
  \scalebox{1}{\input{Bilder/beule4.pstex_t}}
\caption{Beispiel einer Beule, in der man keinen Face-Flip mehr anwenden kann um die Gr��e zu reduzieren}
\label{problembeule}
\end{figure}

sie zeigt ein Beispiel an dem keine solchen Ecken mehr vorhanden sind. Mit dem geraden Kreis als Grundkreis sind wir an dieser Stelle also auch am Ende, da wir im Gegensatz zu den Beulen der Breite 1 diese breiteren nicht durch Facekreise zusammensetzen k�nnen. Bei Breite 2 kommt es bereits auf die L�nge an: ist die L�nge gerade, dann lassen sie sich ebenso wie Beulen der Breite 1 durch Facekreise und den Grundkreis zusammensetzen, ist die L�nge ungerade funktioniert das allerdings auch schon nicht mehr.


%%% FAZIT
\section{Fazit}

Wie wir gesehen haben, lassen sich die positiven Resultate aus dem planaren Fall hier nicht direkt auf den Fall der torischen Domino Tilings erweitern. Insbesondere steigt die Zahl der ben�tigten Kreisklassen von einer (Face-Kreise) auf m�glicherweise beliebig viele abh�ngig von der Gittergr��e. Dadurch ist insbesondere nicht klar, wie man m�glichst g�nstig Flipzusammenhang konstruieren kann. Gleichzeitig konnte ich nicht zeigen, dass eine gro�e Menge ung�nstiger Kreisklassen auch implizieren, dass es keine kleinen Mengen gibt die ausreichend sind. Wir haben im letzten Kapitel gesehen, dass man durchaus Kreise aus anderen zusammensetzen kann, zumindest f�r den Fall von Faces und einem Grundkreis der nicht nullhomolog ist. 

Eine interessante Fragestellung w�re nun zum Beispiel, ob durch das Zusammenspiel mehrerer nicht nullhomologer Kreise mit Facekreisen deutlich mehr zu erreichen ist. Auf allgemeinen $\alpha$-Orientierungen stimmt es leider auch nicht, dass alle nullhomologen Kreise sich immer durch Faces zusammensetzen lassen, es w�re interessant zu sehen ob das auch f�r die Orientierungen gilt, die durch Domino Tilings induziert werden. %TODO geht da nicht mehr???
Ebenfalls konnte ich f�r aus Domino Tilings entstandene Graphen nicht zeigen, ob die Bedingung aus Lemma \ref{notwBed} auch hinreichend ist. 

\subsection*{Dank}

An Kolja Knauer, der sich pers�nlich wie per Mail die Zeit genommen hat um mich zu betreuen, sich immer wieder meine Ideen angeh�rt und dazu wertvolle Ideen entwickelt und mir Hinweise gegeben hat, insbesondere Lemma \ref{notwBed} betreffend - Danke!

%TODO hier weitergehen und genau sagen, wie weit die Gr��e sich reduziert!!!



% \newpage
% \newpage
% \addcontentsline{toc}{section}{Literaturverzeichnis}
% \bibliographystyle{amsplain}
% \bibliography{literatur}
%\nocite{*}

\end{document}
