\documentclass[a4paper]{article}

\usepackage[T1]{fontenc}
\usepackage[utf8]{inputenc}
\usepackage[ngerman]{babel}
\usepackage{times, graphicx, currvita, hyperref, longtable, float, caption, subcaption}
\usepackage{amsmath,amsfonts,amssymb, amsthm}%, dsfont, bbm, mathtools, stmaryrd}
%\usepackage{fancyhdr}
%\usepackage{color}
%\usepackage[english]{babel}
\usepackage{paralist}
%\usepackage{algorithmic}
\usepackage{wasysym}	% verschiedene Symbole, siehe http://rpi.edu/dept/arc/training/latex/LaTeX_symbols.pdf
\graphicspath{{Bilder/}}

\theoremstyle{definition}
\newtheorem{definition}{Definition}[section]
\newtheorem*{Def1}{Definition}
\newtheorem*{ex}{Example}

\newtheoremstyle{Tobi}{10pt}{}{}{}{\bf}{ }{\newline}{}
% die parameter sind: {name}{platz drueber}{platz drunter}{schriftart text}{einrueckung}{schriftart kopf}{Punktierung des Kopfes}{Platz zwischen kopf und text}

\theoremstyle{Tobi}
\newtheorem{prop}[definition]{Proposition}
\newtheorem*{Pro}{Proposition}
\newtheorem{theorem}[definition]{Satz}
%\newtheorem*{le}{Lemma}
\newtheorem{lemma}[definition]{Lemma}
\newtheorem{cor}[definition]{Korollar}
\newtheorem*{conj}{Conjecture}
\newtheorem{obs}[definition]{Beobachtung}

% \theoremstyle{remark}
% \newtheorem*{que}{Questions}
% \newtheorem*{claim}{Claim}
% \newtheorem*{note}{Note:}
% \newtheorem*{remark}{Remark}


\newcommand{\R}{\mathbb{R}}
\newcommand{\N}{\mathbb{N}}
% \newcommand{\F}{\mathbb{F}}
% \newcommand{\1}{\mathbbm{1}}
\newcommand{\Rn}{\mathbb{R}^n}
% \newcommand{\La}{\mathcal{L}}
% \newcommand{\D}{\mathcal{D}}
\newcommand{\Om}{\Omega}
\newcommand{\pa}{\partial}
\newcommand{\C}{\mathcal{C}}
\newcommand{\ph}{\varphi}

\setlength{\parindent}{0pt}

\begin{document}

\title{Entwurf MA - Flussschranken in Netzen ohne Kreisfluss}

\author{Tobias Buchwald}
%\maketitle

%\large{Bachelorarbeit bei Prof. Dr. Stefan Felsner}


% \begin{center} 
%TODO auf Masterarbeit anpassen
% \Huge{Domino Tilings auf dem Torus}\\ \vspace{12 cm}
% \Large{Bachelorarbeit\\ bei Prof. Dr. Stefan Felsner}\\ \vspace{1cm}
% \large{Vorgelegt von Tobias Buchwald}\\
% \large{am Fachbereich Mathematik der \\Technischen Universit�t Berlin}\\
% \vspace{2cm}
% \large{Berlin,  \today}
% 
% \end{center} 
% 
% 
% \textbf{Erkl\"arung}\\

Hiermit versichere ich an Eides statt, dass ich die vorliegende Masterarbeit selbst\"andig und eigenh\"andig sowie
ausschlie\ss lich unter Verwendung der aufgef\"uhrten Quellen und Hilfsmittel angefertigt habe. \\


Berlin, den \today
\newline

\rule[-0.2cm]{10cm}{0.5pt}

\textsl{Tobias Buchwald} 
% \newpage

\tableofcontents
\newpage
\section{Introduction} \subsection{Motivation and Outline}
Today more and more real-world problems in the areas of simulation and optimization are solved by mathematical and 
computational methods. A growing number of these problems can be solved without problems, i.e. even huge instances give 
an optimal or near optimal solution within seconds. Still, there remain problems that even on modern computers are hard 
to solve. For these problems it is important to find ways to increase the efficiency of the algorithms. 

The topic of this thesis arises from the computation of flow in natural gas networks, which is currently developed 
in the FORNE Project in a cooperation of OGE with universities and research insitutes including ZIB.
%TODO genaueres zu FORNE? genaueres zum aufbau des Gasnetzes?
The flow of natural gas in a network is described by nonlinear equations and depends on many parameters, which makes 
the problem hard to solve. If we can find good upper and lower bounds for the flow on an arc during the preprocessing, 
we can hope to improve the behavior of the nonlinear solver by giving these tigther bounds. 

The flow is induced by pressure differences, so in reality there can't be cyclic flow (if we exclude compressor 
stations). Without the condition of acyclic flow, it is sufficient to run a standard min-cost-flow algorithm where the 
maximized arc $e$ gets weight $w_e = -1$ and all others are 0. However, the arising bounds are far from optimal. If arc 
$e$ is contained in any cycle we could decrease the cost by pushing more and more flow around this cycle until the arcs 
capacity is at its limits.

This master thesis will deal with the problem of finding a network flow with no directed cycles (acyclic flow), which at 
the same time maximizes the amount of flow on a specified arc $e$ of the network. We will discuss the complexity, an 
exact algorithm based on a mixed integer program with separation of inequalities that forbid cycles and also a heuristic 
approach that yields results much faster (but not optimal).%TODO am ende genauer schreiben was wirklich gemacht wurde

% \subsection{The gas flow problem}
% Although it is mainly the motivation, not really the topic of this thesis, we want to briefly introduce the gas 
% transport problem. For more a detailed description we refer to LINK 
% %TODO link auf ein entsprechendes ZIB-paper? Nee, Jesco meinte soll nicht so rein 
% 

\newpage
\section{Definitions and Notation}% Als Referenz und Grundlage fuer die Definitionen nutze ich das Lehrbuch Combinatorial Optimization:Theory and 
%Algorithms von Bernhard Korte u Jens Vygen - definiere die wichtigsten Sachen aber erstmal auch selbst

Since there are many definitions, which may differ slightly, we want to introduce now the basic notation and 
definitions that we use throughout this thesis. The definitions in this chapter are mainly taken from the textbook 
about combinatorial optimization from Korte and Vygen \cite{KorteVygenCombOpt2007}.

An undirected graph is a triple (V, E, $\Psi$), where $V$ and $E$ are finite sets and
$\Psi: E\to \{X \subseteq V: |X| = 2\}$. 
A directed graph or digraph is a triple $(V, E, \Psi)$,
where $V$ and $E$ are finite sets and $\Psi : E \to \{(v, w) \in V \times V : v \neq w\}$. In this thesis by a
graph we mean normally the directed graph. If we talk about undirected graphs it will be stated 
explicitly. The elements of $V$ are called vertices, the elements of $E$ are the edges. Edges of undirected graphs can 
also be called arcs to make clear that they are directed.

Two edges $e, e'$ with $\Psi(e) = \Psi ( e')$ are called parallel. Graphs without parallel
edges are called simple. For simple graphs we usually identify an edge $e$ with its
image $\psi(e)$ and write $G = (V(G), E(G))$, where $E(G) \subseteq \{X \subseteq V(G) : |X| = 2\}$
or $E(G) \subseteq V(G) \times V(G)$. We often use this simpler notation even in the presence
of parallel edges, then the ``set'' $E (G)$ may contain several ``identical'' elements. In this thesis all graphs 
are considered simple if nothing different is said. %TODO rausnehmen falls gar keine parallelen Kanten gebraucht werden
$|E(G)|$ denotes the number of edges, and for two edge sets $E$ and $F$ we always
have $|E \cup F | = |E | + |F |$ even if parallel edges arise.

We say that an edge $e = \{v, w\}$ or $e = (v, w)$ joins $v$ and $w$. In this case, $v$ and $w$ are adjacent. $v$ is a 
neighbour of $w$ (and vice versa). $v$ and $w$ are the endpoints of $e$. If $v$ is an endpoint of an edge $e$, we say 
that $v$ is incident with $e$. 
In the directed case we say that $( v, w)$ leaves $v$ and enters $w$, $v$ is the tail and $w$ is the head of the arc 
$e$. Two edges which share at least one endpoint are called adjacent.

For a digraph $G$ we sometimes consider the underlying undirected graph, i.e. the undirected graph $G'$ on the same 
vertex set which contains an edge $\{v, w\}$
for each edge $(v, w)$ of $G$. We also say that $G$ is an orientation of $G'$.
A subgraph of a graph $G = (V(G), E(G))$ is a graph $H = (V(H), E(H))$
with $V(H) \subset V(G)$ and $E(H) \subset E(G)$. We also say that $G$ contains $H$. $H$ is an
induced subgraph of $G$ if it is a subgraph of $G$ and $E (H) = \{ \{x, y\} \textrm{ resp. } (x, y) \in
E(G) : x, y \in V(H)\}$. Here $H$ is the subgraph of $G$ induced by $V(H)$. We also
write $H = G[V(H)]$. A subgraph $H$ of $G$ is called spanning if $V(H) = V(G)$.
If $v \in V(G)$, we write $G- v$ for the subgraph of $G$ induced by $V(G) \setminus {v}$.
If $e \in E(G)$, we define $G- e := (V(G), E(G) \setminus \{e\})$. Furthermore, the addition
of a new edge $e$ is abbreviated by $G + e := (V(G), E(G) \cup {e})$. If $G$ and $H$
are two graphs, we denote by $G + H$ the graph with $V(G +H)= V(G) \cup V(H)$
and $E(G +H)$ being the disjoint union of $E(G)$ and $E(H)$ (parallel edges may arise).

For a graph $G$ and $X, Y\subseteq V(G)$ we define $E(X, Y) := \{\{x, y\} \in E(G) : x \in
X \setminus Y, y \in Y \setminus X\}$ resp. $E^+(X, Y) := \{(x, y) \in E(G) : x \in X\setminus Y, y \in Y \setminus 
X\}$.
For undirected graphs $G$ and $X \subseteq V(G)$ we define $\delta(X) := E(X, V(G) \setminus X)$. The
set of neighbours of $X$ is defined by $ \Gamma(X) := \{v \in V(G) \setminus X : E(X, \{v\})  \neq \emptyset\}$.
For digraphs $G$ and $X \subseteq V(G)$ we define $\delta^+(X) := E^+(x, V(G) \setminus X)$, $\delta^-(x) :=
\delta^+(V(G) \setminus X)$ and $\delta(X) := \delta^+(x) \cup \delta^-(x)$. We use subscripts (e.g. $\delta_G(X)$) to
specify the graph $G$ if necessary.

For singletons, i.e. one-element vertex sets $\{v\} (v \in V(G))$ we write $\delta(v) :=
\delta(\{v\})$, $\Gamma(v) := \Gamma(\{v\}), \delta^+(v) := \delta^+(\{v\})$ and $\delta^-(v) := \delta^-(\{v\})$. The 
degree of a vertex $v$ is $|\delta(v)|$, the number of edges incident to $v$. In the directed case, the
in-degree is $|\delta^-(v)|$, the out-degree is $|\delta^+(v)|$, and the degree is $|\delta^+(v)|+ |\delta^-(v)|$.
A vertex $v$ with zero degree is called isolated. A graph where all vertices have
degree $k$ is called $k$-regular.

An edge progression $W$ in $G$ is a sequence $v_1, e_1, v_2, \dots , v_k, e_k, v_{k+1}$ such that $k \ge 0$,
and $e_i = (v_i, v_{i+ 1}) \in E(G)$ resp. $e_i = \{v_i, v_{i+1}\}\in E(G)$ for $i = 1, \dots , k$. If in
addition $e_i \ne e_j \,\forall\, 1 \le i < j \le k$, $W$ is called a walk in $G$. $W$ is closed if
$v_1 = v_{k+1}$. A path is a graph $P = (\{v_1, ... , v_{k+1}\}, \{e_1, ... , e_k\})$ such that $v_i \ne v_j$ for
$1 \le i < j \le k + 1$ and the sequence $v_1 , e_1 , v_2, \dots , v_k, e_k, v_{k+1}$ is a walk. $P$ is
also called a path from $v_1$ to $v_{k+1}$ or a $v_1 - v_{k+1}$-path. $v_1$ and $v_{k+1}$ are the endpoints
of $P$. By $P_{[x,y]}$ with $x, y \in V(P)$ we mean the (unique) subgraph of $P$ which is
an $x-y$-path. Evidently, there is an edge progression from a vertex $v$ to another
vertex $w$ if and only if there is a $v-w$-path.

A cycle is a graph $(\{v_1, \dots , v_k\}, \{e_1, \dots, e_k\})$ such that the sequence $v_1, e_1, v_2, 
\dots , v_k,e_k,v_1$ is a (closed) walk and $v_i \ne v_j$ for $1 \le i < j\le k$.
An easy induction argument shows that the edge set of a closed walk can be
partitioned into edge sets of cycles.

The length of a path or cycle is the number of its edges. If it is a subgraph
of $G$, we speak of a path or cycle in $G$. A spanning path in $G$ is called a
Hamiltonian path while a spanning cycle in $G$ is called a Hamiltonian cycle
or a tour. A graph containing a Hamiltonian cycle is a Hamiltonian graph.
For two vertices $v$ and $w$ we write $dist(v, w)$ or $dist_G (v, w)$ for the length of
a shortest $v-w$-path (the distance from $v$ to $w$) in $G$. If there is no $v-w$-path at all,
i.e. $w$ is not reachable from $v$, we set $dist(v, w) := \inf$. In the undirected case,
$dist(v, w) = dist(w, v)$ for all $v, w \in V(G)$.

We shall often have a cost function $c : E(G) \to \R$. Then for $F \subseteq E(G)$ we
write $c(F) := \sum_{e\in F} c(e)$ (and $c(\emptyset) = 0$). This extends $c$ to a modular function
$c : 2^{E(G)}\to \R$. Moreover, $dist_{(G,c)}(v, w)$ denotes the minimum $c(E(P))$ over all
$v-w$-paths $P$ in $G$.

\newpage
\section{Heuristic Approach}\newpage
\subsection{A Simple Heuristic Approach}

As we have seen, the computations of the MIP formulation with separation of the Acyclicity Constraints could be 
expensive in running time and ressource consumption. The solved problem is a relaxation of the real-world 
problem anyway, so we can as well think 
about different relaxations and heuristic approaches. This chapter will introduce a simple and easy to 
implement heuristic approach inspired by the idea of the former section, where we showed why path augmentation models 
do not work for the acyclic flowbound problem. This approach uses a graph transformation and computes a Minimum Cost 
Flow on the transformed graph. We show that we get a valid bound for the Acyclic Flowbound Problem from this 
transformed graph.\\

The main idea of this heuristic approach is to avoid that the same flow is cycling over and over again until it reaches 
the capacity bounds. The path augmentation of the section before could achieve this goal. But we need to relax the 
capacity bounds and by this we relax also the problem. The proof that it suffices to double the capacities relies on 
the idea of distinguishing the flow before and after the maximized arc, because in every part of this the capacities 
have to be respected. If we double capacities and compute flow in this network it could happen that flow in one part 
uses 1.5 times of an arcs capacity and the other only half of it. So the bounds get in fact tighter if we separate the 
two parts.

In order to achieve this we make two copies of the original graph and make the arc we maximize the 
bridge between the two. On one part of the graph we only have sources, on the other part we only have sinks. 
In order to make the problem feasible we introduce artificial arcs between the sinks and their counterparts in the other 
copy of the graph. These artificial arcs get a detention cost so they are only used when there is no other way. On this 
modified graph we obtain minimum costs by sending as much flow as possible over the maximized arc. We can add the flow 
in both parts in the end, obtaining a flow for the original graph that might be not acyclic and not respecting the 
capacities but has maximum flow value on the arc we maximize. 

Let us describe the approach in a formal way.

\subsubsection*{The Graph Transformation}

The following algorithm \ref{algo:graphtransform} describes how the graph is transformed to the new graph. It mainly 
splits the graph into two copies $X$ and $Y$, sets costs and introduces links between both parts and the 
supersource and supersink.

%Pseudocode des Algorithmus
\begin{algorithm}
 \caption{graph transformation}
 \label{algo:graphtransform}
 \begin{algorithmic}[5]
  \Function{MakeTransformedGraph}{$G=(V,A), e\in A$}
  \State create empty graph $G':=(V',A'), V'=A'=\emptyset$
  \State create labels capacity $A'\to \R$, cost $A'\to \R$% , demand $V'\to\R$
  \State create $s, t\in V$ \Comment{ supersource and supersink}
  \State create $a_0 :=(s,t) \in A'$
  \State capacity$(a_0)\gets\infty$ 
  \State cost$(a_0)\gets 2$ \Comment{highest arc cost in the network}
  \ForAll{$v\in V$}\Comment{make two copies of each node}
    \State create $v_X, v_Y\in V'$
    \If{$v$ is source of $G$}
      \State create arc $a:=(s,v_X)\in A'$
      \State capacity$(a)\gets$ supply$(v)$
    \ElsIf{$v$ is sink of $G$}
      \State create arc $a:=(v_X, t)\in A'$
      \State create arc $b:=(v_Y,t)\in A'$
      \State cost$(a)\gets 1$
    \EndIf
  \EndFor
  \ForAll{$a=(u,v)\in A$}
    \If{$a=e$}\Comment{for the arc to maximize we only make a bridge, no copy}
      \State create arc $e':=(u_X, v_Y)\in A'$
      \State capacity$(e')\gets$ capacity$(e)$
    \Else
      \State create arcs $a_X:=(u_X, v_X), a_Y:=(u_Y, v_Y)\in A'$
      \State capacity$(a_X)\gets$ capacity$(a)$, capacity$(a_Y)\gets$ capacity$(a)$
      \State cost$(a_X)\gets 0$, cost$(a_Y)\gets0$
    \EndIf
  \EndFor
  \State \Return $G'$
  \EndFunction
 \end{algorithmic}

\end{algorithm}

Figure \ref{bild:graphtransform} shows as an example how the graph from the last section would be transformed by this 
algorithm

\begin{figure}[h!]
\centering
\begin{tikzpicture}[scale=0.6]
\node (a) at (0,3)[vertex]{};
\node (s1) at (1,2.5)[source]{-1};
\node (b) at (2,3)[vertex]{};
\node (d) at (2,2)[vertex]{};
\node (c) at (1.5,0)[vertex]{};
\node (s2) at (3,3)[source]{-1};
\node (t) at (1.5,1)[sink]{2};
% \draw[arcflow, cyan, dashed] (s1)--(d)--(b);
% \draw[arcflow, cyan](b)--(a)--(c)--(t);
% \draw[arcflow, yellow, dashed] (s2)--(b)--(a)--(c)--(t);
\draw[fatarc] (b)--(a)node[pos=0.5, above]{$e$};
\draw[edge] (a)--(s1){};
\draw[edge] (s1)--(d){};
\draw[edge] (d)--(b){};
\draw[edge] (b)--(s2){};
\draw[edge] (s2)--(c){};
\draw[edge] (c)--(a){};
\draw[edge] (c)--(t){};
\draw[edge] (t)--(d){};

%supersource und supersink
\node(sups) at (0,-2.5)[source]{-2};
\node(supt) at (7.5,-2.5)[sink]{2};

\node (la) at (0,-1)[vertex]{};
\node (ls1) at (1,-1.5)[source]{};
\node (lb) at (2,-1)[vertex]{};
\node (ld) at (2,-2)[vertex]{};
\node (lc) at (1.5,-4)[vertex]{};
\node (ls2) at (3,-1)[source]{};
\node (lt) at (1.5,-3)[sink]{};
% \draw[fatarc] (lb)--(la)node[pos=0.5, above]{$e$};
\draw[edge] (la)--(ls1){};
\draw[edge] (ls1)--(ld){};
\draw[edge] (ld)--(lb){};
\draw[edge] (lb)--(ls2){};
\draw[edge] (ls2)--(lc){};
\draw[edge] (lc)--(la){};
\draw[edge] (lc)--(lt){};
\draw[edge] (lt)--(ld){};

\node (ra) at (4,-1)[vertex]{};
\node (rs1) at (5,-1.5)[source]{};
\node (rb) at (6,-1)[vertex]{};
\node (rd) at (6,-2)[vertex]{};
\node (rc) at (5.5,-4)[vertex]{};
\node (rs2) at (7,-1)[source]{};
\node (rt) at (5.5,-3)[sink]{};
\draw[fatarc, bend left] (lb)to(ra);
\draw[edge] (ra)--(rs1){};
\draw[edge] (rs1)--(rd){};
\draw[edge] (rd)--(rb){};
\draw[edge] (rb)--(rs2){};
\draw[edge] (rs2)--(rc){};
\draw[edge] (rc)--(ra){};
\draw[edge] (rc)--(rt){};
\draw[edge] (rt)--(rd){};
\draw[arc, green!70!blue] (sups)--(ls1)node[pos=0.5, left]{$cap=1$};
\draw[arc, green!70!blue] (sups)--(ls2)node[pos=0.5, above]{$cap=1$};
\draw[arc, red] (rt)--(supt)node[pos=0.5, below]{$cap=2,\, cost=0$};
\draw[arc, red] (lt)--(supt)node[pos=0.5, above]{$cap=2,\, cost=1$};
\draw[arc, purple!40!blue, bend right =90] (sups)to(supt)node at (4,-5){$cap=\infty,\,cost=2$};

\end{tikzpicture}
\caption{The transformation of the example network from the former section}
 \label{bild:graphtransform}
\end{figure}


With this transformation and any algorithm for Minimum Cost Flow we can set up an algorithm for our problem. 
For the Maximum Flow and Minimum Cost Flow Problem in graphs there are many standard algorithms we could use. The 
classical Maximum Flow algorithm of Ford and Fulkerson \cite{Ford-Fulkerson_algo} arising from their Max-Flow-Min-Cut 
theorem is based on augmenting flow on source-sink-paths in the network as well as the improved algorithms of Edmonds 
and Karp \cite{EdmondsKarp1972} or Dinic \cite{Dinic1970}. Algorithms for Maximum Flow can be used to first check if 
there is any feasible flow in the network. If there is no feasible flow we can give up at this point, while the flow 
problem on the transformed graph is always feasible by construction.

Deriving from the Max-Flow Algorithms there are many algorithms solving the Min-Cost-Flow Problem. Edmonds and 
Karp described a Successive Shortest Path Algorithm in \cite{EdmondsKarp1972}. Goldberg and Tarjan proposed the Minimum 
Mean Cycle Cancelling Algorithm \cite{minMeanCycleCancelling89} with runnning time of $\mathcal{O}(nm(log 
n)\min\{log(nC), m log n\}) $. %TODO fuer die anderen ebenfalls laufzeiten odr fuer keine
Another algorithm is the Network Simplex by Orlin \cite{NetworkSimplexOrlin97} (which was used in the computational 
study) There are a lot more algorithms available for Minimum Cost Flow.

Our algorithm mainly consists of a feasibility test, a transformation of the graph $G$ and a Minimum Cost Flow 
computation on the transformed graph $G'$, see algorithm \ref{algo:simpleheur}. 

\begin{algorithm}
 \caption{simple heuristic}
 \label{algo:simpleheur}
 \begin{algorithmic}
  \Function{FlowBoundHeur}{$G=(V,A), e\in A$}
  \State mf$\gets$ \Call{MaxFlow}{G}
  \If{mf < nominated ingoing flow}
    \State\Return infeasible
  \EndIf
  \State $G'=$ \Call{MakeTransformedGraph}{$G,e$}
  \State $f\gets$ \Call{MinCostFlow}{$G'$}
  \State ub$\gets f(e')$
  \State backwardflow$\gets f(a_0)$
  \State \Return ub - backwardflow
  \EndFunction
 \end{algorithmic}
\end{algorithm}

We show the correctness of the algorithm: %TODO and running time? dependent on used mincostflow!

\begin{figure}[h!]
\centering
\begin{tikzpicture}[scale=0.6]
%supersource und supersink
\node(sups) at (0,-2.5)[source]{-2};
\node(supt) at (7.5,-2.5)[sink]{2};
%left part
\node (la) at (0,-1)[vertex]{};
\node (ls1) at (1,-1.5)[source]{};
\node (lb) at (2,-1)[vertex]{};
\node (ld) at (2,-2)[vertex]{};
\node (lc) at (1.5,-4)[vertex]{};
\node (ls2) at (3,-1)[source]{};
\node (lt) at (1.5,-3)[sink]{};
%right part
\node (ra) at (4,-1)[vertex]{};
\node (rs1) at (5,-1.5)[source]{};
\node (rb) at (6,-1)[vertex]{};
\node (rd) at (6,-2)[vertex]{};
\node (rc) at (5.5,-4)[vertex]{};
\node (rs2) at (7,-1)[source]{};
\node (rt) at (5.5,-3)[sink]{};

\draw[arcflow, cyan] (sups)--(ls2)--(lb);
\draw[arcflow, cyan, bend left] (lb)to (ra);
\draw[arcflow, cyan](ra)--(rs1)--(rd)--(rt)--(supt);
\draw[arcflow, yellow!80!orange, dashed](sups)--(ls1)--(ld)--(lb);
\draw[arcflow, yellow!80!orange, dashed, bend left](lb)to(ra);
\draw[arcflow, yellow!80!orange, dashed](ra)--(rc)--(rt)--(supt);

\draw[edge] (la)--(ls1){};
\draw[edge] (ls1)--(ld){};
\draw[edge] (ld)--(lb){};
\draw[edge] (lb)--(ls2){};
\draw[edge] (ls2)--(lc){};
\draw[edge] (lc)--(la){};
\draw[edge] (lc)--(lt){};
\draw[edge] (lt)--(ld){};
\draw[fatarc, bend left] (lb)to(ra);
\draw[edge] (ra)--(rs1){};
\draw[edge] (rs1)--(rd){};
\draw[edge] (rd)--(rb){};
\draw[edge] (rb)--(rs2){};
\draw[edge] (rs2)--(rc){};
\draw[edge] (rc)--(ra){};
\draw[edge] (rc)--(rt){};
\draw[edge] (rt)--(rd){};
\draw[arc, green!70!blue] (sups)--(ls1)node[pos=0.5, left]{$cap=1$};
\draw[arc, green!70!blue] (sups)--(ls2)node[pos=0.5, right]{$cap=1$};
\draw[arc, red] (rt)--(supt)node[pos=0.5, below]{$cap=2,\, cost=0$};
\draw[arc, red] (lt)--(supt)node[pos=0.5, above]{$cap=2,\, cost=1$};
\draw[arc, purple!40!blue, bend right =90] (sups)to(supt)node at (4,-5){$cap=\infty,\,cost=2$};
\end{tikzpicture}
\caption{This flow has zero costs, is feasible in the heuristic and has the optimum value $f(e)=2$}
 \label{bild:graphtransformWithFlow}
\end{figure}

\begin{prop}
 The heuristic algorithm \ref{algo:simpleheur} returns a valid upper bound for the possible acyclic flow on a given 
 arc $e$ in the network, or infeasible if there is no feasible network flow for the given flow balances at sources and 
 sinks.
\end{prop}

\begin{proof}
\textbf{Feasibility Test: }We have to show that the ordinary feasibility test with maximum flow is sufficient even for 
acyclic flow: If the flow network nomination (in- and outflow) is infeasible, no algorithm for the 
maximum flow problem will find a flow that is fulfilling these balances completely. Then also for the stricter acyclic 
flow problem this is infeasible. The other direction is also true: if there is a feasible flow, there also is a 
feasible acyclic flow:

Take the maximum flow and choose one flow cycle within (If there is none we are done). On all arcs of the cycle the 
flow is going in the same direction in relation to the cycle. We can choose the minimum flow on the cycles arcs and 
reduce flow on each arc of the cycle by this amount. The resulting flow is still a feasible flow, but we removed the 
flow cycle. This can be done until the flow is acyclic. But if there is any acyclic flow, there also has to be an 
acyclic flow that is maximum on the arc we are interested in. So in order to test feasibility it is enough to run a 
ordinary maximum flow algorithm.\\


\textbf{Relaxation: }To prove that we get an upper bound with this heuristic we have to show that the result we 
get is the maximum value of a relaxation of the problem. 

If a flow is acyclic we can decompose it into flow on a set of simple paths. This means that 
the Acyclic Flowbound Problem could theoretically be solved by an algorithm augmenting simple paths (in the 
right order) where we put some extra constraints and objectives on these paths. These extra constraints have to forbid 
the closing of any flow cycle by the path as well as forcing the path to use arc $e$ if possible. (Section 
\ref{model:pathaugment} describes this idea and shows the difficulties of choosing the right path for augmentation.)
%TODO koennte das modell einzeln hinschreiben mit 
%hinweis auf NP schwere von constrained shortest path. anschließend kommt das relaxierungsargument besser !!!
%Außerdem hätte ich damit shconmal ein gutes modell

The heuristic algorithm is a relaxation of this constrained path augmentation model. It still forces all paths to go 
over $e$ if possible (which might be much more since it is a problem with weaker conditions). 
We can reconstruct a flow in the original graph from the flow in the transformed graph by summing up the flow values on 
both copied arcs, assigning this flow to the original arc: $f(a)=f'(a_X)+f'(a_Y)$ and the assigned flow value of the 
bridge arc to the maximized arc of the original problem. 

All the flow conservation constraints still hold after constructing the flow in the original graph from the two copies:
\begin{align*}
& \sum_{a\in \delta^+(v)}q_a - \sum_{a\in\delta^- (v)}q_a &=& d_v\ &\forall v\in V_X \\
+& \sum_{a\in \delta^+(v)}q_a - \sum_{a\in\delta^- (v)}q_a &=& d_v\ &\forall v\in V_Y \\
=& \sum_{a\in \delta^+(v_X)\cup\delta^+(v_Y)}q_a - \sum_{a\in\delta^- (v_X)\cup\delta^-(v_Y)}q_a &=& d_v\ &\forall 
v_X\in V_X, v_Y\in V_Y \\
=& \sum_{a\in \delta^+(v)}q_a - \sum_{a\in\delta^- (v)}q_a &=& d_v\ &\forall v\in V \\
\end{align*}

The flow balances at sources and sinks are set right by construction.\\

When no more flow can be sent over the maximized arc $e$ the path augmentation model already gives a valid bound 
if the flow is feasible. But in fact this model can determine the amount of flow that is necessary to augment backwards 
on $e$ in a path model.

We distinguish whether the flow is going over the bridge $e$, going to a sink node in already in part $X$ of the 
transformed graph and thus having costs of 1 per unit, 
or is going from $s$ to $t$ directly with a cost of 2 per unit. The flow with costs of 2 cannot be sent over $e$ in 
forward direction neither find any sink in part $X$ of the graph. If there is a feasible flow in $G$ (which we tested) 
the only possible way is that the remaining flow has to go backward over $e$. So we can reduce the flow on the 
bridge by this amount of flow directly going from $s$ to $t$ and the bound is still valid. (In algorithm 
\ref{algo:pathHeur} from the section before this would be the part in line \ref{heur:lineAugCase3} where flow is 
augmented backwards).\\

$\Rightarrow$ Since the problem is just a relaxation of the original problem/the constrained simple path model, the 
obtained bound is always weaker (in this case $\ge$) than the optimal bound for maximum acyclic flow. 
$\Rightarrow$ the heuristic is correct for finding upper bounds for the acyclic flow on an arc.

\end{proof}%TODO der beweis ist lang und unuebersichtlich, man koennte fragen wozu man manche erklaerungen braucht.

% So the algorithm relaxes the capacity constraints and allows weaker kinds of cycling. It still maximizes the amount 
of 
% flow going over the maximized arc $e$ (the bridge): if there is any flow that can be sent over $e$ this is the 
cheapest 
% way. All flows that are not passing $e$ at some point have to use the arcs from part $X$ of the transformed graph 
% directly to the super-sink $t$ (which causes cost $=1$ per flow unit) or even worse from the super-source $s$ 
directly 
% to the super-sink $t$ (which causes cost $=2$ per flow unit). To go over $e$ has no cost at all, therefore 
% the paths with positive costs are only used if there is a directed minimal cut from part $X$ to part $Y$ where no arc 
% has free capacity.

So we have seen that we can compute an upper bound via the optimum for this relaxed problem with a fast standard 
algorithm for maximum flows that can be retranslated into the original problem. 
However, the capacity constraints of the original problem might be violated after retranslation. We set the arcs 
capacities in part $X$ and part $Y$ both to the value of the original capacities. Thus it might happen that actual flow 
value in the original graph sums up to a higher value than the capacity allows (up to twice the capacity). 


\begin{figure}[h!]
\centering
\begin{tikzpicture}
\node (a) at (0,-6)[vertex]{};
\node (s1) at (1,-6.5)[source]{-1};
\node (b) at (2,-6)[vertex]{};
\node (d) at (2,-7)[vertex]{};
\node (c) at (1.5,-9)[vertex]{};
\node (s2) at (3,-6)[source]{-1};
\node (t) at (1.5,-8)[sink]{2};
% \draw[arcflow, cyan, dashed] (s1)--(d)--(b);
\draw[arcflow, cyan](s2)--(b);
\draw[arcflow, cyan](b)--(a)--(s1)--(d);
\draw[arcflow, cyan](d)--(t);
\draw[arcflow, yellow!70!orange, dashed] (s1)--(d);
\draw[arcflow, yellow!70!orange, dashed] (d)--(b)--(a)--(c);
\draw[arcflow, yellow!70!orange, dashed] (c)--(t);
\draw[fatarc] (b)--(a)node[pos=0.5, above]{$e$};
\draw[edge] (a)--(s1){};
\draw[edge] (s1)--(d)node(errpos)[pos=0.5]{};
\draw[edge] (d)--(b){};
\draw[edge] (b)--(s2){};
\draw[edge] (s2)--(c){};
\draw[edge] (c)--(a){};
\draw[edge] (c)--(t){};
\draw[edge] (t)--(d){};
\node (err) at (-1,-7)[ellipse, red, draw=red]{ $f(a)=2>cap(a)=1$};
\draw[->, stealth, semithick, bend right,red](err)to(errpos);
\draw[ultra thick, ->, red!50] (1.55,-6.2) arc (30:330:0.2);
% \draw[pattern=dots, pattern color=red](a)--(s1)--(b);%(0,-6)--(1,-6)--(1,-5);
%TODO man könnte doch kreisfluss umschlossenes gebiet rot füllen?
\end{tikzpicture}
\caption{If we retransform the flow from figure \ref{bild:graphtransformWithFlow} (which is feasible in the heuristic) 
to the original network we get a violated capacity constraint on the arc with $c=1$. The flow is not acyclic anymore 
either.}
 \label{bild:graphretransformWithFlow}
\end{figure}
Also we cannot guarantee the acyclicity anymore. It might happen that we have cyclic flow in our 
original graph $G$ even if the flow in the transformed graph $G'$ was acyclic. This happens when paths from 
the different parts $X$ and $Y$ of $G'$ are crossing after they are brought back to the original graph $G$ to 
build a solution there. For an example see figure \ref{bild:graphretransformWithFlow} which is the retransformation of 
the example flow in figure \ref{bild:graphtransformWithFlow}.
% \newpage
% \subsection{Einschr\"ankung der ben\"otigten Kreisklassen} 
Wie wir nun wissen, kann man nicht unbedingt erwarten mit wenigen Kreisklassen den Flipzusammenhang wieder herzustellen. %TODO Es waere gut nen Beweis zu haben dass es nicht mit konstant vielen Kreisklassen geht...
Allerdings wissen wir auch, dass sich bestimmte Kreisflips durch mehrere andere erzeugen lassen. Wir betrachten deshalb die Kombination von Faces und anderen Kreisklassen.
 Dabei ist das, was in diesem Kapitel zu finden ist, nur ein Anfang, in dem per Hand einige Kreise aus anderen konstruiert werden. W\"unschenswert w\"are nat\"urlich eher ein Ergebnis, dass allgemein angibt wie viele Kreise man sparen kann wenn man eine bestimmte Menge von Kreisen zu flippen erlaubt - leider ist mir so etwas nicht gelungen. Au�erdem ist zu beachten, dass nicht klar ist, ob eine Menge von Kreisen mit denen man den Flipzusammenhang nachweisbar nicht herstellen kann auch bedeutet, dass man dies nicht doch mit einer kleinen Menge von Kreisen bewerkstelligen kann.\\

Face-Flips sind sehr nat\"urlich und erm\"oglichen uns bereits eine sehr gro\ss e Anzahl von Kreisklassen zu flippen, au�erdem ein nat�rlicher Teil der Kreisraumbasis. Deshalb werden wir immer Face-Flips zulassen. Da diese wie gezeigt nicht ausreichen, brauchen wir mindestens eine weitere Klasse von Kreisen deren Flips wir erlauben. Das einfachste Beispiel hierf�r sind die Kreise, die ohne einen Knick immer geradeaus gehen, aus Symmetriegr\"unden nehmen wir gleich sowohl die horizontalen als auch die vertikalen Kreise.

\begin{definition}
Die Menge der Knoten $(x,y)$, die im torischen Gittergraphen der $(3,1)$-Orientierung die jeweils gleiche Koordinate $x$ oder die gleiche Koordinate $y$ haben, nennen wir horizontale oder vertikale Achse des Graphen. Ein Kreis in diesem Graphen hei�t gerade, wenn alle seine Knoten auf der gleichen Achse liegen.\\

Als Beule (bez�glich einer Achse) bezeichnen wir einen Pfad $P$, so dass $P$ nur Start- und Endknoten auf der Achse hat und mit Kanten des geraden Kreises dieser Achse zu einem einfachen nullhomologen Kreis erweitert werden kann. Die L�nge der Beule ist dabei der maximale Abstand eines Punktes auf $P$ zur Achse bez�glich der Seite, zu der $P$ die Achse verl�sst und betritt. Die Breite ist die Anzahl der Kanten, die auf dem geraden Achsenkreis ben�tigt werden, um $P$ zu einem einfachen nullhomologen Kreis zu erweitern.
\end{definition}

%Bild?
Was bringt uns das nun? Wir werden sehen, dass sich Kreise mit schmalen Beulen sich durch gerade und Face-Kreise erzeugen lassen. 
%[schon oben definiert]Eine Ausst�lpung des Kreises ist dabei ein Teil des Kreises, der von der Hauptrichtung zu einer Seite des Kreises abzweigt und auf der gleichen Seite wieder in den Hauptkreis einm\"undet. 

Betrachte als Hauptkreis einen geraden Kreis mit einer Beule der Breite 1, der gerichtet ist, also theoretisch im ganzen flipbar w\"are. Im folgenden Bild sieht man sehr gut, wie leicht sich diese aus Facekreisen und dem Hauptkreis zusammensetzen l�sst. 
\begin{figure}[h!]
  \centering
  \scalebox{1}{\input{Bilder/beule1.pstex_t}}
\caption{Beispiel wie eine Ausst�lpung durch eine Kombination mit nullhomologen Kreisen geflippt wird}
\end{figure}

Gleiches funktioniert nicht nur in diesem Beispiel, sondern f�r Beulen der Breite 1 bei beliebiger L�nge: die Beule an sich ist ja ein nullhomologer Kreis, der bis auf die mit dem Hauptkreis geteilte Kante gerichtet sein muss. Diese Kante kann wenn sie entgegen den anderen gerichtet ist durch das flippen des Hauptkreises umgedreht werden, andernfalls ist sie f�r den Hauptkreis ungerichtet und wird durch den Flip des nullhomologen Kreises gedreht. Dabei muss man nat�rlich die Reihenfolge beachten, das ganze ist in Abbildung \ref{beule2}
schematisch dargestellt.\\

\begin{figure}[h!]

  \centering
  \scalebox{1}{\input{Bilder/beule2.pstex_t}}
\caption{Unterschiedliche Reihenfolge des der Flips, abh�ngig von der Richtung der (roten) Zwischenkante}
\label{beule2}
\end{figure}

Dieses Vorgehen wird schwieriger, wenn die Breite der Beulen zunimmt. Die hier rot gemalte Zwischenkante wird zu einem Pfad von mehreren Zwischenkanten. Diese sind dann aber m�glicherweise nicht gerichtet!  Abbildung \ref{2erbeule}
verdeutlicht das Problem:\\

\begin{figure}[h!]
  \centering
  \scalebox{1}{\input{Bilder/2erbeule.pstex_t}}
\caption{Bei dieser $2\times 1$ Beule ist es nicht mehr direkt m�glich, entlang des schwarz eingezeichneten Kreises einen Flip aus dem geraden Kreis und einem nullhomologen Kreis zusammenzusetzen. Im Domino Tiling (rechte Darstellung) erkennt man, dass es nicht mal klar ist ob �berhaupt Face-Flips m�glich sind.}
\label{2erbeule}
\end{figure}


Wie es scheint k�nnen wir bei dieser Form von Beulen zumindest mit der obigen Methode nichts erreichen. Die gute Nachricht ist aber: Wir sind nicht bei allen gr��eren Ausst�lpungen verloren. H�ufig ist es m�glich, zumindest einen Teil des Kreises �ber Faceflips zu drehen. % und das was �brigbleibt ist dann eben eine solche $2\times 1$ Beule.? ->NEIN
Betrachte die Situation in einer Ecke (Abb \ref{abkuerzung}) 


\begin{figure}[h!]
  \centering
  \scalebox{1}{\input{Bilder/beule3.pstex_t}}
\caption{Eine der beiden gr�nen Kanten muss eine eingehende Kante in den gemeinsamen Knoten sein. Alle nichtgr�nen Kanten folgen zwangsweise aus der Form des Pfades. Je nachdem wie man die Kantenrichtung w�hlt, erh�lt man entweder eine Abk�rzung des gro�en Kreises, oder man erh�lt einen kleinen nullhomologen Kreis aus zwei Faces den man flippen kann. In beiden F�llen reduziert sich die Gr��e des ben�tigten Kreises.}
\label{abkuerzung}
\end{figure}

Analog dazu ist das Bild, wenn der Eckknoten 3 eingehende und eine ausgehende Kante hat, nur dass die Wege dann jeweils in die andere Richtung laufen. Es gen�gt also, einen kleineren Kreis zuzulassen um den gr��eren zu erzeugen. Dieses Argument kann man nun induktiv immer weiter fortf�hren, da in der Regel auch der entsprechende kleinere Kreis eine solche Ecke besitzt. 
Wie weit kann man die Gr��e der Beulen so einschr�nken? Leider kann das Domino Tiling durchaus so aussehen, dass ein gro�er Teil der Beule �brig bleibt. Wir k�nnen also damit nur Ecken ausschlie�en, die aus "' hoch, hoch, rechts, rechts"' oder entsprechend gedrehten/gespiegelten Kanten bestehen.\\ Betrachte dazu Abbildung \ref{problembeule}

\begin{figure}[h!]
  \centering
  \scalebox{1}{\input{Bilder/beule4.pstex_t}}
\caption{Beispiel einer Beule, in der man keinen Face-Flip mehr anwenden kann um die Gr��e zu reduzieren}
\label{problembeule}
\end{figure}

sie zeigt ein Beispiel an dem keine solchen Ecken mehr vorhanden sind. Mit dem geraden Kreis als Grundkreis sind wir an dieser Stelle also auch am Ende, da wir im Gegensatz zu den Beulen der Breite 1 diese breiteren nicht durch Facekreise zusammensetzen k�nnen. Bei Breite 2 kommt es bereits auf die L�nge an: ist die L�nge gerade, dann lassen sie sich ebenso wie Beulen der Breite 1 durch Facekreise und den Grundkreis zusammensetzen, ist die L�nge ungerade funktioniert das allerdings auch schon nicht mehr.


%%% FAZIT
\section{Fazit}

Wie wir gesehen haben, lassen sich die positiven Resultate aus dem planaren Fall hier nicht direkt auf den Fall der torischen Domino Tilings erweitern. Insbesondere steigt die Zahl der ben�tigten Kreisklassen von einer (Face-Kreise) auf m�glicherweise beliebig viele abh�ngig von der Gittergr��e. Dadurch ist insbesondere nicht klar, wie man m�glichst g�nstig Flipzusammenhang konstruieren kann. Gleichzeitig konnte ich nicht zeigen, dass eine gro�e Menge ung�nstiger Kreisklassen auch implizieren, dass es keine kleinen Mengen gibt die ausreichend sind. Wir haben im letzten Kapitel gesehen, dass man durchaus Kreise aus anderen zusammensetzen kann, zumindest f�r den Fall von Faces und einem Grundkreis der nicht nullhomolog ist. 

Eine interessante Fragestellung w�re nun zum Beispiel, ob durch das Zusammenspiel mehrerer nicht nullhomologer Kreise mit Facekreisen deutlich mehr zu erreichen ist. Auf allgemeinen $\alpha$-Orientierungen stimmt es leider auch nicht, dass alle nullhomologen Kreise sich immer durch Faces zusammensetzen lassen, es w�re interessant zu sehen ob das auch f�r die Orientierungen gilt, die durch Domino Tilings induziert werden. %TODO geht da nicht mehr???
Ebenfalls konnte ich f�r aus Domino Tilings entstandene Graphen nicht zeigen, ob die Bedingung aus Lemma \ref{notwBed} auch hinreichend ist. 

\subsection*{Dank}

An Kolja Knauer, der sich pers�nlich wie per Mail die Zeit genommen hat um mich zu betreuen, sich immer wieder meine Ideen angeh�rt und dazu wertvolle Ideen entwickelt und mir Hinweise gegeben hat, insbesondere Lemma \ref{notwBed} betreffend - Danke!

%TODO hier weitergehen und genau sagen, wie weit die Gr��e sich reduziert!!!



% \newpage
% \newpage
% \addcontentsline{toc}{section}{Literaturverzeichnis}
% \bibliographystyle{amsplain}
% \bibliography{literatur}
%\nocite{*}

\end{document}
