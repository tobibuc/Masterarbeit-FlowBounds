\documentclass[a4paper]{article}

\usepackage[T1]{fontenc}
\usepackage[utf8]{inputenc}
\usepackage[ngerman]{babel}
\usepackage{times, graphicx, currvita, hyperref, longtable, float, caption, subcaption}
\usepackage{amsmath,amsfonts,amssymb, amsthm}%, dsfont, bbm, mathtools, stmaryrd}
%\usepackage{fancyhdr}
%\usepackage{color}
%\usepackage[english]{babel}
\usepackage{paralist}
%\usepackage{algorithmic}
\usepackage{wasysym}	% verschiedene Symbole, siehe http://rpi.edu/dept/arc/training/latex/LaTeX_symbols.pdf
\graphicspath{{Bilder/}}

\theoremstyle{definition}
\newtheorem{definition}{Definition}[section]
\newtheorem*{Def1}{Definition}
\newtheorem*{ex}{Example}

\newtheoremstyle{Tobi}{10pt}{}{}{}{\bf}{ }{\newline}{}
% die parameter sind: {name}{platz drueber}{platz drunter}{schriftart text}{einrueckung}{schriftart kopf}{Punktierung des Kopfes}{Platz zwischen kopf und text}

\theoremstyle{Tobi}
\newtheorem{prop}[definition]{Proposition}
\newtheorem*{Pro}{Proposition}
\newtheorem{theorem}[definition]{Satz}
%\newtheorem*{le}{Lemma}
\newtheorem{lemma}[definition]{Lemma}
\newtheorem{cor}[definition]{Korollar}
\newtheorem*{conj}{Conjecture}
\newtheorem{obs}[definition]{Beobachtung}

% \theoremstyle{remark}
% \newtheorem*{que}{Questions}
% \newtheorem*{claim}{Claim}
% \newtheorem*{note}{Note:}
% \newtheorem*{remark}{Remark}


\newcommand{\R}{\mathbb{R}}
\newcommand{\N}{\mathbb{N}}
% \newcommand{\F}{\mathbb{F}}
% \newcommand{\1}{\mathbbm{1}}
\newcommand{\Rn}{\mathbb{R}^n}
% \newcommand{\La}{\mathcal{L}}
% \newcommand{\D}{\mathcal{D}}
\newcommand{\Om}{\Omega}
\newcommand{\pa}{\partial}
\newcommand{\C}{\mathcal{C}}
\newcommand{\ph}{\varphi}

\setlength{\parindent}{0pt}

\begin{document}

\title{Entwurf MA - Flussschranken in Netzen ohne Kreisfluss}

\author{Tobias Buchwald}
%\maketitle

%\large{Bachelorarbeit bei Prof. Dr. Stefan Felsner}


% \begin{center} 
%TODO auf Masterarbeit anpassen
% \Huge{Domino Tilings auf dem Torus}\\ \vspace{12 cm}
% \Large{Bachelorarbeit\\ bei Prof. Dr. Stefan Felsner}\\ \vspace{1cm}
% \large{Vorgelegt von Tobias Buchwald}\\
% \large{am Fachbereich Mathematik der \\Technischen Universit�t Berlin}\\
% \vspace{2cm}
% \large{Berlin,  \today}
% 
% \end{center} 
% 
% 
% \textbf{Erkl\"arung}\\

Hiermit versichere ich an Eides statt, dass ich die vorliegende Masterarbeit selbst\"andig und eigenh\"andig sowie
ausschlie\ss lich unter Verwendung der aufgef\"uhrten Quellen und Hilfsmittel angefertigt habe. \\


Berlin, den \today
\newline

\rule[-0.2cm]{10cm}{0.5pt}

\textsl{Tobias Buchwald} 
% \newpage

\tableofcontents
\newpage
\section{Introduction} \subsection{Motivation and Outline}
Today more and more real-world problems in the areas of simulation and optimization are solved by mathematical and 
computational methods. A growing number of these problems can be solved without problems, i.e. even huge instances give 
an optimal or near optimal solution within seconds. Still, there remain problems that even on modern computers are hard 
to solve. For these problems it is important to find ways to increase the efficiency of the algorithms. 

The topic of this thesis arises from the computation of flow in natural gas networks, which is currently developed 
in the FORNE Project in a cooperation of OGE with universities and research insitutes including ZIB.
%TODO genaueres zu FORNE? genaueres zum aufbau des Gasnetzes?
The flow of natural gas in a network is described by nonlinear equations and depends on many parameters, which makes 
the problem hard to solve. If we can find good upper and lower bounds for the flow on an arc during the preprocessing, 
we can hope to improve the behavior of the nonlinear solver by giving these tigther bounds. 

The flow is induced by pressure differences, so in reality there can't be cyclic flow (if we exclude compressor 
stations). Without the condition of acyclic flow, it is sufficient to run a standard min-cost-flow algorithm where the 
maximized arc $e$ gets weight $w_e = -1$ and all others are 0. However, the arising bounds are far from optimal. If arc 
$e$ is contained in any cycle we could decrease the cost by pushing more and more flow around this cycle until the arcs 
capacity is at its limits.

This master thesis will deal with the problem of finding a network flow with no directed cycles (acyclic flow), which at 
the same time maximizes the amount of flow on a specified arc $e$ of the network. We will discuss the complexity, an 
exact algorithm based on a mixed integer program with separation of inequalities that forbid cycles and also a heuristic 
approach that yields results much faster (but not optimal).%TODO am ende genauer schreiben was wirklich gemacht wurde

% \subsection{The gas flow problem}
% Although it is mainly the motivation, not really the topic of this thesis, we want to briefly introduce the gas 
% transport problem. For more a detailed description we refer to LINK 
% %TODO link auf ein entsprechendes ZIB-paper? Nee, Jesco meinte soll nicht so rein 
% 

\newpage
<<<<<<< HEAD
\section{Definitions and Notation}% Als Referenz und Grundlage fuer die Definitionen nutze ich das Lehrbuch Combinatorial Optimization:Theory and 
%Algorithms von Bernhard Korte u Jens Vygen - definiere die wichtigsten Sachen aber erstmal auch selbst

Since there are many definitions, which may differ slightly, we want to introduce now the basic notation and 
definitions that we use throughout this thesis. The definitions in this chapter are mainly taken from the textbook 
about combinatorial optimization from Korte and Vygen \cite{KorteVygenCombOpt2007}.

An undirected graph is a triple (V, E, $\Psi$), where $V$ and $E$ are finite sets and
$\Psi: E\to \{X \subseteq V: |X| = 2\}$. 
A directed graph or digraph is a triple $(V, E, \Psi)$,
where $V$ and $E$ are finite sets and $\Psi : E \to \{(v, w) \in V \times V : v \neq w\}$. In this thesis by a
graph we mean normally the directed graph. If we talk about undirected graphs it will be stated 
explicitly. The elements of $V$ are called vertices, the elements of $E$ are the edges. Edges of undirected graphs can 
also be called arcs to make clear that they are directed.

Two edges $e, e'$ with $\Psi(e) = \Psi ( e')$ are called parallel. Graphs without parallel
edges are called simple. For simple graphs we usually identify an edge $e$ with its
image $\psi(e)$ and write $G = (V(G), E(G))$, where $E(G) \subseteq \{X \subseteq V(G) : |X| = 2\}$
or $E(G) \subseteq V(G) \times V(G)$. We often use this simpler notation even in the presence
of parallel edges, then the ``set'' $E (G)$ may contain several ``identical'' elements. In this thesis all graphs 
are considered simple if nothing different is said. %TODO rausnehmen falls gar keine parallelen Kanten gebraucht werden
$|E(G)|$ denotes the number of edges, and for two edge sets $E$ and $F$ we always
have $|E \cup F | = |E | + |F |$ even if parallel edges arise.

We say that an edge $e = \{v, w\}$ or $e = (v, w)$ joins $v$ and $w$. In this case, $v$ and $w$ are adjacent. $v$ is a 
neighbour of $w$ (and vice versa). $v$ and $w$ are the endpoints of $e$. If $v$ is an endpoint of an edge $e$, we say 
that $v$ is incident with $e$. 
In the directed case we say that $( v, w)$ leaves $v$ and enters $w$, $v$ is the tail and $w$ is the head of the arc 
$e$. Two edges which share at least one endpoint are called adjacent.

For a digraph $G$ we sometimes consider the underlying undirected graph, i.e. the undirected graph $G'$ on the same 
vertex set which contains an edge $\{v, w\}$
for each edge $(v, w)$ of $G$. We also say that $G$ is an orientation of $G'$.
A subgraph of a graph $G = (V(G), E(G))$ is a graph $H = (V(H), E(H))$
with $V(H) \subset V(G)$ and $E(H) \subset E(G)$. We also say that $G$ contains $H$. $H$ is an
induced subgraph of $G$ if it is a subgraph of $G$ and $E (H) = \{ \{x, y\} \textrm{ resp. } (x, y) \in
E(G) : x, y \in V(H)\}$. Here $H$ is the subgraph of $G$ induced by $V(H)$. We also
write $H = G[V(H)]$. A subgraph $H$ of $G$ is called spanning if $V(H) = V(G)$.
If $v \in V(G)$, we write $G- v$ for the subgraph of $G$ induced by $V(G) \setminus {v}$.
If $e \in E(G)$, we define $G- e := (V(G), E(G) \setminus \{e\})$. Furthermore, the addition
of a new edge $e$ is abbreviated by $G + e := (V(G), E(G) \cup {e})$. If $G$ and $H$
are two graphs, we denote by $G + H$ the graph with $V(G +H)= V(G) \cup V(H)$
and $E(G +H)$ being the disjoint union of $E(G)$ and $E(H)$ (parallel edges may arise).

For a graph $G$ and $X, Y\subseteq V(G)$ we define $E(X, Y) := \{\{x, y\} \in E(G) : x \in
X \setminus Y, y \in Y \setminus X\}$ resp. $E^+(X, Y) := \{(x, y) \in E(G) : x \in X\setminus Y, y \in Y \setminus 
X\}$.
For undirected graphs $G$ and $X \subseteq V(G)$ we define $\delta(X) := E(X, V(G) \setminus X)$. The
set of neighbours of $X$ is defined by $ \Gamma(X) := \{v \in V(G) \setminus X : E(X, \{v\})  \neq \emptyset\}$.
For digraphs $G$ and $X \subseteq V(G)$ we define $\delta^+(X) := E^+(x, V(G) \setminus X)$, $\delta^-(x) :=
\delta^+(V(G) \setminus X)$ and $\delta(X) := \delta^+(x) \cup \delta^-(x)$. We use subscripts (e.g. $\delta_G(X)$) to
specify the graph $G$ if necessary.

For singletons, i.e. one-element vertex sets $\{v\} (v \in V(G))$ we write $\delta(v) :=
\delta(\{v\})$, $\Gamma(v) := \Gamma(\{v\}), \delta^+(v) := \delta^+(\{v\})$ and $\delta^-(v) := \delta^-(\{v\})$. The 
degree of a vertex $v$ is $|\delta(v)|$, the number of edges incident to $v$. In the directed case, the
in-degree is $|\delta^-(v)|$, the out-degree is $|\delta^+(v)|$, and the degree is $|\delta^+(v)|+ |\delta^-(v)|$.
A vertex $v$ with zero degree is called isolated. A graph where all vertices have
degree $k$ is called $k$-regular.

An edge progression $W$ in $G$ is a sequence $v_1, e_1, v_2, \dots , v_k, e_k, v_{k+1}$ such that $k \ge 0$,
and $e_i = (v_i, v_{i+ 1}) \in E(G)$ resp. $e_i = \{v_i, v_{i+1}\}\in E(G)$ for $i = 1, \dots , k$. If in
addition $e_i \ne e_j \,\forall\, 1 \le i < j \le k$, $W$ is called a walk in $G$. $W$ is closed if
$v_1 = v_{k+1}$. A path is a graph $P = (\{v_1, ... , v_{k+1}\}, \{e_1, ... , e_k\})$ such that $v_i \ne v_j$ for
$1 \le i < j \le k + 1$ and the sequence $v_1 , e_1 , v_2, \dots , v_k, e_k, v_{k+1}$ is a walk. $P$ is
also called a path from $v_1$ to $v_{k+1}$ or a $v_1 - v_{k+1}$-path. $v_1$ and $v_{k+1}$ are the endpoints
of $P$. By $P_{[x,y]}$ with $x, y \in V(P)$ we mean the (unique) subgraph of $P$ which is
an $x-y$-path. Evidently, there is an edge progression from a vertex $v$ to another
vertex $w$ if and only if there is a $v-w$-path.

A cycle is a graph $(\{v_1, \dots , v_k\}, \{e_1, \dots, e_k\})$ such that the sequence $v_1, e_1, v_2, 
\dots , v_k,e_k,v_1$ is a (closed) walk and $v_i \ne v_j$ for $1 \le i < j\le k$.
An easy induction argument shows that the edge set of a closed walk can be
partitioned into edge sets of cycles.

The length of a path or cycle is the number of its edges. If it is a subgraph
of $G$, we speak of a path or cycle in $G$. A spanning path in $G$ is called a
Hamiltonian path while a spanning cycle in $G$ is called a Hamiltonian cycle
or a tour. A graph containing a Hamiltonian cycle is a Hamiltonian graph.
For two vertices $v$ and $w$ we write $dist(v, w)$ or $dist_G (v, w)$ for the length of
a shortest $v-w$-path (the distance from $v$ to $w$) in $G$. If there is no $v-w$-path at all,
i.e. $w$ is not reachable from $v$, we set $dist(v, w) := \inf$. In the undirected case,
$dist(v, w) = dist(w, v)$ for all $v, w \in V(G)$.

We shall often have a cost function $c : E(G) \to \R$. Then for $F \subseteq E(G)$ we
write $c(F) := \sum_{e\in F} c(e)$ (and $c(\emptyset) = 0$). This extends $c$ to a modular function
$c : 2^{E(G)}\to \R$. Moreover, $dist_{(G,c)}(v, w)$ denotes the minimum $c(E(P))$ over all
$v-w$-paths $P$ in $G$.

\newpage
\section{Heuristic Approach}\newpage
\subsection{A Simple Heuristic Approach}

As we have seen, the computations of the MIP formulation with separation of the Acyclicity Constraints could be 
expensive in running time and ressource consumption. The solved problem is a relaxation of the real-world 
problem anyway, so we can as well think 
about different relaxations and heuristic approaches. This chapter will introduce a simple and easy to 
implement heuristic approach inspired by the idea of the former section, where we showed why path augmentation models 
do not work for the acyclic flowbound problem. This approach uses a graph transformation and computes a Minimum Cost 
Flow on the transformed graph. We show that we get a valid bound for the Acyclic Flowbound Problem from this 
transformed graph.\\

The main idea of this heuristic approach is to avoid that the same flow is cycling over and over again until it reaches 
the capacity bounds. The path augmentation of the section before could achieve this goal. But we need to relax the 
capacity bounds and by this we relax also the problem. The proof that it suffices to double the capacities relies on 
the idea of distinguishing the flow before and after the maximized arc, because in every part of this the capacities 
have to be respected. If we double capacities and compute flow in this network it could happen that flow in one part 
uses 1.5 times of an arcs capacity and the other only half of it. So the bounds get in fact tighter if we separate the 
two parts.

In order to achieve this we make two copies of the original graph and make the arc we maximize the 
bridge between the two. On one part of the graph we only have sources, on the other part we only have sinks. 
In order to make the problem feasible we introduce artificial arcs between the sinks and their counterparts in the other 
copy of the graph. These artificial arcs get a detention cost so they are only used when there is no other way. On this 
modified graph we obtain minimum costs by sending as much flow as possible over the maximized arc. We can add the flow 
in both parts in the end, obtaining a flow for the original graph that might be not acyclic and not respecting the 
capacities but has maximum flow value on the arc we maximize. 

Let us describe the approach in a formal way.

\subsubsection*{The Graph Transformation}

The following algorithm \ref{algo:graphtransform} describes how the graph is transformed to the new graph. It mainly 
splits the graph into two copies $X$ and $Y$, sets costs and introduces links between both parts and the 
supersource and supersink.

%Pseudocode des Algorithmus
\begin{algorithm}
 \caption{graph transformation}
 \label{algo:graphtransform}
 \begin{algorithmic}[5]
  \Function{MakeTransformedGraph}{$G=(V,A), e\in A$}
  \State create empty graph $G':=(V',A'), V'=A'=\emptyset$
  \State create labels capacity $A'\to \R$, cost $A'\to \R$% , demand $V'\to\R$
  \State create $s, t\in V$ \Comment{ supersource and supersink}
  \State create $a_0 :=(s,t) \in A'$
  \State capacity$(a_0)\gets\infty$ 
  \State cost$(a_0)\gets 2$ \Comment{highest arc cost in the network}
  \ForAll{$v\in V$}\Comment{make two copies of each node}
    \State create $v_X, v_Y\in V'$
    \If{$v$ is source of $G$}
      \State create arc $a:=(s,v_X)\in A'$
      \State capacity$(a)\gets$ supply$(v)$
    \ElsIf{$v$ is sink of $G$}
      \State create arc $a:=(v_X, t)\in A'$
      \State create arc $b:=(v_Y,t)\in A'$
      \State cost$(a)\gets 1$
    \EndIf
  \EndFor
  \ForAll{$a=(u,v)\in A$}
    \If{$a=e$}\Comment{for the arc to maximize we only make a bridge, no copy}
      \State create arc $e':=(u_X, v_Y)\in A'$
      \State capacity$(e')\gets$ capacity$(e)$
    \Else
      \State create arcs $a_X:=(u_X, v_X), a_Y:=(u_Y, v_Y)\in A'$
      \State capacity$(a_X)\gets$ capacity$(a)$, capacity$(a_Y)\gets$ capacity$(a)$
      \State cost$(a_X)\gets 0$, cost$(a_Y)\gets0$
    \EndIf
  \EndFor
  \State \Return $G'$
  \EndFunction
 \end{algorithmic}

\end{algorithm}

Figure \ref{bild:graphtransform} shows as an example how the graph from the last section would be transformed by this 
algorithm

\begin{figure}[h!]
\centering
\begin{tikzpicture}[scale=0.6]
\node (a) at (0,3)[vertex]{};
\node (s1) at (1,2.5)[source]{-1};
\node (b) at (2,3)[vertex]{};
\node (d) at (2,2)[vertex]{};
\node (c) at (1.5,0)[vertex]{};
\node (s2) at (3,3)[source]{-1};
\node (t) at (1.5,1)[sink]{2};
% \draw[arcflow, cyan, dashed] (s1)--(d)--(b);
% \draw[arcflow, cyan](b)--(a)--(c)--(t);
% \draw[arcflow, yellow, dashed] (s2)--(b)--(a)--(c)--(t);
\draw[fatarc] (b)--(a)node[pos=0.5, above]{$e$};
\draw[edge] (a)--(s1){};
\draw[edge] (s1)--(d){};
\draw[edge] (d)--(b){};
\draw[edge] (b)--(s2){};
\draw[edge] (s2)--(c){};
\draw[edge] (c)--(a){};
\draw[edge] (c)--(t){};
\draw[edge] (t)--(d){};

%supersource und supersink
\node(sups) at (0,-2.5)[source]{-2};
\node(supt) at (7.5,-2.5)[sink]{2};

\node (la) at (0,-1)[vertex]{};
\node (ls1) at (1,-1.5)[source]{};
\node (lb) at (2,-1)[vertex]{};
\node (ld) at (2,-2)[vertex]{};
\node (lc) at (1.5,-4)[vertex]{};
\node (ls2) at (3,-1)[source]{};
\node (lt) at (1.5,-3)[sink]{};
% \draw[fatarc] (lb)--(la)node[pos=0.5, above]{$e$};
\draw[edge] (la)--(ls1){};
\draw[edge] (ls1)--(ld){};
\draw[edge] (ld)--(lb){};
\draw[edge] (lb)--(ls2){};
\draw[edge] (ls2)--(lc){};
\draw[edge] (lc)--(la){};
\draw[edge] (lc)--(lt){};
\draw[edge] (lt)--(ld){};

\node (ra) at (4,-1)[vertex]{};
\node (rs1) at (5,-1.5)[source]{};
\node (rb) at (6,-1)[vertex]{};
\node (rd) at (6,-2)[vertex]{};
\node (rc) at (5.5,-4)[vertex]{};
\node (rs2) at (7,-1)[source]{};
\node (rt) at (5.5,-3)[sink]{};
\draw[fatarc, bend left] (lb)to(ra);
\draw[edge] (ra)--(rs1){};
\draw[edge] (rs1)--(rd){};
\draw[edge] (rd)--(rb){};
\draw[edge] (rb)--(rs2){};
\draw[edge] (rs2)--(rc){};
\draw[edge] (rc)--(ra){};
\draw[edge] (rc)--(rt){};
\draw[edge] (rt)--(rd){};
\draw[arc, green!70!blue] (sups)--(ls1)node[pos=0.5, left]{$cap=1$};
\draw[arc, green!70!blue] (sups)--(ls2)node[pos=0.5, above]{$cap=1$};
\draw[arc, red] (rt)--(supt)node[pos=0.5, below]{$cap=2,\, cost=0$};
\draw[arc, red] (lt)--(supt)node[pos=0.5, above]{$cap=2,\, cost=1$};
\draw[arc, purple!40!blue, bend right =90] (sups)to(supt)node at (4,-5){$cap=\infty,\,cost=2$};

\end{tikzpicture}
\caption{The transformation of the example network from the former section}
 \label{bild:graphtransform}
\end{figure}


With this transformation and any algorithm for Minimum Cost Flow we can set up an algorithm for our problem. 
For the Maximum Flow and Minimum Cost Flow Problem in graphs there are many standard algorithms we could use. The 
classical Maximum Flow algorithm of Ford and Fulkerson \cite{Ford-Fulkerson_algo} arising from their Max-Flow-Min-Cut 
theorem is based on augmenting flow on source-sink-paths in the network as well as the improved algorithms of Edmonds 
and Karp \cite{EdmondsKarp1972} or Dinic \cite{Dinic1970}. Algorithms for Maximum Flow can be used to first check if 
there is any feasible flow in the network. If there is no feasible flow we can give up at this point, while the flow 
problem on the transformed graph is always feasible by construction.

Deriving from the Max-Flow Algorithms there are many algorithms solving the Min-Cost-Flow Problem. Edmonds and 
Karp described a Successive Shortest Path Algorithm in \cite{EdmondsKarp1972}. Goldberg and Tarjan proposed the Minimum 
Mean Cycle Cancelling Algorithm \cite{minMeanCycleCancelling89} with runnning time of $\mathcal{O}(nm(log 
n)\min\{log(nC), m log n\}) $. %TODO fuer die anderen ebenfalls laufzeiten odr fuer keine
Another algorithm is the Network Simplex by Orlin \cite{NetworkSimplexOrlin97} (which was used in the computational 
study) There are a lot more algorithms available for Minimum Cost Flow.

Our algorithm mainly consists of a feasibility test, a transformation of the graph $G$ and a Minimum Cost Flow 
computation on the transformed graph $G'$, see algorithm \ref{algo:simpleheur}. 

\begin{algorithm}
 \caption{simple heuristic}
 \label{algo:simpleheur}
 \begin{algorithmic}
  \Function{FlowBoundHeur}{$G=(V,A), e\in A$}
  \State mf$\gets$ \Call{MaxFlow}{G}
  \If{mf < nominated ingoing flow}
    \State\Return infeasible
  \EndIf
  \State $G'=$ \Call{MakeTransformedGraph}{$G,e$}
  \State $f\gets$ \Call{MinCostFlow}{$G'$}
  \State ub$\gets f(e')$
  \State backwardflow$\gets f(a_0)$
  \State \Return ub - backwardflow
  \EndFunction
 \end{algorithmic}
\end{algorithm}

We show the correctness of the algorithm: %TODO and running time? dependent on used mincostflow!

\begin{figure}[h!]
\centering
\begin{tikzpicture}[scale=0.6]
%supersource und supersink
\node(sups) at (0,-2.5)[source]{-2};
\node(supt) at (7.5,-2.5)[sink]{2};
%left part
\node (la) at (0,-1)[vertex]{};
\node (ls1) at (1,-1.5)[source]{};
\node (lb) at (2,-1)[vertex]{};
\node (ld) at (2,-2)[vertex]{};
\node (lc) at (1.5,-4)[vertex]{};
\node (ls2) at (3,-1)[source]{};
\node (lt) at (1.5,-3)[sink]{};
%right part
\node (ra) at (4,-1)[vertex]{};
\node (rs1) at (5,-1.5)[source]{};
\node (rb) at (6,-1)[vertex]{};
\node (rd) at (6,-2)[vertex]{};
\node (rc) at (5.5,-4)[vertex]{};
\node (rs2) at (7,-1)[source]{};
\node (rt) at (5.5,-3)[sink]{};

\draw[arcflow, cyan] (sups)--(ls2)--(lb);
\draw[arcflow, cyan, bend left] (lb)to (ra);
\draw[arcflow, cyan](ra)--(rs1)--(rd)--(rt)--(supt);
\draw[arcflow, yellow!80!orange, dashed](sups)--(ls1)--(ld)--(lb);
\draw[arcflow, yellow!80!orange, dashed, bend left](lb)to(ra);
\draw[arcflow, yellow!80!orange, dashed](ra)--(rc)--(rt)--(supt);

\draw[edge] (la)--(ls1){};
\draw[edge] (ls1)--(ld){};
\draw[edge] (ld)--(lb){};
\draw[edge] (lb)--(ls2){};
\draw[edge] (ls2)--(lc){};
\draw[edge] (lc)--(la){};
\draw[edge] (lc)--(lt){};
\draw[edge] (lt)--(ld){};
\draw[fatarc, bend left] (lb)to(ra);
\draw[edge] (ra)--(rs1){};
\draw[edge] (rs1)--(rd){};
\draw[edge] (rd)--(rb){};
\draw[edge] (rb)--(rs2){};
\draw[edge] (rs2)--(rc){};
\draw[edge] (rc)--(ra){};
\draw[edge] (rc)--(rt){};
\draw[edge] (rt)--(rd){};
\draw[arc, green!70!blue] (sups)--(ls1)node[pos=0.5, left]{$cap=1$};
\draw[arc, green!70!blue] (sups)--(ls2)node[pos=0.5, right]{$cap=1$};
\draw[arc, red] (rt)--(supt)node[pos=0.5, below]{$cap=2,\, cost=0$};
\draw[arc, red] (lt)--(supt)node[pos=0.5, above]{$cap=2,\, cost=1$};
\draw[arc, purple!40!blue, bend right =90] (sups)to(supt)node at (4,-5){$cap=\infty,\,cost=2$};
\end{tikzpicture}
\caption{This flow has zero costs, is feasible in the heuristic and has the optimum value $f(e)=2$}
 \label{bild:graphtransformWithFlow}
\end{figure}

\begin{prop}
 The heuristic algorithm \ref{algo:simpleheur} returns a valid upper bound for the possible acyclic flow on a given 
 arc $e$ in the network, or infeasible if there is no feasible network flow for the given flow balances at sources and 
 sinks.
\end{prop}

\begin{proof}
\textbf{Feasibility Test: }We have to show that the ordinary feasibility test with maximum flow is sufficient even for 
acyclic flow: If the flow network nomination (in- and outflow) is infeasible, no algorithm for the 
maximum flow problem will find a flow that is fulfilling these balances completely. Then also for the stricter acyclic 
flow problem this is infeasible. The other direction is also true: if there is a feasible flow, there also is a 
feasible acyclic flow:

Take the maximum flow and choose one flow cycle within (If there is none we are done). On all arcs of the cycle the 
flow is going in the same direction in relation to the cycle. We can choose the minimum flow on the cycles arcs and 
reduce flow on each arc of the cycle by this amount. The resulting flow is still a feasible flow, but we removed the 
flow cycle. This can be done until the flow is acyclic. But if there is any acyclic flow, there also has to be an 
acyclic flow that is maximum on the arc we are interested in. So in order to test feasibility it is enough to run a 
ordinary maximum flow algorithm.\\


\textbf{Relaxation: }To prove that we get an upper bound with this heuristic we have to show that the result we 
get is the maximum value of a relaxation of the problem. 

If a flow is acyclic we can decompose it into flow on a set of simple paths. This means that 
the Acyclic Flowbound Problem could theoretically be solved by an algorithm augmenting simple paths (in the 
right order) where we put some extra constraints and objectives on these paths. These extra constraints have to forbid 
the closing of any flow cycle by the path as well as forcing the path to use arc $e$ if possible. (Section 
\ref{model:pathaugment} describes this idea and shows the difficulties of choosing the right path for augmentation.)
%TODO koennte das modell einzeln hinschreiben mit 
%hinweis auf NP schwere von constrained shortest path. anschließend kommt das relaxierungsargument besser !!!
%Außerdem hätte ich damit shconmal ein gutes modell

The heuristic algorithm is a relaxation of this constrained path augmentation model. It still forces all paths to go 
over $e$ if possible (which might be much more since it is a problem with weaker conditions). 
We can reconstruct a flow in the original graph from the flow in the transformed graph by summing up the flow values on 
both copied arcs, assigning this flow to the original arc: $f(a)=f'(a_X)+f'(a_Y)$ and the assigned flow value of the 
bridge arc to the maximized arc of the original problem. 

All the flow conservation constraints still hold after constructing the flow in the original graph from the two copies:
\begin{align*}
& \sum_{a\in \delta^+(v)}q_a - \sum_{a\in\delta^- (v)}q_a &=& d_v\ &\forall v\in V_X \\
+& \sum_{a\in \delta^+(v)}q_a - \sum_{a\in\delta^- (v)}q_a &=& d_v\ &\forall v\in V_Y \\
=& \sum_{a\in \delta^+(v_X)\cup\delta^+(v_Y)}q_a - \sum_{a\in\delta^- (v_X)\cup\delta^-(v_Y)}q_a &=& d_v\ &\forall 
v_X\in V_X, v_Y\in V_Y \\
=& \sum_{a\in \delta^+(v)}q_a - \sum_{a\in\delta^- (v)}q_a &=& d_v\ &\forall v\in V \\
\end{align*}

The flow balances at sources and sinks are set right by construction.\\

When no more flow can be sent over the maximized arc $e$ the path augmentation model already gives a valid bound 
if the flow is feasible. But in fact this model can determine the amount of flow that is necessary to augment backwards 
on $e$ in a path model.

We distinguish whether the flow is going over the bridge $e$, going to a sink node in already in part $X$ of the 
transformed graph and thus having costs of 1 per unit, 
or is going from $s$ to $t$ directly with a cost of 2 per unit. The flow with costs of 2 cannot be sent over $e$ in 
forward direction neither find any sink in part $X$ of the graph. If there is a feasible flow in $G$ (which we tested) 
the only possible way is that the remaining flow has to go backward over $e$. So we can reduce the flow on the 
bridge by this amount of flow directly going from $s$ to $t$ and the bound is still valid. (In algorithm 
\ref{algo:pathHeur} from the section before this would be the part in line \ref{heur:lineAugCase3} where flow is 
augmented backwards).\\

$\Rightarrow$ Since the problem is just a relaxation of the original problem/the constrained simple path model, the 
obtained bound is always weaker (in this case $\ge$) than the optimal bound for maximum acyclic flow. 
$\Rightarrow$ the heuristic is correct for finding upper bounds for the acyclic flow on an arc.

\end{proof}%TODO der beweis ist lang und unuebersichtlich, man koennte fragen wozu man manche erklaerungen braucht.

% So the algorithm relaxes the capacity constraints and allows weaker kinds of cycling. It still maximizes the amount 
of 
% flow going over the maximized arc $e$ (the bridge): if there is any flow that can be sent over $e$ this is the 
cheapest 
% way. All flows that are not passing $e$ at some point have to use the arcs from part $X$ of the transformed graph 
% directly to the super-sink $t$ (which causes cost $=1$ per flow unit) or even worse from the super-source $s$ 
directly 
% to the super-sink $t$ (which causes cost $=2$ per flow unit). To go over $e$ has no cost at all, therefore 
% the paths with positive costs are only used if there is a directed minimal cut from part $X$ to part $Y$ where no arc 
% has free capacity.

So we have seen that we can compute an upper bound via the optimum for this relaxed problem with a fast standard 
algorithm for maximum flows that can be retranslated into the original problem. 
However, the capacity constraints of the original problem might be violated after retranslation. We set the arcs 
capacities in part $X$ and part $Y$ both to the value of the original capacities. Thus it might happen that actual flow 
value in the original graph sums up to a higher value than the capacity allows (up to twice the capacity). 


\begin{figure}[h!]
\centering
\begin{tikzpicture}
\node (a) at (0,-6)[vertex]{};
\node (s1) at (1,-6.5)[source]{-1};
\node (b) at (2,-6)[vertex]{};
\node (d) at (2,-7)[vertex]{};
\node (c) at (1.5,-9)[vertex]{};
\node (s2) at (3,-6)[source]{-1};
\node (t) at (1.5,-8)[sink]{2};
% \draw[arcflow, cyan, dashed] (s1)--(d)--(b);
\draw[arcflow, cyan](s2)--(b);
\draw[arcflow, cyan](b)--(a)--(s1)--(d);
\draw[arcflow, cyan](d)--(t);
\draw[arcflow, yellow!70!orange, dashed] (s1)--(d);
\draw[arcflow, yellow!70!orange, dashed] (d)--(b)--(a)--(c);
\draw[arcflow, yellow!70!orange, dashed] (c)--(t);
\draw[fatarc] (b)--(a)node[pos=0.5, above]{$e$};
\draw[edge] (a)--(s1){};
\draw[edge] (s1)--(d)node(errpos)[pos=0.5]{};
\draw[edge] (d)--(b){};
\draw[edge] (b)--(s2){};
\draw[edge] (s2)--(c){};
\draw[edge] (c)--(a){};
\draw[edge] (c)--(t){};
\draw[edge] (t)--(d){};
\node (err) at (-1,-7)[ellipse, red, draw=red]{ $f(a)=2>cap(a)=1$};
\draw[->, stealth, semithick, bend right,red](err)to(errpos);
\draw[ultra thick, ->, red!50] (1.55,-6.2) arc (30:330:0.2);
% \draw[pattern=dots, pattern color=red](a)--(s1)--(b);%(0,-6)--(1,-6)--(1,-5);
%TODO man könnte doch kreisfluss umschlossenes gebiet rot füllen?
\end{tikzpicture}
\caption{If we retransform the flow from figure \ref{bild:graphtransformWithFlow} (which is feasible in the heuristic) 
to the original network we get a violated capacity constraint on the arc with $c=1$. The flow is not acyclic anymore 
either.}
 \label{bild:graphretransformWithFlow}
\end{figure}
Also we cannot guarantee the acyclicity anymore. It might happen that we have cyclic flow in our 
original graph $G$ even if the flow in the transformed graph $G'$ was acyclic. This happens when paths from 
the different parts $X$ and $Y$ of $G'$ are crossing after they are brought back to the original graph $G$ to 
build a solution there. For an example see figure \ref{bild:graphretransformWithFlow} which is the retransformation of 
the example flow in figure \ref{bild:graphtransformWithFlow}.
=======
\section{Basic Notation and Definitions}% Als Referenz und Grundlage fuer die Definitionen nutze ich das Lehrbuch Combinatorial Optimization:Theory and 
%Algorithms von Bernhard Korte u Jens Vygen - definiere die wichtigsten Sachen aber erstmal auch selbst

Since there are many definitions, which may differ slightly, we want to introduce now the basic notation and 
definitions that we use throughout this thesis. The definitions in this chapter are mainly taken from the textbook 
about combinatorial optimization from Korte and Vygen \cite{KorteVygenCombOpt2007}.

An undirected graph is a triple (V, E, $\Psi$), where $V$ and $E$ are finite sets and
$\Psi: E\to \{X \subseteq V: |X| = 2\}$. 
A directed graph or digraph is a triple $(V, E, \Psi)$,
where $V$ and $E$ are finite sets and $\Psi : E \to \{(v, w) \in V \times V : v \neq w\}$. In this thesis by a
graph we mean normally the directed graph. If we talk about undirected graphs it will be stated 
explicitly. The elements of $V$ are called vertices, the elements of $E$ are the edges. Edges of undirected graphs can 
also be called arcs to make clear that they are directed.

Two edges $e, e'$ with $\Psi(e) = \Psi ( e')$ are called parallel. Graphs without parallel
edges are called simple. For simple graphs we usually identify an edge $e$ with its
image $\psi(e)$ and write $G = (V(G), E(G))$, where $E(G) \subseteq \{X \subseteq V(G) : |X| = 2\}$
or $E(G) \subseteq V(G) \times V(G)$. We often use this simpler notation even in the presence
of parallel edges, then the ``set'' $E (G)$ may contain several ``identical'' elements. In this thesis all graphs 
are considered simple if nothing different is said. %TODO rausnehmen falls gar keine parallelen Kanten gebraucht werden
$|E(G)|$ denotes the number of edges, and for two edge sets $E$ and $F$ we always
have $|E \cup F | = |E | + |F |$ even if parallel edges arise.

We say that an edge $e = \{v, w\}$ or $e = (v, w)$ joins $v$ and $w$. In this case, $v$ and $w$ are adjacent. $v$ is a 
neighbour of $w$ (and vice versa). $v$ and $w$ are the endpoints of $e$. If $v$ is an endpoint of an edge $e$, we say 
that $v$ is incident with $e$. 
In the directed case we say that $( v, w)$ leaves $v$ and enters $w$, $v$ is the tail and $w$ is the head of the arc 
$e$. Two edges which share at least one endpoint are called adjacent.

For a digraph $G$ we sometimes consider the underlying undirected graph, i.e. the undirected graph $G'$ on the same 
vertex set which contains an edge $\{v, w\}$
for each edge $(v, w)$ of $G$. We also say that $G$ is an orientation of $G'$.
A subgraph of a graph $G = (V(G), E(G))$ is a graph $H = (V(H), E(H))$
with $V(H) \subset V(G)$ and $E(H) \subset E(G)$. We also say that $G$ contains $H$. $H$ is an
induced subgraph of $G$ if it is a subgraph of $G$ and $E (H) = \{ \{x, y\} \textrm{ resp. } (x, y) \in
E(G) : x, y \in V(H)\}$. Here $H$ is the subgraph of $G$ induced by $V(H)$. We also
write $H = G[V(H)]$. A subgraph $H$ of $G$ is called spanning if $V(H) = V(G)$.
If $v \in V(G)$, we write $G- v$ for the subgraph of $G$ induced by $V(G) \setminus {v}$.
If $e \in E(G)$, we define $G- e := (V(G), E(G) \setminus \{e\})$. Furthermore, the addition
of a new edge $e$ is abbreviated by $G + e := (V(G), E(G) \cup {e})$. If $G$ and $H$
are two graphs, we denote by $G + H$ the graph with $V(G +H)= V(G) \cup V(H)$
and $E(G +H)$ being the disjoint union of $E(G)$ and $E(H)$ (parallel edges may arise).

For a graph $G$ and $X, Y\subseteq V(G)$ we define $E(X, Y) := \{\{x, y\} \in E(G) : x \in
X \setminus Y, y \in Y \setminus X\}$ resp. $E^+(X, Y) := \{(x, y) \in E(G) : x \in X\setminus Y, y \in Y \setminus 
X\}$.
For undirected graphs $G$ and $X \subseteq V(G)$ we define $\delta(X) := E(X, V(G) \setminus X)$. The
set of neighbours of $X$ is defined by $ \Gamma(X) := \{v \in V(G) \setminus X : E(X, \{v\})  \neq \emptyset\}$.
For digraphs $G$ and $X \subseteq V(G)$ we define $\delta^+(X) := E^+(x, V(G) \setminus X)$, $\delta^-(x) :=
\delta^+(V(G) \setminus X)$ and $\delta(X) := \delta^+(x) \cup \delta^-(x)$. We use subscripts (e.g. $\delta_G(X)$) to
specify the graph $G$ if necessary.

For singletons, i.e. one-element vertex sets $\{v\} (v \in V(G))$ we write $\delta(v) :=
\delta(\{v\})$, $\Gamma(v) := \Gamma(\{v\}), \delta^+(v) := \delta^+(\{v\})$ and $\delta^-(v) := \delta^-(\{v\})$. The 
degree of a vertex $v$ is $|\delta(v)|$, the number of edges incident to $v$. In the directed case, the
in-degree is $|\delta^-(v)|$, the out-degree is $|\delta^+(v)|$, and the degree is $|\delta^+(v)|+ |\delta^-(v)|$.
A vertex $v$ with zero degree is called isolated. A graph where all vertices have
degree $k$ is called $k$-regular.

An edge progression $W$ in $G$ is a sequence $v_1, e_1, v_2, \dots , v_k, e_k, v_{k+1}$ such that $k \ge 0$,
and $e_i = (v_i, v_{i+ 1}) \in E(G)$ resp. $e_i = \{v_i, v_{i+1}\}\in E(G)$ for $i = 1, \dots , k$. If in
addition $e_i \ne e_j \,\forall\, 1 \le i < j \le k$, $W$ is called a walk in $G$. $W$ is closed if
$v_1 = v_{k+1}$. A path is a graph $P = (\{v_1, ... , v_{k+1}\}, \{e_1, ... , e_k\})$ such that $v_i \ne v_j$ for
$1 \le i < j \le k + 1$ and the sequence $v_1 , e_1 , v_2, \dots , v_k, e_k, v_{k+1}$ is a walk. $P$ is
also called a path from $v_1$ to $v_{k+1}$ or a $v_1 - v_{k+1}$-path. $v_1$ and $v_{k+1}$ are the endpoints
of $P$. By $P_{[x,y]}$ with $x, y \in V(P)$ we mean the (unique) subgraph of $P$ which is
an $x-y$-path. Evidently, there is an edge progression from a vertex $v$ to another
vertex $w$ if and only if there is a $v-w$-path.

A cycle is a graph $(\{v_1, \dots , v_k\}, \{e_1, \dots, e_k\})$ such that the sequence $v_1, e_1, v_2, 
\dots , v_k,e_k,v_1$ is a (closed) walk and $v_i \ne v_j$ for $1 \le i < j\le k$.
An easy induction argument shows that the edge set of a closed walk can be
partitioned into edge sets of cycles.

The length of a path or cycle is the number of its edges. If it is a subgraph
of $G$, we speak of a path or cycle in $G$. A spanning path in $G$ is called a
Hamiltonian path while a spanning cycle in $G$ is called a Hamiltonian cycle
or a tour. A graph containing a Hamiltonian cycle is a Hamiltonian graph.
For two vertices $v$ and $w$ we write $dist(v, w)$ or $dist_G (v, w)$ for the length of
a shortest $v-w$-path (the distance from $v$ to $w$) in $G$. If there is no $v-w$-path at all,
i.e. $w$ is not reachable from $v$, we set $dist(v, w) := \inf$. In the undirected case,
$dist(v, w) = dist(w, v)$ for all $v, w \in V(G)$.

We shall often have a cost function $c : E(G) \to \R$. Then for $F \subseteq E(G)$ we
write $c(F) := \sum_{e\in F} c(e)$ (and $c(\emptyset) = 0$). This extends $c$ to a modular function
$c : 2^{E(G)}\to \R$. Moreover, $dist_{(G,c)}(v, w)$ denotes the minimum $c(E(P))$ over all
$v-w$-paths $P$ in $G$.

\newpage
\section{The Acyclic Flowbound Problem}We already described the gas flow problem, where the motivation for this thesis came from. Here we want to define the 
problem as a general combinatorial flow problem with specific contraints. We will also give the formulation as a Mixed 
Integer Program (MIP) and as well some natural relaxations, which might be easier to solve.

We will represent our originally undirected graph by a directed graph where flow is allowed to go over edges backward 
and forward as well. This allows us to specify directions forward and backward on every edge in a consistant way. 


\begin{definition}
 Let $G=(V,A)$ be a directed Graph and $e \in A$ a specific arc of $G$. For every vertex $v\in V$ let there be a 
prescribed range for the amount of flow demand $[d_l(v), d_u(v)]\subset \R \setminus \{0\}$. Vertices with demand 
values greater than 0 are sinks, vertices with negative demand value are sources. The flow demands are assumed to be 
balanced on their (absolute) higher bound values 
$$\sum_{v \in V\textrm{ sink}}d_u(v)-\sum_{v \in V\textrm{ source}}d_l(v)=0$$ 
%TODO am ende kontrollieren, dass demands (was positiv, was negativ) konsistent verwendet wurde!

Let there be a capacity function $c:A\to \mathcal{P}(\R), \, c(a)=[c_l(a), c_u(a)],\, c_l(a)\le 0 \le c_u(a) \, \forall 
a\in A$, and a flow  $f: A\to \R $ with $c_l(a)\le f(a)\le 
c_u(a)\, \forall a\in A$ and $\sum_{a\in \delta^+(v)}f(a)-\sum_{a\in\delta^-(v)}f(a)+d(v) = 0 \, \forall v\in V$.\\
We call this flow a \textit{feasible network flow on G}.
\end{definition}

The standard definition of Flow Problems has fixed demand values $d(v)=const$ instead of an interval. These two 
definitions are equivalent in the way that you can transform them into each other easily in polynomial time. Obviously 
we can see fixed demands $d(v)$ as intervals with $d_l(v)=d_u(v)=d(v)$ consisting of only one point $[d(v), d(v)]$. The 
other way around we use a construction that adds only one node and $|\{v\in V| d_u(v)<0\}|+|\{v\in V| d_l(v)>0\}|$ arcs 
to the network.
\begin{prop}
 There is a (linear time) transformation from the flow problem with demand intervals on the vertices to the standard 
flow problem with fixed demands. 
\end{prop}
\begin{proof}
 Given a graph $G=(V,A)$ with demand intervals at the entries and exits, we construct a graph $G'=(V',A')$ as follows:
 $$V' := V\cup \{z\} \textrm{, and } A' := A\cup \{ (v,z)| d_u(v)<0 \}\cup\{ (z,v)|d_l(v)>0\}$$ On $G'$ we then have 
to redefine the arc capacity and node demands: 
$$c':A\to \mathcal{P}(\R)  \,, c'(a)= c'(u,v) = \begin{cases} \{x|x\in [0,d_u(u)-d_l(u)]\} &\textrm{if }v=z\\ 
\{x|x\in [0,d_u(v)-d_l(v)]\}&\textrm{if }u=z\\ \{x|x\in[c_l(a), c_u(a)]\} &\textrm{else } \end{cases}
$$ 
The node demands now are 
$$ d'(v)=\begin{cases} d_l(v) & \textrm{if v is source, i.e. }d_u(v)<0\\
         d_u(v) &\textrm{if v is sink, i.e. }d_u(v)>0
        \end{cases}
$$
If we compute a solution of the flow problem on $G'$ and restrict the flow to the arcs of $G$, we get the solution of 
the problem on $G$ that respects the demand intervals:\\ 
We call the demands in the solution restricted to arcs of $G$ $d^s$ and have $ d^s(u)=d'(u)+f'((u,z))$. The flow balance 
constraints in normal nodes 
(nodes with demand 0) do not change at all, since the only new node $z$ is not in their neighborhood. All vertices that 
are sources or sinks (per definition thy cannot be both at the same time) have $z$ as neighbor. The flow balance on a 
source $u$ before and after restriction to $G$ differs only by the flow on the artificial arc $(u,z)$. This flow 
$f'((u,z))$ is in the capacity range $[0,d_u(u)-d_l(u)]$. Hence for sources $u$ ($d'(u)<0$) with 
$d^s(u)=d'(u)+f'((u,z))$ we can conclude for the demands $d^s(u)$ of the solution restricted to $G$ 
$$d_l(u)\le d^s(u)=d'(u)+f'((u,z))=d_l(u)+\underbrace{f'((u,z))}_{0\le f'((u,z))\le d_u(u)-d_l(u)} \le d_u(u)$$
and as well for sinks ($d'(v)>0$)
$$d_l(v)\le d_u(v)-\underbrace{f'((z,v))}_{0\le f'((z,v))\le d_u(v)-d_l(v)}=d'(v)-f'((z,v))=d^s(v) \le d_u(v)$$
\end{proof}



 %cycle was already defined in the Definitions chapter
\begin{definition}
A \textit{cyclic flow} within such a feasible network flow $f$ is a flow $f':C\to \R$, where $C\subseteq A$ is a cycle, 
$\sum_{a\in \delta^+(v)\cap C}f(a)-\sum_{a\in\delta^-(v)\cap C}f(a) = 0 \, \forall v\in C$ , for all arcs the flow 
directions in $f'$ and $f$ are the same, i.e. $f(a)\ge 0\Rightarrow f'(a)\ge 0,\, f(a)\le 0 \Rightarrow f'(a)\le 0$ and 
there is really a flow, i.e. $f'(a) \ne 0 \,\forall a\in C$. 

If there is any cyclic flow $f'$ in a network flow $f$ we say that $f$ contains a flow cycle. If there is no cyclic 
flow, we say $f$ is \textit{acyclic} or an \textit{acyclic flow} on $G$.\\

The conditions imply, that a cyclic flow is one that has no sources or sinks and the same value of flow on each arc.

The \textit{size of a cyclic flow} is the absolute amount of of flow contributing to the cycle, that means it is the 
maximum value $|f'(a)|$ a cyclic flow $f'$ can reach. 
%flow $ f: C\to \R ,\, C\subseteq A \textrm{ a cycle, }\,c_l(a)\le 
%f(a)\le c_u(a)\, \forall a\in C$ with the property that 
\end{definition}
%TODO Kreisflussmenge, Kreisflusskapazität, Augmentierung


\begin{definition}
  Given a graph $G=(V,A)$ like above, with $e \in A$ a specific arc of $G$. 
  
  We call the problem of finding an acyclic flow $f:A\to \R$ with $f(e)\ge f'(e)\,\forall f':A\to \R \textrm{ s.t.} f' 
\textrm{ is an acyclic flow on }G$ the \textit{ Edge-Maximizing Acyclic Flow Problem}.
\end{definition}

\begin{prop}
  The maximal flow value on the maximized arc $e$ in the Edge-Maximizing Acyclic Flow Problem is the same as the 
  maximal flow value if only cycles containing $e$ are forbidden. 
\end{prop}
\begin{proof}
 Let $f_{acyc}$ be the flow of a solution of an all-acyclic flow problem with maximum flow $f_{acyc}(e)$ and let 
$f_{cyc}$ be a flow of the relaxation where we allow cyclic flow that is not going over $e$. 

Since $f_{cyc}$ is a relaxation of the acyclic problem we always have $f_{cyc}(e)\ge f_{acyc}(e)$. But we have to show 
that also $f_{cyc}(e)\le f_{acyc}(e))$.
% Every two flows differ only by a sequence of cyclic flow augmentations in the network. Thus if not $f_{cyc}(e)\le 
% f_{acyc}(e)$ we can choose a pair $f,f'$ such that $f$ is completely acyclic and $f'$ is not, $f'(e)>f(e)$ and that 
% $f'$ differs from $f$ by just one cyclic shift of flow (take the first flow of the sequence where $f'(e)>f(e)$).

%%%%%%%%%%%%%%%%%%%%%%%%%%%%%%%%%%%%%%%%%%%%%%%
%observation: it is always easier to look at a cycle and ask how much flow can you send from one node to another. this 
%not changing, never, by a cyclic flow. but a cyclic flow could affect other cycles. But since we do not have any 
% lower bounds higher than 0 we can always shift a cyclic flow until it is acyclic!!!
%%%%%%%%%%%%%%%%%%%%%%%%%%%%%%%%%%%%%%%%%%%%%%%
In $f_{cyc}$ there are no cyclic flows on the maximized arc $e$. Take a solution of $f_{cyc}$ with cyclic flow on a 
simple cycle $C\in A(G)$. There is no minimum flow higher than 0. Thus it is always possible to shift the cyclic flow 
backwards on the cycle by the minimum amount of flow on this cycle. This is a cyclic flow, so on every arc $a\in C$ the 
flow value $f(a)$ is decreased by this. However, there is no arc where flow directions are changed. We can do this 
operation as long as there is cyclic flow without changing flow value on $e$ or creating new flow cycles. In the 
end we get an acyclic flow with the same flow value.
\end{proof}


Now that we defined our problem, we can investigate it deeper. It is obviously closely related to other flow problems, 
though the special constraints we built make it more complicated. One way to deal with this is to look at related 
problems, which could give bounds on our problem or even the same solution in special cases.

\subsubsection{Relation To Other Flow Problems/Relaxations}

If we look for related problems, it is natural to just drop the constraints of our problem. In this case we could for 
example easily forget the contraint that flows have to be acyclic. If we do this, our problem becomes a special 
instance of a Min-Cost-Flow where we allow edge weights to be negative:

\begin{definition}\label{def:mincostflow}
 The \textit{Minimum Cost Flow Problem} is the problem to determine a feasible network flow with the least possible 
cost $c_f$. The cost function $c:A\to \R$ (or often $c:A\to \R_{\ge 0}$) is defined on every arc, the cost of the flow 
is $c_f = \sum_{a\in A}|f(a)|\cdot c(a)$. So for a given network and cost function we look for a flow $f$ s.t. $c_f\le 
c_{f'}\forall f':A\to \R$.
\end{definition}

This problem is normally defined with nonnegative cost functions, like in \cite{EdmondsKarp1972} where 
Jack Edmonds and Richard Karp present their well known $O(V\cdot E^2)$ flow algorithm. At least cycles of negative 
weight should be excluded, since they lead to problems in the algorithm's subroutines like shortest path. Nevertheless 
there are algorithms that can handle negative cycles without a problem - but they will of course give a solution 
where as much flow as possible is just flowing around these negative cycles. 
%, and make the problem hard. We will proof the hardness of this problem later.TODO auch ohne kreisfreiheit?

In our application we want to find upper and lower bounds on the possible flow over each arc. For this we set the 
weight on the arc $e$ where we want to maximize flow to -1 and compute a Minimum Cost Flow in the graph. If $e$ is 
contained in a cycle of $G$ whose arcs all have a very high upper capacity bound, $e$ will get a very high bound as 
well. The reason is that any cyclic flow over $e$ can reduce the cost of the flow. Hence as much cyclic flow as 
possible will be used to obtain a Minimum Cost Flow. 

So in most cases the bound computed with a Min-Cost Flow algorithm is not sharp. Still it gives an upper bound for the 
possible flow. Hence one could ask how good this bound will be, and whether we can use it as an approximation. It turns 
out that the gap between the amount of flow on an arc $e$ in an acyclic flow and in a possibly cyclic flow can be 
arbitrarily large:

\begin{prop}
 The gap between the maximum flow values $f(a)$ on an arc $a\in A$ in a normal and in an acyclic flow problem is 
unbounded. 
\end{prop}
\begin{proof}
 As a proof look at the two examples. 
 \begin{figure}[h!]
\centering
\begin{tikzpicture}
[scale=2, vertex/.style={circle,fill=black}]
\node (s) at (0,0)[vertex]{};
\node (stext) at (-0.1, -0.2){s, $d(s)=1$};
\node (t) at (2,0)[vertex]{} ;
\node (ttext) at (2.1, -0.2){t, $(d(t)=-1$};
\node (3) at (1,1.5)[vertex]{};
\node(etext) at (1.7,0.9){e};

\draw (s) -- (3);
\draw (s) -- (t);
  \begin{scope}[
    decoration={markings,mark=at position 0.5 with {\arrow{triangle 60}}}
    ]
    \draw[postaction={decorate}] (3) -- (t);
  \end{scope}
\end{tikzpicture}
 \end{figure}
In this first picture we see that the flow on the maximized arc $e$ in the acyclic case has to be 1. In the normal case 
it could be any number we choose as capacities for all arcs in the network. Hence the relative gap
$ \frac{f_{cyc}(e)}{f_{acyclic}(e)}$ is $\frac{\textrm{min capacity on the cycle}}{1}$ which is unbounded.
  
\begin{figure}[h!]
\centering
\begin{tikzpicture}
[scale=2, vertex/.style={circle,fill=black}]
\node (s) at (0,0)[vertex]{};
\node (stext) at (-0.1, 0.2){s};
\node (t) at (0,2)[vertex]{} ;
\node (ttext) at (-0.1, 1.8){t};
\node (3) at (1,1)[vertex]{};
\node (4) at (2,2)[vertex]{};
\node (5) at (2,0)[vertex]{};
\node(etext) at (1.4,1.6){e};

\draw (s) -- (3);
\draw (t) -- (3);
\draw (4) -- (5);
\draw (3) -- (5);
  \begin{scope}[
    decoration={markings,mark=at position 0.5 with {\arrow{triangle 60}}}
    ]
    \draw[postaction={decorate}] (3) -- (4);
  \end{scope}
\end{tikzpicture}
 
\end{figure}

The second picture shows a graph where the flow on $e$ in the acyclic case is $0$. If we do not forbid cyclic flows, 
the flow on $e$ will always be $\min_{a\in \textrm{ cycle }C}c_u(a)$. So the gap in relative numbers $ 
\frac{f_{cyc}(e)}{f_{acyclic}(e)}$ is not even defined!
\end{proof}

As a different relaxation we could drop the capacity constraints, but still insist on an acyclic flow that is maximum 
on an arc $e$. In the normal Max-Flow Problem this reduces the problem to finding a shortest path between the source 
and the sink. 

In our case , if there is only one source and one sink, we have to decide whether there is a simple path from the 
source to the sink that contains the maximized arc $e$. If there are more sources and sinks, we already have the 
problem of finding a set of simple paths. %TODO haerte des Problems? Wie loest man es?

%
Is this problem easy to solve, and how to solve it? TODO
%


\newpage
\section{Complexity}In this chapter we will discuss the computational complexity of the \textit{Acyclic Flowbound Problem}. We show that 
Cycle-free Min-Cost-Flow with negative weights is $\mathcal{NP}$-complete, and that our problem is just a special case 
of this problem. \\

$\mathcal{NP}$-completeness is maybe the most important concept in the classification of combinatorial optimization 
problems (we will define it in a second). The $\mathcal{P-NP}$-Problem is an open question for many years now. Many 
computer scientists conjecture that in fact $\mathcal{P}\neq \mathcal{NP}$ and the $\mathcal{NP}$-complete algorithmical 
problems are hard to solve. 

The classical formal definition of the class $\mathcal{NP}$ in complexity theory uses formal languages and automata 
theory (namely Turing Machines). For this formal definition we refer to the textbook of Korte and Vygen 
\cite{KorteVygenCombOpt2007}. If we accept to be a bit less formal we can define it the following way:
\begin{definition}
%  A decision problem is an algorithmic problem that has yes or no as solution.
%  A polynomial algorithm is a procedure that 

% A combinatorial decision problem is in the complexity class $\mathcal{NP}$ if there is any nondeterministic algorithm 
% which computes the output "yes" in a polynomial number of steps.

A combinatorial decision problem (a problem with the possible solutions "yes" and "no") is in the complexity class 
$\mathcal{NP}$ iff there exists an algorithm that, given the problem and an additional input (the certificate), 
computes "yes" in a polynomial bounded number of steps if and only if the instance is a "yes"-instance of the problem 
\textit{and} the right certificate is given, otherwise false.\\
The set of problems where there exists an algorithm that always finds the correct answer in a polynomial (in the input 
size) bounded number of steps is called the complexity class $\mathcal{P}$. 
\end{definition}

Obviously all problems for which polynomial algorithms are known are in $\mathcal{NP}$, so 
$\mathcal{P}\subseteq\mathcal{NP}$. We do not know if polynomial algorithms (without the "certificate") exist for all 
problems in $\mathcal{NP}$. There are problems in $\mathcal{NP}$ for which no subexponential algorithm is known so far, 
for example the SAT problem. For some of them one can show that a polynomial algorithm for this problem already would 
imply a polynomial algorithm for all others. 

\begin{definition}
 A problem \textit{P} is called $\mathcal{NP}$-complete if \textit{P}$\in\mathcal{NP}$ and for every problem 
\textit{P'} $\in \mathcal{NP}$ there exists an algorithm with polynomial bounded running time if it can call an 
oracle for \textit{P}, i.e. a blackbox algorithm solving \textit{P} in a single step.
\end{definition}

Stephen Cook proved that every problem in $\mathcal{NP}$ can be solved by a polynomial algorithm if the SAT (boolean 
satisfiability) problem can be solved polynomially \cite{Cook:1971:CTP:800157.805047}. 
Richard Karp presented the result that the hamiltonian cycle problem (and by this also the hamiltonian path problem) 
can be reduced to SAT (\cite{Karp1972}). Thus these problems are $\mathcal{NP}$-complete.

We want to show that the acyclic minimum cost flow problem, of which the acyclic flowbound problem is a special case, 
is an $\mathcal{NP}$-complete problem. This justifies to deal with algorithms that either are not provable to have 
polynomial worst case running time or else are no exact algorithms but rather heuristics to find a weaker bound.


%TODO kann man für das Unterproblem selbst auch etwas zeigen? Das 
%könnte ja durchaus auch Polynomiell sein!

\begin{definition}
 We call the problem of finding a Minimum Cost Flow as in Definition \ref{def:mincostflow} in a given 
network $G$ with the additional constraint  that there is no cyclic flow the \textit{Acyclic Minimum Cost Flow} 
Problem. 

The corresponding decision problem is to decide the question if there can be any acyclic flow with total cost of at 
most $x$ in the network. With binary search on the values of $x$ we can solve any optimization problem by iteratively 
solving the corresponding decision problem - so since both have the same complexity, we can speak of $\mathcal{NP}$ 
completeness of the optimization problem as well . 
\end{definition}

\begin{theorem}
 The \textit{Acyclic Minimum Cost Flow Problem} with arbitrary arc weights is $\mathcal{NP}$ complete.
\end{theorem}
\begin{proof}
 We have to show two things: The problem is in the complexity class $\mathcal{NP}$ , and the problem is indeed 
$\mathcal{NP}$ hard. We do this via two lemmas:
\begin{lemma}
 The Acyclic Minimum Cost Flow Problem is in the complexity class $\mathcal{NP}$
\end{lemma}
\begin{proof}
It is easy to see that the problem is in $\mathcal{NP}$. Given a valid solution to the decision problem we can check if 
it is a feasible network flow by just checking the flow balances on all nodes, which needs only linear time 
$\mathcal{O}(|V|+|A|)$. We can also easily check if the total cost is smaller or equal than demanded and that all 
capacities are respected by running over all arcs and checking them, thus this part is also linear.
We can also check the acyclicity in polynomial time: On each of the arcs we can start a graph search, considering arcs 
only in their flow direction. If this graph search comes back to the original arc again, there is a cycle. %TODOprove?
If graph search does not find a way to the tail of the original arc, there is no cyclic flow on this arc. The graph 
search is also a linear algorithm. If we do this for each arc, the running time is still quadratic.
\end{proof}

\begin{lemma}
 The Acyclic Minimum Cost Flow Problem is $\mathcal{NP}$-hard.
\end{lemma}
\begin{proof}

To show the $\mathcal{NP}$ hardness, we reduce the problem to the \textit{undirected $s-t$-Hamiltonian Path Problem} 
which is well known as a standard example for $\mathcal{NP}$ completeness. Hamiltonian cycle is $\mathcal{NP}$ 
complete both in the directed and undirected case, see \cite{Karp1972}. To find an $s-t$-hamiltonian path in a graph 
$G$ for some vertices $s,t\in V(G)$ we add one extra vertex $x$ and the edges $(t,x)$ and $(x,u)$ to $G$. This new 
graph has a hamiltonian cycle if and only if $G$ has a hamiltonian path from $s$ to $t$. \\
%done zitieren wo hampath vorkommt,NP-haertebeweis

Given an instance of the $s-t$-Hamiltonian Path problem, we have to transform it into an instance of our problem. Then 
we show that the reduced problem is a yes-instance of the Acyclic Min Cost Flow decision problem if and only if the 
original instance of the Hamiltonian Path problem is a yes-instance. Further the reduction must be computable in 
polynomial time.
 
\begin{figure}[h!]
\centering
\begin{tikzpicture}
\node (s) at (0,0)[vertex, label=left:s]{};
\node (t) at (0,1)[vertex, label=left:t]{} ;
\node (3) at (1,0)[vertex]{};
\node (4) at (1,1.7)[vertex]{};
\node (5) at (1.7,1)[vertex]{};

% \draw[edgeflow] (s)--(3)--(4)--(5)--(t);
\draw[edge] (s)--(3)--(4)--(5)--(t);
\draw[edge] (t)--(s)--(5)--(3);

%now draw the transformed graph
\node (s2) at (3,0)[vertex, label=left:s]{};
\node (t2) at (3,1)[vertex, label=left:t]{} ;
\node (32) at (4,0)[vertex]{};
\node (42) at (4,1.7)[vertex]{};
\node (52) at (4.7,1)[vertex]{};

\draw[edgeflow] (s2)--(32)--(42)--(52)--(t2);
\draw[edge] (s2)--(32)--(42)--(52)--(t2);
\draw[edge] (t2)--(s2)--(52)--(32);
\end{tikzpicture}
\caption{This graph has an $s-t$-hamilton path contained, marked in blue in the second picture}
\label{bild:hampath}
\end{figure}

An $s-t$-Hamiltonian Path is a path in a network starting at node $s$ and ending at node $t$ that is visiting every 
node of $G$ exactly once like in the example in figure \ref{bild:hampath}. 

This can only work in connected graphs, so we know that if $n$ is the number of nodes in the 
graph, the Hamiltonian Path has to have size $n-1$. Our reduction is the following:

Transform the $s-t$-Hamiltonian Path instance into an instance of a flow network where for each edge $\{uv\}$ there 
are two directed arcs $(u,v),(v,u)$, the arcs $a\in A$ all have capacities $c(a)=[0,1]$ and costs of $-1$ 
with the same underlying graph as before. The nodes $s$ and $t$ become the source and sink of the flow.
The decision we ask for is whether there is a flow with total costs of at most $-(n-1)$ or not. (Figure 
\ref{bild:hampathreduct} shows an example of such a transformation)

 
\begin{figure}[h!]
\centering
\begin{tikzpicture}
\node (s) at (0,0)[vertex, label=left:s]{};
\node (t) at (0,1)[vertex, label=left:t]{} ;
\node (3) at (1,0)[vertex]{};
\node (4) at (1,1.7)[vertex]{};
\node (5) at (1.7,1)[vertex]{};

\draw[edgeflow] (s)--(3)--(4)--(5)--(t);
\draw[edge] (s)--(3)--(4)--(5)--(t);
\draw[edge] (t)--(s)--(5)--(3);

%now draw the transformed graph
\node (s2) at (3,0)[vertex, label=left:s]{};
\node (t2) at (3,1)[vertex, label=left:t]{} ;
\node (32) at (4,0)[vertex]{};
\node (42) at (4,1.7)[vertex]{};
\node (52) at (4.7,1)[vertex]{};

\draw[curvedarcflow](s2)to(32);
\draw[curvedarcflow](32)to(42);
\draw[curvedarcflow](42)to(52);
\draw[curvedarcflow](52)to(t2);
\draw[curvedarc](s2)to(32);
\draw[curvedarc](32)to(s2);
\draw[curvedarc](42)to(32);
\draw[curvedarc](32)to(42);
\draw[curvedarc](42)to(52);
\draw[curvedarc](52)to(42);
\draw[curvedarc](t2)to(52);
\draw[curvedarc](52)to(t2);
\draw[curvedarc](s2)to(32);
\draw[curvedarc](s2)to(52);
\draw[curvedarc](52)to(s2);
\draw[curvedarc](52)to(32);
\draw[curvedarc](32)to(52);
\draw[curvedarc](t2)to(s2);
\draw[curvedarc](s2)to(t2);

\end{tikzpicture}
\caption{A graph with an $s-t$-hamilton path is transformed to an acyclic minimum cost flow instance with costs of $-1$ 
per arc. Hence in this example we find a minimum acyclic flow of cost $-4=-n+1$}
\label{bild:hampathreduct}
\end{figure}

It is obviously a polynomial time transformation because only a linear number of flags and arcs is added to the 
network, while the edges are deleted when they get replaced with the new arcs.

It yields the same decision as the original Hamiltonian Path instance:
\begin{itemize}
 \item $\Rightarrow:$ Given an instance of the Acyclic Min Cost Flow which has a acyclic flow from $s$ to $t$ with cost 
$-n+1$.
Flow can have a value of at most $1$ on each arc, so each arc the flow is going through can only contribute up to $-1$ 
costs to the cost function. The flow could be split up into different paths from $s$ to $t$, but we know that there are 
no cycles. That means on every of these paths every node is visited only once - if we see it twice we have closed a 
cycle. The cost of the flow is minus the average length of the paths of flow. So we can conclude there is at least one 
path with length $n-1$ that is seeing no node twice - hence it has to see every node exactly once, so we found (at 
least) one $s-t$-Hamiltonian Path and know it is a yes-instance of Hamiltonian Path too.

\item $\Leftarrow:$ Given a graph together with an $s-t$-Hamiltonian Path. In the transformed instance of Min Cost Flow 
we get immediately a feasible $s-t$-Flow by setting the flow value on all forward arcs on the given hamiltonian path 
equals $1$. This flow also has total costs of $-(n-1)$.
\end{itemize}
\end{proof}

These two lemmas together show that our problem of finding an acyclic flow with minimum costs and arbitrary weight 
function is in general $\mathcal{NP}$-complete.
\end{proof}

\newpage
\section{Implementation}
\newpage
\section{Practical Results}
\newpage
\section{Conclusion}
>>>>>>> 46d867bf17e898b083e9dc984cd04d8ff4217fb3
% \newpage
% \section*{Literature}

\newpage
\newpage
\addcontentsline{toc}{section}{Literaturverzeichnis}
\bibliographystyle{amsplain}
\bibliography{literatur}
\nocite{*}

\end{document}
