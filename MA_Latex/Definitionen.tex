% Wir betrachten folgendes Problem:\\
% Gegeben sei ein gerichteter Graph $\overrightarrow{G}=(V,A)$. Die Knoten des Graphen haben ein beliebig großes 
% Intervall möglicher dort ein- oder ausgespeister Gasmengen $\in [-\infty, \infty]$. Die Kanten des Graphen haben ein 
% Intervall zulässigen Flusses $f_a\in [\underline{c_a}, \overline{c_a}]$ gegeben, das nicht verletzt werden darf.
% 
% \begin{definition}
%  Das Problem, in obigem Graphen für jede Kante die maximale und die minimale mögliche Flussmenge zu bestimmen, so dass 
% die Ein- und Ausspeisemengen erfüllt werden, wird nachfolgend das Flussschrankenproblem genannt.\\
% Mit der zusätzlichen Bedingung, dass es keinen Kreisfluss geben darf, nennen wir es azyklisches Flussschrankenproblem. 
% \end{definition}


% möglicherweise einfach den entsprechenden Teil aus 
% http://www.zib.de/groetschel/teaching/WS1213/Skriptum_ADM_I_130326.pdf kopieren?