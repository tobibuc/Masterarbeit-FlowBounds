%zusammenfassung: was war Fragestellung, was wurde gemacht und welche Ergebnisse! hat die Arbeit gebracht? 
%sollte so geschrieben sein, dass jemand der die Arbeit nicht komplett gelesen hat/lesen will die Ergebnisse versteht

In this thesis we wanted to see if the bounds on flow values can be improved if we take into account that the flow has 
to be acyclic. 
We therefor defined the acyclic flowbound problem which turns out to be an interesting mathematical and algorithmical 
problem. We could show that the gap between cyclic and acyclic flow is not bounded and exhibit relations to known 
mathematical problems. Two different mixed integer programs modelling this problem as well as a heuristic were 
described, implemented and tested. 

For the more general Acyclic Minimum Cost Flow Problem with arbitrary weights we showed the $\mathcal{NP}$ hardness, 
however it remains an open question if this is only due to the unlimited number of arcs with negative arc weights. If 
so, there 
might be a chance to find a direct, efficient algorithm.
Despite this the implemented MIP models showed a huge increase in running time even for moderate problem sizes. This 
holds for two different exact MIP models tested as well as for their relaxations. The heuristics results were 
relatively weak but still improve the bounds used until now.

Small problems are solvable with both MIP models described and implemented to optimality. With 
enough time and effort also a limitation by solvers running time or MIP nodes yields good bounds. \\


There are more practical improvements to the algorithms one could try to implement: 

In a graph there can be induced paths (i.e. neighboring vertices with degree 2).
On a path the flow value has to be the same on each edge, so also the bounds have to be the same and it is sufficient 
to compute the bound on one arc of a path and automatically set the same bound on the other arcs of this path.

Another improvement could be to immediately write the computed bounds into the network for all further computations. 
Currently every arcs bounds are computed on the same network with default flow bounds. It might improve the algorithm 
if the already computed bounds are set directly into the network and used when another arcs bound is computed. It could 
also work to compute bounds only on a small selection of interesting arcs and propagate these bounds. In an incomplete 
acyclicity constraint separation algorithm like what we used for the gaslib-582 networks there are sometimes arcs where 
the MIP solver did find a good solution, while it did not find one on neighboring arcs. Since the heuristic in a way 
propagates flow bounds it even could improve the bounds to let the heuristic run again after setting the bounds already 
found.

A smaller network of course would help a lot for the solvers, since the number of variables could be decreased by large 
numbers. The interesting part for the acyclic flowbound problem are strongly connected components, i.e. 
components where for each pair of nodes there are at least 2 different paths connecting them with no arcs in common. 
Thus one idea to make the models work in practice would be to just isolate the strongly connected components and 
compute acyclic flow bounds on each of them seperately. While the worst case complexity stays the same, real networks 
have a nice structure (for example planar and with a small tree width).\\

There might also be models or algorithms than the ones described in this thesis. The results show that in 
practice some relevant improvements on the flow bounds can be achieved by taking into account that gas flow has to be 
acyclic. It also shows the differences between cyclic and acyclic flow bounds can be huge not only in theory (the gap 
can be unbounded as we have shown) but also in practice.\\

On the theoretical side it would of course be interesting to have a direct proof of $\mathcal{NP}$ hardness or a better 
algorithm. 

Another interesting question could be if the number of acyclicity constraints can be bounded by proving more cases for 
redundancies or finding an example that needs exponential numbers of them. 

