
Today there are various real-world problems that are (tried to be) solved by computational methods. One of them which 
is currently attacked at the Konrad Zuse Zentrum für Informationstechnik Berlin (ZIB) is the Problem of transporting 
natural gas through a network of pipelines. Motivated by this challenging problem the question of how to produce 
acyclic network flows arises. This problem is not specific for gas networks but for general flow in networks with arc 
capacities and indeed can be seen as just a special case of the famous minimum cost flow problem. 

To my knowledge there is no literature about acyclic network flows despite wide research on various flow problems. 
Thus this thesis will try to develop models and algorithms to handle it and compute acyclic flows.
Whereever we have a network with nonpositive lower capacity bounds, we can make any flow acyclic by shifting flow 
back on the cycles until an edge has zero flow. 
However, we are interested in bounds on the flow which means that it is not enough to find just any acyclic flow. 
For each arc we have a maximum 
capacity and want to know the maximum flow value this arc could have in a feasible acyclic network flow. Thus 
we have to find the \textit{best} acyclic flow for each arc. This problem turns out to be much more challenging.


At first glance it may seem that running a normal Min-Cost-Flow algorithm could give an appropriate bound 
on an arc. If we set the weight $w_e$ on an arc $e$ to 1 or -1 and all other weights to zero, we get 
a valid bound on the maximum or minimum possible flow on the arc. However, this bound is for most of the arcs far from 
reality. The reason is that a Min-Cost-Flow could have cycles within it and the algorithm could maximize the flow by 
sending as much flow as possible on a cycle. In natural gas networks this can not happen (except from compressor 
stations) since the flow is always induced by pressure differences.\\

\section{Outline}
The intention of this thesis is to investigate the problem of flow bounds in a network where the flow is assumed to be 
acyclic. 

We will first define the mathematical/algorithmical problem and examine connections to known problems 
and relaxations of the acyclic flowbound problem. We show that the minimum cost flow problem with general (possibly 
negative) weights becomes $\mathcal{NP}$-hard if we demand acyclic flow. 
Justified by this we model the problem as a mixed integer program. We present two different mixed integer programs 
that model our exact problem. One of the MIP models relies on constraints which each forbid cyclic flow on a cycle of 
the network. A network can have an exponential number of cycles. We show a condition under which an acyclicity 
constraint becomes redundant. Depending on the network we are given this could reduce the number of these constraints 
to be linear in the input size. Still on some networks the condition is not applicable.
We furthermore discuss the idea of an approach that tries to modify and adjust path based algorithms for maximum flow 
or minimum cost flow in order to solve this problem. We present examples to show that this can only produce a 
heuristic algorithm or else requires to solve another hard problem as subroutine. A heuristic algorithm close to the 
one we would get there is described and analysed.

The algorithms/models we discussed were also implemented. In the end we present details of the algorithms that were 
implemented and compare results achieved by them. 

\subsection{Motivation}
Natural gas is one of the most common ressources of energy in germany and makes up more than 20 percent of energy 
consumption. Most of this gas is produced in ressource-rich countries like Russia or Norway and transported to Germany 
through special pipelines. 
%TODO wie sollte ich das projekt am besten referenzieren?
This thesis evolved from a joint research project of the Konrad Zuse Zentrum fuer Informationstechnik Berlin (ZIB) with 
Open Grid Europe (OGE) who operate the largest network of gas pipelines in germany. 
A general overview over the work done at ZIB on the gas network planning in this project is given in 
\cite{FuegenschuhGeisslerGollmeretal.2013}. The problem of finding settings for all active components of the network 
such that given (static) demands and supplies can be sent through the network with respect to all constraints is the 
\textit{nomination validation} problem. A description of the model used and work done at ZIB for the nomination 
validation problem can be found in \cite{PfetschFuegenschuhGeissleretal.2012}. 

The flow of gas through a pipeline network is determined by physical conditions such as pressure and temperature. 
Pressure can be increased in compressor stations and controlled by elements like valves. Pressure loss along a pipe is 
described by ordinary differential equations. 

The computation of such a physically exact flow is difficult due to numerical and algorithmical reasons. Good 
preprocessing routines can help a lot by reducing the domains and model sizes before the main computation is started.

Since the flow of gas in the network is always determined by pressure differences and flow is always going from higher 
pressure to lower pressure we know that flow in the network in general has to be acyclic. The only exception of this 
rule is a compressor station. While  active compressor stations increase
pressure it is possible that they cause cyclic flow. But of course in real gas networks a compressor station would 
normally be placed at a point where it has the highest impact. This is naturally at a place where it is unlikely to 
waste energy by sending flow on a cycle. The pressure would rather be controlled by resistors, valves and control valves 
in order to avoid cycing. So we will assume that the acyclic flow model is reasonable for the gas flow problem.\\

%erklärung warum wir engere Bounds brauchen können
The bounds are expected to be useful for two things in the solving process of the nomination validation problem 
described in \cite{PfetschFuegenschuhGeissleretal.2012}. If we have stronger bounds on the flow values in the network, 
it might improve the models that solve the actual nomination validation problem. 
%TODO detaillierter! Erst das Modell mit dem MIP beschreiben und dass es durch kleinere Intervalle die Zahl der vars 
%verringert
For the MILP approach described in \cite{PfetschFuegenschuhGeissleretal.2012} a linearization of the 
nonlinear model is computed. If the bounds are smaller or flow is even fixed to a specific value, the interval to be 
linearized is smaller. The smaller the interval, the less variables are needed. A small model might improve the running 
time and solvability of the model.

We also hope for improvements on the outer approximation with spatial branching approach. Here improved bounds could 
result in a stronger relaxation and thereby make solving the problem faster. \\
% TODO das MINLP beschreiben und wie sich im MINLP die Modellgroesse verkleinert?

So to have good bounds and reduce problem size is a goal set in order to reduce the overall running time of the 
nomination validation computations. To achieve this it is necessary to find the bounds quickly. This is why we also 
present relaxations/heuristics for the acyclic flowbound problem. The effects of the bounds on the MILP model reduction 
are shown in the last section of this thesis.



% We simplified the problem by assuming that there are no compressor stations or at least they do not cause cyclic flow. 
% At the same time we just ignore the physical properties of the gas. We do not compute or take care of any gas density, 
% pressure or possible pressure loss. Pressure is only used to justify the assumption of acyclicity. This abstraction is 
% justified by the fact that it is weaker than the original problem which also has to fulfill the flow conditions we use. 
% The bounds are only needed for preprocessing and to improve the model size in the end. 

\subsection{Related Work}
The diploma thesis of Thea Göllner \cite{DiplomarbeitTheaGoellner} describes preprocessing techniques for stationary gas 
network optimization models. This is the same model as the one we use here. In her thesis she also describes an 
algorithm for determining flow directions on the arcs of the network and determines cases in which we can fix the flow 
direction even on arcs contained in cycles of a certain kind. To achieve this she makes use of the acyclicity property 
of gas flow as well. 
Fixing a flow is basically saying that the lower flow bound in one direction is zero. Thus her work is a special case 
of the much more general approach on preprocessing acyclic bounds we have in here. The fixing of flow directions that 
she proposes works on induced paths of the network and on cycles where flow can only be coming in from one side. 
She also shows the restrictions of this approach and says that for many cases it is necessary to take pressure or flow 
quantity into account in order to see if one can fix the direction variable. 

Here we aim at computing more general acyclic flow bounds. So we can not only fix the direction of flow in some special 
cases but also give upper bounds on possible flow and bounds in cases were it is not clear which direction flow has to 
take. It still yields flow directions in the cases described. But in practice a combination of our models with the 
flow direction fixing Göllner did could improve runtime and quality, because for big networks we can only use heuristic 
algorithms in practice.



% [TODO kurzer Abriss pber den Inhalt der Arbeit]
% Today more and more real-world problems in the areas of simulation and optimization are solved by mathematical and 
% computational methods. A growing number of these problems can be solved without problems, i.e. even huge instances give 
% an optimal or near optimal solution within seconds. Still, there remain problems that even on modern computers are hard 
% to solve. For these problems it is important to find ways to increase the efficiency of the algorithms. 
% 
% The topic of this thesis arises from the computation of flow in natural gas networks, which is currently developed 
% in the FORNE Project in a cooperation of OGE with universities and research insitutes including ZIB.
% %TODO genaueres zu FORNE? genaueres zum aufbau des Gasnetzes?
% The flow of natural gas in a network is described by nonlinear equations and depends on many parameters, which makes 
% the problem hard to solve. If we can find good upper and lower bounds for the flow on an arc during the preprocessing, 
% we can hope to improve the behavior of the nonlinear solver by giving these tigther bounds. 
% 
% The flow is induced by pressure differences, so in reality there can't be cyclic flow (if we exclude compressor 
% stations). Without the condition of acyclic flow, it is sufficient to run a standard min-cost-flow algorithm where the 
% maximized arc $e$ gets weight $w_e = -1$ and all others are 0. However, the arising bounds are far from optimal. If arc 
% $e$ is contained in any cycle we could decrease the cost by pushing more and more flow around this cycle until the arcs 
% capacity is at its limits.
% 
% This master thesis will deal with the problem of finding a network flow with no directed cycles (acyclic flow), which at 
% the same time maximizes the amount of flow on a specified arc $e$ of the network. We will discuss the complexity, an 
% exact algorithm based on a mixed integer program with separation of inequalities that forbid cycles and also a heuristic 
% approach that yields results much faster (but not optimal).%TODO am ende genauer schreiben was wirklich gemacht wurde
% 
% \subsection{The gas flow problem}
% Although it is mainly the motivation, not really the topic of this thesis, we want to briefly introduce the gas 
% transport problem. For more a detailed description we refer to LINK %TODO link auf ein entsprechendes ZIB-paper!!! 
