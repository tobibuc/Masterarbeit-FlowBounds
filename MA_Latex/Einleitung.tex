Hier kommen Motivation und Hintergrunderklärungen rein. TODO


Today there are various real-world problems that are (tried to be) solved by computational methods. One of them which 
is currently attacked at the Konrad Zuse Zentrum für Informationstechnik Berlin (ZIB) is the Problem of transporting 
natural gas through a network of pipelines. Due to the physics of gas flows this turns out to be a nonlinear, nonconvex 
mixed-integer optimization problem, which is is quite hard to solve. 
For solving the nonlinear constraints, we compute a [linearization... TODO ]
On first glance it may seem that a relaxation to a normal Min-Cost-Flow algorithm could yield an appropriate bound 
on one arc to the problem. If you set the weight on the arc to 1 or -1 and all other weights on zero, you get a 
valid bound on the maximum or minimum possible flow on the arc. However, this bound is for most of the arcs far from 
reality. The reason is, that a Min-Cost-Flow could have cycles within it, and the algorithm could maximize the flow by 
sending it on a cycle. In natural gas networks this cannot happen, since the flow is always induced by pressure 
differences. So (ignoring that compressor stations could increase pressure) there is no cyclic flow. 
The intention of this thesis now is to investigate the problem of flow bounds in a network where the flow is assumed 
to be acyclic. 

[TODO kurzer Abriss pber den Inhalt der Arbeit]