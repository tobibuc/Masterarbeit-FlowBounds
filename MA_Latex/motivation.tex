\subsection{Motivation}
Natural gas is one of the most common ressources of energy in germany and makes up more than 20 percent of energy 
consumption. Most of this gas is produced in ressource-rich countries like Russia or Norway and transported to Germany 
through special pipelines. 
%TODO wie sollte ich das projekt am besten referenzieren?
This thesis evolved from a joint research project of the Konrad Zuse Zentrum fuer Informationstechnik Berlin (ZIB) with 
Open Grid Europe (OGE) who operate the largest network of gas pipelines in germany. 
A general overview over the work done at ZIB on the gas network planning in this project is given in 
\cite{FuegenschuhGeisslerGollmeretal.2013}. The problem of finding settings for all active components of the network 
such that given (static) demands and supplies can be sent through the network with respect to all constraints is the 
\textit{nomination validation} problem. A description of the model used and work done at ZIB for the nomination 
validation problem can be found in \cite{PfetschFuegenschuhGeissleretal.2012}. 

The flow of gas through a pipeline network is determined by physical conditions such as pressure and temperature. 
Pressure can be increased in compressor stations and controlled by elements like valves. Pressure loss along a pipe is 
described by ordinary differential equations. 

The computation of such a physically exact flow is difficult due to numerical and algorithmical reasons. Good 
preprocessing routines can help a lot by reducing the domains and model sizes before the main computation is started.

Since the flow of gas in the network is always determined by pressure differences and flow is always going from higher 
pressure to lower pressure we know that flow in the network in general has to be acyclic. The only exception of this 
rule is a compressor station. While  active compressor stations increase
pressure it is possible that they cause cyclic flow. But of course in real gas networks a compressor station would 
normally be placed at a point where it has the highest impact. This is naturally at a place where it is unlikely to 
waste energy by sending flow on a cycle. The pressure would rather be controlled by resistors, valves and control valves 
in order to avoid cycing. So we will assume that the acyclic flow model is reasonable for the gas flow problem.\\

%erklärung warum wir engere Bounds brauchen können
The bounds are expected to be useful for two things in the solving process of the nomination validation problem 
described in \cite{PfetschFuegenschuhGeissleretal.2012}. If we have stronger bounds on the flow values in the network, 
it might improve the models that solve the actual nomination validation problem. 
%TODO detaillierter! Erst das Modell mit dem MIP beschreiben und dass es durch kleinere Intervalle die Zahl der vars 
%verringert
For the MILP approach described in \cite{PfetschFuegenschuhGeissleretal.2012} a linearization of the 
nonlinear model is computed. If the bounds are smaller or flow is even fixed to a specific value, the interval to be 
linearized is smaller. The smaller the interval, the less variables are needed. A small model might improve the running 
time / solvability of the model.

We also hope for improvements on the outer approximation with spatial branching approach. Here improved bounds could 
result in a stronger relaxation and thereby make solving the problem faster. \\
% TODO das MINLP beschreiben und wie sich im MINLP die Modellgroesse verkleinert?

So we set the goal to have good bounds and reduce problem size in order to reduce the overall running time of the 
nomination validation computations. To achieve this it is necessary to find the bounds quickly. This is why we also 
present relaxations/heuristics for the acyclic flowbound problem. The effects of the bounds on the MILP model reduction 
are shown in the last section of this thesis.



% We simplified the problem by assuming that there are no compressor stations or at least they do not cause cyclic flow. 
% At the same time we just ignore the physical properties of the gas. We do not compute or take care of any gas density, 
% pressure or possible pressure loss. Pressure is only used to justify the assumption of acyclicity. This abstraction is 
% justified by the fact that it is weaker than the original problem which also has to fulfill the flow conditions we use. 
% The bounds are only needed for preprocessing and to improve the model size in the end. 

\subsection{Related Work}
The diploma thesis of Thea Göllner \cite{DiplomarbeitTheaGoellner} describes preprocessing techniques for stationary gas 
network optimization models. This is the same model as the one we use here. In her thesis she also describes an 
algorithm for determining flow directions on the arcs of the network and determines cases in which we can fix the flow 
direction even on arcs contained in cycles of a certain kind. To achieve this she makes use of the acyclicity property 
of gas flow as well. 
Fixing a flow is basically saying that the lower flow bound in one direction is zero. Thus her work is a special case 
of the much more general approach on preprocessing acyclic bounds we have in here. The fixing of flow directions that 
she proposes works on induced paths of the network and on cycles where flow can only be coming in from one side. 
She also shows the restrictions of this approach and says that for many cases it is necessary to take pressure or flow 
quantity into account in order to see if one can fix the direction variable. 

Here we aim at computing more general acyclic flow bounds. So we can not only fix the direction of flow in some special 
cases but also give upper bounds on possible flow and bounds in cases were it is not clear which direction flow has to 
take. It still yields flow directions in the cases described. But in practice a combination of our models with the 
flow direction fixing Göllner did could improve runtime and quality, because for big networks we can only use heuristic 
algorithms in practice.
