\subsection{Motivation}
Natural gas is one of the most common ressources of energy in germany and makes up more than 20 percent of energy 
consumption. Most of this gas is produced in ressource-rich countries like Russia or Norway and transported to Germany 
through special pipelines. 
%TODO wie sollte ich das projekt am besten referenzieren?
This thesis evolved from a joint research project of the Konrad Zuse Zentrum fuer Informationstechnik Berlin(ZIB) with 
Open Grid Europe (OGE) who operate the largest network of gas pipelines in germany. 
A general overview over the work on the gas network planning problem is given in 
\cite{FuegenschuhGeisslerGollmeretal.2013}. The problem of finding settings for all active components of the network 
such that given demands and supplies can be sent through the network with respect to all constraints is the nomination 
validation problem. A description of the model used and work done for the nomination validation problem can be 
found in \cite{PfetschFuegenschuhGeissleretal.2012}. 

The flow of gas through a pipeline network is determined by physical conditions such as pressure and temperature. 
Pressure can be increased in compressor stations and controlled by elements like valves. Pressure loss along a pipe is 
described by ordinary differential equations. 

The computation of the flow is difficult due to numerical and algorithmical reasons. Good preprocessing 
routines can help a lot by reducing the domains and model sizes before the main computation is started.
Since the flow of gas in the network is always determined by pressure differences and flow is always going from higher 
pressure to lower pressure we know that we can exclude any cyclic flow in the network. We can compute a flow with 
maximum value on a specified arc in the network with a standard Minimum Cost Flow algorithm. 
However a Minimum Cost Flow Algorithm usually is allowed to have cyclic flows and thus will maximize flow by shifting 
it around a cycle containing the maximized arc as long 
