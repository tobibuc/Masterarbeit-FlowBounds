\newpage
\subsection{A Simple Heuristic Approach}

So as we have seen, the computations of the MIP formulation with separation of the Acyclicity Constraints could be 
expensive in running time and ressource consumption. The solved problem is a relaxation anyway, so we can as well think 
about different relaxations and heuristic approaches. This chapter will introduce a simple and easy to 
implement heuristic approach. This approach uses a graph transformation and computes a Minmum Cost Flow on the 
transformed graph. We show that we get a valid bound for the Acyclic Flowbound Problem from this transformed graph.\\

The main idea of this approach is to avoid that the same flow is cycling over and over again until it reaches the 
capacity bounds. In order to achieve this we make a copy of the original graph and make the maximized arc the bridge 
between the two. On one part of the graph we only have sources, on the other part we only have sinks. In order to make 
the problem feasible we introduce artificial arcs between the sinks and their counterparts in the other copy of the 
graph. These artificial arcs get a detention cost so they are only used when there is no other way. On this modified 
graph we obtain minimum costs by sending as much flow as possible over the maximized arc. However this flow does not 
respect the capacity constraints neither completely forbids the cyclic flow. 

Now let us describe the approach in a formal way.

\subsubsection*{The Graph Transformation}

The following algorithm \ref{algo:graphtransform} describes how the graph is transformed to the new graph.

%Pseudocode des Algorithmus
\begin{algorithm}
 \caption{graph transforming heuristic}
 \label{algo:graphtransform}
 \begin{algorithmic}[5]
  \Function{MakeTransformedGraph}{$G=(V,A), e\in A$}
  \State create empty graph $G':=(V',A'), V'=A'=\emptyset$
  \State make labels capacity, cost $A'\to \R$% , demand $V'\to\R$
  \State create $s, t\in V$ \Comment{ supersource and supersink}
  \State create $a_0 :=(s,t) \in A'$
  \State capacity$(a_0)\gets\infty$ 
  \State cost$(a_0)\gets 2$ \Comment{highest arc cost in the network}
  \ForAll{$v\in V$}\Comment{make two copies of each node}
    \State create $v_A, v_B\in V'$
    \If{$v$ is source of $G$}
      \State create arc $a:=(s,v_A)\in A'$
      \State capacity$(a)\gets$ supply$(v)$
    \ElsIf{$v$ is sink of $G$}
      \State create arc $a:=(v_B, t)\in A'$
      \State create arc $b:=(v_A,t)\in A'$
      \State cost$(b)\gets 1$
    \EndIf
  \EndFor
  \ForAll{$a=(u,v)\in A$}
    \If{$a=e$}\Comment{for the arc to maximize we only make a bridge, no copy}
      \State create arc $a':=(u_A, v_B)\in A'$
      \State capacity$(a')\gets$ capacity$(e)$
    \Else
      \State create arcs $a_A:=(u_A, v_A), a_B:=(u_B, v_B)\in A'$
      \State capacity$(a_A)=$capacity$(a)$, capacity$(a_b)=$capacity$(a)$
      \State cost$(a_A)=$cost$(a_B)=0$
    \EndIf
  \EndFor
  \EndFunction
 \end{algorithmic}

\end{algorithm}




% \begin{algorithm}
%  \caption{path based heuristic flow bound algorithm}
% \label{algo:pathHeur}
%  %TODO gehe gerade von vorwärts gerichteter e aus, und maximierung
%  \begin{algorithmic}[5]%[0] means no line numbers, n means every nth line number is displayed
%   \Function{Heuristic}{$G=(V,A),\, e=(u,v)\in A,\, b:V\to \R $}%hier übergabeparameter
%     \State define $b':V\to \R,\, b'(v)\gets 0 \forall v\in V$ \Comment{temporary balances}
%     \State define $f: A\to\R,\, f(v)\gets 0 \forall a\in A$
%     \While {$\neg (b+b'\equiv 0) $}\Comment{while there are active source-sink-pairs}
