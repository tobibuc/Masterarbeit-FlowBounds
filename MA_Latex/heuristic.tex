\newpage
\subsection{A Simple Heuristic Approach}

As we have seen, the computations of the MIP formulation with separation of the Acyclicity Constraints could be 
expensive in running time and ressource consumption. The solved problem is a relaxation of the real-world 
problem anyway, so we can as well think 
about different relaxations and heuristic approaches. This chapter will introduce a simple and easy to 
implement heuristic approach inspired by the idea of the former section, where we showed why path augmentation models 
do not work for the acyclic flowbound problem. This approach uses a graph transformation and computes a Minimum Cost 
Flow on the transformed graph. We show that we get a valid bound for the Acyclic Flowbound Problem from this 
transformed graph.\\

The main idea of this heuristic approach is to avoid that the same flow is cycling over and over again until it reaches 
the capacity bounds. The path augmentation of the section before could achieve this goal. But we need to relax the 
capacity bounds and by this we relax also the problem. The proof that it suffices to double the capacities relies on 
the idea of distinguishing the flow before and after the maximized arc, because in every part of this the capacities 
have to be respected. If we double capacities and compute flow in this network it could happen that flow in one part 
uses 1.5 times of an arcs capacity and the other only half of it. So the bounds get in fact tighter if we separate the 
two parts.

In order to achieve this we make two copies of the original graph and make the arc we maximize the 
bridge between the two. On one part of the graph we only have sources, on the other part we only have sinks. 
In order to make the problem feasible we introduce artificial arcs between the sinks and their counterparts in the other 
copy of the graph. These artificial arcs get a detention cost so they are only used when there is no other way. On this 
modified graph we obtain minimum costs by sending as much flow as possible over the maximized arc. We can add the flow 
in both parts in the end, obtaining a flow for the original graph that might be not acyclic and not respecting the 
capacities but has maximum flow value on the arc we maximize. 

Let us describe the approach in a formal way.

\subsubsection*{The Graph Transformation}

The following algorithm \ref{algo:graphtransform} describes how the graph is transformed to the new graph. It mainly 
splits the graph into two copies $X$ and $Y$, sets costs and introduces links between both parts and the 
supersource and supersink.

%Pseudocode des Algorithmus
\begin{algorithm}
 \caption{graph transformation}
 \label{algo:graphtransform}
 \begin{algorithmic}[5]
  \Function{MakeTransformedGraph}{$G=(V,A), e\in A$}
  \State create empty graph $G':=(V',A'), V'=A'=\emptyset$
  \State create labels capacity $A'\to \R^{|A|}$, cost $A'\to \R^{|A|}$% , demand $V'\to\R$
  \State create $s, t\in V$ \Comment{ supersource and supersink}
  \State create $a_0 :=(s,t) \in A'$
  \State capacity$(a_0)\gets\infty$ 
  \State cost$(a_0)\gets 2$ \Comment{highest arc cost in the network}
  \ForAll{$v\in V$}\Comment{make two copies of each node}
    \State create $v_X, v_Y\in V'$
    \If{$v$ is source of $G$}
      \State create arc $a:=(s,v_X)\in A'$
      \State capacity$(a)\gets$ supply$(v)$
    \ElsIf{$v$ is sink of $G$}
      \State create arc $a:=(v_X, t)\in A'$
      \State create arc $b:=(v_Y,t)\in A'$
      \State cost$(a)\gets 1$
    \EndIf
  \EndFor
  \ForAll{$a=(u,v)\in A$}
    \If{$a=e$}\Comment{for the arc to maximize we only make a bridge, no copy}
      \State create arc $e':=(u_X, v_Y)\in A'$
      \State capacity$(e')\gets$ capacity$(e)$
    \Else
      \State create arcs $a_X:=(u_X, v_X), a_Y:=(u_Y, v_Y)\in A'$
      \State capacity$(a_X)\gets$ capacity$(a)$, capacity$(a_Y)\gets$ capacity$(a)$
      \State cost$(a_X)\gets 0$, cost$(a_Y)\gets0$
    \EndIf
  \EndFor
  \State \Return $G'$
  \EndFunction
 \end{algorithmic}

\end{algorithm}

Figure \ref{bild:graphtransform} shows as an example how the graph from the last section would be transformed by this 
algorithm

\begin{figure}[h!]
\centering
\begin{tikzpicture}[scale=0.6]
\node (a) at (0,3)[vertex]{};
\node (s1) at (1,2.5)[source]{-1};
\node (b) at (2,3)[vertex]{};
\node (d) at (2,2)[vertex]{};
\node (c) at (1.5,0)[vertex]{};
\node (s2) at (3,3)[source]{-1};
\node (t) at (1.5,1)[sink]{2};
% \draw[arcflow, cyan, dashed] (s1)--(d)--(b);
% \draw[arcflow, cyan](b)--(a)--(c)--(t);
% \draw[arcflow, yellow, dashed] (s2)--(b)--(a)--(c)--(t);
\draw[fatarc] (b)--(a)node[pos=0.5, above]{$e$};
\draw[edge] (a)--(s1){};
\draw[edge] (s1)--(d){};
\draw[edge] (d)--(b){};
\draw[edge] (b)--(s2){};
\draw[edge] (s2)--(c){};
\draw[edge] (c)--(a){};
\draw[edge] (c)--(t){};
\draw[edge] (t)--(d){};

%supersource und supersink
\node(sups) at (0,-2.5)[source]{-2};
\node(supt) at (7.5,-2.5)[sink]{2};

\node (la) at (0,-1)[vertex]{};
\node (ls1) at (1,-1.5)[source]{};
\node (lb) at (2,-1)[vertex]{};
\node (ld) at (2,-2)[vertex]{};
\node (lc) at (1.5,-4)[vertex]{};
\node (ls2) at (3,-1)[source]{};
\node (lt) at (1.5,-3)[sink]{};
% \draw[fatarc] (lb)--(la)node[pos=0.5, above]{$e$};
\draw[edge] (la)--(ls1){};
\draw[edge] (ls1)--(ld){};
\draw[edge] (ld)--(lb){};
\draw[edge] (lb)--(ls2){};
\draw[edge] (ls2)--(lc){};
\draw[edge] (lc)--(la){};
\draw[edge] (lc)--(lt){};
\draw[edge] (lt)--(ld){};

\node (ra) at (4,-1)[vertex]{};
\node (rs1) at (5,-1.5)[source]{};
\node (rb) at (6,-1)[vertex]{};
\node (rd) at (6,-2)[vertex]{};
\node (rc) at (5.5,-4)[vertex]{};
\node (rs2) at (7,-1)[source]{};
\node (rt) at (5.5,-3)[sink]{};
\draw[fatarc, bend left] (lb)to(ra);
\draw[edge] (ra)--(rs1){};
\draw[edge] (rs1)--(rd){};
\draw[edge] (rd)--(rb){};
\draw[edge] (rb)--(rs2){};
\draw[edge] (rs2)--(rc){};
\draw[edge] (rc)--(ra){};
\draw[edge] (rc)--(rt){};
\draw[edge] (rt)--(rd){};
\draw[arc, green!70!blue] (sups)--(ls1)node[pos=0.5, left]{$cap=1$};
\draw[arc, green!70!blue] (sups)--(ls2)node[pos=0.5, above]{$cap=1$};
\draw[arc, red] (rt)--(supt)node[pos=0.5, below]{$cap=2,\, cost=0$};
\draw[arc, red] (lt)--(supt)node[pos=0.5, above]{$cap=2,\, cost=1$};
\draw[arc, purple!40!blue, bend right =90] (sups)to(supt)node at (4,-5){$cap=\infty,\,cost=2$};

\end{tikzpicture}
\caption{The transformation of the example network from the former section}
 \label{bild:graphtransform}
\end{figure}


With this transformation and any algorithm for Minimum Cost Flow we can set up an algorithm for our problem. 
For the Maximum Flow and Minimum Cost Flow Problem in graphs there are many standard algorithms we could use. The 
classical Maximum Flow algorithm of Ford and Fulkerson \cite{Ford-Fulkerson_algo} arising from their Max-Flow-Min-Cut 
theorem is based on augmenting flow on source-sink-paths in the network as well as the improved algorithms of Edmonds 
and Karp \cite{EdmondsKarp1972} or Dinic \cite{Dinic1970}. Algorithms for Maximum Flow can be used to first check if 
there is any feasible flow in the network. If there is no feasible flow we can give up at this point, while the flow 
problem on the transformed graph is always feasible by construction.

Deriving from the Max-Flow Algorithms there are many algorithms solving the Min-Cost-Flow Problem. Edmonds and 
Karp described a Successive Shortest Path Algorithm in \cite{EdmondsKarp1972}. Goldberg and Tarjan proposed the Minimum 
Mean Cycle Cancelling Algorithm \cite{minMeanCycleCancelling89} with runnning time of $\mathcal{O}(nm(log 
n)\min\{log(nC), m log n\}) $. %TODO fuer die anderen ebenfalls laufzeiten odr fuer keine
Another algorithm is the Network Simplex by Orlin \cite{NetworkSimplexOrlin97} (which was used in the computational 
study) There are a lot more algorithms available for Minimum Cost Flow.

Our algorithm mainly consists of a feasibility test, a transformation of the graph $G$ and a Minimum Cost Flow 
computation on the transformed graph $G'$, see algorithm \ref{algo:simpleheur}. 

\begin{algorithm}
 \caption{simple heuristic}
 \label{algo:simpleheur}
 \begin{algorithmic}
  \Function{FlowBoundHeur}{$G=(V,A), e\in A$}
  \State mf$\gets$ \Call{MaxFlow}{G}
  \If{mf < nominated ingoing flow}
    \State\Return infeasible
  \EndIf
  \State $G'=$ \Call{MakeTransformedGraph}{$G,e$}
  \State $f\gets$ \Call{MinCostFlow}{$G'$}
  \State ub$\gets f(e')$
  \State backwardflow$\gets f(a_0)$
  \State \Return ub - backwardflow
  \EndFunction
 \end{algorithmic}
\end{algorithm}

We show the correctness of the algorithm: %TODO and running time? dependent on used mincostflow!

\begin{figure}[h!]
\centering
\begin{tikzpicture}[scale=0.6]
%supersource und supersink
\node(sups) at (0,-2.5)[source]{-2};
\node(supt) at (7.5,-2.5)[sink]{2};
%left part
\node (la) at (0,-1)[vertex]{};
\node (ls1) at (1,-1.5)[source]{};
\node (lb) at (2,-1)[vertex]{};
\node (ld) at (2,-2)[vertex]{};
\node (lc) at (1.5,-4)[vertex]{};
\node (ls2) at (3,-1)[source]{};
\node (lt) at (1.5,-3)[sink]{};
%right part
\node (ra) at (4,-1)[vertex]{};
\node (rs1) at (5,-1.5)[source]{};
\node (rb) at (6,-1)[vertex]{};
\node (rd) at (6,-2)[vertex]{};
\node (rc) at (5.5,-4)[vertex]{};
\node (rs2) at (7,-1)[source]{};
\node (rt) at (5.5,-3)[sink]{};

\draw[arcflow, cyan] (sups)--(ls2)--(lb);
\draw[arcflow, cyan, bend left] (lb)to (ra);
\draw[arcflow, cyan](ra)--(rs1)--(rd)--(rt)--(supt);
\draw[arcflow, yellow!80!orange, dashed](sups)--(ls1)--(ld)--(lb);
\draw[arcflow, yellow!80!orange, dashed, bend left](lb)to(ra);
\draw[arcflow, yellow!80!orange, dashed](ra)--(rc)--(rt)--(supt);

\draw[edge] (la)--(ls1){};
\draw[edge] (ls1)--(ld){};
\draw[edge] (ld)--(lb){};
\draw[edge] (lb)--(ls2){};
\draw[edge] (ls2)--(lc){};
\draw[edge] (lc)--(la){};
\draw[edge] (lc)--(lt){};
\draw[edge] (lt)--(ld){};
\draw[fatarc, bend left] (lb)to(ra);
\draw[edge] (ra)--(rs1){};
\draw[edge] (rs1)--(rd){};
\draw[edge] (rd)--(rb){};
\draw[edge] (rb)--(rs2){};
\draw[edge] (rs2)--(rc){};
\draw[edge] (rc)--(ra){};
\draw[edge] (rc)--(rt){};
\draw[edge] (rt)--(rd){};
\draw[arc, green!70!blue] (sups)--(ls1)node[pos=0.5, left]{$cap=1$};
\draw[arc, green!70!blue] (sups)--(ls2)node[pos=0.5, right]{$cap=1$};
\draw[arc, red] (rt)--(supt)node[pos=0.5, below]{$cap=2,\, cost=0$};
\draw[arc, red] (lt)--(supt)node[pos=0.5, above]{$cap=2,\, cost=1$};
\draw[arc, purple!40!blue, bend right =90] (sups)to(supt)node at (4,-5){$cap=\infty,\,cost=2$};
\end{tikzpicture}
\caption{This flow has zero costs, is feasible in the heuristic and has the optimum value $f(e)=2$}
 \label{bild:graphtransformWithFlow}
\end{figure}

\begin{prop}
 The heuristic algorithm \ref{algo:simpleheur} returns a valid upper bound for the possible acyclic flow on a given 
 arc $e$ in the network, or infeasible if there is no feasible network flow for the given flow balances at sources and 
 sinks.
\end{prop}

\begin{proof}
\textbf{Feasibility Test: }We have to show that the ordinary feasibility test with maximum flow is sufficient even for 
acyclic flow: If the flow network nomination (in- and outflow) is infeasible, no algorithm for the 
maximum flow problem will find a flow that is fulfilling these balances completely. Then also for the stricter acyclic 
flow problem this is infeasible. The other direction is also true: if there is a feasible flow, there also is a 
feasible acyclic flow:

Take the maximum flow and choose one flow cycle within (If there is none we are done). On all arcs of the cycle the 
flow is going in the same direction in relation to the cycle. We can choose the minimum flow on the cycles arcs and 
reduce flow on each arc of the cycle by this amount. The resulting flow is still a feasible flow, but we removed the 
flow cycle. This can be done until the flow is acyclic. But if there is any acyclic flow, there also has to be an 
acyclic flow that is maximum on the arc we are interested in. So in order to test feasibility it is enough to run a 
ordinary maximum flow algorithm.\\


\textbf{Relaxation: }To prove that we get an upper bound with this heuristic we have to show that the result we 
get is the maximum value of a relaxation of the problem. 

If a flow is acyclic we can decompose it into flow on a set of simple paths. This means that 
the Acyclic Flowbound Problem could theoretically be solved by an algorithm augmenting simple paths (in the 
right order) where we put some extra constraints and objectives on these paths. These extra constraints have to forbid 
the closing of any flow cycle by the path as well as forcing the path to use arc $e$ if possible. (Section 
\ref{model:pathaugment} describes this idea and shows the difficulties of choosing the right path for augmentation.)
%TODO koennte das modell einzeln hinschreiben mit 
%hinweis auf NP schwere von constrained shortest path. anschließend kommt das relaxierungsargument besser !!!
%Außerdem hätte ich damit shconmal ein gutes modell

The heuristic algorithm is a relaxation of this constrained path augmentation model. It still forces all paths to go 
over $e$ if possible (which might be much more since it is a problem with weaker conditions). 
We can reconstruct a flow in the original graph from the flow in the transformed graph by summing up the flow values on 
both copied arcs, assigning this flow to the original arc: $f(a)=f'(a_X)+f'(a_Y)$ and the assigned flow value of the 
bridge arc to the maximized arc of the original problem. 

All the flow conservation constraints still hold after constructing the flow in the original graph from the two copies:
\begin{align*}
& \sum_{a\in \delta^+(v)}q_a - \sum_{a\in\delta^- (v)}q_a &=& d_v\ &\forall v\in V_X \\
+& \sum_{a\in \delta^+(v)}q_a - \sum_{a\in\delta^- (v)}q_a &=& d_v\ &\forall v\in V_Y \\
=& \sum_{a\in \delta^+(v_X)\cup\delta^+(v_Y)}q_a - \sum_{a\in\delta^- (v_X)\cup\delta^-(v_Y)}q_a &=& d_v\ &\forall 
v_X\in V_X, v_Y\in V_Y \\
=& \sum_{a\in \delta^+(v)}q_a - \sum_{a\in\delta^- (v)}q_a &=& d_v\ &\forall v\in V \\
\end{align*}

The flow balances at sources and sinks are set right by construction.\\

When no more flow can be sent over the maximized arc $e$ the path augmentation model already gives a valid bound 
if the flow is feasible. But in fact this model can determine the amount of flow that is necessary to augment backwards 
on $e$ in a path model.

We distinguish whether the flow is going over the bridge $e$, going to a sink node in already in part $X$ of the 
transformed graph and thus having costs of 1 per unit, 
or is going from $s$ to $t$ directly with a cost of 2 per unit. The flow with costs of 2 cannot be sent over $e$ in 
forward direction neither find any sink in part $X$ of the graph. If there is a feasible flow in $G$ (which we tested) 
the only possible way is that the remaining flow has to go backward over $e$. So we can reduce the flow on the 
bridge by this amount of flow directly going from $s$ to $t$ and the bound is still valid. (In algorithm 
\ref{algo:pathHeur} from the section before this would be the part in line \ref{heur:lineAugCase3} where flow is 
augmented backwards).\\

$\Rightarrow$ Since the problem is just a relaxation of the original problem/the constrained simple path model, the 
obtained bound is always weaker (in this case $\ge$) than the optimal bound for maximum acyclic flow. 
$\Rightarrow$ the heuristic is correct for finding upper bounds for the acyclic flow on an arc.

\end{proof}%TODO der beweis ist lang und unuebersichtlich, man koennte fragen wozu man manche erklaerungen braucht.

% So the algorithm relaxes the capacity constraints and allows weaker kinds of cycling. It still maximizes the amount 
of 
% flow going over the maximized arc $e$ (the bridge): if there is any flow that can be sent over $e$ this is the 
cheapest 
% way. All flows that are not passing $e$ at some point have to use the arcs from part $X$ of the transformed graph 
% directly to the super-sink $t$ (which causes cost $=1$ per flow unit) or even worse from the super-source $s$ 
directly 
% to the super-sink $t$ (which causes cost $=2$ per flow unit). To go over $e$ has no cost at all, therefore 
% the paths with positive costs are only used if there is a directed minimal cut from part $X$ to part $Y$ where no arc 
% has free capacity.

So we have seen that we can compute an upper bound via the optimum for this relaxed problem with a fast standard 
algorithm for maximum flows that can be retranslated into the original problem. 
However, the capacity constraints of the original problem might be violated after retranslation. We set the arcs 
capacities in part $X$ and part $Y$ both to the value of the original capacities. Thus it might happen that actual flow 
value in the original graph sums up to a higher value than the capacity allows (up to twice the capacity). 


\begin{figure}[h!]
\centering
\begin{tikzpicture}
\node (a) at (0,-6)[vertex]{};
\node (s1) at (1,-6.5)[source]{-1};
\node (b) at (2,-6)[vertex]{};
\node (d) at (2,-7)[vertex]{};
\node (c) at (1.5,-9)[vertex]{};
\node (s2) at (3,-6)[source]{-1};
\node (t) at (1.5,-8)[sink]{2};
% \draw[arcflow, cyan, dashed] (s1)--(d)--(b);
\draw[arcflow, cyan](s2)--(b);
\draw[arcflow, cyan](b)--(a)--(s1)--(d);
\draw[arcflow, cyan](d)--(t);
\draw[arcflow, yellow!70!orange, dashed] (s1)--(d);
\draw[arcflow, yellow!70!orange, dashed] (d)--(b)--(a)--(c);
\draw[arcflow, yellow!70!orange, dashed] (c)--(t);
\draw[fatarc] (b)--(a)node[pos=0.5, above]{$e$};
\draw[edge] (a)--(s1){};
\draw[edge] (s1)--(d)node(errpos)[pos=0.5]{};
\draw[edge] (d)--(b){};
\draw[edge] (b)--(s2){};
\draw[edge] (s2)--(c){};
\draw[edge] (c)--(a){};
\draw[edge] (c)--(t){};
\draw[edge] (t)--(d){};
\node (err) at (-1,-7)[ellipse, red, draw=red]{ $f(a)=2>cap(a)=1$};
\draw[->, stealth, semithick, bend right,red](err)to(errpos);
\draw[ultra thick, ->, red!50] (1.55,-6.2) arc (30:330:0.2);
% \draw[pattern=dots, pattern color=red](a)--(s1)--(b);%(0,-6)--(1,-6)--(1,-5);
%TODO man könnte doch kreisfluss umschlossenes gebiet rot füllen?
\end{tikzpicture}
\caption{If we retransform the flow from figure \ref{bild:graphtransformWithFlow} (which is feasible in the heuristic) 
to the original network we get a violated capacity constraint on the arc with $c=1$. The flow is not acyclic anymore 
either.}
 \label{bild:graphretransformWithFlow}
\end{figure}
Also we cannot guarantee the acyclicity anymore. It might happen that we have cyclic flow in our 
original graph $G$ even if the flow in the transformed graph $G'$ was acyclic. This happens when paths from 
the different parts $X$ and $Y$ of $G'$ are crossing after they are brought back to the original graph $G$ to 
build a solution there. For an example see figure \ref{bild:graphretransformWithFlow} which is the retransformation of 
the example flow in figure \ref{bild:graphtransformWithFlow}.