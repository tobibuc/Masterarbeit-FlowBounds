\newpage
\subsection{A Simple Heuristic Approach}

As we have seen, the computations of the MIP formulation with separation of the Acyclicity Constraints could be 
expensive in running time and ressource consumption. The solved problem is a relaxation of the real-world 
problem anyway, so we can as well think 
about different relaxations and heuristic approaches. This chapter will introduce a simple and easy to 
implement heuristic approach. This approach uses a graph transformation and computes a Minimum Cost Flow on the 
transformed graph. We show that we get a valid bound for the Acyclic Flowbound Problem from this transformed graph.\\

The main idea of this approach is to avoid that the same flow is cycling over and over again until it reaches the 
capacity bounds. In order to achieve this we make two copies of the original graph and make the maximized arc the 
bridge between the two. On one part of the graph we only have sources, on the other part we only have sinks. 
In order to make the problem feasible we introduce artificial arcs between the sinks and their counterparts in the other 
copy of the graph. These artificial arcs get a detention cost so they are only used when there is no other way. On this 
modified graph we obtain minimum costs by sending as much flow as possible over the maximized arc. However this flow 
does not respect the capacity constraints neither completely forbids the cyclic flow. 

Let us describe the approach in a formal way.

\subsubsection*{The Graph Transformation}

The following algorithm \ref{algo:graphtransform} describes how the graph is transformed to the new graph.

%Pseudocode des Algorithmus
\begin{algorithm}
 \caption{graph transformation}
 \label{algo:graphtransform}
 \begin{algorithmic}[5]
  \Function{MakeTransformedGraph}{$G=(V,A), e\in A$}
  \State create empty graph $G':=(V',A'), V'=A'=\emptyset$
  \State create labels capacity $A'\to \R$, cost $A'\to \R$% , demand $V'\to\R$
  \State create $s, t\in V$ \Comment{ supersource and supersink}
  \State create $a_0 :=(s,t) \in A'$
  \State capacity$(a_0)\gets\infty$ 
  \State cost$(a_0)\gets 2$ \Comment{highest arc cost in the network}
  \ForAll{$v\in V$}\Comment{make two copies of each node}
    \State create $v_A, v_B\in V'$
    \If{$v$ is source of $G$}
      \State create arc $a:=(s,v_A)\in A'$
      \State capacity$(a)\gets$ supply$(v)$
    \ElsIf{$v$ is sink of $G$}
      \State create arc $a:=(v_A, t)\in A'$
      \State create arc $b:=(v_B,t)\in A'$
      \State cost$(a)\gets 1$
    \EndIf
  \EndFor
  \ForAll{$a=(u,v)\in A$}
    \If{$a=e$}\Comment{for the arc to maximize we only make a bridge, no copy}
      \State create arc $e':=(u_A, v_B)\in A'$
      \State capacity$(e')\gets$ capacity$(e)$
    \Else
      \State create arcs $a_A:=(u_A, v_A), a_B:=(u_B, v_B)\in A'$
      \State capacity$(a_A)\gets$ capacity$(a)$, capacity$(a_b)\gets$ capacity$(a)$
      \State cost$(a_A)\gets 0$, cost$(a_B)\gets0$
    \EndIf
  \EndFor
  \Return $G'$
  \EndFunction
 \end{algorithmic}

\end{algorithm}

%TODO Beispiel mit Bild
TODO BILD-Beispiel 

With this transformation and any algorithm for Mininimum Cost Flow we can set up an algorithm for our problem. 
For the Maximum Flow and Minimum Cost Flow Problem in graphs there are many standard algorithms we could use. The 
classical Maximum Flow algorithm of Ford and Fulkerson \cite{Ford-Fulkerson_algo} arising from their Max-Flow-Min-Cut 
theorem is based on augmenting flow on source-sink-paths in the network as well as the improved algorithms of Edmonds 
and Karp \cite{EdmondsKarp1972} or Dinic \cite{Dinic1970}. Algorithms for Maximum Flow can be used to first check if 
there is any feasible flow in the network. If there is no feasible flow we can give up at this point, while the flow 
problem on the transformed graph is always feasible by construction.

Deriving from the Max-Flow Algorithms there are many algorithms solving the Min-Cost-Flow Problem. Edmonds and 
Karp described a Successive Shortest Path Algorithm in \cite{EdmondsKarp1972}. Goldberg and Tarjan proposed the Minimum 
Mean Cycle Cancelling Algorithm \cite{minMeanCycleCancelling89} with runnning time of $\mathcal{O}(nm(log 
n)\min\{log(nC), m log n\}) $.%TODO fuer die anderen ebenfalls laufzeiten odr fuer keine
Another algorithm is the Network Simplex by Orlin \cite{NetworkSimplexOrlin97} (which was used in the computational 
study) and there are a lot more algorithms available for Minimum Cost Flow.

Our algorithm mainly applies a feasibility test, a transformation of the graph $G$ and a Minimum Cost Flow computation 
on the transformed graph $G'$, see algorithm \ref{algo:simpleheur}

\begin{algorithm}
 \caption{simple heuristic}
 \label{algo:simpleheur}
 \begin{algorithmic}
  \Function{FlowBoundHeur}{$G=(V,A), e\in A$}
  \State mf$\gets$ \Call{MaxFlow}{G}
  \If{mf < nominated ingoing flow}
    \State\Return infeasible
  \EndIf
  \State $G'=$ \Call{MakeTransformedGraph}{$G,e$}
  \State $f\gets$ \Call{MinCostFlow}{$G'$}
  \State ub$\gets f(e')$
  \State backwardflow$\gets f(a_0)$
  \State \Return ub - backwardflow
  \EndFunction
 \end{algorithmic}
\end{algorithm}

Let us show the correctness of the algorithm: %TODO and running time? dependent on used mincostflow!

\begin{prop}
 The heuristic algorithm \ref{algo:simpleheur} returns a valid upper bound for the possible acyclic flow on a given 
  arc $e$ in the network.
\end{prop}

\begin{proof}
 If a flow is acyclic we can decompose it into flow on a set of simple paths. This means that the 
Acyclic Flowbound Problem could theoretically be solved by an algorithm augmenting simple shortest paths where we put 
some extra constraints and objectives on these paths. These constraints have to forbid the closing of any flow cycle by 
the path as well as forcing the path to use arc $e$ if possible. \\ %TODO koennte das modell einzeln hinschreiben mit 
%hinweis auf NP schwere von constrained shortest path. anschließend kommt das relaxierungsargument besser !!!
%Außerdem hätte ich damit shconmal ein gutes modell

Our heuristic algorithm is a relaxation of this constrained simple path model. It still forces all paths to go 
over $e$ if possible. 
We can reconstruct a flow in the original graph from the flow on the transformed graph by adding the flow values on 
both copied arcs and assigning this flow to the original arc: $f(a)=f'(a_A)+f'(a_B)$ and the assigned flow value of the 
bridge arc to the maximized arc of the original problem. All the flow conservation constraints still hold in this flow:
\begin{align*}
& \sum_{a\in \delta^+(v)}x_a - \sum_{a\in\delta^- (v)}x_a &=& d_v\ &\forall v\in V_A \\
+& \sum_{a\in \delta^+(v)}x_a - \sum_{a\in\delta^- (v)}x_a &=& d_v\ &\forall v\in V_B \\
=& \sum_{a\in \delta^+(v_A)\cup\delta^+(v_B)}x_a - \sum_{a\in\delta^- (v_A)\cup\delta^-(v_B)}x_a &=& d_v\ &\forall 
v_A\in V_A, v_B\in V_B \\
=& \sum_{a\in \delta^+(v)}x_a - \sum_{a\in\delta^- (v)}x_a &=& d_v\ &\forall v\in V \\
\end{align*}
However, the capacity constraints of the original problem might be violated! We set the arcs capacities in part A and 
part B both to the value of the original capacities. Thus it might happen that actual flow value in the original graph 
amounts to a higher value than the capacity allows (up to twice the capacity). 

Another Problem is that we cannot guarantee the acyclicity anymore. It might occur that we have cyclic flow in our 
original graph $G$ even if the flow in the transformed graph $G'$ is acyclic. This happens when paths from 
the different parts A and B of $G'$ are crossing when they are brought back to the original graph $G$ to 
build a solution there. For an example see picture TODO
%TODO BILD einfuegen zu kreuzenden kreiserzeugenden pfaden

So the algorithm relaxes the capacity constraints and allows weaker kinds of cycling. It still maximizes the amount of 
flow going over the maximized arc $e$ (the bridge): if there is any flow that can be sent over $e$ this is the cheapest 
way. All flows that are not passing $e$ at some point have to use the arcs from part $A$ of the transformed graph 
directly to the super-sink $t$ or even worse from the super-source $s$ directly to the super-sink $t$. The first way 
has costs of 1 per flow unit and the second has costs of 2 per flow unit. To go over $e$ is for free, therefore the 
paths with positive costs are only used if there is a directed minimal cut where no arc has free capacity.


$\Rightarrow$ Since the problem is just a relaxation of the original problem/the  constrained simple path model, the 
obtained bound is always weaker (in this case $\ge$) than the optimal bound for maximum acyclic flow. 
$\Rightarrow$ the heuristic is correct for finding upper bounds for the acyclic flow on some arc.


This basic heuristic can still be improved without changing the main algorithm. We distinguish whether the flow is 
going over the bridge, going to a sink node in part $A$ of the transformed graph and thus having costs of 1 per unit, 
or is going from $s$ to $t$ directly with a cost of 2 per unit. The flow with costs of 2 cannot be sent over $e$ in 
forward direction neither find any sink in part $A$ of the graph. If there is a feasible flow in $G$ (which we tested) 
the only possible way is that the remaining flow has to go backward over $e$. So we can just reduce the flow on the 
bridge by this amount of flow directly going from $s$ to $t$ and the bound is still valid.
\end{proof}
