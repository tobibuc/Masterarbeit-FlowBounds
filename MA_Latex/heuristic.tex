\newpage
\subsection{A Path-Based Heuristic Approach}

So as we have seen, the computations of the MIP formulation with separation of the Acyclicity Constraints could be 
expensive in running time and ressource consumption. The solved problem is a relaxation anyway, so we can as well think 
about different relaxations and heuristic approaches. This chapter will introduce a simple and easy to 
implement heuristic approach. This approach uses a graph transformation and computes a Minmum Cost Flow on the 
transformed graph. We show that we get a valid bound for the Acyclic Flowbound Problem from this transformed graph.\\

The main idea of this approach is to avoid that the same flow is cycling over and over again until it reaches the 
capacity bounds. In order to achieve this we make a copy of the original graph and make the maximized arc the bridge 
between the two. On one part of the graph we only have sources, on the other part we only have sinks. In order to make 
the problem feasible we introduce artificial arcs between the sinks and their counterparts in the other copy of the 
graph. These artificial arcs get a detention cost so they are only used when there is no other way.