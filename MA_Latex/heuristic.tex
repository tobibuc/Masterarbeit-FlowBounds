\newpage
\subsection{A Simple Heuristic Approach}

So as we have seen, the computations of the MIP formulation with separation of the Acyclicity Constraints could be 
expensive in running time and ressource consumption. The solved problem is a relaxation anyway, so we can as well think 
about different relaxations and heuristic approaches. This chapter will introduce a simple and easy to 
implement heuristic approach. This approach uses a graph transformation and computes a Minmum Cost Flow on the 
transformed graph. We show that we get a valid bound for the Acyclic Flowbound Problem from this transformed graph.\\

The main idea of this approach is to avoid that the same flow is cycling over and over again until it reaches the 
capacity bounds. In order to achieve this we make a copy of the original graph and make the maximized arc the bridge 
between the two. On one part of the graph we only have sources, on the other part we only have sinks. In order 
to make 
the problem feasible we introduce artificial arcs between the sinks and their counterparts in the other copy of the 
graph. These artificial arcs get a detention cost so they are only used when there is no other way. On this modified 
graph we obtain minimum costs by sending as much flow as possible over the maximized arc. However this flow does not 
respect the capacity constraints neither completely forbids the cyclic flow. 

Now let us describe the approach in a formal way.

\subsubsection*{The Graph Transformation}

The following algorithm \ref{algo:graphtransform} describes how the graph is transformed to the new graph.

%Pseudocode des Algorithmus
\begin{algorithm}
 \caption{graph transformation}
 \label{algo:graphtransform}
 \begin{algorithmic}[5]
  \Function{MakeTransformedGraph}{$G=(V,A), e\in A$}
  \State create empty graph $G':=(V',A'), V'=A'=\emptyset$
  \State create labels capacity $A'\to \R$, cost $A'\to \R$% , demand $V'\to\R$
  \State create $s, t\in V$ \Comment{ supersource and supersink}
  \State create $a_0 :=(s,t) \in A'$
  \State capacity$(a_0)\gets\infty$ 
  \State cost$(a_0)\gets 2$ \Comment{highest arc cost in the network}
  \ForAll{$v\in V$}\Comment{make two copies of each node}
    \State create $v_A, v_B\in V'$
    \If{$v$ is source of $G$}
      \State create arc $a:=(s,v_A)\in A'$
      \State capacity$(a)\gets$ supply$(v)$
    \ElsIf{$v$ is sink of $G$}
      \State create arc $a:=(v_B, t)\in A'$
      \State create arc $b:=(v_A,t)\in A'$
      \State cost$(b)\gets 1$
    \EndIf
  \EndFor
  \ForAll{$a=(u,v)\in A$}
    \If{$a=e$}\Comment{for the arc to maximize we only make a bridge, no copy}
      \State create arc $e':=(u_A, v_B)\in A'$
      \State capacity$(e')\gets$ capacity$(e)$
    \Else
      \State create arcs $a_A:=(u_A, v_A), a_B:=(u_B, v_B)\in A'$
      \State capacity$(a_A)\gets$ capacity$(a)$, capacity$(a_b)\gets$ capacity$(a)$
      \State cost$(a_A)\gets 0$, cost$(a_B)\gets0$
    \EndIf
  \EndFor
  \Return $G'$
  \EndFunction
 \end{algorithmic}

\end{algorithm}

%TODO Beispiel mit Bild
TODO BILD-Beispiel 

With this transformation and any algorithm for Min Cost Flow we can set up an algorithm for our problem. 
For the Maximum Flow and Minimum Cost Flow Problem in graphs there are many standard algorithms we could use. The 
classical Maximum Flow algorithm of Ford and Fulkerson \cite{Ford-Fulkerson_algo} arising from their Max-Flow-Min-Cut 
theorem is based on augmenting flow on source-sink-paths in the network as well as the improved algorithms of Edmonds 
and Karp \cite{EdmondsKarp1972} or Dinic \cite{Dinic1970}. Algorithms for Maximum Flow can be used to first check if 
there is any feasible flow in the network. If there is no feasible flow we can give up at this point, while the flow 
problem on the transformed graph is always feasible by construction.

Deriving from the Max-Flow Algorithms there are many algorithms solving the Min-Cost-Flow Problem. Edmonds and 
Karp described a Successive Shortest Path Algorithm in \cite{EdmondsKarp1972}. Goldberg and Tarjan proposed the Minimum 
Mean Cycle Cancelling Algorithm \cite{minMeanCycleCancelling89} with runnning time of $\mathcal{O}(nm(log 
n)\min\{log(nC), m log n\}) $.
Another algorithm is the Network Simplex by Orlin \cite{NetworkSimplexOrlin97} (which was used in the computational 
study) and there are a lot more algorithms available for Minimum Cost Flow.

Our algorithm mainly applies a feasibility test, a transformation of the graph $G$ and a Minimum Cost Flow computation 
on the transformed graph $G'$, see algorithm \ref{algo:simpleheur}

\begin{algorithm}
 \caption{simple heuristic}
 \label{algo:simpleheur}
 \begin{algorithmic}
  \Function{FlowBoundHeur}{$G=(V,A), e\in A$}
  \State mf$\gets$ \Call{MaxFlow}{G}
  \If{mf < nominated ingoing flow}
    \State\Return infeasible
  \EndIf
  \State $G'=$ \Call{MakeTransformedGraph}{$G,e$}
  \State $f\gets$ \Call{MinCostFlow}{$G'$}
  \State ub$\gets f(e')$
  \State \Return ub
  \EndFunction
 \end{algorithmic}
\end{algorithm}

We have to show the correctness of the algorithm: %TODO and running time? dependent on used mincostflow!

\begin{prop}
 The heuristic algorithm \ref{algo:simpleheur} returns a valid upper bound for the possible acyclic flow on a given 
  arc $e$ in the network.
\end{prop}

\begin{proof}
 If a flow is acyclic we can decompose it into flow on a set of simple paths. On the other side this means that the 
Acyclic Flowbound Problem could theoretically be solved by an algorithm augmenting simple shortest paths where we put 
some extra constraints and objectives on these paths. These constraints have to forbid the closing of any flow cycle by 
the path as well as forcing the path to use arc $e$ if possible. \\ %TODO koennte das modell einzeln hinschreiben mit 
%hinweis auf NP schwere von constrained shortest path. anschließend kommt das relaxierungsargument besser !!!
%Außerdem hätte ich damit shconmal ein gutes modell

Our algorithm is a relaxation of this constrained simple path model. It still forces all paths that possibly can to go 
over $e$. But it relaxes the capacity constraints and allows weaker kinds of cycling.
%TODO das genauer ausführen!!!
Since it is a relaxation of the problem, the objective 

%  First we show that the value $f(e')$ of the Min-Cost-Flow on $e'$ is the maximum flow that can go over this arc if we 
% are in a setting where flow "particles" cannot pass over $e'$ twice. Second we show that this is an upper bound for the 
% Acyclic Flowbound Problem on the arc $e'$.\\
% 
% For the first part we assume to have an algorithm using simple source-sink-paths and augmenting as much flow as 
% possible on them. A simple path cannot go over any arc twice, so it is a specialization of 
\end{proof}



% \begin{algorithm}
%  \caption{path based heuristic flow bound algorithm}
% \label{algo:pathHeur}
%  %TODO gehe gerade von vorwärts gerichteter e aus, und maximierung
%  \begin{algorithmic}[5]%[0] means no line numbers, n means every nth line number is displayed
%   \Function{Heuristic}{$G=(V,A),\, e=(u,v)\in A,\, b:V\to \R $}%hier übergabeparameter
%     \State define $b':V\to \R,\, b'(v)\gets 0 \forall v\in V$ \Comment{temporary balances}
%     \State define $f: A\to\R,\, f(v)\gets 0 \forall a\in A$
%     \While {$\neg (b+b'\equiv 0) $}\Comment{while there are active source-sink-pairs}
