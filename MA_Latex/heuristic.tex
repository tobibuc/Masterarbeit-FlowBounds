So as we have seen, the computational results of the MIP formulation with separation of the Acyclicity Constraints 
are needing a lot of computing time and memory, which makes them too bad for using in practice. The solved problem is a 
relaxation anyway, so we can as well think about different relaxations and heuristic approaches. They might not yield 
optimal results, but as long as the bounds are valid and the computations take much less time they might be the better 
way in practice.

\caption{Flow Algorithms based on Path Augmentations}

Among the well-known algorithms for Max-Flow and Min-Cost-Flow Problems there are a number of which are based on 
searching for paths between sources and sinks in the network and augmenting the maximal possible flow among them until 
alls demands and supplies are satisfied. For a detailed description of these algorithms see [TODO FordFulkerson and 
EdmondsKarp and?]. 
Now you can come up with the following idea: If you always augment (simple) paths going over the arc $e$ we want to 
maximize, the algorithm can't send arbitrary amounts of flow in a cycle. Obviously one single path can't produce a 
cycle, but a combination of paths still can do, as example [TODO] illustrates. \\

%TODO das ist falsch,  man erhält keine 2-Approx. Kann man eventuell beweisen dass es keine gibt?

% But in the case that the nomination can be fulfilled by \textbf{only} augmenting paths including the arc $e$, we even 
% can prove that this gives a $2$-Approximation of the maximal acyclic flow on $e$:
% 
% \begin{prop}
%  Let $f: A\to \R$ be a flow, that comes from augmenting only simple paths going over an arc $e\in A$, and $f':A\to \R$ 
% an acyclic flow with $f'(e)\ge \overline{f(e)}$ for all acyclic flows $\overline{f(e)}:A\to \R$ ie $f'$ is a maximum 
% acyclic flow on $e$. Then $$ f'(e)\le f(e)\le 2f'(e)$$.
% \end{prop}
% 
% \begin{proof}
%  Since all flow was sent over the arc $e$, the amount of flow is at least as much as the maximum acyclic flow, so it 
% really leads to a valid upper bound.\\
% We have to show, that it is at most twice as much: Consider a cyclic flow that includes the arc $e$ and the paths as 
% they were when flow was augmented along them. A single path can't form a cycle, so there are at least two different 
% paths involved in building any cyclic flow. Take the involved path with the lowest augmented value. Just removing this 
% path would mean, that this cyclic flow disappears. But on the arc $e$ it cost only at most half of the value of the 
% involved arcs. Iterating this we can show that only half of the total flow is lost in this process. Of course, the 
% resulting flow is not maximal anymore, but is acyclic.
% %TODO
% \end{proof}
% 
% While a 2-Approximation would be a nice thing, there are still two major issues: 
\begin{itemize}
 \item What can we say in the case where the nomination is not fulfilled and we do not find any source-sink path over 
$e$?
 \item How do we ensure to find the simple paths that are \textit{going over one special arc $e$}? 
\end{itemize}

The first issue is kind of problematic, since in this case we can't say anything about approximation, as [TODO Bild 
machen] the simply example [...] is showing. As it is likely to happen in reality that we are not able to send all flow 
over one arc, we have to deal with fullfilling the nomination in this case, but without losing maximality. 
