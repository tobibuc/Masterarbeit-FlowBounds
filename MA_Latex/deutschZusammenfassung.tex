%TODO deutsche Zusammenfassung der Arbeit (ist vorgabe der studienordnung)
Die vorliegende Arbeit beschäftigt sich mit dem Problem, den maximalen azyklischen Fluss in einem 
Flussnetzwerk mit gegebenen Kapazitäten und Ein- bzw. Ausspeisungen an Quellen und Senken zu finden. 
Eine möglichst gute Schranke für den azyklischen Fluss ist ein Baustein für gutes Preprocessing 
für die Nominierungsvalidierung in Gasnetzen, denn Gasfluss ist druckinduziert und daher abgesehen von 
Verdichterstationen azyklisch. 
Das Problem wird definiert und Bezüge zu verwandten algorithmischen Problemen hergestellt. 
Für eine leichte Verallgemeinerung des Problems wird NP-Vollständigkeit bewiesen und eine 
Reduktion dieses Problems angegeben. 
Zur Lösung des Azyklischen Flusschrankenproblems werden verschiedene Modelle als (Quadratic) Mixed Integer Program vorgestellt
sowie eine Heuristik beschrieben. Es werden Ungleichungen eines der MIP-Modelle
untersucht und Kriterien für deren Redundanz gefunden. 
Die verschiedenen Lösungsansätze wurden implementiert und werden in ihren Ergebnissen, Auswirkungen auf
die Modellgröße im Preprocessing und Laufzeitverhalten auf Testinstanzen verglichen.