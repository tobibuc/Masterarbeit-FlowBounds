%TODO deutsche Zusammenfassung der Arbeit (ist vorgabe der studienordnung)
Die vorliegende Arbeit beschäftigt sich mit dem Problem, den maximalen azyklischen Fluss in einem 
Flussnetzwerk mit gegebenen Kapazitäten sowie Ein- und Ausspeisungen an Quellen und Senken zu finden. 

Eine möglichst gute Schranke für den azyklischen Fluss ist ein Baustein für gutes Preprocessing 
für die Nominierungsvalidierung in Gasnetzen. Gasfluss ist druckinduziert und daher (abgesehen von Verdichterstationen) 
azyklisch. Es ist möglich, einen Fluss mit minimalen Kosten in einem Netzwerk mit einem negativen Gewicht auf der zu 
maximierenden Kante zu berechnen um den Flusswert der Kante zu maximieren. Dieser Ansatz führt allerdings zu wesentlich 
schwächeren Schranken als der azyklische (in der Tat ist die Lücke zu azyklischen Flussschranken unbeschränkt, wie 
bewiesen wird). Gleichzeitig ist das Erzwingen von Azyklizität des Flusses vor allem in dem Fall der Extrema des 
Flusswertes interessant und in sonstigen Anwendungen nicht nötig. Einen beliebigen azyklischen Fluss kann man in einem 
Netzwerk immer erhalten wenn man irgendeinen Fluss gegeben hat. Dieser erfüllt aber nicht die Maximalitätsbedingungen.


Die Arbeit ist folgendermaßen strukturiert: Das Problem und wichtige mathematische Grundlagen werden zunächst definiert 
und Bezüge zu verwandten algorithmischen Problemen hergestellt. 
Für das Problem des kostenminimalen Flusses mit negativen Gewichten wird NP-Vollständigkeit bewiesen und eine Reduktion 
dieses Problems angegeben. 
Zur Lösung des Azyklischen Flussschrankenproblems werden verschiedene Modelle als (Quadratic) 
Mixed Integer Program vorgestellt. Es werden Ungleichungen zur Erzwingung azyklischen Flusses aus einem der Modelle 
untersucht und Kriterien für deren Redundanz/benötigte Aanzahl angegeben und bewiesen. Ein theoretisches Modell für ein 
pfadbasierte Flussaugmentierung wird untersucht und eine Heuristik für das Problem entwickelt.
Die verschiedenen Lösungsansätze wurden implementiert und werden in ihren Ergebnissen, Auswirkungen auf
die Modellgröße im Preprocessing und Laufzeitverhalten auf Testinstanzen verglichen.