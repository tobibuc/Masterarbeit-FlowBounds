%main section about the different models 



\subsection{MIP Formulations}
For many combinatorial Problems it is the best practical solution to formulate them as a Mixed Integer Problem and just 
solve this problem with modern MIP Solvers such as CPlex, Gurobi, SCIP etc. Often there are different possible 
formulations as 
MIP, which might yield very different running times due to numerical or algorithmical reasons. In our problem, it is 
easy to model the flow conservation and the ingoing and outgoing flow on vertices. Like described before, we can assign 
a negative weight to an arc variable and this way maximize flow over this arc by minimizing the overall cost. This 
would be the typical MIP formulation of a min cost flow:

\begin{align*}
  &\min \sum_{a\in A} w_a\cdot x_a  \\
  %TODO an die balance-intervall formulierung anpassen
 s.t. & \sum_{a\in \delta^+(v)}x_a - \sum_{a\in\delta^- (v)}x_a = d_v\ &\forall v\in V \\
  & c_l(a)\le x_a \le c_u(a) & \forall a\in A
\end{align*}

This model still allows cyclic flow, so we have to find constraints to avoid cyclicity. 
\subsubsection{Model 1: Node Potentials}
%TODO welche formulierungen koennte es noch geben fuer dieses problem?
%TODO PORTA results and polytope of the problem?

One idea (that is unfortunately not a linear problem formulation anymore) would be to set potentials on the nodes and 
allow only flow from higher to lower potential. This is quite close to the application of our problem in gas flow 
computation, where gas will only flow from places with high pressure to places with lower pressure in the network. So 
each vertex $v$ would get a variable $\pi_v$ for its potential, each arc $a$ as before a variable $x_a$ for the amount 
of flow. Then we add the constraint $$x_a\cdot (\pi_v -\pi_w)\ge 0\,\forall a=(v,w)\in A $$
We have to ensure that the values of the potentials are all 
different, or that there is no flow between nodes of the same potential. Otherwise any solution where all potentials 
are set to the same value would fulfill this constraint, regardless if it is acyclic or not. So in practice we could 
set a constant $\varepsilon > 0$ to describe the minimum distance between the potentials. A feasible 
solution now has to fulfill the constraint above on every arc. If we set up the constraint as above, we get the Mixed 
Integer Nonlinear Program

\begin{align*}
  &\min \sum_{a\in A} w_a\cdot x_a & \\
 s.t. & \sum_{a\in \delta^+(v)}x_a - \sum_{a\in\delta^- (v)}x_a &=& d_v\ &\forall v\in V \\
 & x_a &\le& c_u(a) & \forall a\in A\\
 & -x_a &\le& c_l(a) & \forall a\in A\\
 & -x_a\cdot (\pi_v -\pi_w)&\le& 0 &\forall a=(v,w)\in A\\
 & (\pi_v - \pi_w)^2 &\ge& \varepsilon &\forall v,w \in V\\
 & x_a \in \R & & &\forall a\in A\\
 & \pi_v \in \R & & & \forall v\in V\\
 & d_a \in \{0,1\} & & &\forall a\in A\\
\end{align*}

We show that this flow indeed is an acyclic one:

\begin{prop}
 A flow which fulfills the constraints of the above nonlinear program is always acyclic.
\end{prop}
\begin{proof}
 Assume a flow fulfilling the above constraints would have cyclic flow on a cycle $C$. Without loss of generality 
we label the vertices on the cycle from $1$ to $n$ and assume all arcs are directed forward on this cycle. This means 
there are arcs $a_1=(v_1,v_2),a_2=(v_2,v_3), \dots a_n=(v_n, v_1)\in C$ such that $x_{a_i} > 0 \,\forall i={1,\dots , 
n}$. Also we know from the constraints  $x_a\cdot (\pi_v -\pi_w)\ge 0$ and $\pi_v - \pi_w \neq 0$ (Hence $\Rightarrow 
\pi_v\neq\pi_w)\, \forall v,w\in V$. So we conclude an ordering of the vertices potentials: $x_{a_1}\cdot (\pi_{v_1} 
-\pi_{v_2})\ge 0 \Rightarrow \pi_{v_1}>\pi_{v_2}$ and so on. This yields a sequence $\pi_{v_1}>\pi_{v_2}>\dots 
>\pi_{v_n}>\pi_{v_1}$, which is a contradiction. \lightning
\end{proof}

However, solving nonlinear mixed integer programs comes with many difficulties, so we will use indicator constraints 
or an equivalent Big-M constraint to get a mixed integer formulation avoiding these difficulties. For this we  replace 
$$x_a\cdot (\pi_w -\pi_v)\le 0 \,\forall a=(v,w)\in A $$ and $$(\pi_v - \pi_w)^2 \ge \varepsilon \,\forall v,w \in V$$

%introduce rho (=richtung) variables
Instead we introduce a new binary variable $\rho_{vw}$ to indicate flow direction on $a=(v,w)$, and say this variable 
has to be 1 if flow is from $v$ to $w$ and 0 if it is from $w$ to $v$. Also if $\rho_{vw}$ is $1$ the potential 
$\pi_v$ of $v$ has to be greater than the potential $\pi_w$ of $w$ and the other way around. So with indicator 
constraints we get
\begin{align*}
 &\rho_{vw}=1 &\Rightarrow &\pi_v\ge\pi_w +1\\
 &\rho_{vw}=0 &\Rightarrow &\pi_w\ge\pi_v +1\\
 &\rho_{vw}=1 &\Rightarrow &f((v,w))\ge 0\\
 &\rho_{vw}=0 &\Rightarrow &f((v,w))\le 0\\
 &\rho_{vw} \in \{0,1\}&&
\end{align*}
We can express these indicator constraints by
\begin{align*}
 &\pi_v-\pi_w &\ge & N(\rho_{vw}-1)+1\\
 &\pi_v-\pi_w&\le &N\rho_{vw}-1 \\
 &f((v,w))&\ge& M(\rho_{vw}-1)\\
 &f((v,w))&\le & M\rho_{vw}\\
 &\rho_{vw} \in \{0,1\}&&\\
\end{align*}
where $N\in \R$ and $M\in \R$ are sufficient big constants. Sufficient big in this case could mean we choose 
$N=|V|\, ,M=\sum_{v\in V}|d(v)|$. \\
So the complete Mixed Integer Program is:

\begin{align*}
  &\min \sum_{a\in A} w_a\cdot x_a & \\
 s.t. & \sum_{a\in \delta^+(v)}x_a - \sum_{a\in\delta^- (v)}x_a &=& d_v\ &\forall v\in V \\
 & x_a &\le& c_u(a) & \forall a\in A\\
 & x_a &\ge& c_l(a) & \forall a\in A\\
 & \pi_v-\pi_w &\ge & N(\rho_{vw}-1)+1& \forall a=(v,w)\in A\\
 & \pi_v-\pi_w&\le &N\rho_{vw}-1& \forall a=(v,w)\in A\\
 & x_a&\ge& M(\rho_{vw}-1)& \forall a=(v,w)\in A\\
 & x_a&\le & M\rho_{vw}& \forall a=(v,w)\in A\\
 & x_a \in \R & & &\forall a\in A\\
 & \pi_v \in \R & & & \forall v\in V\\
 & d_a \in \{0,1\} & & &\forall a\in A\\
 & \rho_{vw} \in \{0,1\}&&& \forall a=(v,w)\in A\\
\end{align*}
%TODO show that these constraints are equivalent to the nonlinear ones? or make a comment at least?

\subsubsection{Model 2: Acyclicity Constraints On All Cycles}

Another idea (which was implemented in Lamatto by Robert Schwarz) is the following: Every arc has a direction and the 
sign of the flow on this arc tells us in which direction flow is send over this arc. Again we introduce variables 
$\rho_a\in \{0,1\}$ for each arc $a\in A$ that indicate the direction of flow and are coupled with the flow variables:
\begin{align*}
\rho_a=1 & \Rightarrow x_a\ge 0 \\
\rho_a=0 & \Rightarrow x_a\le 0
\end{align*}
These indicator constraints can be handled the way they are by standard MIP solvers, but to make them real MIP 
constraints we have to formulate them as follows (with $M$ a constant big enough, that means bigger than any possible 
flow $x_a$ , e.g. $M=\sum_{v\in V}|d_v|$) :
\begin{align*}
 x_a + M\cdot (1-\rho_a) &\ge 0\\
 x_a - M\cdot \rho_a & \le 0
\end{align*}



Now we have decision variables for the flow direction, we add constraints to avoid cycles. There should be at least 
one arc in each direction, so we get $$ 1\le\sum_{a\in C\textrm{ forward }} \rho_a + \sum_{a\in C\textrm{ backward 
}}1-\rho_a\le n-1$$
Let us formulate this in a slightly different way to get only one sum: For each cycle of size $C_n=C_l+C_m$ 
let $C_m$ be the number of arcs directed forward and $C_l$ the number of backward directed arcs. We define that always 
$C_l\le C_m$, so forward is defined as the direction where more arcs are pointing towards (left or right would only 
make 
sense in a planar embedding). Then we get the constraint $$1-l \le \sum_{a\in C}\rho_a\le n-(l+1)$$ that forbids any
cyclic flow on $C$. We will show later under which conditions such constraints also forbid cyclic flow on other cycles. 
%TODO was ist wenn gleich viele? Ist das als Definition so ueberhaupt ok?

So our MIP formulation of the model is finally
\begin{align*}
 &\min \sum_{a\in A} w_a\cdot x_a & \\
 s.t. & \sum_{a\in \delta^+(v)}x_a - \sum_{a\in\delta^- (v)}x_a = d_v\ &\forall v\in V \\
  & c_l(a)\le x_a \le c_u(a) & \forall a\in A\\
 &x_a + M\cdot (1-\rho_a) \ge 0 & \forall a\in A\\
 &x_a - M\cdot \rho_a \le 0 & \forall a\in A\\
 &1-l \le \sum_{a\in C}\rho_a \le n-(l+1) & \forall \textrm{ cycle }C\in G\\
 & x_a \in \R &\forall a\in A\\
 & \rho_a \in \{0,1\} &\forall a\in A
\end{align*}
or if we normalize the MIP to only $\le$ inequalities:
\begin{align}
 &\min \sum_{a\in A} w_a\cdot x_a & \\
 s.t. & \sum_{a\in \delta^+(v)}x_a - \sum_{a\in\delta^- (v)}x_a &=& b_v\ &\forall v\in V \\
 & x_a &\le& c_u(a) & \forall a\in A\\
 & -x_a &\le& c_l(a) & \forall a\in A\\
 &-x_a - M\cdot (1-\rho_a) &\le& 0 & \forall a\in A\\
 &x_a - M\cdot \rho_a &\le& 0 & \forall a\in A\\
 &1-l - \sum_{a\in C}\rho_a &\le& 0 & \forall \textrm{ cycle }C\in G\\
 & \sum_{a\in C}\rho_a +(l+1)-n &\le& 0 & \forall \textrm{ cycle }C\in G\\
 & x_a \in \R & & &\forall a\in A\\
 & \rho_a \in \{0,1\} & & &\forall a\in A
\end{align}
%TODO gibt es noch andere Möglichkeiten als die von Robert? bestimmt!?



\subsection{The number of cycles we have to forbid}

\newpage
\subsection{A Path-Based Heuristic Approach}