We already described the gas flow problem, where the motivation for this thesis came from. Here we want to define the 
problem as a general combinatorial flow problem with specific contraints. We will also give the formulation as a Mixed 
Integer Program (MIP) and as well some natural relaxations, which might be easier to solve.

We will represent our originally undirected graph by a directed graph where flow is allowed to go over arcs backward 
and forward as well. This allows us to specify directions forward and backward on every arc in a consistant way. 
\section{Problem Definition}

\begin{definition}
 Let $G=(V,A)$ be a directed Graph and $e \in A$ a specific arc of $G$. For every vertex $v\in V$ let there be a 
prescribed range for the amount of flow demand $[d_l(v), d_u(v)]\subset \R \setminus \{0\}$. Vertices with demand 
values greater than 0 are sinks, vertices with negative demand value are sources. In a feasible solution flow 
demands are assumed to be balanced on their (absolute) higher bound values 
$$\sum_{v \in V\textrm{ sink}}d_u(v)-\sum_{v \in V\textrm{ source}}d_l(v)=0.$$ 
%TODO am ende kontrollieren, dass demands (was positiv, was negativ) konsistent verwendet wurde!

Let there be a capacity function 
$$c:A\to \mathcal{P}(\R^{|A|}), \, c(a)=[c_l(a), c_u(a)],\, c_l(a)\le 0 \le c_u(a) \, \forall a\in A$$
and a flow  $f: A\to\R^{|A|} $ with $$c_l(a)\le f(a)\le 
c_u(a)\, \forall a\in A$$ and $$\sum_{a\in\delta^-(v)}f(a)-\sum_{a\in\delta^+(v)}f(a)-d(v) = 0 \, \forall v\in V$$
We call this a \textit{feasible network flow on G}, i.e. a flow with balances of 0 and respected capacity bounds.
We also call $$\sum_{a\in \delta^-(v)}f(a)-\sum_{a\in\delta^+(v)}f(a)-d(v)$$ the flow balance, which might be nonzero 
in incomplete flows during a solving algorithm.\\
\end{definition}
Note that a flow value can have a negative sign in this definition. This is due to the fact that we want to have a 
consistant notion of direction on an arc but also allow flow in both directions. A negative sign is simply saying that 
flow is going in backward direction. We will speak of maximizing flow on an arc, which in this setting means maximizing 
the absolute value of flow in a prescribed direction. So either we look for the maximum value in forward direction or 
the (usually nagative) minimum value in backward direction.\\

The standard definition of Flow Problems has fixed demand values $d(v)=const$ instead of an interval. These two 
definitions are equivalent in the way that you can transform them into each other easily in polynomial time. Obviously 
we can see fixed demands $d(v)$ as intervals with $d_l(v)=d_u(v)=d(v)$ consisting of only one point $[d(v), d(v)]$. The 
other way around we use a construction that adds only one node and $|\{v\in V| d_u(v)<0\}|+|\{v\in V| d_l(v)>0\}|$ arcs 
to the network.
\begin{prop}
 There is a (linear time) transformation from the flow problem with demand intervals on the vertices to the standard 
flow problem with fixed demands. 
\end{prop}
\begin{proof}
 Given a graph $G=(V,A)$ with demand intervals at the entries and exits, we construct a graph $G'=(V',A')$ as follows:
 $$V' := V\cup \{z\} \textrm{, and } A' := A\cup \{ (v,z)| d_u(v)<0 \}\cup\{ (z,v)|d_l(v)>0\}$$ On $G'$ we then have 
to redefine the arc capacity and node demands: 
$$c':A\to \mathcal{P}(\R)  \,, c'(a)= c'(u,v) = \begin{cases} \{x|x\in [0,d_u(u)-d_l(u)]\} &\textrm{if }v=z\\ 
\{x|x\in [0,d_u(v)-d_l(v)]\}&\textrm{if }u=z\\ \{x|x\in[c_l(a), c_u(a)]\} &\textrm{else } \end{cases}
$$ 
The node demands now are 
$$ d'(v)=\begin{cases} d_l(v) & \textrm{if v is source, i.e. }d_u(v)<0\\
         d_u(v) &\textrm{if v is sink, i.e. }d_u(v)>0
        \end{cases}
$$
If we compute a solution of the flow problem on $G'$ and restrict the flow to the arcs of $G$, we get the solution of 
the problem on $G$ that respects the demand intervals:\\ 
We call the demands in the solution restricted to arcs of $G$ $d^s$ and have $ d^s(u)=d'(u)+f'((u,z))$. The flow balance 
constraints in normal nodes 
(nodes with demand 0) do not change at all, since the only new node $z$ is not in their neighborhood. All vertices that 
are sources or sinks (per definition thy cannot be both at the same time) have $z$ as neighbor. The flow balance on a 
source $u$ before and after restriction to $G$ differs only by the flow on the artificial arc $(u,z)$. This flow 
$f'((u,z))$ is in the capacity range $[0,d_u(u)-d_l(u)]$. Hence for sources $u$ ($d'(u)<0$) with 
$d^s(u)=d'(u)+f'((u,z))$ we can conclude for the demands $d^s(u)$ of the solution restricted to $G$ 
$$d_l(u)\le d^s(u)=d'(u)+f'((u,z))=d_l(u)+\underbrace{f'((u,z))}_{0\le f'((u,z))\le d_u(u)-d_l(u)} \le d_u(u)$$
and as well for sinks ($d'(v)>0$)
$$d_l(v)\le d_u(v)-\underbrace{f'((z,v))}_{0\le f'((z,v))\le d_u(v)-d_l(v)}=d'(v)-f'((z,v))=d^s(v) \le d_u(v)$$
\end{proof}



 %cycle was already defined in the Definitions chapter
\begin{definition}
Given a feasible network flow $f:A\to \R^{|A|}$ in a network $G=(V,A)$. Let us say $f$ can be \textit{split} or we 
\textit{split the network flow} $f$ into flows $f_1$ and $f_2$, $f=f_1+f_2$, if there exist flows $f_1, f_2:A\to 
\R^{|A|}$ such that 
\begin{enumerate}
 \item both divided flows $f_1$ and $f_2$ are again feasible network flows for demands $d_1, d_2$
 \item For all arcs the flow directions in $f_1, f_2$ and $f$ are the same, i.e. 
  $$f(a)\ge 0\Rightarrow f_i(a)\ge 0,\, f(a)\le 0 \Rightarrow f_i(a)\le 0 \,\textrm{ for } i=\{1,2\}$$
 \item The demands of both flows sum up to the demand of $f$ and only at sources and sinks of $f$ there are sources 
 or sinks of the split flows: 
  $$d(v)=0\Rightarrow d_1(v)=d_2(v)=0$$ $$d(v)=d_1(v)+d_2(v) \textrm{ for all }v\in V$$
 \item The flow values of both flows sum up to the flow value of $f$:$$f(a)=f_1(a)+f_2(a) \textrm{ for all }a\in A$$
\end{enumerate}
\end{definition}

\begin{definition}
 We say a network flow $f:A\to \R^{|A|}$ in $G=(V,A)$ contains \textit{cyclic flow} if $f$ can be split into two flows 
$f_1$ and $f_2$ in a way such that one of the flows (say $f_1$) is a nonzero flow, but all demands $d_1$ for the flow 
are zero. This part of the flow is called \textit{cyclic flow}. 
Formally this is:
$ \exists f_1, f_2:A\to \R^{|A|}$ such that: 
\begin{enumerate}
 \item $f=f_1+f_2$
 \item $ d_1\equiv 0 $
 \item $ f_1\neq 0$
\end{enumerate}
% If the cyclic flow is maximal the other part of the flow is acyclic. 
A flow is called acyclic if it does not contain cyclic flow.\\

Any cyclic flow can be split into different cyclic flows in a way such that in each of these the arcs with nonzero 
flow value form a simple cycle and all have the same absolute flow value. We call this part a flow cycle and the value 
on the arcs its size.

% A \textit{cyclic flow} within a feasible network flow $f$ is a maximal flow $f':C\to \R$ where $C\subseteq A$ is a 
% cycle, $$\sum_{a\in \delta^+(v)\cap C}f(a)-\sum_{a\in\delta^-(v)\cap C}f(a) = 0 \, \forall v\in C$$ for all arcs the 
% flow directions in $f'$ and $f$ are the same, i.e. $$f(a)\ge 0\Rightarrow f'(a)\ge 0,\, f(a)\le 0 \Rightarrow f'(a)\le 
% 0$$ and there is really a nonzero flow, i.e. $$f'(a) \ne 0 \,\forall a\in C$$. 
% 
% If we can find any cyclic flow $f'$ in a network flow $f$ we say that $f$ contains a flow cycle. If there is no cyclic 
% flow, we say $f$ is \textit{acyclic} or an \textit{acyclic flow} on $G$.\\

% The \textit{size of a cyclic flow} we call the maximal absolute amount of flow contributing to the cycle. This means 
% it is the value $|f'(a)|$ a cyclic flow $f'$ has on an arc and the amount of flow that is really cycling. If we reduce 
% flow on each arc of the cycle by the size of the flow there should be at least one arc with zero flow resulting since 
% we said that it is maximal.

%flow $ f: C\to \R ,\, C\subseteq A \textrm{ a cycle, }\,c_l(a)\le 
%f(a)\le c_u(a)\, \forall a\in C$ with the property that 
\end{definition}
%BILD fuer die illustration der definition mit der Aufteilung machen
\begin{figure}[h!]
\centering
\begin{tikzpicture}
\node (s) at (0,0)[source]{s};
\node (t) at (3,0)[sink]{t};
\node (1) at (1,0)[vertex]{};
\node (2) at (2,0)[vertex]{};
\node (3) at (1,1)[vertex]{};
\node (4) at (2,1)[vertex]{};
\draw[arcflow] (s)--(1);
\draw[arcflow] (1)--(2);
\draw[arcflow] (2)--(t);
\draw[arcflow] (2)--(4)--(3)--(1);
\draw[arc] (s)--(1)node [pos=0.5, below, blue]{$3$};
\draw[arc] (1)--(2)node [pos=0.5, below, blue]{$5$};
\draw[arc] (2)--(4)node [pos=0.5, right, blue]{$2$};
\draw[arc] (4)--(3)node [pos=0.5, below, blue]{$2$};
\draw[arc] (3)--(1)node [pos=0.5, left, blue]{$2$};
\draw[arc] (2)--(t)node [pos=0.5, below, blue]{$3$};
\end{tikzpicture}
\caption{This flow is cyclic - as figure \ref{bild:dividedFlow} shows it contains a cycle}
 \label{bild:undividedFlow}
\end{figure}

\begin{figure}[h!]
\centering
\begin{tikzpicture}[scale=0.7]
\node (s) at (0,0)[source]{s};
\node (t) at (3,0)[sink]{t};
\node (1) at (1,0)[vertex]{};
\node (2) at (2,0)[vertex]{};
\node (3) at (1,1)[vertex]{};
\node (4) at (2,1)[vertex]{};
\draw[arcflow] (s)--(1);
\draw[arcflow] (1)--(2);
\draw[arcflow] (2)--(t);
% \draw[arcflow] (2)--(4)--(3)--(1);
\draw[arc] (s)--(1)node [pos=0.5, below, blue]{$3$};
\draw[arc] (1)--(2)node [pos=0.5, below, blue]{$3$};
\draw[arc] (2)--(4)node [pos=0.5, right, blue]{$0$};
\draw[arc] (4)--(3)node [pos=0.5, below, blue]{$0$};
\draw[arc] (3)--(1)node [pos=0.5, left, blue]{$0$};
\draw[arc] (2)--(t)node [pos=0.5, below, blue]{$3$};

\node (s) at (4,0)[source]{s};
\node (t) at (7,0)[sink]{t};
\node (1) at (5,0)[vertex]{};
\node (2) at (6,0)[vertex]{};
\node (3) at (5,1)[vertex]{};
\node (4) at (6,1)[vertex]{};
% \draw[arcflow] (s)--(1);
\draw[arcflow] (1)--(2);
% \draw[arcflow] (2)--(t);
\draw[arcflow] (2)--(4)--(3)--(1);
\draw[arc] (s)--(1)node [pos=0.5, below, blue]{$0$};
\draw[arc] (1)--(2)node [pos=0.5, below, blue]{$2$};
\draw[arc] (2)--(4)node [pos=0.5, right, blue]{$2$};
\draw[arc] (4)--(3)node [pos=0.5, below, blue]{$2$};
\draw[arc] (3)--(1)node [pos=0.5, left, blue]{$2$};
\draw[arc] (2)--(t)node [pos=0.5, below, blue]{$0$};
\end{tikzpicture}
\caption{ This is the flow of the above figure \ref{bild:undividedFlow} split into an acyclic and a cyclic part.}
 \label{bild:dividedFlow}
\end{figure}

\begin{definition}
  Given a graph $G=(V,A)$ like above, with $e \in A$ a specific arc of $G$. 
  
We call the problem of finding an acyclic flow 
$$f:A\to\R^{|A|}\textrm{ with }f(e)\ge f'(e)\,\forall f':A\to\R^{|A|}\textrm{ s.t.} f'\textrm{ is an acyclic flow on 
}G$$
  the \textit{ Acyclic Flowbound Problem}.
\end{definition}

For describing the algorithms and manipulations of network flows we will need further definitions. 
\begin{definition}
 Assume we are in the situation that we  have a current flow $f$ in the given network $G$ and we have a set 
$S\subseteq A$ of arcs with a direction forward or backward for each - for example a path from a source to a sink. We 
call the act of setting flow values to
$$f'(a)= f(a)+x \,\forall a\in S:\textrm{ a is forward, }f'(a)= f(a)-x \,\forall a\in S:\textrm{ a is backward}$$
in a way that $f'$ again is a feasible flow for a (maybe different) set of flow demands $d':V\to\R^{|V|}$ 
\textit{ augmenting a value $x$ on $S$}. 

If we augment a cycle flow the demand values do not change. We also call this \textit{shifting flow on a cycle} in $G$.

\end{definition}


\begin{prop}
  The maximum flow value on the specified arc $e$ in the Acyclic Flow Problem is the same as the 
  maximum flow value on this arc in the case where only cyclic flow on cycles containing $e$ is forbidden. 
\end{prop}
This proposition is stating that only a small part of the cycles in the network are important. It is not necessary 
to add the acyclicity constraints on all cycles - they do only make a difference if the arc we maximize is contained in 
the cycle. If we have any feasible flow in a network, then we also have an acyclic flow there. We can just shift flow 
along cycles until one arc has zero flow, and repeat this procedure until the flow is acyclic. The proof does the same 
with cycles not containing the specified arc $e$, reasoning these cycles do not impact the optimal solution on this
maximized arc.
\begin{proof}
 Let $F_{acyc}$ be the set of flows that are completely acyclic and $F_{cyc}^e$ the flows that can have cyclic flow on 
 cycles not containing $e$. Since $F_{acyc}\subseteq F_{cyc}^e $ (i.e. the partially cyclic flow problem is a 
 relaxation of the acyclic problem) we always have $f_{cyc}(e)\ge f_{acyc}(e)$ for all $f_{cyc}\in F_{cyc}^e, 
f_{acyc}\in F_{acyc}$ that are maximizing flow on $e$. We have to show that also $f_{cyc}(e)\le f_{acyc}(e))$.
% Every two flows differ only by a sequence of cyclic flow augmentations in the network. Thus if not $f_{cyc}(e)\le 
% f_{acyc}(e)$ we can choose a pair $f,f'$ such that $f$ is completely acyclic and $f'$ is not, $f'(e)>f(e)$ and that 
% $f'$ differs from $f$ by just one cyclic shift of flow (take the first flow of the sequence where $f'(e)>f(e)$).

%%%%%%%%%%%%%%%%%%%%%%%%%%%%%%%%%%%%%%%%%%%%%%%
%observation: it is always easier to look at a cycle and ask how much flow can you send from one node to another. this 
%not changing, never, by a cyclic flow. but a cyclic flow could affect other cycles. But since we do not have any 
% lower bounds higher than 0 we can always shift a cyclic flow until it is acyclic!!!
%%%%%%%%%%%%%%%%%%%%%%%%%%%%%%%%%%%%%%%%%%%%%%%
% In $f_{cyc}$ there are no cyclic flows on the maximized arc $e$. Take a solution of $f_{cyc}$ with cyclic flow on a 
% simple cycle $C\in A(G)$. There is no minimum flow higher than 0. Thus it is always possible to shift the cyclic flow 
% backwards on the cycle by the minimum amount of flow on this cycle. This is a cyclic flow, so on every arc $a\in C$ the 
% flow value $f(a)$ is decreased by this. However, there is no arc where flow directions are changed. We can do this 
% operation as long as there is cyclic flow without changing flow value on $e$ or creating new flow cycles. In the 
% end we get an acyclic flow with the same flow value.
We show how we can produce a completely acyclic flow $f'\in F_{acyc}$ from any given flow $f_{cyc}\in F_{cyc}^e$. 

This flow will have the same flow value on $e$: $f'(e)=f_{cyc}(e)$ , so we prove for each flow $f\in F_{cyc}^e$ there 
is an acyclic flow with at least the same flow value on $e$.\\

Take a given flow $f\in F_{cyc}^e\setminus F_{acyc}$. Consider a cycle $C$ where $f$ has cyclic flow. On $C$ all flow 
is going in the same direction. Choose the minimum flow value $\alpha(C)=\min(f(a)|a\in C)$ on an edge of $C$ and shift 
flow by $-\alpha$ (this means decrease flow on every edge of $C$ by $\alpha$). On no arc the direction of flow was 
reversed, so there are no new cyclic flows created by this. But on at least one arc there is no flow at all now, 
so on the cycle $C$ we have no cyclic flow anymore. 
The resulting flow is still a valid flow: On each node ingoing and outgoing flow is decreased by the same value $\alpha$ 
so flow balances still hold. The capacity constraints are still fulfilled because we know that no minimum flow bound is 
higher than 0 in our problem. The flow on the special arc $e$ is unchanged, since $e$ must not be contained in any 
cyclic flow.

We can continue this until there is no cyclic flow left and we transformed our flow into a flow $f'\in F_{acyc}$ with 
the same flow value on $e$ as before. The algorithm terminates after less than $|A(G)|$ repetitions because every arc 
can be set to zero only once.\\
$\Rightarrow $ We conclude that $$\max(f(e)|f\in F_{cyc}^e)\le \max(f'(e)|f'\in F_{acyc})$$ and together with 
$F_{acyc}\subseteq F_{cyc}^e $ this gives us
$$\max(f(e)|f\in F_{cyc}^e)= \max(f'(e)|f'\in F_{acyc}) $$
\end{proof}


Now we can investigate the problem deeper. The Acyclic Flowbound Problem is obviously closely related to other flow 
problems though the special constraints we built make it more complicated. One way to deal with this is to look at 
related problems, which could give bounds on our problem or even the same solution in special cases.

\section{Relation To Other Flow Problems/Relaxations}
\label{sect:mincostflow}
If we look for related problems, it is natural to just drop the constraints of our problem. In this case we could for 
example easily forget the contraint that flows have to be acyclic. If we do this, our problem becomes a special 
instance of a Min-Cost-Flow where we allow edge weights to be negative:

\begin{definition}\label{def:mincostflow}
 The \textit{Minimum Cost Flow Problem} is the problem to determine a feasible network flow with the least possible 
cost $c_f$. The cost function $c:A\to\R^{|A|}$ (or often $c:A\to\R^{|A|}_{\ge 0}$) is defined on every arc, the cost of 
the flow 
is $$c_f := \sum_{a\in A}|f(a)|\cdot c(a)$$ So for a given network and cost function we look for a feasible network 
flow $f$ with $$c_f\le c_{f'}\,\forall f':A\to\R^{|A|}$$ i.e. with minimum overall costs.
\end{definition}

This problem is normally defined with nonnegative cost functions, like in \cite{EdmondsKarp1972} where 
Jack Edmonds and Richard Karp present their well known $O(V\cdot E^2)$ flow algorithm. At least cycles of negative 
weight should be excluded, since they lead to problems in the algorithm's subroutines like shortest path. Nevertheless 
there are algorithms that can handle negative cycles without a problem - but they will of course give a solution 
where as much flow as possible is just going around these negative cycles. 
%, and make the problem hard. We will proof the hardness of this problem later.TODO auch ohne kreisfreiheit?

In our application we want to find upper and lower bounds on the possible flow over each arc, or equivalently the 
maximum possible flow on this arc in forward and backward direction. For this we set the 
weight on the arc $e$ where we want to maximize flow to -1 and compute a Minimum Cost Flow in the graph. If $e$ is 
contained in a cycle of $G$ whose arcs all have a very high upper capacity bound, $e$ will get a very high bound as 
well. The reason is that any cyclic flow over $e$ can reduce the cost of the flow. Hence as much cyclic flow as 
possible will be used to obtain a Minimum Cost Flow. 

So in most cases the bound computed with a Min-Cost Flow algorithm is not sharp. Still it gives an upper bound for the 
possible flow. One could ask how good this bound will be, and whether we can use it as an approximation. It turns 
out that the gap between the amount of flow on an arc $e$ in an acyclic flow and in a possibly cyclic flow can be 
arbitrarily large:

\begin{prop}
 The gap between the maximum flow values $f(a)$ on an arc $a\in A$ in a normal and in an acyclic flow problem is 
unbounded. 
\end{prop}
\begin{proof}
 In order to show this we look at the two examples:
 \begin{figure}[h!]
  \centering
  \begin{tikzpicture}
  \node (s) at (0,0)[source, label=below:{s, $d(s)=-1$}]{-1};
%   \node (stext) at (-0.1, -0.2){s, $d(s)=1$};
    \node (t) at (2,0)[sink, label=below:{t, $d(t)=1$}]{1};
%   \node (ttext) at (2.1, -0.2);
  \node (3) at (1,1.5)[vertex]{};
%   \node(etext) at (1.7,0.9){e};

  % \draw [edgeflow] (s)--(3);
  \draw[arcflow] (s)--(3);
  \draw[arcflow] (3)--(t);
  \draw [edge](s) -- (3);
  \draw [edge](s) -- (t);
  \draw[fatarc] (3) -- (t)node [pos=0.6, above] {$e$};;
  \end{tikzpicture}
 \caption{A simple network with acyclic flow}
 \label{bild:acyclicTriangle}
 \end{figure}
 
 \begin{figure}[h!]
  \centering
  \begin{tikzpicture}
  \node (t) at (2,0)[sink, label=below:{t, $d(t)=1$}]{1};
  \node (s) at (0,0)[source, label=below:{s, $d(s)=-1$}]{-1};
  \node (3) at (1,1.5)[vertex]{};

  \draw[arcflow=red] (s)--(3)--(t)--(s)--(3)--(t);
  \draw [edge](s) -- (3);
  \draw [edge](s) -- (t);
  \draw[fatarc] (3) -- (t)node [pos=0.6, above] {$e$};
  \end{tikzpicture}
 \caption{The same simple network with cycling flow that can pass $e$ several times}
 \label{bild:cyclicTriangle}
 \end{figure}
In figure \ref{bild:acyclicTriangle} we see that the flow on the maximized arc $e$ in the acyclic case has to be 1. 
If we allow cyclic flow as in figure \ref{bild:cyclicTriangle} it could be any number - it is the minimum we choose as 
capacity for an arc in the network. Hence the relative gap is
$$ \frac{f_{cyc}(e)}{f_{acyclic}(e)}=\frac{\textrm{min capacity on the cycle}}{1}$$ which is unbounded.


Figure \ref{bild:acycliczeroflow} shows a graph where the flow on $e$ in the acyclic case is $0$. If we do not forbid 
cyclic flows, 
the flow on $e$ will always be $\min_{a\in \textrm{ cycle }C}c_u(a)$ like in the first picture. So the gap in relative 
numbers $$ \frac{f_{cyc}(e)}{f_{acyclic}(e)}=\frac{\min_{a\in \textrm{ cycle }C}(c_u(a))}{0}$$ is not even defined!
  
\begin{figure}[h!]
\centering
\begin{tikzpicture}
\node (s) at (0,0)[source]{s};
% \node (stext) at (-0.1, 0.2){s};
\node (t) at (0,2)[sink]{t} ;
% \node (ttext) at (-0.1, 1.8){t};
\node (3) at (1,1)[vertex]{};
\node (4) at (2,2)[vertex]{};
\node (5) at (2,0)[vertex]{};

\draw[arcflow] (s)--(3)--(t);
\draw [edge](s) -- (3)--(5)--(4);
\draw [edge](t) -- (3);
\draw[fatarc] (3) -- (4)node [pos=0.6, above] {$e$};
\end{tikzpicture}
\caption{In the acyclic case there is no flow at all on the arc $e$ in this network}
\label{bild:acycliczeroflow}
\end{figure}
\end{proof}

As a different relaxation we could weaken the capacity constraints, but still insist on an acyclic flow that is maximum 
on an arc $e$. In the normal Max-Flow Problem dropping the capacity constraints reduces the problem to finding a 
shortest path between the source and the sink. 

In our case, if there is only one source and one sink, we have to decide whether there is a simple path from the 
source to the sink that contains the maximized arc $e$. If there are more sources and sinks, we already have the 
problem of finding a set of simple paths containing $e$. Since the general constrained shortest path problem is 
$\mathcal{NP}$ hard this seems not to be promising and won't be disscussed here.
%TODO haerte des Problems? Wie loest man es?

%
% Is this problem easy to solve, and how to solve it? TODO
%

